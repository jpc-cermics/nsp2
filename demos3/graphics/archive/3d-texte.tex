Les macros pour faire des plot3d et manipuler les donn�es 

-1-  premiere facon de faire 
     x,y,z  ou x et y sont deux vecteurs et z une matrice 

-2-  On peut avoir envie de generer une version facettes de 
     cela i.e x,y,z trois matrices dont chaque colonne est 
     une facette carr�e.

eval3dp 
eval3d 
genfac3d 
nf3d

plot3d2
plot3d3

dans demos/graphics/surface on trouve 

eval3d evaluates a function on a grid assuming that the function 
   accept arrays inputs 
   ie is able to compute 
   f(x,y) -> f(x(i),y(i))

eval3dp calcule en plus les facettes 
   c'est donc normalement eval3d+genfac3d

genfac3d: genere des facettes a partir de x,y,z 
   ou x et y sont deux vecteurs et z une matrice 

nf3d: passe d'un mode facettes aproprie au surfaces parametrique 
      a des facettes standard 

quelques examples 
------------------

function z=f(x,y)
  z=cos(x).*sin(y)
endfunction

x=%pi*(-1:0.2:1);
y=%pi*(-1:0.2:1);
z=eval3d(f,x,y);
plot3d(x,y,z);
[xx,yy,zz]=genfac3d(x,y,z);
plot3d(xx,yy,zz);

// cas de la sphere 
     u = linspace(-%pi/2,%pi/2,10);
     v = linspace(0,2*%pi,10);
     x= cos(u)'*cos(v);
     y= cos(u)'*sin(v);
     z= sin(u)'*ones(v);
 
//ici chacun des x,y,z pourrait etre obtenu par un 
//eval3d 
//a partir des trois matrices on veut des facettes 

[xx,yy,zz]=nf3d(x,y,z);

plot3d(xx,yy,zz)

// couleurs par sommet -> interpolated shading 

plot3d(xx,yy,list(zz,30*(zz-min(zz))/(max(zz)-min(zz))+1))










