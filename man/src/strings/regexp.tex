% -*- mode: latex -*-

\mansection{regexp}
\begin{mandesc}
  \short{regexp}{convert letter to upper case} \\
  \short{regsub}{convert letter to lower case} \\
\end{mandesc}
% -- Calling sequence section
\begin{calling_sequence}
\begin{verbatim}
[b,A]=regexp(str,regexp)
str=regsub(str,regexp,regsub)
\end{verbatim}
\end{calling_sequence}

% -- Parameters
\begin{parameters}
  \begin{varlist}
    \vname{str} : a string.
    \vname{b}:  a boolean.
    \vname{A}:  a scalar Matrix.
    \vname{regexp,regsub}: two strings
  \end{varlist}
\end{parameters}

\begin{mandescription}
\noindent \verb+regexp+ and \verb+regsub+ are used to search or substitute regular expression
in a given string. The syntax for specifying regular expression is the \verb+tcl+ 
regexp syntax. 
\verb+regexp+ returns \verb+%t+ in a boolean \verb+b+ if a match was found and the 
position of the matched string in the original string is returned in 
the first line of matrix \verb+A+ (first character, last character). 
Note that position of sub-expressions if asked (see example below) 
are also returned in the other lines of \verb+A+. 

In the \verb+regsub+ string matched sub expressions can be specified with the 
syntax \verb+\\n+ where \verb+n+ stands for the \verb+n+-th sub-expression of 
\verb+regexp+.
\end{mandescription}

\begin{examples}
  \begin{program}
    [a,b]=regexp('pipopopopa','p(i[po]+)(pa)')
    regsub('pipopopopa','p(i[po]+)(pa)','poo\\1')
  \end{program}
\end{examples}

% -- see also

