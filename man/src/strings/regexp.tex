% -*- mode: latex -*-

\mansection{regexp}
\begin{mandesc}
  \short{regexp}{search regular expressions} \\
  \short{regsub}{substitutions based on regular expression pattern matching} 
\end{mandesc}
% -- Calling sequence section
\begin{calling_sequence}
\begin{verbatim}
[b,A]=regexp(str,regexp, nocase=%t|%f)
[Str,A]=regsub(Str,regexp,regsub, all=%t|%f, nocase=%t|%f)
\end{verbatim}
\end{calling_sequence}

% -- Parameters
\begin{parameters}
  \begin{varlist}
    \vname{str} : a string
    \vname{Str} : a string matrix.
    \vname{b}:  a boolean.
    \vname{A}:  a scalar Matrix.
    \vname{regexp,regsub}: two strings
  \end{varlist}
\end{parameters}

\begin{mandescription}
\noindent \verb+regexp+ and \verb+regsub+ are used to search or substitute regular expression
in a given string. The syntax for specifying regular expression follows the \verb+tcl+ 
regexp syntax.

\verb+regexp+ returns \verb+%t+ in a boolean \verb+b+ if a match was found and the 
position of the matched string in the original string is returned in 
the first line of matrix \verb+A+ (first character, last character). 
Note that position of sub-expressions if asked (see example below) 
are also returned in the other lines of \verb+A+. If the optional argument \verb+nocase+ 
is set to \verb+%t+, both \verb+Str+ and \verb+regexp+ are converted to lower case before string matching.

\verb+regsub+ returns the string matrix obtained after pattern matching substitutions. 
If requested \verb+A+ is a numerical matrix which contains for each string the number of successful matches. 
In the \verb+regsub+ string, matched sub-expressions can be specified using the 
syntax \verb+\\n+ where \verb+n+ stands for the \verb+n+-th sub-expression of 
\verb+regexp+. The optional argument \verb+all+ (default value \verb+%t+) can be set to 
\verb+%f+ to limit substitution in each string to the first match. If the optional argument \verb+nocase+ 
is set to \verb+%t+, both \verb+Str+ and \verb+regexp+ are converted to lower case before string matching 
substitutions. 

\end{mandescription}

\begin{examples}
  \begin{program}
    [a,b]=regexp('pipopopopa','p(i[po]+)(pa)')
    regsub('pipopopopa','p(i[po]+)(pa)','poo\\1')
  \end{program}
\end{examples}

% -- see also

