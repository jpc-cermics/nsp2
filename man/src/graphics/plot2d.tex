% -*- mode: latex -*-

\mansection{plot2d}
\begin{mandesc}
  \short{plot2d}{draw two dimensional graphs}
\end{mandesc}

% -- Calling sequence section
\begin{calling_sequence}
\begin{verbatim}
plot2d(x,y,[leg=str1, leg_pos=str2, style=val1, rect=val2, axesflag=val3])
\end{verbatim}
\end{calling_sequence}
% -- Parameters
\begin{parameters}
  \begin{varlist}
    \vname{x} : abscissa vector. It can be empty, in this case \verb+[1:length(y)]+ is used.
    \vname{y} : vector or matrix of numbers.
    \vname{style=val1} : optional named argument, sets the style for each curve. \verb+val1+
    must be a vector of integers.
    \vname{rect=val2} : optional named argument, sets the bounds of the plotting
    window. \verb+val2+ must be a vector of integers: \verb+[xmin, ymin, xmax, ymax]+.
    \vname{leg=str1} : optional named argument, str1 must be a string of the form
    \verb+'leg1@leg2@...'+ where \verb+legi+ is the legend associated with the i-th curve.
    \vname{leg_pos=str2} : optional named argument, only useful when \verb+leg=str1+
    is specified. \verb+str2+ is one of these strings ``dr'' (downer right corner), ``ur'' (upper
    right corner), ``dl'' (downer left corner), ``ul'' (upper left corner)
    indicating the position of the legend.
    \vname{axesflag=val3} : optional named argument, specified how axes are
    drown. \verb+val3+ is an integer between \verb+0+ and \verb+5+
    \begin{itemize}
    \item \verb+val3=0+ : no axes are drawn.
    \item \verb+val3=1+ : axes are drawn with the y axis on the left.
    \item \verb+val3=2+ : a box is drawn around the plot.
    \item \verb+val3=3+ : axes are drawn with the y axis on the right.
    \item \verb+val3=4+ : axes are drawn centred in the middle of the plotting window
    \item \verb+val3=5+ : axes are drawn so to cross at the point \verb+(0,0)+.
    \end{itemize}
  \end{varlist}
\end{parameters}

\begin{mandescription}
  plot2d plots a set of two dimensional curves.

  If \verb+y+ is a matrix, each column of \verb+y+ is plotted against vector
  \verb+x+. In this case.

  If \verb+x+ is empty and \verb+y+ is a vector, \verb+plot2d([], y, opt_arg)+ is
  equivalent to \verb|plot2d((1:size(y,'*')), y, opt_arg)|.

  If \verb+x+ is empty and \verb+y+ is a matrix, \verb+plot2d([], y, opt_arg)+ is
  equivalent to \verb|plot2d((1:size(y,'r')), y, opt_arg)|.  
\end{mandescription}

\begin{examples}
  \begin{program}
    x=(-%pi:0.01:%pi);
    //simple plot
    plot2d(x, sin(x));

    //multiple plot
    plot2d(x, [sin(x)', cos(x)'], leg='sin@cos', style=[3,4], leg_pos='ul');

    x=(-3:0.01:3)
    xtitle('density of the normal distribution')
    plot2d(x, exp(-x.*x/2)/sqrt(2*%pi), rect=[-3,0,3,1]);
  \end{program}
\end{examples}

\begin{manseealso}
  xtitle
\end{manseealso}

% -- Authors
\begin{authors}
   jl
\end{authors}
