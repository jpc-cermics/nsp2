% -*- mode: latex -*-
% Copyright Lelong
\mansection{plot2d}
\begin{mandesc}
  \short{plot2d}{two dimensional plot}
\end{mandesc}

% -- Calling sequence section
\begin{calling_sequence}
\begin{verbatim}
 plot2d(x,y,options)
 plot2d(x,f,options)
\end{verbatim}
\end{calling_sequence}
% -- Parameters
\begin{parameters}
  \begin{varlist}
    \vname{x}: abscissa vector or matrix. It can be empty and in that case
    \verb+x=[1:length(y)]+ is used.
    \vname{y}: vector or matrix of numbers. When \verb!y! is a matrix,
    each column of the matrix describes a curve. When \verb!y! is a row or
    column vector then \verb!y! describes a curve.
    The matrix \verb!x! can be emty, can have the same dimension as \verb!y!
    or can be a vector of \verb!length(size(y,'r'))! when \verb!y! is a matrix.
    \vname{f}: a nsp function.
    \vname{axesflag}: optional named argument which specify how axes are drawn. Its is
    an integer between \verb+0+ and \verb+5+:
    \begin{varlist}
    \item \verb+0+ : no axes are drawn.
    \item \verb+1+ : axes are drawn with the y axis on the left.
    \item \verb+2+ : a box is drawn around the plot.
    \item \verb+3+ : axes are drawn with the y axis on the right.
    \item \verb+4+ : axes are drawn centred in the middle of the plotting window
    \item \verb+5+ : axes are drawn so to cross at the point \verb+(0,0)+.
    \end{varlist}
    \vname{style}: optional argument which gives the color to use for each curve or
    the mark. It can be replaced by the following optional arguments which are
    vectors giving one value for each curve.
    \begin{varlist}
    \vname{line_color}: a vector giving the color to be used for each curve (\verb!-1! means use the
    default color, \verb!-2! means do not draw whith colored line).
    \vname{line_thickness}: the thickness to be used for each curve.
    \vname{mark}:  a vector giving the mark to be used for each curve (\verb!-1! means use the
    default mark, \verb!-2! means do not draw whith marks).
    \vname{mark_color}:  a vector giving the mark color to be used for each curve (\verb!-1! means use the
    default  color).
    \vname{mark_size}: a vector giving the size to be used for each mark (\verb!-1! means use the
    default size).
    \end{varlist}
    \vname{leg}: optional named argument which must be a string of the form
    \verb!'leg1@leg2@...'! where \verb!legi! is the legend associated with the i-th curve.
    \vname{leg_pos}: optional named argument used with \verb!leg! to specify the legend position.
    It can be one of the strings \verb!'dr'! (down right corner), \verb!'ur'! (up right corner), \verb!'dl'! (down left corner), \verb!'ul'!  (up left corner),
    \verb!urm! (up right margin), \verb!drm! (down right margin).
    \vname{logflag}: a string of length two (\verb!'nn'!, or \verb!'nl'!, or \verb!'ln'! or
    \verb!'ll'!) where \verb!n! stands for normal and \verb!l! stands for log.
    \vname{nax}: a vector \verb![stx,tx,sty,ty]! which is used when
    \verb!auto_axis! is false. It gives the number of intervals \verb!tx! and
    \verb!ty! and subintervals \verb!stx! and \verb!sty! used for drawing x-axis
    and y-axis.
    \vname{rect}: optional named argument, sets the bounds of the plot. It is a vector of length four
    \verb+[xmin, ymin, xmax, ymax]+.
    \vname{strf}: deprecated replaced by specifying \verb!rect!, \verb!axesflag!.
    \vname{auto_axis}: a boolean if true axes are graduated in automatic mode if
    false \verb!nax! is used.
    \vname{iso}: when true iso mode is used.
    \vname{mode}: a string which can be \verb!"std"!, or \verb!"stairs"!, or \verb!"stem"!,
    or \verb!"arrow"!, or \verb!"fill"!, or \verb!"stairs_fill"!. It give the drawing mode
    used for curves.
  \end{varlist}
\end{parameters}

\begin{mandescription}
  plot2d plots a set of two dimensional curves.

  If \verb+y+ is a matrix, each column of \verb+y+ is plotted against vector
  \verb+x+.

  If \verb+x+ is empty and \verb+y+ is a vector, \verb+plot2d([], y, opt_arg)+ is
  equivalent to \verb|plot2d((1:size(y,'*')), y, opt_arg)|.

  If \verb+x+ is empty and \verb+y+ is a matrix, \verb+plot2d([], y, opt_arg)+ is
  equivalent to \verb|plot2d((1:size(y,'r')), y, opt_arg)|.
\end{mandescription}

\begin{examples}

\noindent A simple example

\begin{mintednsp}{nsp}
x=-%pi:0.01:%pi;
plot2d(x, sin(x));
\end{mintednsp}

\noindent multiple plots

\begin{mintednsp}{nsp}
x=-%pi:0.01:%pi;
plot2d(x, [sin(x)', cos(x)'], leg='sin@cos', style=[3,4], leg_pos='ul');
\end{mintednsp}

\noindent using \verb!rect!

\begin{mintednsp}{nsp}
x=-3:0.01:3;
xtitle('density of the normal distribution')
plot2d(x, exp(-x.*x/2)/sqrt(2*%pi), rect=[-3,0,3,1]);
\end{mintednsp}
\end{examples}

\begin{manseealso}
  \manlink{xtitle}{xtitle}
\end{manseealso}
% -- Authors
\begin{authors}
  J.-Ph. C.
\end{authors}
