% -*- mode: latex -*-

\mansection{contour}
\begin{mandesc}
  \short{contour}{level curves on a 3D surface}\\ % @mandesc@
\end{mandesc}
%-- Calling sequence section
\begin{calling_sequence}
\begin{verbatim}
  contour(x,y,z,nz,options)
\end{verbatim}
\end{calling_sequence}

%-- Parameters
\begin{parameters}
  \begin{varlist}
    \vname{x,y} vectors of respective size \verb!n1! and \verb!n2! and in that case they give x-axis
    and y-axis coordinates of the surface.
    \vname{z}:  a matrix of size \verb!n1 x n2! (\verb!z(i,j)! is the
    value of the surface at the point \verb!(x(i),y(j))!) or a nsp function.
    \vname{nz}: the level values or the number of levels.
    \begin{varlist}
      \vname{-}If \verb!nv! is scalar, its value gives the
      number of level curves equally spaced from zmin to zmax as
      follows \verb!z= zmin + (1:nv)*(zmax-zmin)/(nv+1)!
      Note that the \verb!zmin! and \verb!zmax!
      levels are not drawn (generically they are reduced to points)
      but they can be added with
      \begin{Verbatim}
        [im,jm] = find(z == zmin); // or zmax
        plot2d(x(im)',y(jm)',-9)
      \end{Verbatim}
      \vname{-}If \verb!nv! is a vector, \verb!nv(i)! gives
      the value of the ith level curve.
    \end{varlist}
    \vname{alpha}: see \manlink{plot3d}{plot3d}
    \vname{ebox}: see \manlink{plot3d}{plot3d}
    \vname{flag}: see \manlink{plot3d}{plot3d}
    \vname{leg}: see \manlink{plot3d}{plot3d}
    \vname{theta}: see \manlink{plot3d}{plot3d}
    \vname{zlevel}: position at which the level curves are drawn on a 3d plot.
  \end{varlist}
\end{parameters}

\begin{mandescription}
  \verb!contour! draws level curves of a surface \verb!z=f(x,y)!. The level curves are
  drawn on a 3D surface. The optional arguments, except \verb!zlev!, are the same as
  for the function \verb!plot3d!. The parameter \verb!flag(1)=mode! has a special meaning
  described now:
  \begin{description}
  \item[mode=0] : the level curves are drawn on the surface defined by (x,y,z).
  \item[mode=1] : the level curves are drawn on a 3D plot and on the plan defined
    by  the equation z=zlev.
  \item[mode=2] : the level curves are drawn on a 2D plot.
  \end{description}

  You can change the format of the floating point number printed on the levels
  by using \verb!xset("fpf",string)! where \verb!string! gives the
  format in C format syntax (for example \verb!string="%.3f"!). Use
  \verb!string=""! to switch back to default format and Use
  \verb!string=" "! to suppress printing.
  Usually we use \verb!contour2d! to draw levels curves on a 2D plot.
  Enter the command \verb!contour()! to see a demo.
\end{mandescription}

% --example
\begin{examples}
  \begin{Verbatim}
    t=linspace(-%pi,%pi,30);
    function z=my_surface(x,y),z=x*sin(x)^2*cos(y),endfunction
    contour(t,t,my_surface,10,flag=[0,2,1])
  \end{Verbatim}
\end{examples}
% -- see also
\begin{manseealso}
  \manlink{contour2d}{contour2d} \manlink{plot3d}{plot3d}
\end{manseealso}
% -- Author
\begin{authors}
  J.-Ph. C.
\end{authors}
