% -*- mode: latex -*-
\mansection{champ}
\begin{mandesc}
  \short{champ}{2D vector field plot}\\ % @mandesc@
  \short{champ1}{2D vector field plot}\\ % @mandesc@
\end{mandesc}
%-- Calling sequence section
\begin{calling_sequence}
\begin{verbatim}
  champ(x,y,fx,fy,arfact=,rect=,strf=)
  champ1(x,y,fx,fy,arfact=,rect=,strf=)
\end{verbatim}
\end{calling_sequence}
%-- Parameters
\begin{parameters}
  \begin{varlist}
    \vname{x,y}: two vectors which define the grid points
    \vname{fx,fy}: two matrices which describes the x and y component of the vector field.
    \vname{arfact}: an optional argument of type real which gives a scale factor
    for the drawing of the arrows head (default value is 1.0)
    \vname{rect}: a vector \verb!rect=[xmin,ymin,xmax,ymax]! which gives
    the boundaries of the graphics frame to use.
    \vname{strf}: a string of length 3 "xyz" which has the same meaning as the
    \verb!strf! parameter of \verb!plot2d!. The first
    character x has no effect with \verb!champ!.
  \end{varlist}
\end{parameters}

\begin{mandescription}
  The functions \verb!champ! and \verb!champ1! are used to draw two dimensional
  vector fields. When using \verb!champ!, the length of the arrows is proportional to the intensity of the field.
  When using \verb!champ1! the color id used to draw the arrow is proportional to the intensity of the field.
\end{mandescription}
%--example
\begin{examples}
  \noindent A simple example with \verb!champ!
  \begin{Verbatim}
    champ(-5:5,-5:5,rand(11,11),rand(11,11));
  \end{Verbatim}
  \noindent A simple example with \verb!champ1!
  \begin{Verbatim}
    champ1(-5:5,-5:5,rand(11,11),rand(11,11),arfact=2,rect=[-10,-10,10,10]);
  \end{Verbatim}

  \noindent Add a grid and a mark at each point of the grid.
  \begin{Verbatim}
    x=linspace(-%pi,%pi,20);
    y=x;
    gray=xget('lastpattern')+3;
    xsegs([x;x],[min(y)*ones(size(x));max(y)*ones(size(x))],style=gray);
    xsegs([min(x)*ones(size(y));max(x)*ones(size(y))],[y;y],style=gray);
    champ1(x,y,sin(x'*y),cos(x'*y));
    [X,Y]=ndgrid(x,y);
    plot2d(X(:),Y(:),line_color=-2,mark=13,mark_size=3,mark_color=1);
  \end{Verbatim}

\end{examples}
%-- see also
%-- Author
\begin{authors}
  J.-Ph. C.
\end{authors}
