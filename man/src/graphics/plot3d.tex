% -*- mode: latex -*-
\mansection{plot3d}
\begin{mandesc}
  \short{plot3d}{3D plot of a surface}\\
  \short{plot3d1}{3D plot of a surface}\\
\end{mandesc}
% -- Calling sequence section
\begin{calling_sequence}
\begin{verbatim}
  plot3d(x,y,z,args=,alpha=,colormap=,colors=,ebox=,
         flag=,leg=,theta=,mesh=,mesh_only=,shade=);

  plot3d(x,y,f,args=,alpha=,colormap=,colors=,ebox=,
         flag=,leg=,theta=,mesh=,mesh_only=,shade=);
\end{verbatim}
\end{calling_sequence}

%-- Parameters
\begin{parameters}
  \begin{varlist}
    \vname{x,y,z} first possible case: \verb!x! and \verb!y! can be vectors of
    respective size \verb!n1! and \verb!n2! and in that case they give x-axis
    and y-axis coordinates of the surface and \verb!z! must be a matrix of size
    \verb!n1 x n2!.  \verb!z(i,j)! is the value of the surface at the point
    (x(i),y(j)).
    %
    \vname{x,y,z} second possible case: \verb!x! and \verb!y! and \verb!z!  are
    matrices of size \verb!nf x n!.  In that case, they define the \verb!n!
    facets which compose the surface. The i-th gacet is defined by a polygon
    with \verb!nf! points.  The coordinates of the points of the i-th facet are
    given respectively by \verb!xf(:,i)!, \verb!yf(:,i)! and \verb!zf(:,i)!.
    %
    \vname{colors}: a vector of size \verb!n! giving the color of each facets or
    a matrix of size \verb!nf x n! giving color at each facet extreme
    points. Facet colors are then interpolated or averaged depending on the
    optional \verb!shade! parameter.
    %
    \vname{theta, alpha}: optional arguments giving in degree the spherical coordinates of the
    observation point.
    %
    \vname{leg}: optional string defining the labels for each axis with \verb!@! as a field
    separator, for example \verb!"X@Y@Z"!.
    %
    \vname{flag}: optional real vector of size three.
    \verb!flag=[mode,type,box]!.
    \begin{varlist}
      \vname{mode}: an integer (surface color).
      \begin{varlist}
        \vname{mode$>$0}: the surface is painted with color
        \verb!"mode"! ; the boundary of the facet is drawn
        with current line style and color.\vname{mode=0:}a mesh of the surface is drawn.
        \vname{mode$<$0:}the surface is painted with color \verb!"-mode"! ; the boundary of the facet is not
        drawn.
      \end{varlist}
      \vname{type}: an integer (scaling).
      \begin{varlist}
        \vname{type=0}: the plot is made using the current 3D scaling.
        \vname{type=1}: rescales automatically 3d boxes with extreme aspect
        ratios, the boundaries are specified by the value of the
        optional argument \verb!ebox!.
        \vname{type=2}: rescales automatically 3d boxes with extreme aspect
        ratios, the boundaries are computed using the given
        data.
        \vname{type=3}: 3d isometric with box bounds given by optional \verb!ebox!, similarily to
        \verb!type=1!.
        \vname{type=4}: 3d isometric bounds derived from the data, similarily tp \verb!type=2!.
        \vname{type=5}: 3d expanded isometric bounds with box bounds given by optional \verb!ebox!, similarily to
        \verb!type=1!.
        \vname{type=6}: 3d expanded isometric bounds derived from the data,
        similarily to \verb!type=2!.
      \end{varlist}
      \vname{box}: an integer (frame around the plot).
      \begin{varlist}
        \vname{box=0}: nothing is drawn around the plot.
        \vname{box=1}: unimplemented (like box=0).
        \vname{box=2}: only the axes behind the surface are drawn.
        \vname{box=3}: a box surrounding the surface is drawn and captions are added.
        \vname{box=4}: a box surrounding the surface is drawn, captions and
        axes are added.
      \end{varlist}
    \end{varlist}
    %
    \vname{ebox}: vector \verb![xmin,xmax,ymin,ymax,zmin,zmax]! used to fix the
    boundaries of the plot. This argument is used when the optional argument
    \verb!flag! is given with type set to \verb!1!, or \verb!3!, or \verb!5!.
    \vname{mesh}: a boolean specifying if the mesh is to be drawn.
    \vname{mesh_only}: a boolean specifying if only the mesh is to be drawn.
  \end{varlist}
\end{parameters}

\begin{mandescription}
  When \verb!x! and \verb!y! are vectors of respective size \verb!n1! and
  \verb!n2! and \verb!z! is a matrix of size \verb!n1 x n2!  a parametric
  surface which value $z(i,j)$ at point $(x(i),y(j)$ is drawn.  holes can be
  inserted in the surface by setting some values to \verb!%nan!.

  When \verb!x!, \verb!y!, and \verb!z! are matrices of the same size. They
  describe a set of facets which are drawn.  In that case, specific color for
  each facet can be given through the optional parameter \verb!color!.

  When \verb!plot3d1! is used instead of \verb!plot3d!, z-values are mapped to
  colors of the current colormap.
\end{mandescription}
%--example
\begin{examples}

\noindent a first simple example with \verb!plot3d!

\begin{Verbatim}
  t=-%pi:0.3:%pi;plot3d(t,t,sin(t)'*cos(t));
\end{Verbatim}

\noindent a first simple example with \verb!plot3d1! (the default
colormap is a jetcolormap).

\begin{Verbatim}
  t=-%pi:0.3:%pi;plot3d1(t,t,sin(t)'*cos(t));
\end{Verbatim}

\noindent same example with mesh only

\begin{Verbatim}
  t=-%pi:0.3:%pi;plot3d(t,t,sin(t)'*cos(t),mesh_only=%t);
\end{Verbatim}

\noindent same example without mesh

\begin{Verbatim}
  t=-%pi:0.3:%pi;plot3d(t,t,sin(t)'*cos(t),mesh=%f);
\end{Verbatim}

\noindent same example with a function

\begin{Verbatim}
  function z=f(x,y,args) z=sin(args(1)*x)*cos(args(2)*y);endfunction;
  // the optional argument arg is used to pass extra arguments to function f.
  t=-%pi:0.3:%pi;plot3d(t,t,f,args=list(1,2));
\end{Verbatim}

\noindent computing facets and plot facets with colors

\begin{Verbatim}
  t=-%pi:0.3:%pi;
  z=sin(t)'*cos(t);
  [xx,yy,zz]=genfac3d(t,t,z);
  plot3d(xx,yy,zz,colors=linspace(1,40,400),colormap=hotcolormap(40));
  // remode shade i.e one color per face
  xclear();
  plot3d(xx,yy,zz,colors=linspace(1,40,400),colormap=hotcolormap(40),shade=%f);
\end{Verbatim}

multiple plots by combining facets

\begin{Verbatim}
  t=-%pi:0.3:%pi;
  z=sin(t)'*cos(t);
  [xx,yy,zz]=genfac3d(t,t,z);
  z=sin(2*t)'*cos(t);
  [m,n]=size(xx);
  [xx1,yy1,zz1]=genfac3d(t,t,z);
  plot3d([xx,xx1],[yy,yy1],[zz,zz1],colors=[20*ones(1,n),30*ones(1,n)],colormap=hotcolormap(40));
\end{Verbatim}

multiple plots using opengl

\begin{Verbatim}
  xinit(opengl=%t);
  z=sin(t)'*cos(t);
  [xx,yy,zz]=genfac3d(t,t,z);
  [m,n]=size(xx);
  plot3d(xx,yy,zz,colors=[20*ones(1,n)],colormap=hotcolormap(40));
  z=sin(2*t)'*cos(t);
  [xx,yy,zz]=genfac3d(t,t,z);
  plot3d(xx,yy,zz,colors=[30*ones(1,n)],colormap=hotcolormap(40));
\end{Verbatim}
\end{examples}

%-- see also
\begin{manseealso}
  \manlink{eval3dp}{eval3dp} \manlink{genfac3d}{genfac3d} \manlink{geom3d}{geom3d}
  \manlink{param3d}{param3d} \manlink{plot3d1}{plot3d1} \manlink{fec}{fec}
  \manlink{xdel}{xdel}
\end{manseealso}

%-- Author

\begin{authors}
  J.Ph.C and B. Pin�on
\end{authors}
