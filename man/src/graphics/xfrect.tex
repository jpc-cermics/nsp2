% -*- mode: latex -*-
\mansection{xfrect}
\begin{mandesc}
  \short{xfrect}{fill a rectangle}\\ % @mandesc@
\end{mandesc}
%-- Calling sequence section
\begin{calling_sequence}
\begin{verbatim}
  R=xfrect(x,y,w,h,options=value);
  R=xfrect(rect,options=value) 
\end{verbatim}
\end{calling_sequence}

%-- Parameters
\begin{parameters}
  \begin{varlist}
    \vname{x,y,w,h}: four real values defining the rectangle.
    \vname{rect}: a vector of size \verb!4! giving the rectangle
    \verb![x,y,w,h]!.
    \vname{color, stroke_color, thickness}: optional arguments (default values -1, -2, -1).
    \vname{R}: a graphical rectangle object.
  \end{varlist}
\end{parameters}

\begin{mandescription}
  \verb!xfrect! fills a rectangle defined by \verb![x,y,w,h]! 
  (upper-left point, width, height) using the current scale and style.
  The fill color, stroke color and thickness can be given by optional arguments
  \verb!color!, \verb!stroke color! and \verb!thickness!. 

  If \verb!color! is positive it gives the color to be used for painting,
  if \verb!color=-1! the default color is used, if  \verb!color=-2! the
  rectangle is not filled.

  If \verb!stroke_color! is positive it gives the color to be used for drawing the rectangle 
  boundary. if \verb!stroke_color=-1! the default color is used, if  \verb!stroke_color=-2! the rectangle 
  boundary is not drawn. The parameter \verb!thickness! control the line thickness for drawing boundaries.
\end{mandescription}

%--example
\begin{examples}

A first example

  \begin{mintednsp}{nsp}
    xsetech(frect=[-2,-2,2,2]);
    xfrect(-1,1,2,2,color=6,thickness=4,stroke_color=3);
  \end{mintednsp}

A second example

  \begin{mintednsp}{nsp}
    xsetech(frect=[-2,-2,2,2]);
    xfrect(-1,1,2,2,color=-1,thickness=4,stroke_color=3);
    // we change the default color
    xset('color',10);
    F=get_current_figure();
    F.invalidate[];// force redrawing;
  \end{mintednsp}
\end{examples}
%-- see also
\begin{manseealso}
  \manlink{xrect}{xrect} \manlink{xrects}{xrects} 
\end{manseealso}
%-- Author
\begin{authors}
  J.-Ph. C.  
\end{authors}

