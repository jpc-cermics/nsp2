% -*- mode: latex -*-

\mansection{xs2ps}
\begin{mandesc}
  \short{xexport}{export graphics to a file}\\ % @mandesc@
\end{mandesc}
%-- Calling sequence section
\begin{calling_sequence}
\begin{verbatim}
  xexport(winid,filename,color=,mode=figure_background=);
  xs2ps(winid,filename,color=,mode=figure_background=);
  xs2fig(winid,filename,color=,mode=figure_background=);
  xs2pdf(winid,filename,color=,mode=figure_background=);
  xs2png(winid,filename,color=,mode=figure_background=);
  xs2svg(winid,filename,color=,mode=figure_background=);
  xs2ps(winid,filename,color=,mode=figure_background=);
  xs2eps(winid,filename,color=,mode=figure_background=);
  xs2ps_old(winid,filename,color=,mode=figure_background=);
\end{verbatim}
\end{calling_sequence}
%-- Parameters
\begin{parameters}
  \begin{varlist}
    \vname{winid}: integer scalar.
    \vname{filename}: a file name given as a string.
    \vname{color}: optional boolean. 
    \vname{figure\_background}: a boolean. 
  \end{varlist}
\end{parameters}
\begin{mandescription}
  The function \verb!xexport! export graphics of the graphic window \verb!winid! to 
  the file \verb!filename! in a format selected from the filename suffix. 
  The possible suffixes are ".svg", ".pdf", ".eps", ".ps", ".fig", ".png". 
  The optional parameter \verb!color! is used to select between colored or black and 
  white graphics. The optional parameter \verb!figure\_background! is used to select 
  or not in the export of the graphic the background rectangle. 

  Specialized functions can be used to generate graphic file with a specific format. 
\end{mandescription}
%--example
\begin{examples}
  \begin{Verbatim} 
    if ~new_graphics() then switch_graphics();end;xclear();
    plot2d();
    filename='foo'; // ! no extension
    xs2ps(0,filename)
    xexport(0,'foo.svg')
  \end{Verbatim}
\end{examples}
%-- see also


