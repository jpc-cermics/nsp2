% -*- mode: latex -*-
\mansection{xs2ps}
\begin{mandesc}
  \short{xexport}{export graphics to a file}\\
\end{mandesc}
%-- Calling sequence section
\begin{calling_sequence}
\begin{verbatim}
  xexport(winid,filename,color=,figure_background=);
  xs2eps(winid,filename,color=,figure_background=);
  xs2fig(winid,filename,color=,figure_background=);
  xs2pdf(winid,filename,color=,figure_background=);
  xs2png(winid,filename,color=,figure_background=);
  xs2ps(winid,filename,color=,figure_background=);
  xs2svg(winid,filename,color=,figure_background=);
  xs2ps_old(winid,filename,color=,figure_background=);
\end{verbatim}
\end{calling_sequence}
%-- Parameters
\begin{parameters}
  \begin{varlist}
    \vname{winid}: the id of the graphic window as an integer.
    \vname{filename}: a string giving the file name.
    \vname{color}: optional boolean.
    \vname{figure\_background}: optional boolean.
    %% To be done:    \vname{mode}: a string taken among 'd', 'l', 'n', 'p', or 'k'.
  \end{varlist}
\end{parameters}
\begin{mandescription}
  The function \verb!xexport! export the graphic contents of window \verb!winid! to
  the file \verb!filename! in a format selected from the filename suffix.
  The possible suffixes are ".svg", ".pdf", ".eps", ".ps", ".fig", ".png".
  The optional parameter \verb!color! is used to select between colored or black and
  white graphics. The optional parameter \verb!figure\_background! is used to select
  or not in the export of the graphic the background rectangle.
  %%To be done: The optional parameter  \verb!mode! is

  Specialized functions can be used to generate graphic file with a specific format.
\end{mandescription}
%--example
\begin{examples}
  \begin{Verbatim}
    plot2d();
    xexport(0,'foo.svg')
  \end{Verbatim}
\end{examples}
%-- see also
