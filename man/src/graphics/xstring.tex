% -*- mode: latex -*-

\mansection{xstring}
\begin{mandesc}
  \short{xstring}{draw a string matrix} \\
  \short{xstringl}{draw a string matrix} \\
\end{mandesc}
%-- Calling sequence section
\begin{calling_sequence}
\begin{verbatim}
  R=xstring(x,y,S,options=value);
  rect=xstringl(x,y,S,options=value);
\end{verbatim}
\end{calling_sequence}
%-- Parameters
\begin{parameters}
  \begin{varlist}
    \vname{x,y}: real scalars, coordinates of a point.
    \vname{S}: a string matrix.
    \vname{angle}: a double, clockwise angle in degree; default is 0.
    \vname{box}: a boolean (default is false).
    \vname{color}: font color.
    \vname{fill}: a boolean.
    \vname{h,w}: two scalar doubles.
    \vname{posx}: a string selected among \verb!'left'!, \verb!'center'!, or \verb!'right'! (default value is \verb!'right'!)
    \vname{posy}: a string selected among \verb!'bottom'!, \verb!'center'!, \verb!'baseline'!, or \verb!'up'! (default value is \verb!'bottom'!).
    \vname{size}: a scalar int giving the font size to use in pixel.
    \vname{R}: a string graphic object
    \vname{rect}: a \verb!1x4! real matrix, \verb![x,y,w,h]!, giving a rectangle coordinates (upper-left point,width,height).
  \end{varlist}
\end{parameters}

\begin{mandescription}
  \verb!xstring! draws the matrix of strings \verb!S! at location \verb!x,y!
  in the current graphic window. Each row of the matrix
  stands for a line of text where row elements are separated by a white space.
  Optional parameters can be given to control the position, size and color of the
  text.

  \begin{description}
  \item[angle]: it gives the slope in degree used for drawing the strings.
    The rotation is around the point \verb!(x,y)! or around the center of the box \verb![x,y,w,h]!
    when  \verb!w! and \verb!h! are given.
  \item[size]: give the font size to use.
  \item[w,h,fill]: when \verb!w! and \verb!h! are given and are not both equal to zero,
    then the string matrix is drawn centered in the box \verb![x,y,w,h]! where
    \verb!(x,y)! is the lower-left point of the box. Then, if \verb!fill! is true
    then the font size is adapter for the string to be contained in the box.
  \item[posx, posy]: The parameters \verb!posx! and \verb!posy! give the relative
    position of the string matrix display from the point \verb!(x,y)!. They can be used
    when \verb!w!, \verb!h! and \verb!fill! are not given.
  \end{description}

  \verb!xstringl! returns a \verb!1x4! matrix giving the coordinates of an enclosing rectangle.

\end{mandescription}

%--example
\begin{examples}

\noindent A first example

  \begin{mintednsp}{nsp}
    xsetech(frect=[0,0,2,2],iso=%t);
    xstring(1,1,["Nsp";"Graphics"])
  \end{mintednsp}

\noindent A second example where the string rotation is around \verb!(x,y)!.

  \begin{mintednsp}{nsp}
    xsetech(frect=[0,0,2,2],iso=%t);
    xset('font size',3);
    for A=[0:30:330];
      x=1+0.5*cos(-%pi*A/180);y=1+0.5*sin(-%pi*A/180);
      xstring(x,y,'Nsp',A,0,color=A/10);
      xsegs([1;x],[1;y]);
    end
  \end{mintednsp}

\noindent A second example where the string rotation is around the center of the
  given box\verb!(x,y,w,h)!.

  \begin{mintednsp}{nsp}
    xsetech(frect=[-1.0,-1.0,1.0,1.0],iso=%t);
    w=0.2,h=0.2;
    for A=[0:30:330];
       x=0.5*cos(-%pi*A/180);y=0.5*sin(-%pi*A/180);
       // (x,y) is upper-left point for xrect
       R=xrect(-w/2,h/2,w,h);
       R.rotate[[cos(-%pi*A/180),sin(-%pi*A/180)]];
       R.translate[[x,y]];
       // (x,y) is lower-left point for xstring
       xstring(x-w/2,y-h/2,'Nsp',fill=%t,w=w,h=h,angle=A,color=A/10);
       xsegs([0;x],[0;y]);
    end
  \end{mintednsp}

\end{examples}

%-- see also
\begin{manseealso}
  \manlink{xnumb}{xnumb} \manlink{xstringb}{xstringb} \manlink{xstringl}{xstringl} \manlink{xtitle}{xtitle}
\end{manseealso}

%-- Author

\begin{authors}
  J.-Ph. C.
\end{authors}
