% -*- mode: latex -*-
\mansection{xpolys}
\begin{mandesc}
  \short{xpolys}{draw a set of polylines or polygons}\\ % @mandesc@
\end{mandesc}
% -- Calling sequence section
\begin{calling_sequence}
\begin{verbatim}
  A=xpolys(xpols,ypols,draw,options=values)
  A=xpolys(xpols,ypols,options=values)
\end{verbatim}
\end{calling_sequence}

%-- Parameters
\begin{parameters}
  \begin{varlist}
    \vname{xpols,ypols}: two real matrices of size \verb!pxn!.
    \vname{draw}: optional vector of size \verb!n!.
    \vname{A}: if requested \verb!A! is a graphic polyline object or a graphic compound object.
    \vname{close}: a boolean flag specifying to close or not the polylines.
    \vname{color}: a scalar or a vector of size \verb!n!
    \vname{compound}: a boolean  specifying if returned value is a compound or the last polyline.
    \vname{mark}:  a scalar or a vector of size \verb!n!
    \vname{mark_color}: a scalar or a vector of size \verb!n!
    \vname{mark_size}:  a scalar or a vector of size \verb!n!
    \vname{thickness}:  a scalar or a vector of size \verb!n!
  \end{varlist}
\end{parameters}
\begin{mandescription}
  \verb!xpolys! draws a set of \verb!n! polylines using marks and/or colored lines.
  The coordinates of polyline \verb!i! is given by the \verb!i!-th
  column of matrices \verb!xpols! and \verb!ypols!.
  The style of polyline \verb!i! can be given by the optional argument \verb!draw(i)!
  or can be given by optional named arguments.

  \begin{description}
  \item[-] If \verb!draw(i)! is negative, the mark with id \verb!-draw(i)! is
    used to draw polyline i (marks are drawn using the current pattern). Use
    \verb!xset()! to see the meaning of the ids.
  \item[-] If \verb!draw(i)! is strictly positive, the line style (or color)
    with id \verb!draw(i)! is used to draw polyline i. Use \verb!xset()! to see
    the meaning of the ids.
  \end{description}

  Negative values for some optional named parameters like \verb!color! or
  \verb!mark! have special meaning.
  Parameters are ignored when they are equal to \verb!-2!
  and are set to default values if they are equal to \verb!-1!.
\end{mandescription}

%--example
\begin{examples}
  \begin{Verbatim}
    if ~new_graphics() then switch_graphics();end;
    N=9;z=linspace(0,1,N)';
    x=0.05*cos(2*%pi*z);
    y=0.05*sin(2*%pi*z);
    xsetech(frect=2*[-1,-1,1,1]);
    xset('colormap',jetcolormap(32));
    // three polylines
    A=xpolys([x,x+0.2,x],[y,y,y+0.2],color=[1,1,1],close=%t,thickness=[1:3]);
    // animate the set of polylines
    for i=1:32
      A.invalidate[];
      for j=1:length(A.children),
        A.children(j).color=i;
        A.children(j).rotate[[cos(%pi/10),sin(%pi/10)]];
        A.children(j).translate[[0.05*j,0]];
        A.children(j).scale[[1.05,1.05]];
      end
      A.invalidate[];
      xpause(100000,%t)
    end
  \end{Verbatim}
\end{examples}
%-- see also
\begin{manseealso}
  \manlink{xfpoly}{xfpoly} \manlink{xfpolys}{xfpolys} \manlink{xpoly}{xpoly}
\end{manseealso}
%-- Author
\begin{authors}
  J.-Ph. C.
\end{authors}
