% -*- mode: latex -*-
\mansection{param3d}
\begin{mandesc}
  \short{param3d}{3D plot of a parametric curve}\\
\end{mandesc}
%-- Calling sequence section
\begin{calling_sequence}
\begin{verbatim}
  param3d(x,y,z,style=,alpha=,colormap=,ebox=,flag=,leg=,theta=);
\end{verbatim}
\end{calling_sequence}

%-- Parameters
\begin{parameters}
  \begin{varlist}
    \vname{x,y,z}: three vectors of the same size (points of the parametric
    curve).

    \vname{style}: is a vector of size the number of curves \verb!nc!. If
    \verb!style(i)! is negative the \verb!i!-th curve is plotted using mark with id equal to \verb!abs(style(i))! and
    if \verb!style(i)! is positive, the \verb!i!-th curve is drawn with color id equal to \verb!style(i)!.

    \vname{theta, alpha}: optional arguments giving in degree the spherical coordinates of the
    observation point.
    \vname{leg}: optional string defining the labels for each axis with \verb!@! as a field
    separator, for example \verb!"X@Y@Z"!.
    \vname{flag}: optional real vector of size three.
    \verb!flag=[type,box]!.
    \begin{varlist}
      \vname{type}: an integer (scaling).
      \begin{varlist}
        \vname{type=0}: the plot is made using the current 3D scaling.
        \vname{type=1}: rescales automatically 3d boxes with extreme aspect
        ratios, the boundaries are specified by the value of the
        optional argument \verb!ebox!.
        \vname{type=2}: rescales automatically 3d boxes with extreme aspect
        ratios, the boundaries are computed using the given
        data.
        \vname{type=3}: 3d isometric with box bounds given by optional \verb!ebox!, similarily to
        \verb!type=1!.
        \vname{type=4}: 3d isometric bounds derived from the data, similarily tp \verb!type=2!.
        \vname{type=5}: 3d expanded isometric bounds with box bounds given by optional \verb!ebox!, similarily to
        \verb!type=1!.
        \vname{type=6}: 3d expanded isometric bounds derived from the data,
        similarily to \verb!type=2!.
      \end{varlist}
      \vname{box}: an integer (frame around the plot).
      \begin{varlist}
        \vname{box=0}: nothing is drawn around the plot.
        \vname{box=1}: unimplemented (like box=0).
        \vname{box=2}: only the axes behind the surface are drawn.
        \vname{box=3}: a box surrounding the surface is drawn and captions are added.
        \vname{box=4}: a box surrounding the surface is drawn, captions and
        axes are added.
      \end{varlist}
    \end{varlist}
    \vname{ebox}: vector \verb![xmin,xmax,ymin,ymax,zmin,zmax]! used to fix the boundaries of the plot. This argument is used
    when the optional argument \verb!flag! is given with type set to \verb!1!, or \verb!3!, or \verb!5!.
  \end{varlist}
\end{parameters}

\begin{mandescription}
  \verb!param3d! is used to plot a 3D curve defined by its
  coordinates \verb!x!, \verb!y! and \verb!z!. Note that
  data can also be set or get through the surface entity properties (see
  \manlink{surface\_properties}{surface-properties} ).Note that properties like \verb!rotation angles!,
  \verb!colors! and \verb!thickness! of the plotted curves can
  also be set through the param3d entity properties (see
  \manlink{param3d\_properties}{param3d-properties} ).Use \verb!param3d1! to do multiple plots.Enter the command \verb!param3d()! to see a demo.
\end{mandescription}

%--example
\begin{examples}
\noindent A simple example
\begin{mintednsp}{nsp}
  t=linspace(0,5*%pi,100)';
  param3d(sin(t),cos(t),t/10);
\end{mintednsp}
\noindent using \verb!flag! and \verb!style!

\begin{mintednsp}{nsp}
  t=linspace(0,5*%pi,100)';
  param3d(sin(t),cos(t),t/10,flag=[2,1],style=-7);
  param3d(sin(t)/2,cos(t)/2,t/10,flag=[2,1],style=7);
\end{mintednsp}
\end{examples}

%-- see also
\begin{manseealso}
  \manlink{param3d1}{param3d1} \manlink{plot3d}{plot3d}
\end{manseealso}

%-- Author

\begin{authors}
  J.Ph.C.

\end{authors}
