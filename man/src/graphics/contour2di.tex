% -*- mode: latex -*-

\mansection{contour2di}
\begin{mandesc}
  \short{contour2di}{compute points of level curves of a surface}\\
\end{mandesc}
% -- Calling sequence section
\begin{calling_sequence}
\begin{verbatim}
  [xc,yc]=contour2di(x,y,z,nz)
\end{verbatim}
\end{calling_sequence}

%-- Parameters
\begin{parameters}
  \begin{varlist}
    \vname{x,y}: two real row vectors of size n1 and n2: the grid.
    \vname{z}: real matrix of size (n1,n2), the values of the function.
    \vname{nz}: the level values or the number of levels.
    \begin{varlist}
      \vname{-}If \verb!nz! is an integer, its value gives the number of
      level curves  equally spaced from zmin to zmax as follows:
      \verb!z= zmin + (1:nz)*(zmax-zmin)/(nz+1)!
      \vname{-} If \verb!nz! is a vector, \verb!nz(i)! gives the value of the ith level curve.
    \end{varlist}
    \vname{xc,yc}: vectors of same sizes containing the contours definitions.
  \end{varlist}
\end{parameters}

\begin{mandescription}
  \verb!contour2di! computes points defining level curves of a surface \verb!z=f(x,y)!.
  The values of \verb!f(x,y)! are given by the matrix \verb!z! at the grid points
  defined by \verb!x! and \verb!y!.
  \verb!xc(1)! contains the level associated with first contour path,
  \verb!yc(1)! contains the number \verb!N1! of points defining this contour path
  and (\verb!xc(1+(1:N1))!, \verb!yc(1+(1:N1))! ) contain the coordinates
  of the paths points.
  The second path begin at \verb!xc(2+N1)! and \verb!yc(2+N1)! and so on.
\end{mandescription}

%--example
\begin{examples}
  \begin{mintednsp}{nsp}
    [xc,yc]=contour2di(1:10,1:10,rand(10,10),5);
    k=1;n=yc(k);c=1;
    while k+yc(k) < size(xc,'*')
    n=yc(k);
    plot2d(xc(k+(1:n)),yc(k+(1:n)),line_color=c)
    c=c+1;
    k=k+n+1;
    end
  \end{mintednsp}
\end{examples}

%-- see also
\begin{manseealso}
  \manlink{contour}{contour}
  \manlink{fcontour}{fcontour}
  \manlink{fcontour2d}{fcontour2d}
  \manlink{contour2d}{contour2d}
  \manlink{plot2d}{plot2d}
  \manlink{xset}{xset}
\end{manseealso}
%-- Author
\begin{authors}
  J.-Ph. C.
\end{authors}
