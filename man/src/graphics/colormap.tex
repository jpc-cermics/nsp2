% -*- mode: latex -*-
\mansection{colormap}
\begin{mandesc}
  \short{colormap}{using colormaps in graphics}\\ % @mandesc@
  \short{graycolormap}{linear gray colormap from black to white}\\ % @mandesc@
  \short{greencolormap}{linear green colormap}\\ % @mandesc@
  \short{hotcolormap}{red to yellow colormap}\\ % @mandesc@
  \short{jetcolormap}{blue to red colormap}\\ % @mandesc@
\end{mandesc}
\begin{calling_sequence}
\begin{verbatim}
  cmap=graycolormap(n)
  cmap=greencolormap(n)
  cmap=hotcolormap(n)
  cmap=jetcolormap(n)
\end{verbatim}
\end{calling_sequence}
\begin{parameters}
  \begin{varlist}
    \vname{n}: an integer giving the colormap size.
    \vname{cmap}: a real matrix of size \verb!nx3!.
\end{varlist}
\end{parameters}
\begin{mandescription}
  A colormap \verb!cmap! is defined by a \verb!mx3! real matrix where \verb!m!
  gives the number of colors. The \verb!i!-th row of the colormap gives the
  red, green and blue intensity (numbers in $[0,1]$) of the \verb!i!-th color.
  The functions \manlink{graycolormap}{graycolormap},
  \manlink{greencolormap}{greencolormap}, \manlink{hotcolormap}{hotcolormap},
  and \manlink{jetcolormap}{jetcolormap} provide default colormaps with
  continuous variation of colors.
\end{mandescription}
%--example
\begin{examples}
  \begin{mintednsp}{nsp}
    // set the current colormap
    xset('colormap',hotcolormap(64));
    A=1:64;plot2d([0,10],[0,10],style=0);Matplot1(A,[1,1,9,9]);
    cmap=xget('colormap');
    [n,m]=size(cmap)
    // number of colors in the current colormap
    n1= xget('lastpattern');
    n1 == n
  \end{mintednsp}
\end{examples}
%-- see also
%\begin{manseealso}
%\end{manseealso}
