% -*- mode: latex -*-

\mansection{grayplot}
\begin{mandesc}
  \short{grayplot}{2D plot of a surface using colors}\\
\end{mandesc}
%-- Calling sequence section
\begin{calling_sequence}
\begin{verbatim}
  grayplot(x,y,z,options=)
  grayplot(x,y,f,options=)
\end{verbatim}
\end{calling_sequence}

%-- Parameters
\begin{parameters}
  \begin{varlist}
    \vname{x,y}: two real vectors of size \verb!n1! and \verb!n2!.
    \vname{z}: a real matrix of size \verb!n1xn2!. The value of the z-level of the
    surface at point \verb!(x(i),y(j))! is given by \verb!z(i,j)!.
    \vname{f}: a nsp function. The value of the z-level of the
    surface at point \verb!(x(i),y(j))! is given by \verb!f(x(i),y(j))!.
    \vname{axesflag}: see \manlink{plot2d}{plot2d}
    \vname{colminmax}: see \manlink{fec}{fec}
    \vname{colout}: see \manlink{fec}{fec}
    \vname{frameflag}: see \manlink{plot2d}{plot2d}
    \vname{leg_pos}: see \manlink{plot2d}{plot2d}
    \vname{leg}: see \manlink{plot2d}{plot2d}
    \vname{logflag}: see \manlink{plot2d}{plot2d}
    \vname{nax}: see \manlink{plot2d}{plot2d}
    \vname{rect}: see \manlink{plot2d}{plot2d}
    \vname{remap}: If true then the \verb!z! values are scaled to match colors
    of the current colormap.  If false the values of \verb!z! are directly
    considered as colorid. The default value for \verb!remap! is true.
    \vname{shade}: If true the color in each rectangle of the surface are
    interpolated using colors at the corners. If false the color of each
    rectangle of the surface is the average of the colors of the rectangle
    corners.
    \vname{strf}: see \manlink{plot2d}{plot2d}
    \vname{zminmax}: see \manlink{fec}{fec}
    \end{varlist}
\end{parameters}
\begin{mandescription}
  \verb!grayplot! makes a 2D plot of the surface described by its values
  \verb!z! or by a function \verb!f!  on a grid defined by \verb!x! and
  \verb!y!.  Each rectangle on the grid is filled with a color level depending
  on the average value of \verb!z! on the corners of the rectangle (If
  \verb!shade! is false) or by interpolated colors (If \verb!shade! is
  true). The way \verb!z! values are mapped to the colors of the current
  colormap is explained in the \verb!fec! manual.
\end{mandescription}

%--example
\begin{examples}

\noindent A first example.
  \begin{mintednsp}{nsp}
    x=-10:10; y=-10:10;m =rand(21,21);
    grayplot(x,y,m,rect=[-20,-20,20,20])
  \end{mintednsp}

\noindent A second example. The values of \verb!m! are not remapped and
thus values above \verb!32! are ignored.

  \begin{mintednsp}{nsp}
    x=-10:10; y=-10:10;m = 64*rand(21,21);
    xset('colormap',jetcolormap(32));
    grayplot(x,y,m,rect=[-20,-20,20,20] ,remap=%f);
  \end{mintednsp}

\noindent A third example with a function. The z-values of the surface are remapped.
Remapped values in the range given by \verb!zminmax! and associated to colors in the range
\verb!colminmax! and values above or below are painted with colors given by \verb!colout!

  \begin{mintednsp}{nsp}
    function z=f(x,y); z = 30*sin(x)*cos(y);endfunction
    t=-%pi:0.1:%pi;
    xset('colormap',jetcolormap(32));
    grayplot(t,t,f,remap=%t,shade=%t,zminmax=[6,24],colminmax=[6,24],colout=[36,35])
  \end{mintednsp}
\end{examples}

%-- see also
\begin{manseealso}
  \manlink{fgrayplot}{fgrayplot} \manlink{plot2d}{plot2d} \manlink{Sgrayplot}{Sgrayplot} \manlink{Sfgrayplot}{Sfgrayplot}
\end{manseealso}

%-- Author

\begin{authors}
  J.-Ph. C.
\end{authors}
