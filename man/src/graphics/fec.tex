% -*- mode: latex -*-

\mansection{fec}
\begin{mandesc}
  \short{fec}{Plot of a function defined on a triangular mesh}\\
\end{mandesc}
% -- Calling sequence section
\begin{calling_sequence}
\begin{verbatim}
  fec(x,y,triangles,z,[option=value])
\end{verbatim}
\end{calling_sequence}
%-- Parameters
\begin{parameters}
  \begin{varlist}
    \vname{x,y}: two real vectors of size \verb!n! defining \verb!n! nodes.
    The \verb!i!-th node has coordinates \verb!(x(i),y(i))!.
    \vname{z}: a vector of size \verb!n! where \verb!z(i)!
    gives the value at node \verb!i! of the function to be ploted.
    \vname{triangles}: a \verb!mx5! matrix where in each row of the \verb!triangles! matrix
    a triangle of the mesh is described by five numbers \verb![number,node1,node2,node3,flag]!.
    \verb!node1,node2,node3! are the ids of the nodes composing the triangle.
    \verb!number! is the number of the triangle and \verb!flag! is an unused value.
    \vname{axesflag}:
    \vname{colminmax}: vector of 2 positives integers [colmin colmax]
    \vname{colout}: vector of 2 integers [under\_min\_col upper\_max\_col]
    \vname{frameflag}:
    \vname{leg}: see \manlink{plot2d}{plot2d}
    \vname{leg_pos}:
    \vname{logflag}:
    \vname{mesh}: a boolean scalar (\verb!%f! as default value) controling the display of the grid mesh.
    \vname{nax}: see \manlink{plot2d}{plot2d}
    \vname{rect}: see \manlink{plot2d}{plot2d}
    \vname{strf}: see \manlink{plot2d}{plot2d}
    \vname{style}:
    \vname{zminmax}: vector with 2 components [zmin zmax] (useful in particular for animation)
    \vname{colorbar}: a boolean scalar controling the display of a side colorbar
    \vname{paint}:
  \end{varlist}
\end{parameters}
\begin{mandescription}
  This function can be used to draw linear triangular finite element solutions
  or more generally to draw a function defined on a triangulation. The color interpolation
  is done through software computation and so it is not too fast.
  The \verb!zminmax! argument gives the z values to be associated with the first and the last
  color (of the current colormap). More precisely, if the colormap have nc colors and if we note
  $dz = (zmax-zmin)/nc$, then the part of the triangulation where
  $zmin + (i-1)dz \le z < zmin + i dz$ is filled with the color $i$.
  By default \verb!zmin = min(z)! and \verb!zmax = max(z)!. When performing
  an animation with time varying \verb!z! values, zmin and zmax should be set to
  the global minimum and maximum of the varying \verb!z! values or something close.

  The \verb!colout! argument is used to specify the colors to be used when drawing the
  two extreme regions \verb!z < min! and \verb!z > zmax!, \verb!under_min_col! and
  \verb!upper_max_col! may be equal (independently) to:
  \begin{description}
  \item[-1]: in this case the same color than in the neighbouring zone is used (CAUTION:
    you don't see that the corresponding limit is crossed), this is the default case.
  \item[0]: in this case the extreme region is not painted at all.
  \item[k]: (k being a valid index to a color of the current colormap) the extreme region
    is painted with color k.
  \end{description}
  If you don't want to use the complete colormap you may use the \verb!colminmax!
  argument with \verb! 1 <= colmin < colmax <= nc! (nc being the number of colors
  of the current colormap) so as to use only the \verb![colmin,colmax]!  sub-part of the colormap.
  (by default all the colors of the colormap are used).

  Some demo files are given in the \verb!NSP/demos/graphics/fec! directory.
\end{mandescription}

% --example
\begin{examples}

\noindent a first simple in which a triangulation and a function are
built and given to \verb!fec!
\begin{Verbatim}
if ~new_graphics() then switch_graphics();end
N=20;
n=1:N;
x=cos(n*2*%pi/N);
y=sin(n*2*%pi/N);
noeul=[(1:(N))',x',y',0*ones(N,1)];
noeul=[noeul;(N+1),0,0,0];
trianl=[(1:N)',[(2:N)';1],(N+1)*ones(N,1)];
triangles=[(1:N)',trianl,zeros(N,1)];
rect=[-1.2,-1.2,1.2,1.2];
z=(1:N+1)';
if new_graphics() then z(1:4)=%nan;end
xset('colormap',jetcolormap(32));
fec(noeul(:,2),noeul(:,3),triangles,z,strf='030',axesflag=2,rect=rect,mesh=%t);
\end{Verbatim}

\noindent a second example

\begin{Verbatim}
if ~new_graphics() then  switch_graphics(); end
// define a mini triangulation (4 vertices, 2 triangles)
x = [0 1 0 -1];
y = [0 0 1  1];
T = [1 1 2 3 1;
     2 3 4 1 1];
// values of the function at each vertices of the
// triangulation;
z = [0 1 0 -1];
xclear();
subplot(1,2,1)
xset("colormap",jetcolormap(64))
fec(x,y,T,z,strf="040",mesh=%t)
xtitle("fec example (with the mesh)")

subplot(1,2,2)
fec(x,y,T,z,strf="040")  // rmq: mesh=%f by default
xtitle("fec example (without the mesh)")
\end{Verbatim}

\noindent this example shows the effect of zminmax
and needs the data of the previous example.

\begin{Verbatim}
if ~new_graphics() then  switch_graphics(); end
xclear();
xset("colormap",jetcolormap(64))
fec(x,y,T,z,strf="040", zminmax=[-0.5 0.5], mesh=%t)
xtitle("fec example : using zminmax argument")
\end{Verbatim}

\noindent this example shows the effect of zminmax
and colout. Itneeds the data of the previous example.

\begin{Verbatim}
if ~new_graphics() then  switch_graphics(); end
xclear();
xset("colormap",jetcolormap(64))
subplot(2,2,1)
fec(x,y,T,z,strf="040", zminmax=[-0.5 0.5], colout=[0 0], mesh=%t)
xtitle("fec example : using zminmax and colout =[0 0]")

subplot(2,2,2)
fec(x,y,T,z,strf="040", zminmax=[-0.5 0.5], colout=[67 67], mesh=%t)
xtitle("fec example : using zminmax and colout =[67 67]")

subplot(2,2,3)
fec(x,y,T,z,strf="040", zminmax=[-0.5 0.5], colout=[-1 0], mesh=%t)
xtitle("fec example : using zminmax and colout =[-1 0]")

subplot(2,2,4)
fec(x,y,T,z,strf="040", zminmax=[-0.5 0.5], colout=[0 -1], mesh=%t)
xtitle("fec example : using zminmax and colout =[0 -1]")
xselect()
\end{Verbatim}

\noindent this example shows how to use two colormaps one for each subplots. It
also uses the first example data.

\begin{Verbatim}
if ~new_graphics() then  switch_graphics(); end
xclear();
xset("colormap",[hotcolormap(64);greencolormap(64)])
subplot(1,2,1)
fec(x,y,T,z,strf="040", colminmax=[1 64], mesh=%t)
xtitle("fec using the hot colormap")

subplot(1,2,2)
fec(x,y,T,z,strf="040", colminmax=[65 128], mesh=%t)
xtitle("fec using the greencolormap")
xselect()
\end{Verbatim}
\end{examples}
% -- see also
  \begin{manseealso}
    \manlink{colorbar}{colorbar} \manlink{Sfgrayplot}{Sfgrayplot} \manlink{Sgrayplot}{Sgrayplot}
  \end{manseealso}
