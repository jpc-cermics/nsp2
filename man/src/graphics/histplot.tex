% -*- mode: latex -*-
% Copyright Lelong 

\mansection{histplot}
\begin{mandesc}
  \short{histplot}{draw an histogram of the components of a vector of numbers}
\end{mandesc}

% -- Calling sequence section
\begin{calling_sequence}
\begin{verbatim}
 histplot(n,data,[normalize=%t|%f,style=val1,rect=val2,axesflag=val3,fill=%t|%f])
 histplot(x,data,[normalize=%t|%f,style=val1,rect=val2,axesflag=val3,fill=%t|%f])
\end{verbatim}
\end{calling_sequence}
% -- Parameters
\begin{parameters}
  \begin{varlist}
    \vname{n} : number of classes for the histogram
    \vname{x} : an increasing vector defining the classes.
    \vname{y} : vector of data.
    \vname{style=val1} : optional named argument, sets the style for each curve. \verb+val1+
    must be a vector of integers.
    \vname{rect=val2} : optional named argument, sets the bounds of the plotting
    window. \verb+val2+ must be a vector of integers: \verb+[xmin, ymin, xmax, ymax]+.
    \vname{nax} : not implemented
    \vname{axesflag=val3} : optional named argument, specified how axes are
    drown. \verb+val3+ is an integer between \verb+0+ and \verb+5+
    \begin{itemize}
    \item \verb+val3=0+ : no axes are drawn.
    \item \verb+val3=1+ : axes are drawn with the y axis on the left.
    \item \verb+val3=2+ : a box is drawn around the plot.
    \item \verb+val3=3+ : axes are drawn with the y axis on the right.
    \item \verb+val3=4+ : axes are drawn centred in the middle of the plotting window
    \item \verb+val3=5+ : axes are drawn so to cross at the point \verb+(0,0)+.
    \end{itemize}

\end{varlist}
\end{parameters}

\begin{mandescription}
  This function plots an histogram of the vector data using the
  classes x. If \verb+n+ is given, x is defined by x=[min(data):h:max(data)] with
  h=(max(data)-min(data))/n.

  The optional argument \verb+normalise+ (default value \verb+%t+) can be set to
  \verb+%f+.
  optional argument \verb+fill+ (default value \verb+%t+).


  
\end{mandescription}

\begin{examples}
  \begin{program}
    \HCode{
    g = rand(1,10000, 'normal');\Hnewline
    histplot(100, g)
    }
  \end{program}

\end{examples}

\begin{manseealso}
  xtitle, plot2d
\end{manseealso}

% -- Authors
%% \begin{authors}
%%    jl
%% \end{authors}
