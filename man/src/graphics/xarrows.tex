% -*- mode: latex -*-

\mansection{xarrows}
\begin{mandesc}
  \short{xarrows}{draw a set of arrows}\\ % @mandesc@
\end{mandesc}
%-- Calling sequence section
\begin{calling_sequence}
\begin{verbatim}
  xarrows(nx,ny,arsize=,style=)  
\end{verbatim}
\end{calling_sequence}
%-- Parameters
\begin{parameters}
  \begin{varlist}
    \vname{nx,ny}: real vectors or matrices of same size.
    \vname{arsize}: real scalar, size of the arrow head. The default value can be obtained  by setting arsize to -1.
    \vname{style}: matrix or scalar. If \verb!style! is a positive scalar it
    gives the color to use for all arrows. If it is a negative scalar then the
    current color is used. If it is a vector \verb!style(i)! gives the color to
    use for arrow \verb!i!.
  \end{varlist}
\end{parameters}

\begin{mandescription}
  The function \verb!xarrows! draws a set of arrows given by \verb!nx! and \verb!ny!.
  If \verb!nx! and \verb!ny! are vectors, the \verb!i!-th arrow is defined by
  \verb!(nx(i),ny(i))--$>$(nx(i+1),ny(i+1))!.
  If \verb!nx! and \verb!ny! are matrices:
  \begin{Verbatim}
    nx=[xi_1 x1_2 ...; xf_1 xf_2 ...]
    ny=[yi_1 y1_2 ...; yf_1 yf_2 ...]
  \end{Verbatim}

  The \verb!k!-th arrow is defined by \verb!(xi_k,yi_k)--$>$(xf_k,yf_k)!. 
  The function \verb!xarrows! uses the current graphics scale 
  which can be set by calling a high level drawing function such as \verb!xsetech!.
\end{mandescription}
%--example
\begin{examples}
  \begin{mintednsp}{nsp}
    x=2*%pi*linspace(0,1,30);x($)=[];
    x1=[sin(x);(1.5+x).*sin(x)];
    y1=[cos(x);(1.5+x).*cos(x)];
    xsetech(frect=[-8,-8,6,10],fixed=%t,clip=%t,iso=%t);
    xset('colormap',jetcolormap(32));
    xarrows(x1,y1,arsize=0.5,style=1:length(x));
  \end{mintednsp}
\end{examples}
%-- Author
\begin{authors}
  J.Ph.C.  
\end{authors}

