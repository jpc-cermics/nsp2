% -*- mode: latex -*-
\mansection{xfpoly}
\begin{mandesc}
  \short{xfpoly}{fill a polygon with color}\\
\end{mandesc}
% -- Calling sequence section
\begin{calling_sequence}
\begin{verbatim}
  p=xfpoly(x,y,fill_color=,color=,thickness)
\end{verbatim}
\end{calling_sequence}
%-- Parameters
\begin{parameters}
  \begin{varlist}
    \vname{x,y}: two real vectors of same size \verb!pxn!.
    \vname{color}: color to be used for drawing boundary (default \verb!-2!).
    \vname{fill_color}: color for filling the polyline (default \verb!-1!).
    \vname{thickness}: thickness of boundary (default \verb!-1!).
    \vname{p}: a graphics polyline object.
  \end{varlist}
\end{parameters}

\begin{mandescription}
  \verb!xfpoly! fills a polygon with color and draws the boundary of
  the polygon if color is not equal to \verb!-2!.

  Negative values for some parameters have special meaning.
  Parameters are ignored when they are equal to \verb!-2!
  and are set to default values if they are equal to \verb!-1!.
\end{mandescription}

%--example
\begin{examples}
  \begin{Verbatim}
    if ~new_graphics() then switch_graphics();end;xclear();
    x=sin(2*%pi*(0:4)/5);
    y=cos(2*%pi*(0:4)/5);
    xsetech(frect=[-2,-2,2,2])
    xfpoly(x,y,color=6,fill_color=4,thickness=3);
  \end{Verbatim}
\end{examples}

%-- see also
\begin{manseealso}
  \manlink{xfpolys}{xfpolys} \manlink{xpoly}{xpoly} \manlink{xpolys}{xpolys}
\end{manseealso}

%-- Author

\begin{authors}
  J.Ph.C.

\end{authors}
