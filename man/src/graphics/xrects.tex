% -*- mode: latex -*-

\mansection{xrects}
\begin{mandesc}
  \short{xrects}{draw or fill a set of rectangles}\\ % @mandesc@
\end{mandesc}
%-- Calling sequence section
\begin{calling_sequence}
\begin{verbatim}
  R=xrects(rects,[colors])
  R=xrects(rects,color=..,thickness=..,background=..,compound=%t|%f);
\end{verbatim}
\end{calling_sequence}

%-- Parameters
\begin{parameters}
  \begin{varlist}
    \vname{rects}: matrix of size (4,n).
    \vname{colors}: vector of size n.
    \vname{background}: vector of size n.
    \vname{color}: vector of size n.
    \vname{thickness}: vector of size n.
    \vname{compound}: a boolean.
    \vname{R}: a graphic object.
  \end{varlist}
\end{parameters}

\begin{mandescription}
  \verb!xrects(rects,colors)! draws or fills a set of rectangles.
  Each column of the variable \verb!rects! describes a rectangle
  (upper-left point, width, height): \verb!rects=[x1 y1 w1 h1;x2 y2 w2 h2;...]'!.
  The value of \verb!color(i)! gives the pattern or color to use for filling or drawing rectangle
  \verb!i!:
  \begin{itemize}
  \item if \verb!fill(i) < 0!, rectangle \verb!i! is drawn using the line style (or
    color) \verb!-fill(i)!
  \item if \verb!fill(i) > !, rectangle \verb!i! is filled using the pattern (or
    color) \verb!fill(i)!
  \item if \verb!fill(i)=0!, rectangle \verb!i! is drawn using the current
    line style (or color).
  \end{itemize}

  \verb!xrects(rects,color=..,thickness=..,background=..,compound=%t|%f);! draws a set of
  rectangles. The color and thickness of drawing line can be given by optional arguments
  color and thickness. If optional background argument is given it is used as color to fill
  the rectangle. A graphic object is returned in argument \verb!R! if requested and by default
  its is a compound which contains all the graphics rectangle objects.

\end{mandescription}

%--example
\begin{examples}

A first example \verb!xrects(rects,cols)!

\begin{Verbatim}
  plot2d([-100,500],[-50,50],style=[-1,-1],strf="022")
  cols=[-34,-33,-32,-20:5:20,32,33,34];
  x=400*(0:14)/14; step=20;
  rects=[x;10*ones(size(x));step*ones(size(x));30*ones(size(x))];
  xrects(rects,cols)
  xnumb(x,15*ones(size(x)),cols)
\end{Verbatim}

A second example with optional arguments

\begin{Verbatim}
  if ~new_graphics() then switch_graphics();end
  plot2d([-100,500],[-50,50],style=[-1,-1],strf="022")
  x=400*(0:14)/14; step=20;
  rects=[x;10*ones(size(x));step*ones(size(x));30*ones(size(x))];
  background=[4*ones(1,5),1:5,6*ones(1,5)];
  color=[6*ones(1,5),ones(1,5),4*ones(1,5)];
  thickness=[4*ones(1,5),ones(1,5),4*ones(1,5)];
  xrects(rects,background=background,color=color,thickness=thickness);
\end{Verbatim}

\end{examples}

%-- see also
\begin{manseealso}
  \manlink{xfrect}{xfrect} \manlink{xrect}{xrect}
\end{manseealso}

%-- Author

\begin{authors}
  J.-Ph. C.
\end{authors}
