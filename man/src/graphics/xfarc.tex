% -*- mode: latex -*-
\mansection{xfarc}
\begin{mandesc}
  \short{xfarc}{fill a part of an ellipse}\\ % @mandesc@
\end{mandesc}
%-- Calling sequence section
\begin{calling_sequence}
\begin{verbatim}
  A=xfarc(x,y,w,h,a1,a2,options=value)
  A=xfarc(arc,options=value)
\end{verbatim}
\end{calling_sequence}
%-- Parameters
\begin{parameters}
  \begin{varlist}
    \vname{x,y,w,h}: four real values defining a rectangle.
    \vname{a1,a2}: real values defining a sector.
    \vname{arc}: a vector of size \verb!6! 
    \vname{background, color, thickness}: optional arguments.
    \vname{A}: a graphical rectangle object.
  \end{varlist}
\end{parameters}

\begin{mandescription}
  \verb!xfarc! fills a part of an ellipse contained in the rectangle 
  \verb!(x,y,w,h)!
  (upper-left point, width, height), and in the sector defined by 
  the angle \verb!alpha1! and the angle \verb!alpha1+alpha2!. 
  \verb!alpha1! and \verb!alpha2! are 
  given respectively by \verb!a1/64! degrees and \verb!a2/64! degrees.
  This function uses the current color and graphics scale.

  The color and thickness of drawing line can be given by optional arguments
  \verb!color! and \verb!thickness!. If optional \verb!background!
  argument is given it is used as color to fill the rectangle.

  If \verb!color! is positive it gives the color to be used for drawing,
  if \verb!color=-1! the default color is used, if  \verb!color=-2! the
  rectangle is not drawn.

  If \verb!background! is positive it gives the color to be used for filling,
  if \verb!background=-1! the default filling color is used,
  if  \verb!background=-2! the rectangle is not filled.
  
\end{mandescription}

%--example
\begin{examples}
  \begin{Verbatim}
    if ~new_graphics() then switch_graphics();end;xclear();
    xsetech(frect=[-2,-2,2,2],iso=%t,axesflag=0);
    xfarc(-1,1,2,2,0,90*64,background=7,color=3);
    xarc(-1.5,1.5,3,3,0,360*64)
  \end{Verbatim}
\end{examples}

%-- see also
\begin{manseealso}
  \manlink{xarc}{xarc} \manlink{xarcs}{xarcs} \manlink{xfarcs}{xfarcs} 
\end{manseealso}

%-- Author

\begin{authors}
  J.Ph.C.;   

\end{authors}

