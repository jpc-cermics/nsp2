% -*- mode: latex -*-
\mansection{Sgrayplot}
\begin{mandesc}
  \short{Sgrayplot}{smooth 2D plot of a surface using colors}\\
  \short{Sfgrayplot}{smooth 2D plot of a surface using colors}\\
\end{mandesc}
%-- Calling sequence section
\begin{calling_sequence}
\begin{verbatim}
  Sgrayplot(x,y,z,options=value)
  Sfgrayplot(x,y,f,options=value)
\end{verbatim}
\end{calling_sequence}
%-- Parameters
\begin{parameters}
  \begin{varlist}
    \vname{x,y}: real row vectors of size \verb!n1! and \verb!n2!.
    \vname{z}: real matrix of size \verb!n1xn2!.
    The z-value of the surface at \verb!(x(i),y(j))! is given by \verb!z(i,j)!
    \vname{f}: a nsp function. The z-value of the surface at point \verb!(x(i),y(j))!
    is given by \verb!f(x(i),y(j))!
    \vname{options=value}: a sequence of optional names values (See \manlink{fec}{fec}).
  \end{varlist}
\end{parameters}

\begin{mandescription}
  \verb!Sgrayplot! is similar to \verb!grayplot! but the plot is done using the
  \verb!fec! function. The surface is plotted assuming that it is linear on a set of
  triangles built from the grid (here with n1=5, n2=3):
  \begin{Verbatim}
    _____________
    | /| /| /| /|
    |/_|/_|/_|/_|
    | /| /| /| /|
    |/_|/_|/_|/_|
  \end{Verbatim}
  Optional parameters are described in the manual page of function \manlink{fec}{fec}.
\end{mandescription}

%--example
\begin{examples}
\noindent A first example
  \begin{mintednsp}{nsp}
    xclear();
    x=-10:10; y=-10:10;m =rand(21,21);
    xset("colormap",hotcolormap(64))
    Sgrayplot(x,y,m, colorbar=%t);
  \end{mintednsp}

\noindent A second example

  \begin{mintednsp}{nsp}
    xclear();
    n = 30;
    nt = 100;
    x = linspace(0,2*%pi,n);
    y = linspace(0,%pi,n/2);
    z = sin(x')*sin(y);
    t = linspace(0,4*%pi,nt);
    xset("colormap",jetcolormap(64));
    Sgrayplot(x,y,cos(t(1))*z, colorbar=%t);
    F=get_current_figure[];
    fecd=F.children(1).children(1);
    for i=1:nt
      fecd.func = cos(t(i))*z;
      fecd.invalidate[];
      xpause(10000,%t);
    end
  \end{mintednsp}

  \noindent Using functions for describing surfaces

  \begin{mintednsp}{nsp}
    xclear();
    function z=surf1(x,y), z=x*y, endfunction
    function z=surf2(x,y), z=x^2-y^2, endfunction
    function z=surf3(x,y), z=x^3+y^2, endfunction
    function z=surf4(x,y), z=x^2+y^2, endfunction
    xset("colormap",[jetcolormap(64);hotcolormap(64)])
    x = linspace(-1,1,60);
    y = linspace(-1,1,60);
    subplot(2,2,1)
    Sfgrayplot(x,y,surf1,axesflag=0,colorbar=%t,colminmax=[1,64]);
    xtitle("f(x,y) = x*y")
    subplot(2,2,2)
    Sfgrayplot(x,y,surf2,axesflag=0,colorbar=%t,colminmax=[65,128])
    xtitle("f(x,y) = x^2-y^2")
    subplot(2,2,3)
    Sfgrayplot(x,y,surf3,axesflag=0,colorbar=%t,colminmax=[65,128])
    xtitle("f(x,y) = x^3+y^2")
    subplot(2,2,4)
    Sfgrayplot(x,y,surf4,axesflag=0,colorbar=%t,colminmax=[1,64])
    xtitle("f(x,y) = x^2+y^2")
  \end{mintednsp}

  % \noindent Adding some level curves
  % \begin{nspcode}
  %   function z=surf3(x,y), z=x^3+y^2, endfunction
  %   x = linspace(-1,1,60);
  %   y = linspace(-1,1,60);
  %   xset("colormap",hotcolormap(128))
  %   Sfgrayplot(x,y,surf3,strf="041")
  %   fcontour2d(x,y,surf3,rect=[-0.1, 0.025, 0.4],style=[1 1 1],strf="000")
  %   xtitle("f(x,y) = x^3+y^2")
  % \end{nspcode}

\end{examples}
%-- see also
\begin{manseealso}
  \manlink{fec}{fec}
  \manlink{fgrayplot}{fgrayplot}
  \manlink{grayplot}{grayplot}
\end{manseealso}
%-- Author
