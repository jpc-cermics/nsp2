% -*- mode: latex -*-

\mansection{polarplot}
\begin{mandesc}
  \short{polarplot}{draw a curve given with polar coordinates}\\
\end{mandesc}
%-- Calling sequence section
\begin{calling_sequence}
\begin{verbatim}
  polarplot(theta,rho,grid_color=,color=,rect=)
\end{verbatim}
\end{calling_sequence}

%-- Parameters
\begin{parameters}
  \begin{varlist}
    \vname{rho}: the radius values as a real vector.
    \vname{theta}: the angle values as a real vector.
    \vname{grid_color}: color of the underlying grid.
    \vname{color}: color of the curve.
    \vname{rect}: enclosing rectangle.
  \end{varlist}
\end{parameters}

\begin{mandescription}
  The function \verb!polarplot! draw the curve described by the polar coordinates \verb!(theta,rho)!.
  The vector \verb!theta! is the angle from the x-axis to the radius vector specified in radians.
  The vector \verb!rho! is the length of the radius vector.
\end{mandescription}

%--example
\begin{examples}
  \begin{Verbatim}
    t= 0:.01:2*%pi;
    polarplot(sin(3*t),cos(2*t),color=10)
    // add a second curve
    theta= 2*%pi*linspace(0,1,50); rho=linspace(0,1,50);
    xpoly(rho.*cos(theta),rho.*sin(theta),color=7,thickness=3);
  \end{Verbatim}
\end{examples}
