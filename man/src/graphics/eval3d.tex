% -*- mode: latex -*-
\mansection{eval3d}
\begin{mandesc}
  \short{eval3d}{evaluates a function on a grid}\\
\end{mandesc}
%-- Calling sequence section
\begin{calling_sequence}
\begin{verbatim}
  [z]=eval3d(fun,x,[y])
\end{verbatim}
\end{calling_sequence}
%-- Parameters
\begin{parameters}
  \begin{varlist}
    \vname{fun}: function accepting vectors as arguments.
    \vname{x,y}: 2 vectors of size \verb!n1! and \verb!n2!. Vector \verb!y! can be omitted and
    is then taken as equal to \verb!x!.
    \vname{z}: matrix of size \verb!n1xn2!.
  \end{varlist}
\end{parameters}
\begin{mandescription}
  The function \verb!eval3d! returns a matrix \verb!z! of size \verb!n1xn2!
  which is such that \verb!z(i,j)=fun(x(i),y(j))!. The evaluation uses
  the fact that function \verb!fun! accept vectors as arguments, if this is not
  the case one can use the primitive \verb!feval! or directly transmit the function
  \verb!fun! to plot3d is the aim was to prepare data for 3d ploting.

  Note that \verb!eval3d! is obtained by \verb![xx,yy] = ndgrid(x,y);z=fun(xx,yy)\verb!
\end{mandescription}

%--example
\begin{examples}
  \begin{Verbatim}
    if ~new_graphics() then switch_graphics();end;xclear();
    x=-5:5;y=x;
    function [z]=f(x,y); z= x.*y; endfunction;
    z=eval3d(f,x,y);
    plot3d(x,y,z);
  \end{Verbatim}
\end{examples}

%-- see also
\begin{manseealso}
  \manlink{feval}{feval}
\end{manseealso}

%-- Author

\begin{authors}
  Steer S.; ; ;

\end{authors}
