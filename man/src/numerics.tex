\chapter*{numerics (ode, f(x)=0, integrals, interpolation,
  optimization,...)}
\addcontentsline{toc}{chapter}{numerics (ode, f(x)=0, integrals, interpolation,
  optimization,...)}


\begin{quote}
\noindent
\hyperlink{linear_interpn}{linear\_interpn} - n dimensional linear interpolation \\
\hyperlink{splin}{splin} - build a cubic interpolation spline or sub-spline \\
\hyperlink{interp}{interp} - eval a cubic interpolation spline or sub-spline \\
\hyperlink{splin2d}{splin2d} - build a bicubic interpolation spline or sub-spline \\
\hyperlink{interp2d}{interp2d} - eval a bicubic interpolation spline or sub-spline \\
\hyperlink{intg}{intg} - one dimensional integration over a finite or infinite interval \\
\hyperlink{int2d}{int2d} - two dimensional integration over a rectangle or a set of triangles \\
 \hyperlink{int3d}{int3d} - three dimensional integration over a parallelepiped  or a set of tetrahedra \\
\hyperlink{ode}{ode} - ordinary differential equations solver \\
\hyperlink{fsolve}{fsolve} - non linear equations solver \\
\hyperlink{fsolve_lsq}{fsolve\_lsq} - non linear least square solver \\
\hyperlink{derivative}{derivative} - approximation of first and second derivatives of a function\\
\hyperlink{quapro}{quapro} - linear quadratic programming solver \\
\end{quote}


\input numerics/interp.tex
\input numerics/interp2d.tex  
\input numerics/linear_interpn.tex
\input numerics/quapro.tex
\input numerics/splin.tex 
\input numerics/splin2d.tex  
\input numerics/intg.tex 
\input numerics/int2d.tex 
\input numerics/int3d.tex 
\input numerics/ode.tex 
\input numerics/fsolve.tex 
\input numerics/fsolve_lsq.tex 
\input numerics/derivative.tex 
