% -*- mode: latex -*-
\mansection{cls2dls}
\begin{mandesc}
  \short{cls2dls}{bilinear transform} \\ % 
\end{mandesc}
%\index{cls2dls}\label{cls2dls}
%-- Calling sequence section
\begin{calling_sequence}
\begin{verbatim}
  [sl1]=cls2dls(sl,T [,fp])  
\end{verbatim}
\end{calling_sequence}
%-- Parameters
\begin{parameters}
  \begin{varlist}
    \vname{sl,sl1}: linear systems (\verb!syslin! lists)
    \vname{T}: real number, the sampling period
    \vname{fp}: prevarping frequency in hertz
  \end{varlist}
\end{parameters}
\begin{mandescription}
  given \verb!sl=[A,B,C,D]! (\verb!syslin! list),a continuous time system 
  \verb!cls2dls! returns the sampled system obtained by the 
  bilinear transform \verb!s=(2/T)*(z-1)/(z+1)!.
\end{mandescription}
%--example 
\begin{examples}
  \begin{mintednsp}{nsp}
    s=poly(0,'s');z=poly(0,'z');
    sl=syslin('c',(s+1)/(s^2-5*s+2));  //Continuous-time system in transfer form
    slss=tf2ss(sl);  //Now in state-space form
    sl1=cls2dls(slss,0.2);  //sl1= output of cls2dls
    sl1t=ss2tf(sl1) // Converts in transfer form
    sl2=horner(sl,(2/0.2)*(z-1)/(z+1))   //Compare sl2 and sl1
  \end{mintednsp}
\end{examples}
%-- see also
\begin{manseealso}
  \manlink{horner}{horner}  
\end{manseealso}
