% -*- mode: latex -*-
\mansection{sident}
\begin{mandesc}
  \short{sident}{discrete-time state-space realization and Kalman gain} \\ % 
\end{mandesc}
%\index{sident}\label{sident}
%-- Calling sequence section
\begin{calling_sequence}
\begin{verbatim}
  [(A,C)(,B(,D))(,K,Q,Ry,S)(,rcnd)] = sident(meth,job,s,n,l,R(,tol,t,Ai,Ci,printw))  
\end{verbatim}
\end{calling_sequence}
%-- Parameters
\begin{parameters}
  \begin{varlist}
    \vname{meth}: integer option to determine the method to use:
    \begin{varlist}
      \vname{1}: MOESP method with past inputs and outputs;
      \vname{2}: N4SID method;
      \vname{3}: combined method: A and C via MOESP, B and D via N4SID.
    \end{varlist}
    \vname{job}: integer option to determine the calculation to be performed:
    \begin{varlist}
      \vname{1}: compute all system matrices, A, B, C, D;
      \vname{2}: compute the matrices A and C only;
      \vname{3}: compute the matrix B only;
      \vname{4}: compute the matrices B and D only.
    \end{varlist}
    \vname{s}: the number of block rows in the processed input and output block Hankel matrices.  s $>$ 0.
    \vname{n}: integer, the order of the system
    \vname{l}: integer, the number of the system outputs
    \vname{R}: the 2*(m+l)*s-by-2*(m+l)*s part of  R  contains the processed upper triangular factor  R  from the QR factorization of the concatenated block-Hankel matrices, and further details needed for computing system matrices.
    \vname{tol}: (optional) tolerance used for estimating the rank of matrices. If  tol $>$ 0,  then the given value of  tol  is used as a lower bound for the reciprocal condition number; an m-by-n matrix whose estimated condition number is less than  1/tol  is considered to be of full rank. Default:    m*n*epsilon\_machine where epsilon\_machine is the relative machine precision.
    \vname{t}: (optional) the total number of samples used for calculating the covariance matrices.  Either t = 0, or t $>$= 2*(m+l)*s. This parameter is not needed if the covariance matrices and/or the Kalman predictor gain matrix are not desired. If t = 0, then K, Q, Ry, and S are not computed. Default:    t = 0.
    \vname{Ai}: real matrix
    \vname{Ci}:  real matrix
    \vname{printw}: (optional) switch for printing the warning messages.
    \begin{varlist}
      \vname{=  } 1:  print warning messages;
      \vname{=  } 0:  do not print warning messages.
    \end{varlist}
    Default:    printw = 0.
    \vname{A}: real matrix
    \vname{C}: real matrix
    \vname{B}: real matrix
    \vname{D}: real matrix
    \vname{K}: real matrix, kalman gain
    \vname{Q}: (optional) the n-by-n positive semidefinite state covariance matrix used as state weighting matrix when computing the Kalman gain.
    \vname{RY}: (optional) the l-by-l positive (semi)definite output covariance matrix used as output weighting matrix when computing the Kalman gain.
    \vname{S}: (optional) the n-by-l state-output cross-covariance matrix used as cross-weighting matrix when computing the Kalman gain.
    \vname{rcnd}: (optional) vector of length lr, containing estimates of the reciprocal condition numbers of the matrices involved in rank decisions, least squares, or Riccati equation solutions, where   lr = 4,  if Kalman gain matrix K is not required, and  lr = 12, if Kalman gain matrix K is required.
  \end{varlist}
\end{parameters}
\begin{mandescription}
  SIDENT function for computing a discrete-time state-space realization
  (A,B,C,D) and Kalman gain K using SLICOT routine IB01BD.
\begin{verbatim}
  [A,C,B,D] = sident(meth,1,s,n,l,R)
  [A,C,B,D,K,Q,Ry,S,rcnd] = sident(meth,1,s,n,l,R,tol,t)
  [A,C] = sident(meth,2,s,n,l,R)
  B = sident(meth,3,s,n,l,R,tol,0,Ai,Ci)
  [B,K,Q,Ry,S,rcnd] = sident(meth,3,s,n,l,R,tol,t,Ai,Ci)
  [B,D] = sident(meth,4,s,n,l,R,tol,0,Ai,Ci)
  [B,D,K,Q,Ry,S,rcnd] = sident(meth,4,s,n,l,R,tol,t,Ai,Ci)
\end{verbatim}
SIDENT computes a state-space realization (A,B,C,D) and the Kalman
predictor gain K of a discrete-time system, given the system
order and the relevant part of the R factor of the concatenated 
block-Hankel matrices, using subspace identification techniques 
(MOESP, N4SID, or their combination).
The model structure is:
\begin{verbatim}
  x(k+1) = Ax(k) + Bu(k) + Ke(k),   k $>$= 1,
  y(k)   = Cx(k) + Du(k) + e(k),
\end{verbatim}
where  
x(k)  is the  n-dimensional state vector (at time k),
u(k)  is the  m-dimensional input vector,
y(k)  is the  l-dimensional output vector,
e(k)  is the  l-dimensional disturbance vector,
and  A, B, C, D, and K  are real matrices of appropriate dimensions.
\end{mandescription}
%-- section-Comments
\paragraph{Comments}
1. The n-by-n system state matrix A, and the p-by-n system output  matrix C are computed for job $<$= 2.
2. The n-by-m system input matrix B is computed for job $<$$>$ 2.
3. The l-by-m system matrix D is computed for job = 1 or 4.
4. The n-by-l Kalman predictor gain matrix K and the covariance matrices Q, Ry, and S are computed for t $>$ 0.
%--example 
\begin{examples}
  \begin{mintednsp}{nsp}
    //generate data from a given linear system
    A = [ 0.5, 0.1,-0.1, 0.2;
      0.1, 0,  -0.1,-0.1;      
      -0.4,-0.6,-0.7,-0.1;  
      0.8, 0,  -0.6,-0.6];      
    B = [0.8;0.1;1;-1];
    C = [1 2 -1 0];
    SYS=syslin(0.1,A,B,C);
    nsmp=100;
    U=prbs_a(nsmp,nsmp/5);
    Y=(flts(U,SYS)+0.3*randn(1,nsmp));
    S = 15;
    N = 3;
    METH=1;
    [R,N1] = findR(S,Y',U',METH);
    [A,C,B,D,K] = sident(METH,1,S,N,1,R);
    SYS1=syslin(1,A,B,C,D);
    SYS1.X0 = inistate(SYS1,Y',U');
    Y1=flts(U,SYS1);
    xbasc();plot2d((1:nsmp)',[Y',Y1'])
    METH = 2;
    [R,N1,SVAL] = findR(S,Y',U',METH);
    tol = 0;
    t = size(U',1)-2*S+1;
    [A,C,B,D,K] = sident(METH,1,S,N,1,R,tol,t)
    SYS1=syslin(1,A,B,C,D)
    SYS1.X0 = inistate(SYS1,Y',U');
    Y1=flts(U,SYS1);
    xbasc();plot2d((1:nsmp)',[Y',Y1'])
  \end{mintednsp}
\end{examples}
%-- see also
\begin{manseealso}
  \manlink{findBD}{findBD}, \manlink{sorder}{sorder}  
\end{manseealso}
%-- Author
\begin{authors}
  V. Sima, Research Institute for Informatics, Bucharest, Oct. 1999. 
  Revisions: May 2000, July 2000. 
\end{authors}
