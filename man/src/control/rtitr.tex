% -*- mode: latex -*-
\mansection{rtitr}
\begin{mandesc}
  \short{rtitr}{discrete time response (transfer matrix)  } \\ % 
\end{mandesc}
%\index{rtitr}\label{rtitr}
%-- Calling sequence section
\begin{calling_sequence}
\begin{verbatim}
  [y]=rtitr(Num,Den,u [,up,yp])  
\end{verbatim}
\end{calling_sequence}
%-- Parameters
\begin{parameters}
  \begin{varlist}
    \vname{Num,Den}: polynomial  matrices (resp. dimensions: \verb!n!x\verb!m! and \verb!n!x\verb!n!)
    \vname{u}: real matrix (dimension \verb!m!x\verb!(t+1)!
    \vname{up,yp}: real matrices (\verb!up! dimension \verb!m!x\verb!(maxi(degree(Den)))! (default values=\verb!0!) ,
    \verb!yp! dimension \verb!n!x 
    \verb!(maxi(degree(Den)))!)
    \vname{y}: real matrix
  \end{varlist}
\end{parameters}
\begin{mandescription}
  \verb!y=rtitr(Num,Den,u [,up,yp])! returns the time response of
  the discrete time linear system with transfer matrix \verb!Den^-1 Num! 
  for the input \verb!u!, i.e \verb!y! and \verb!u! are such that \verb!Den y = Num u! at t=0,1,...
  If \verb!d1=maxi(degree(Den))!, and \verb!d2=maxi(degree(Num))! the polynomial 
  matrices  \verb!Den(z)! and \verb!Num(z)! may be written respectively as:
\begin{verbatim}
  D(z)= D_0  + D_1  z + ... + D_d1   z^d1
  N(z)= N_0  + N_1  z + ... + N_d2   z^d2
\end{verbatim}
and \verb!Den y = Num u! is interpreted as the recursion:
\begin{verbatim}
  D(0)y(t)+D(1)y(t+1)+...+ D(d1)y(t+d1)= N(0) u(t) +....+ N(d2) u(t+d2)
\end{verbatim}
It is assumed that \verb!D(d1)! is non singular.
The columns of u are the inputs of the system at t=0,1,...,T:
\begin{verbatim}
  u=[u(0) , u(1),...,u(T)]
\end{verbatim}
The outputs at \verb!t=0,1,...,T+d1-d2! are the columns of the matrix \verb!y!:
\begin{verbatim}
  y=[y(0), y(1),  .... y(T+d1-d2)]
\end{verbatim}
\verb!up! and \verb!yp! define the initial conditions for t $<$ 0 i.e
\begin{verbatim}
  up=[u(-d1), ..., u(-1)  ]
  yp=[y(-d1), ...  y(-1)  ]
\end{verbatim}
Depending on the relative values of \verb!d1! and \verb!d2!, some of the
leftmost components of \verb!up!, \verb!yp! are ignored.
The default values of \verb!up! and \verb!yp! are zero:
\verb!up = 0*ones(m,d1), yp=0*ones(n,d1)!
\end{mandescription}
%--example 
\begin{examples}
  \begin{mintednsp}{nsp}
    z=poly(0,'z');
    Num=1+z;Den=1+z;u=[1,2,3,4,5];
    rtitr(Num,Den,u)-u
    //Other examples
    //siso
    //causal
    n1=1;d1=poly([1 1],'z','coeff');       // y(j)=-y(j-1)+u(j-1)
    r1=[0 1 0 1 0 1 0 1 0 1 0];
    r=rtitr(n1,d1,ones(1,10));norm(r1-r,1)
    //hot restart
    r=rtitr(n1,d1,ones(1,9),1,0);norm(r1(2:11)-r)
    //non causal
    n2=poly([1 1 1],'z','coeff');d2=d1;    // y(j)=-y(j-1)+u(j-1)+u(j)+u(j+1)
    r2=[2 1 2 1 2 1 2 1 2];
    r=rtitr(n2,d2,ones(1,10));norm(r-r2,1)
    //hot restart
    r=rtitr(n2,d2,ones(1,9),1,2);norm(r2(2:9)-r,1)
    //
    //MIMO example
    //causal
    d1=d1*diag([1 0.5]);n1=[1 3 1;2 4 1];r1=[5;14]*r1;
    r=rtitr(n1,d1,ones(3,10));norm(r1-r,1)
    //
    r=rtitr(n1,d1,ones(3,9),[1;1;1],[0;0]);
    norm(r1(:,2:11)-r,1)
    //polynomial n1  (same ex.)
    n1=n1*poly(1,'z','c');r=rtitr(n1,d1,ones(3,10));norm(r1-r,1)
    //
    r=rtitr(n1,d1,ones(3,9),[1;1;1],[0;0]);
    norm(r1(:,2:11)-r,1)
    //non causal
    d2=d1;n2=n2*n1;r2=[5;14]*r2;
    r=rtitr(n2,d2,ones(3,10));norm(r2-r)
    //
    r=rtitr(n2,d2,ones(3,9),[1;1;1],[10;28]);
    norm(r2(:,2:9)-r,1)
    //
    //  State-space or transfer
    a = [0.21 , 0.63 , 0.56 , 0.23 , 0.31
      0.76 , 0.85 , 0.66 , 0.23 , 0.93
      0 , 0.69 , 0.73 , 0.22 , 0.21
      0.33 , 0.88 , 0.2 , 0.88 , 0.31
      0.67 , 0.07 , 0.54 , 0.65 , 0.36];
    b = [0.29 , 0.5 , 0.92
      0.57 , 0.44 , 0.04
      0.48 , 0.27 , 0.48
      0.33 , 0.63 , 0.26
      0.59 , 0.41 , 0.41];
    c = [0.28 , 0.78 , 0.11 , 0.15 , 0.84
      0.13 , 0.21 , 0.69 , 0.7 , 0.41];
    d = [0.41 , 0.11 , 0.56
      0.88 , 0.2 , 0.59];
    s=syslin('d',a,b,c,d);
    h=ss2tf(s);num=h.num;den=h.den;den=den(1,1)*eye(2,2);
    u=1;u(3,10)=0;r3=flts(u,s);
    r=rtitr(num,den,u);norm(r3-r,1)
  \end{mintednsp}
\end{examples}
%-- see also
\begin{manseealso}
  \manlink{ltitr}{ltitr} \manlink{exp}{exp} \manlink{flts}{flts}  
\end{manseealso}
