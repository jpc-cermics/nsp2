% -*- mode: latex -*-
\mansection{flts}
\begin{mandesc}
  \short{flts}{time response (discrete time, sampled system)  } \\ % 
\end{mandesc}
%\index{flts}\label{flts}
%-- Calling sequence section
\begin{calling_sequence}
\begin{verbatim}
  [y [,x]]=flts(u,sl [,x0])  
  [y]=flts(u,sl [,past])    
\end{verbatim}
\end{calling_sequence}
%-- Parameters
\begin{parameters}
  \begin{varlist}
    \vname{u}: matrix (input vector)
    \vname{sl}: list (linear system \verb!syslin!)
    \vname{x0}: vector (initial state ; default value=\verb!0!)
    \vname{past}: matrix (of the past ; default value=\verb!0!)
    \vname{x,y}: matrices (state and output)
  \end{varlist}
\end{parameters}
\begin{mandescription}
  State-space form:\verb!sl! is a \verb!syslin! list containing the matrices of the 
  following linear system\verb!sl=syslin('d',A,B,C,D)! (see \verb!syslin!):
\begin{verbatim}
  x[t+1] = A x[t] + B u[t]
  y[t] = C x[t] + D u[t]
\end{verbatim}
or, more generally, if \verb!D! is a polynomial matrix (\verb!p = degree(D(z))!):
Transfer form:\verb! y=flts(u,sl[,past])!. Here \verb!sl! is a linear system in 
transfer matrix  representation i.e\verb!sl=syslin('d',transfer_matrix)! (see \verb!syslin!).
is the matrix of past values of u and y.
nd is the maximum of degrees of lcm's of each row of the denominator
matrix of sl.
\begin{verbatim}
  u=[u0 u1 ... un]  (input)
  y=[y0 y1 ... yn]  (output)
\end{verbatim}
p is the difference between maximum degree of numerator and 
maximum degree of denominator
\end{mandescription}
%--example 
\begin{examples}
  \begin{mintednsp}{nsp}
    sl=syslin('d',1,1,1);u=1:10;
    y=flts(u,sl); 
    plot2d2((1:size(u,'c'))',y')
    [y1,x1]=flts(u(1:5),sl);y2=flts(u(6:10),sl,x1);
    y-[y1,y2]
    //With polynomial D:
    z=poly(0,'z');
    D=1+z+z^2; p =D.degree[];
    sl=syslin('d',1,1,1,D);
    y=flts(u,sl);[y1,x1]=flts(u(1:5),sl);
    y2=flts(u(5-p+1:10),sl,x1);  // (update)
    y-[y1,y2]
    //Delay (transfer form): flts(u,1/z)
    // Usual responses
    z=poly(0,'z');
    h=(1-2*z)/(z^2+0.3*z+1)
    u=zeros(1,20);u(1)=1;
    imprep=flts(u,tf2ss(h));   //Impulse response
    plot2d2((1:size(u,'c'))',imprep')
    u=ones(1,20);
    stprep=flts(u,tf2ss(h));   //Step response
    plot2d2((1:size(u,'c'))',stprep')
    //
    // Other examples
    A=[1 2 3;0 2 4;0 0 1];B=[1 0;0 0;0 1];C=eye(3,3);Sys=syslin('d',A,B,C);
    H=ss2tf(Sys); u=[1;-1]*(1:10);
    //
    yh=flts(u,H); ys=flts(u,Sys);
    norm(yh-ys,1)    
    //hot restart
    [ys1,x]=flts(u(:,1:4),Sys);ys2=flts(u(:,5:10),Sys,x);
    norm([ys1,ys2]-ys,1)
    //
    yh1=flts(u(:,1:4),H);yh2=flts(u(:,5:10),H,[u(:,2:4);yh(:,2:4)]);
    norm([yh1,yh2]-yh,1)
    //with D$<$$>$0
    D=[-3 8;4 -0.5;2.2 0.9];
    Sys=syslin('d',A,B,C,D);
    H=ss2tf(Sys); u=[1;-1]*(1:10);
    rh=flts(u,H); rs=flts(u,Sys);
    norm(rh-rs,1)
    //hot restart
    [ys1,x]=flts(u(:,1:4),Sys);ys2=flts(u(:,5:10),Sys,x);
    norm([ys1,ys2]-rs,1)
    //With H:
    yh1=flts(u(:,1:4),H);yh2=flts(u(:,5:10),H,[u(:,2:4); yh1(:,2:4)]);
    norm([yh1,yh2]-rh)
  \end{mintednsp}
\end{examples}
%-- see also
\begin{manseealso}
  \manlink{ltitr}{ltitr} \manlink{dsimul}{dsimul} \manlink{rtitr}{rtitr}  
\end{manseealso}
