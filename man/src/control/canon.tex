% -*- mode: latex -*-
\mansection{canon}
\begin{mandesc}
  \short{canon}{canonical controllable form} \\ % 
\end{mandesc}
%\index{canon}\label{canon}
%-- Calling sequence section
\begin{calling_sequence}
\begin{verbatim}
  [Ac,Bc,U,ind]=canon(A,B)    
\end{verbatim}
\end{calling_sequence}
%-- Parameters
\begin{parameters}
  \begin{varlist}
    \vname{Ac,Bc}: canonical form
    \vname{U}: current basis (square nonsingular matrix)
    \vname{ind}: vector of integers, controllability indices
  \end{varlist}
\end{parameters}
\begin{mandescription}
  gives the canonical controllable form of the pair \verb!(A,B)!.\verb!Ac=inv(U)*A*U, Bc=inv(U)*B!
  The vector \verb!ind! is made of the \verb!epsilon_i!'s indices
  of the pencil \verb![sI - A ,  B]! (decreasing order).
  For example with \verb!ind=[3,2]!, \verb!Ac! and \verb!Bc! are as follows:
\begin{verbatim}
        [*,*,*,*,*]           [*]
        [1,0,0,0,0]           [0]
  Ac=   [0,1,0,0,0]        Bc=[0]
        [*,*,*,*,*]           [*]
        [0,0,0,1,0]           [0]
\end{verbatim}
  If \verb!(A,B)! is controllable, by an appropriate choice 
  of \verb!F! the \verb!*! entries of \verb!Ac+Bc*F! 
  can be arbitrarily set to desired values (pole placement).
\end{mandescription}
%--example 
\begin{examples}
  \begin{mintednsp}{nsp}
    A=[1,2,3,4,5;
      1,0,0,0,0;
      0,1,0,0,0;
      6,7,8,9,0;
      0,0,0,1,0];
    B=[1,2;
      0,0;
      0,0;
      2,1;
      0,0];
    X=rand(5,5);A=X*A*inv(X);B=X*B;    //Controllable pair 
    [Ac,Bc,U,ind]=canon(A,B);  //Two indices --$>$ ind=[3.2];
    index=1;for k=1:size(ind,'*')-1,index=[index,1+sum(ind(1:k))];end
    Acstar=Ac(index,:);Bcstar=Bc(index,:);
    s=poly(0,'s');
    p1=s^3+2*s^2-5*s+3;p2=(s-5)*(s-3);   
    //p1 and p2 are desired closed-loop polynomials with degrees 3,2
    c1=coeff(p1);c1=c1($-1:-1:1);c2=coeff(p2);c2=c2($-1:-1:1);
    Acstardesired=[-c1,0,0;0,0,0,-c2];  
    //Acstardesired(index,:) is companion matrix with char. pol=p1*p2
    F=Bcstar\(Acstardesired-Acstar);   //Feedbak gain
    Ac+Bc*F         // Companion form 
    spec(A+B*F/U)   // F/U is the gain matrix in original basis.
  \end{mintednsp}
\end{examples}
%-- see also
\begin{manseealso}
  \manlink{obsv\_mat}{obsv-mat} \manlink{cont\_mat}{cont-mat} \manlink{ctr\_gram}{ctr-gram} \manlink{contrss}{contrss} \manlink{ppol}{ppol} \manlink{contr}{contr} \manlink{stabil}{stabil}  
\end{manseealso}
%-- Author
\begin{authors}
  F.D.  
\end{authors}
