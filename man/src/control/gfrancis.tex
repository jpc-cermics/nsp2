% -*- mode: latex -*-
\mansection{gfrancis}
\begin{mandesc}
  \short{gfrancis}{Francis equations for tracking} \\ % 
\end{mandesc}
%\index{gfrancis}\label{gfrancis}
%-- Calling sequence section
\begin{calling_sequence}
\begin{verbatim}
  [L,M,T]=gfrancis(Plant,Model)  
\end{verbatim}
\end{calling_sequence}
%-- Parameters
\begin{parameters}
  \begin{varlist}
    \vname{Plant}: \verb!syslin! list
    \vname{Model}: \verb!syslin! list
    \vname{L,M,T}: real matrices
  \end{varlist}
\end{parameters}
\begin{mandescription}
  Given the the linear plant:
\begin{verbatim}
  x'= F*x + G*u 
  y = H*x + J*u
\end{verbatim}
and the linear model
\begin{verbatim}
  xm'= A*xm + B*um
  ym = C*xm + D*um
\end{verbatim}
the goal is for the plant to track the model i.e. \verb!e = y - ym ---$>$ 0!
while keeping stable the state x(t) of the plant. 
\verb!u! is given by feedforward and feedback
\begin{verbatim}
  u = L*xm + M*um + K*(x-T*xm) = [K , L-K*T] *(x,xm) + M*um
\end{verbatim}
The matrices T,L,M satisfy generalized Francis equations
\begin{verbatim}
  F*T + G*L = T*A
  H*T + J*L = C
  G*M = T*B
  J*M = D
\end{verbatim}
The matrix \verb!K! must be chosen as stabilizing the pair \verb!(F,G)!
See example of use in directory \verb!demos/tracking!.
\end{mandescription}
%--example 
\begin{examples}
  \begin{mintednsp}{nsp}
    Plant=ssrand(1,3,5);
    [F,G,H,J]=abcd(Plant);
    nw=4;nuu=2;A=rand(nw,nw);
    st=max(real(spec(A)));A=A-st*eye(size(A));
    B=rand(nw,nuu);C=2*rand(1,nw);D=0*rand(C*B);
    Model=syslin('c',A,B,C,D);
    [L,M,T]=gfrancis(Plant,Model);
    if norm(F*T+G*L-T*A,1) > 1000*%eps then pause;end
    if norm(H*T+J*L-C,1)> 1000*%eps then pause;end
    if norm(G*M-T*B,1)> 1000*%eps then pause;end
    if norm(J*M-D,1)> 1000*%eps then pause;end
  \end{mintednsp}
\end{examples}
%-- see also
\begin{manseealso}
  \manlink{lqg}{lqg} \manlink{ppol}{ppol}  
\end{manseealso}
