% -*- mode: latex -*-
\mansection{projsl}
\begin{mandesc}
  \short{projsl}{linear system projection} \\ % 
\end{mandesc}
%\index{projsl}\label{projsl}
%-- Calling sequence section
\begin{calling_sequence}
\begin{verbatim}
  [slp]=projsl(sl,Q,M)  
\end{verbatim}
\end{calling_sequence}
%-- Parameters
\begin{parameters}
  \begin{varlist}
    \vname{sl,slp}: \verb!syslin! lists
    \vname{Q,M}: matrices (projection factorization)
  \end{varlist}
\end{parameters}
\begin{mandescription}
  \verb!slp!= projected model of \verb!sl! where \verb!Q*M! is the full rank 
  factorization of the projection.
  If \verb!(A,B,C,D)! is the representation of \verb!sl!, the projected model
  is given by \verb!(M*A*Q,M*B,C*Q,D)!.
  Usually, the projection \verb!Q*M! is obtained as the spectral projection
  of an appropriate auxiliary matrix \verb!W! e.g. \verb!W! = product
  of (weighted) gramians or product of Riccati equations.
\end{mandescription}
%--example 
\begin{examples}
  \begin{Verbatim}
    sl=ssrand(2,2,5);[A,B,C,D]=abcd(sl);poles=spec(A)
    [Q,M]=pbig(A,0,'c');  //keeping unstable poles
    slred=projsl(sl,Q,M);spec(slred.A)
    sl.D=rand(2,2);  //making proper system
    trzeros(sl)  //zeros of sl
    wi=inv(sl);  //wi=inverse in state-space
    [q,m]=psmall(wi.A,2,'d');  //keeping small zeros (poles of wi) i.e. abs(z)$<$2
    slred2=projsl(sl,q,m);
    trzeros(slred2)  //zeros of slred2 = small zeros of sl
    //  Example keeping second order modes
    A=diag([-1,-2,-3]);
    sl=syslin('c',A,rand(3,2),rand(2,3));[nk2,W]=hankelsv(sl)
    [Q,M]=pbig(W,nk2(2)-%eps,'c');    //keeping 2 eigenvalues of W
    slr=projsl(sl,Q,M);  //reduced model
    hankelsv(slr)
  \end{Verbatim}
\end{examples}
%-- see also
\begin{manseealso}
  \manlink{pbig}{pbig}  
\end{manseealso}
%-- Author
\begin{authors}
  F. D.;   
\end{authors}
