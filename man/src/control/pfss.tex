% -*- mode: latex -*-
\mansection{pfss}
\begin{mandesc}
  \short{pfss}{partial fraction decomposition} \\ % 
\end{mandesc}
%\index{pfss}\label{pfss}
%-- Calling sequence section
\begin{calling_sequence}
\begin{verbatim}
  elts=pfss(Sl)  
  elts=pfss(Sl,rmax)  
  elts=pfss(Sl,'cord')  
  elts=pfss(Sl,rmax,'cord')  
\end{verbatim}
\end{calling_sequence}
%-- Parameters
\begin{parameters}
  \begin{varlist}
    \vname{Sl} \verb!syslin! list (state-space or transfer linear system)
    rmax: real number controlling the conditioning of block
    diagoanalization cord: character string \verb!'c'!
    or \verb!'d'!.
  \end{varlist}
\end{parameters}
\begin{mandescription}
  Partial fraction decomposition of the linear system \verb!Sl! (in state-space 
  form, transfer matrices are automatically converted to state-space form
  by \verb!tf2ss!):\verb!elts! is the list of linear systems which add up to \verb!Sl!
  i.e. \verb!elts=list(S1,S2,S3,...,Sn)! with:\verb!Sl = S1 + S2 +... +Sn!.
  Each \verb!Si! contains some poles of \verb!!S according to the 
  block-diagonalization of the \verb!A! matrix of \verb!S!.
  For non proper systems the polynomial part of \verb!Sl! is put
  in the last entry of \verb!elts!.
  If \verb!Sl! is given in transfer form, it is first converted into state-space
  and each subsystem \verb!Si! is then converted in transfer form.
  The A matrix is of the state-space is put into block diagonal form
  by function \verb!bdiag!. The optional parameter \verb!rmax! is sent to
  \verb!bdiag!. If \verb!rmax! should be set to a large number to enforce
  block-diagonalization.
  If the optional flag \verb!cord='c'! is given the elements in \verb!elts!
  are sorted according to the real part (resp. magnitude if \verb!cord='d'!)
  of the eigenvalues of A matrices.
\end{mandescription}
%--example 
\begin{examples}
  \begin{mintednsp}{nsp}
    W=ssrand(1,1,6);
    elts=pfss(W); 
    W1=0;for k=1:size(elts), W1=W1+ss2tf(elts(k));end
    clean(ss2tf(W)-W1)
  \end{mintednsp}
\end{examples}
%-- see also
\begin{manseealso}
  \manlink{pbig}{pbig} \manlink{bdiag}{bdiag} \manlink{coffg}{coffg} \manlink{dtsi}{dtsi}  
\end{manseealso}
%-- Author
\begin{authors}
  F.D.;   
\end{authors}
