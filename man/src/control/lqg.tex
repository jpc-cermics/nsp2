% -*- mode: latex -*-
\mansection{lqg}
\begin{mandesc}
  \short{lqg}{LQG compensator} \\ % 
\end{mandesc}
%\index{lqg}\label{lqg}
%-- Calling sequence section
\begin{calling_sequence}
\begin{verbatim}
  [K]=lqg(P,r)  
\end{verbatim}
\end{calling_sequence}
%-- Parameters
\begin{parameters}
  \begin{varlist}
    \vname{P}: \verb!syslin! list (augmented plant) in state-space form
    \vname{r}: 1x2 row vector = (number of measurements, number of inputs)
    (dimension of the 2,2 part of \verb!P!)
    \vname{K}: \verb!syslin! list (controller)
  \end{varlist}
\end{parameters}
\begin{mandescription}
  \verb!lqg!  computes the linear optimal \verb+LQG+ (H2) controller for the
  ``augmented'' plant \verb!P=syslin('c',A,B,C,D)! (continuous time) or
  \verb!P=syslin('d',A,B,C,D)! (discrete time).  The function \verb!lqg2stan!
  returns \verb!P! and \verb!r! given the nominal plant, weighting terms and
  variances of noises. The matrix \verb!K! is given by the following \verb+ABCD+ matrices:
  \verb![A+B*Kc+Kf*C+Kf*D*Kc,-Kf,Kc,0]! where \verb!Kc=lqr(P12)!  is the
  controller gain and \verb!Kf=lqe(P21)! is the filter gain.  See example in
  \verb!lqg2stan!.
\end{mandescription}
%-- see also
\begin{manseealso}
  \manlink{lqg2stan}{lqg2stan} \manlink{lqr}{lqr} \manlink{lqe}{lqe} \manlink{h\_inf}{h_inf} \manlink{obscont}{obscont}  
\end{manseealso}
%-- Author
\begin{authors}
  F.D.  
\end{authors}
