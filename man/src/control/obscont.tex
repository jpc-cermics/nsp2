% -*- mode: latex -*-
\mansection{obscont}
\begin{mandesc}
  \short{obscont}{observer based controller} \\ % 
\end{mandesc}
%\index{obscont}\label{obscont}
%-- Calling sequence section
\begin{calling_sequence}
\begin{verbatim}
  [K]=obscont(P,Kc,Kf)  
  [J,r]=obscont(P,Kc,Kf)  
\end{verbatim}
\end{calling_sequence}
%-- Parameters
\begin{parameters}
  \begin{varlist}
    \vname{P}: \verb!syslin! list (nominal plant) in state-space form,
    continuous or discrete time
    \vname{Kc}: real matrix, (full state) controller gain
    \vname{Kf}: real matrix, filter gain
    \vname{K}: \verb!syslin! list (controller)
    \vname{J}: \verb!syslin! list (extended controller)
    \vname{r}: 1x2 row vector
  \end{varlist}
\end{parameters}
\begin{mandescription}
  \verb!obscont! returns the observer-based controller associated with a
  nominal plant \verb!P! with matrices \verb![A,B,C,D]!.
  The full-state control gain is \verb!Kc! and the filter gain is \verb!Kf!.
  These gains can be computed, for example, by pole placement.\verb!A+B*Kc! and
  \verb!A+Kf*C! are (usually) assumed stable.\verb!K! is a state-space
  representation of the compensator \verb!K: y-->u!
  in:\verb! xdot = A x + B u!, \verb!y=C x + D u!, \verb!zdot= (A + Kf C)z -Kf y +B u!, \verb!u=Kc z!. 
  Where \verb!K! is a linear system (\verb!syslin! list) with matrices given by:
  \verb!K=[A+B*Kc+Kf*C+Kf*D*Kc,Kf,-Kc]!.  The closed loop feedback system
  \verb! Cl: v -$>$y! with (negative) feedback \verb!K! (i.e.
  \verb!y = P u, u = v - K y!, or
\begin{verbatim}
  xdot = A x + B u, 
  y = C x + D u, 
  zdot = (A + Kf C) z - Kf y + B u, 
  u = v -F z
\end{verbatim}
) is given by \verb!Cl = P/.(-K)!

The poles of \verb!Cl! (\verb! spec(cl('A')) !) are located at the eigenvalues
of \verb!A+B*Kc!  and \verb!A+Kf*C!.  Invoked with two output arguments
\verb!obscont! returns a (square) linear system \verb!K! which parametrizes all
the stabilizing feedbacks via a LFT.  Let \verb!Q! an arbitrary stable linear
system of dimension \verb!r(2)!x\verb!r(1)!  i.e. number of inputs x number of
outputs in \verb!P!.  Then any stabilizing controller \verb!K! for \verb!P! can
be expressed as \verb!K=lft(J,r,Q)!. The controller which corresponds to
\verb!Q=0! is \verb!K=J(1:nu,1:ny)! (this \verb!K! is returned by
\verb!K=obscont(P,Kc,Kf)!).  \verb!r! is \verb!size(P)! i.e the vector [number
of outputs, number of inputs];
\end{mandescription}
%--example 
\begin{examples}
  \begin{mintednsp}{nsp}
    ny=2;nu=3;nx=4;P=ssrand(ny,nu,nx);[A,B,C,D]=abcd(P);
    Kc=-ppol(A,B,[-1,-1,-1,-1]);  //Controller gain
    Kf=-ppol(A',C',[-2,-2,-2,-2]);Kf=Kf';    //Observer gain
    cl= -P/.(obscont(P,Kc,Kf));spec(cl.A)   //closed loop system
    [J,r]=obscont(P,Kc,Kf);
    Q=ssrand(nu,ny,3);Q.A=Q.A-(max(real(spec(Q.A)))+0.5)*eye(size(Q.A))
    //Q is a stable parameter
    K=lft(J,r,Q);
    spec(h_cl(P,K))  // closed-loop A matrix (should be stable);
  \end{mintednsp}
\end{examples}
%-- see also
\begin{manseealso}
  \manlink{ppol}{ppol} \manlink{lqg}{lqg} \manlink{lqr}{lqr} \manlink{lqe}{lqe} \manlink{h\_inf}{h-inf} \manlink{lft}{lft} \manlink{syslin}{syslin} \manlink{feedback}{feedback} \manlink{observer}{observer}  
\end{manseealso}
%-- Author
\begin{authors}
  F.D.
\end{authors}
