% -*- mode: latex -*-
\mansection{svplot}
\begin{mandesc}
  \short{svplot}{singular-value sigma-plot} \\ % 
\end{mandesc}
%\index{svplot}\label{svplot}
%-- Calling sequence section
\begin{calling_sequence}
\begin{verbatim}
  [SVM]=svplot(sl,[w])  
\end{verbatim}
\end{calling_sequence}
%-- Parameters
\begin{parameters}
  \begin{varlist}
    \vname{sl}: \verb!syslin! list (continuous, discrete or sampled system)
    \vname{w}: real vector (optional parameter)
  \end{varlist}
\end{parameters}
\begin{mandescription}
  computes for the system \verb!sl=(A,B,C,D)!
  the singular values of its transfer function matrix:
\begin{verbatim}
  G(jw) = C(jw*I-A)B^-1+D
  or
  G(exp(jw)) = C(exp(jw)*I-A)B^-1+D
  or
  G(exp(jwT)) = C(exp(jw*T)*I-A)B^-1+D
\end{verbatim}
  evaluated over the frequency range specified by \verb!w!. (T is the sampling
  period, \verb!T=sl('dt')! for sampled systems).\verb!sl! is a \verb!syslin! list representing the system
  \verb![A,B,C,D]! in state-space form. \verb!sl! can be continous or
  discrete time or sampled system.
  The \verb!i!-th column of the output matrix \verb!SVM! contains the singular
  values of \verb!G! for the \verb!i!-th frequency value \verb!w(i)!.
\begin{verbatim}
  SVM = svplot(sl)
\end{verbatim}
  is equivalent to
\begin{verbatim}
  SVM = svplot(sl,logspace(-3,3))  (continuous)
\end{verbatim}
\begin{verbatim}
  SVM = svplot(sl,logspace(-3,%pi)) (discrete)
\end{verbatim}
\end{mandescription}
%--example 
\begin{examples}
  \begin{Verbatim}
    x=logspace(-3,3,10);
    y=svplot(ssrand(2,2,4),x);
    xbasc();plot2d1(x',20*log(y')/log(10),logflag='ln');
    xgrid(12)
    xtitle("Singular values plot","(Rd/sec)", "Db");
  \end{Verbatim}
\end{examples}
%-- Author
\begin{authors}
  F.D; ;   
\end{authors}
