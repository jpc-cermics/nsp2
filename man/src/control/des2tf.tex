% -*- mode: latex -*-
\mansection{des2tf}
\begin{mandesc}
  \short{des2tf}{descriptor to transfer function conversion} \\ % 
\end{mandesc}
% \index{des2tf}\label{des2tf}
%-- Calling sequence section
\begin{calling_sequence}
\begin{verbatim}
  [S]=des2tf(sl)  
  [Bfs,Bis,chis]=des2tf(sl)  
\end{verbatim}
\end{calling_sequence}
%-- Parameters
\begin{parameters}
  \begin{varlist}
    \vname{sl}: list (linear system in descriptor form)
    \vname{Bfs, Bis}: two polynomial matrices
    \vname{chis}: polynomial
    \vname{S}: rational matrix
  \end{varlist}
\end{parameters}
\begin{mandescription}
  Given the linear system in descriptor form, 
  \verb!des2tf! converts \verb!sl! into
  its transfer function representation:
\begin{verbatim}
  S=C*(s*E-A)^(-1)*B+D
\end{verbatim}

  Called with 3 outputs arguments \verb!des2tf! returns
  \verb!Bfs! and \verb!Bis! two polynomial matrices, and \verb!chis!
  polynomial such that \verb!S=Bfs/chis - Bis!
  where \verb!chis! is the determinant of \verb!(s*E-A)! (up to a constant);
\end{mandescription}
% --example 
\begin{examples}
  \begin{Verbatim}
    s=poly(0,'s');
    G=[1/(s+1),s;1+s^2,3*s^3];
    De=tf2des(G);
    Tf1=des2tf(De)
    De2=tf2des(G,"withD");
    Tf2=des2tf(De2)
    A=De2.A;B=De2.B;C=De2.C;D=De2.D;E=De2.E;
    Tf3=C*inv(s*E-A)*B+D
  \end{Verbatim}
\end{examples}
%-- see also
\begin{manseealso}
  \manlink{glever}{glever} \manlink{pol2des}{pol2des} \manlink{tf2des}{tf2des} \manlink{ss2tf}{ss2tf} \manlink{des2ss}{des2ss} \manlink{rowshuff}{rowshuff}  
\end{manseealso}
%-- Author
\begin{authors}
  F. D.  
\end{authors}
