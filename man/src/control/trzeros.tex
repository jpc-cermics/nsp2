% -*- mode: latex -*-
%
% Scilab ( http://www.scilab.org/ ) - This file is part of Scilab
% Copyright (C) 2008 - INRIA -  
% 
% This file must be used under the terms of the CeCILL.
% This source file is licensed as described in the file COPYING, which
% you should have received as part of this distribution.  The terms
% are also available at    
% http://www.cecill.info/licences/Licence_CeCILL_V2-en.txt
%



% -*- mode: latex -*-
\mansection{trzeros}
\begin{mandesc}
  \short{trzeros}{transmission zeros and normal rank} \\ % 
\end{mandesc}
%\index{trzeros}\label{trzeros}
%-- Calling sequence section
\begin{calling_sequence}
\begin{verbatim}
  [tr]=trzeros(Sl)  
  [nt,dt,rk]=trzeros(Sl)  
\end{verbatim}
\end{calling_sequence}
%-- Parameters
\begin{parameters}
  \begin{varlist}
    \vname{Sl}: linear system (\verb!syslin! list)
    \vname{nt}: complex vectors
    \vname{dt}: real vector
    \vname{rk}: integer (normal rank of Sl)
  \end{varlist}
\end{parameters}
\begin{mandescription}
  Called with one output argument, \verb!trzeros(Sl)! returns the 
  transmission zeros of the linear system \verb!Sl!.\verb!Sl! may have a polynomial (but square) \verb!D! matrix.
  Called with 2 output arguments, \verb!trzeros! returns the 
  transmission zeros of the linear system \verb!Sl! as \verb!tr=nt./dt!;
  (Note that some components of \verb!dt! may be zeros)
  Called with 3 output arguments, \verb!rk!  is the normal rank of \verb!Sl!
  Transfer matrices are converted to state-space.
  If \verb!Sl! is a (square) polynomial matrix \verb!trzeros! returns the 
  roots of its determinant.
  For usual state-space system \verb!trzeros! uses the state-space 
  algorithm of Emami-Naeni and Van Dooren.
  If \verb!D! is invertible the transmission zeros are the eigenvalues
  of the "\verb!A! matrix" of the inverse system: \verb!A - B*inv(D)*C!;
  If \verb!C*B! is invertible the transmission zeros are the eigenvalues
  of \verb!N*A*M! where \verb!M*N! is a full rank factorization of 
  \verb!eye(A)-B*inv(C*B)*C!;
  For systems with a polynomial \verb!D! matrix zeros are 
  calculated as the roots of the determinant of the system matrix.
  Caution: the computed zeros are not always reliable, in particular
  in case of repeated zeros.
\end{mandescription}
%--example 
\begin{examples}
  \begin{mintednsp}{nsp}
    W1=ssrand(2,2,5);trzeros(W1)    //call trzeros
    roots(det(systmat(W1)))         //roots of det(system matrix)
    s=poly(0,'s');W=[1/(s+1);1/(s-2)];W2=(s-3)*W*W';[nt,dt,rk]=trzeros(W2);
    St=systmat(tf2ss(W2));[Q,Z,Qd,Zd,numbeps,numbeta]=kroneck(St);
    St1=Q*St*Z;rowf=(Qd(1)+Qd(2)+1):(Qd(1)+Qd(2)+Qd(3));
    colf=(Zd(1)+Zd(2)+1):(Zd(1)+Zd(2)+Zd(3));
    roots(St1(rowf,colf)), nt./dt     //By Kronecker form
  \end{mintednsp}
\end{examples}
%-- see also
\begin{manseealso}
  \manlink{gspec}{gspec} \manlink{kroneck}{kroneck}  
\end{manseealso}
