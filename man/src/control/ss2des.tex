% -*- mode: latex -*-
\mansection{ss2des}
\begin{mandesc}
  \short{ss2des}{(polynomial) state-space to descriptor form} \\ % 
\end{mandesc}
% \index{ss2des}\label{ss2des}
% -- Calling sequence section
\begin{calling_sequence}
\begin{verbatim}
  S=ss2des(Sl [,flag])  
\end{verbatim}
\end{calling_sequence}
%-- Parameters
\begin{parameters}
  \begin{varlist}
    \vname{Sl}: a \verb!syslin! object. 
    \vname{flag}: a character string \verb!"withD"!
    \vname{S}: a hash table 
  \end{varlist}
\end{parameters}
\begin{mandescription}
  Given the linear system in state-space representation 
  \verb!Sl!, with a \verb!D! matrix which is either
  polynomial or constant, but not zero \verb!ss2des!
  returns a descriptor system as a hash table containing keys \verb!(A,B,C,D,E)!
  such that: \verb!Sl=C*(s*E-A)^(-1)*B!.

  If the flag \verb!"withD"! is given, the returned hash table contains 
  a \verb!D! matrix of maximal rank.
\end{mandescription}
% --example 
\begin{examples}
  \begin{mintednsp}{nsp}
    s=poly(0,'s');
    G=[1/(s+1),s;1+s^2,3*s^3];
    Sl=tf2ss(G);
    S=ss2des(Sl)
    S1=ss2des(Sl,"withD")
    Des=des2ss(S);
    clean(Des.D)
    Des1=des2ss(S1)
  \end{mintednsp}
\end{examples}
%-- see also
\begin{manseealso}
  \manlink{pol2des}{pol2des} \manlink{tf2des}{tf2des} \manlink{des2ss}{des2ss}  
\end{manseealso}
%-- Author
\begin{authors}
  Fran�ois Delebecque
\end{authors}
