% -*- mode: latex -*-
\mansection{p\_margin}
\begin{mandesc}
  \shortunder{p\_margin}{p_margin}{phase margin} \\ % 
\end{mandesc}
%\index{p\_margin}\label{p-margin}
%-- Calling sequence section
\begin{calling_sequence}
\begin{verbatim}
  [phm,fr]=p_margin(h)  
  phm=p_margin(h)  
\end{verbatim}
\end{calling_sequence}
%-- Parameters
\begin{parameters}
  \begin{varlist}
    \vname{h}: SISO linear system (\verb!syslin! list).
    \vname{phm}: phase margin (in degree)
    \vname{fr}: corresponding frequency (hz)
  \end{varlist}
\end{parameters}
\begin{mandescription}
  The phase margin is the values of the phase at points where the
  nyquist plot of \verb!h! crosses the unit circle.
\end{mandescription}
%--example 
\begin{examples}
  \begin{Verbatim}
    s=poly(0,'s');
    h=syslin('c',-1+s,3+2*s+s^2)  
    [p,fr]=p_margin(h)  
    [p,fr]=p_margin(h+0.7)  
    h1= h+0.7;
    nyquist(h1.num,h1.den);
    t=(0:0.1:2*%pi)';plot2d(sin(t),cos(t),line_color=-3);
  \end{Verbatim}
\end{examples}
%-- see also
\begin{manseealso}
  \manlink{chart}{chart} \manlink{black}{black} 
  \manlink{g\_margin}{g-margin} \manlink{nyquist}{nyquist}  
\end{manseealso}
%-- Author
\begin{authors}
  Serge Steer
\end{authors}
