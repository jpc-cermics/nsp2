% -*- mode: latex -*-
\mansection{ddp}
\begin{mandesc}
  \short{ddp}{disturbance decoupling} \\ % 
\end{mandesc}
%\index{ddp}\label{ddp}
%-- Calling sequence section
\begin{calling_sequence}
\begin{verbatim}
  [Closed,F,G]=ddp(Sys,zeroed,B1,D1)  
  [Closed,F,G]=ddp(Sys,zeroed,B1,D1,flag,alfa,beta)  
\end{verbatim}
\end{calling_sequence}
%-- Parameters
\begin{parameters}
  \begin{varlist}
    \vname{Sys}: \verb!syslin! list containing the matrices \verb!(A,B2,C,D2)!.
    \vname{zeroed}: integer vector, indices of outputs of \verb!Sys! which are zeroed.
    \vname{B1}: real matrix
    \vname{D1}: real matrix. \verb!B1! and \verb!D1! have the same number of columns.
    \vname{flag}: string \verb!'ge'! or \verb!'st'! (default) or \verb!'pp'!.
    \vname{alfa}: real or complex vector (loc. of closed loop poles)
    \vname{beta}: real or complex vector (loc. of closed loop poles)
  \end{varlist}
\end{parameters}
\begin{mandescription}
  Exact disturbance decoupling (output nulling algorithm).
  Given a linear system, and a subset of outputs, z, which are to
  be zeroed, characterize the inputs w of Sys such that the 
  transfer function from w to z is zero.
  \verb!Sys! is a linear system {A,B2,C,D2} with one input and two outputs 
  ( i.e.  Sys: u--$>$(z,y) ), part the following system defined from \verb!Sys!
  and \verb!B1,D1!:
\begin{verbatim}
  xdot =  A x + B1  w + B2  u
  z = C1 x + D11 w + D12 u
  y = C2 x + D21 w + D22 u
\end{verbatim}
outputs of Sys are partitioned into (z,y) where z is to be zeroed,
i.e. the matrices C and D2 are:
\begin{verbatim}
  C=[C1;C2]         D2=[D12;D22]
  C1=C(zeroed,:)    D12=D2(zeroed,:)
\end{verbatim}
The matrix \verb!D1! is partitioned similarly as \verb!D1=[D11;D21]!
with \verb!D11=D1(zeroed,:)!.
The control is u=Fx+Gw and one  looks for matriced \verb!F,G! such that the 
closed loop system: w--$>$z given by
\begin{verbatim}
  xdot= (A+B2*F)  x + (B1 + B2*G) w
  z = (C1+D12F) x + (D11+D12*G) w
\end{verbatim}
has zero transfer transfer function.\verb!flag='ge'!: no stability constraints.
\verb!flag='st'!: look for stable closed loop system (A+B2*F stable).
\verb!flag='pp'!: eigenvalues of A+B2*F are assigned to \verb!alfa! and 
\verb!beta!.
Closed is a realization of the \verb!w--$>$y! closed loop system
\begin{verbatim}
  xdot= (A+B2*F)  x + (B1 + B2*G) w
  y = (C2+D22*F) x + (D21+D22*G) w
\end{verbatim}
Stability (resp. pole placement) requires stabilizability 
(resp. controllability) of (A,B2).
\end{mandescription}
%--example 
\begin{examples}
  \begin{mintednsp}{nsp}
    nx=6;nz=3;nu=2;ny=1;
    A=diag(1:6);A(2,2)=-7;A(5,5)=-9;B2=[1,2;0,3;0,4;0,5;0,0;0,0];
    C1=[zeros(nz,nz),eye(nz,nz)];D12=[0,1;0,2;0,3];
    Sys12=syslin('c',A,B2,C1,D12);
    C=[C1;rand(ny,nx)];D2=[D12;rand(ny,size(D12,2))];
    Sys=syslin('c',A,B2,C,D2);
    [A,B2,C1,D12]=abcd(Sys12);  //The matrices of Sys12.
    alfa=-1;beta=-2;flag='ge';
    [X,dims,F,U,k,Z]=abinv(Sys12,alfa,beta,flag);
    clean(X'*(A+B2*F)*X)
    clean(X'*B2*U)
    clean((C1+D12*F)*X)
    clean(D12*U);
    //Calculating an ad-hoc B1,D1
    G1=rand(size(B2,2),3);
    B1=-B2*G1;
    D11=-D12*G1;
    D1=[D11;rand(ny,size(B1,2))];
    [Closed,F,G]=ddp(Sys,1:nz,B1,D1,'st',alfa,beta);
    closed=syslin('c',A+B2*F,B1+B2*G,C1+D12*F,D11+D12*G);
    ss2tf(closed)
  \end{mintednsp}
\end{examples}
%-- see also
\begin{manseealso}
  \manlink{abinv}{abinv} \manlink{ui\_observer}{ui-observer}  
\end{manseealso}
%-- Author
\begin{authors}
  F.D.  
\end{authors}
