% -*- mode: latex -*-
\mansection{calfrq}
\begin{mandesc}
  \short{calfrq}{frequency response discretization} \\ % 
\end{mandesc}
%\index{calfrq}\label{calfrq}
%-- Calling sequence section
\begin{calling_sequence}
\begin{verbatim}
  [frq,split]=calfrq(h,[fmin,fmax])  
\end{verbatim}
\end{calling_sequence}
%-- Parameters
\begin{parameters}
  \begin{varlist}
    \vname{h}: SISO linear system (\verb!syslin! list)
    \vname{fmin,fmax}: real scalars (min and max frequencies)
    \vname{frq}: row vector (discretization of interval)
    \vname{split}: vector of frq splitting points indexes
  \end{varlist}
\end{parameters}
\begin{mandescription}

  frequency response discretization ; \verb!frq! is the discretization of
  \verb![fmin,fmax]!  such that the peaks in the frequency response are well
  represented.  Default values for \verb!fmin! and \verb!fmax! are \verb!1.d-3!,
  \verb!1.d+3! if \verb!h! is continuous-time or \verb!1.d-3!,
  \verb!1/(2*h('dt'))! if \verb!h! is discrete-time.  Singularities are located
  between \verb!frq(split(k))! and \verb!frq(split(k)+1)!  for \verb!k > 1!.

\end{mandescription}
%--example 
\begin{examples}
  \begin{mintednsp}{nsp}
    s=poly(0,'s')
    h=syslin('c',(s^2+2*0.9*10*s+100)/(s^2+2*0.3*10.1*s+102.01))
    h1=h*syslin('c',(s^2+2*0.1*15.1*s+228.01)/(s^2+2*0.9*15*s+225)) 
    [f1,spl]=calfrq(h1,0.01,1000);
    rf=repfreq(h1,frq=f1);
    plot2d(real(rf)',imag(rf)')
  \end{mintednsp}
\end{examples}
%-- see also
\begin{manseealso}
  \manlink{bode}{bode} \manlink{black}{black} \manlink{nyquist}{nyquist} \manlink{freq}{freq} \manlink{repfreq}{repfreq} \manlink{logspace}{logspace}  
\end{manseealso}
