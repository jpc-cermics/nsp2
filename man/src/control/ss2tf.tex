% -*- mode: latex -*-
\mansection{ss2tf}
\begin{mandesc}
  \short{ss2tf}{conversion from state-space to transfer function} \\ % 
\end{mandesc}
%\index{ss2tf}\label{ss2tf}
%-- Calling sequence section
\begin{calling_sequence}
\begin{verbatim}
  [h]=ss2tf(sl)   
  [Ds,NUM,chi]=ss2tf(sl)  
  [h]=ss2tf(sl,"b")   
  [Ds,NUM,chi]=ss2tf(sl,"b")  
  [h]=ss2tf(sl,rmax)   
  [Ds,NUM,chi]=ss2tf(sl,rmax)  
\end{verbatim}
\end{calling_sequence}
%-- Parameters
\begin{parameters}
  \begin{varlist}
    \vname{sl}: linear system (\verb!syslin! list)
    \vname{h}: transfer matrix
  \end{varlist}
\end{parameters}
\begin{mandescription}
  Called with three outputs \verb![Ds,NUM,chi]=ss2tf(sl)! returns 
  the numerator polynomial matrix \verb!NUM!, the characteristic
  polynomial \verb!chi! and the polynomial part \verb!Ds! separately i.e.:
\begin{verbatim}
  h=NUM/chi + Ds
\end{verbatim}
Method:
One uses the characteristic polynomial and 
\verb!det(A+Eij)=det(A)+C(i,j)! where \verb!C! is the adjugate 
matrix of \verb!A!.
With \verb!rmax! or \verb!"b"! argument uses a block diagonalization of
sl.A matrix and applies "Leverrier" algorithm on blocks.
If given, \verb!rmax! controls the conditionning (see bdiag).
\end{mandescription}
%--example 
\begin{examples}
  \begin{mintednsp}{nsp}
    s=poly(0,'s');
    h=[1,1/s;1/(s^2+1),s/(s^2-2)]
    sl=tf2ss(h);
    h=clean(ss2tf(sl))
    [Ds,NUM,chi]=ss2tf(sl)
  \end{mintednsp}
\end{examples}
%-- see also
\begin{manseealso}
  \manlink{tf2ss}{tf2ss} \manlink{syslin}{syslin} 
  \manlink{nlev}{nlev} \manlink{glever}{glever}  
\end{manseealso}
