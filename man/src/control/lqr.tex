% -*- mode: latex -*-
\mansection{lqr}
\begin{mandesc}
  \short{lqr}{LQ compensator (full state)} \\ % 
\end{mandesc}
%\index{lqr}\label{lqr}
%-- Calling sequence section
\begin{calling_sequence}
\begin{verbatim}
  [K,X]=lqr(P12)  
\end{verbatim}
\end{calling_sequence}
%-- Parameters
\begin{parameters}
  \begin{varlist}
    \vname{P12}: \verb!syslin! list (state-space linear system)
    \vname{K,X}: two real matrices
  \end{varlist}
\end{parameters}
\begin{mandescription}
  \verb!lqr!  computes the linear optimal LQ full-state gain
  for the plant \verb!P12=[A,B2,C1,D12]! in continuous or
  discrete time.\verb!P12! is a \verb!syslin! list (e.g. \verb!P12=syslin('c',A,B2,C1,D12)!).
  The cost function is l2-norm of \verb!z'*z! with \verb!z=C1 x + D12 u!
  i.e. \verb![x,u]' * BigQ * [x;u]! where
\begin{verbatim}
        [C1' ]               [Q  S]
  BigQ= [    ]  * [C1 D12] = [    ]
        [D12']               [S' R]
\end{verbatim}
The gain \verb!K! is such that \verb!A + B2*K! is stable.\verb!X! is the stabilizing solution of the Riccati equation.
For a continuous plant:
\begin{verbatim}
  (A-B2*inv(R)*S')'*X+X*(A-B2*inv(R)*S')-X*B2*inv(R)*B2'*X+Q-S*inv(R)*S'=0
\end{verbatim}
\begin{verbatim}
  K=-inv(R)*(B2'*X+S)
\end{verbatim}
For a discrete plant:
\begin{verbatim}
  X=A'*X*A-(A'*X*B2+C1'*D12)*pinv(B2'*X*B2+D12'*D12)*(B2'*X*A+D12'*C1)+C1'*C1;
\end{verbatim}
\begin{verbatim}
  K=-pinv(B2'*X*B2+D12'*D12)*(B2'*X*A+D12'*C1)
\end{verbatim}
An equivalent form for \verb!X! is
\begin{verbatim}
  X=Abar'*inv(inv(X)+B2*inv(r)*B2')*Abar+Qbar
\end{verbatim}
with \verb!Abar=A-B2*inv(R)*S'! and \verb!Qbar=Q-S*inv(R)*S'!
The 3-blocks matrix pencils associated with these Riccati equations are:
\begin{verbatim}
  discrete                               continuous
   |I   0    0|   | A    0    B2|         |I   0   0|   | A    0    B2|
  z|0   A'   0| - |-Q    I    -S|        s|0   I   0| - |-Q   -A'   -S|
   |0   B2'  0|   | S'   0     R|         |0   0   0|   | S'  -B2'   R|
\end{verbatim}
Caution: It is assumed that matrix R is non singular. In particular,
the plant must be tall (number of outputs $>$= number of inputs).
\end{mandescription}
%--example 
\begin{examples}
  \begin{mintednsp}{nsp}
    A=rand(2,2);B=rand(2,1);   //two states, one input
    Q=diag([2,5]);R=2;     //Usual notations x'Qx + u'Ru
    Big=sysdiag(Q,R);    //Now we calculate C1 and D12
    [w,wp]=fullrf(Big);C1=wp(:,1:2);D12=wp(:,3:$); 
    //[C1,D12]'*[C1,D12]=Big
    P=syslin('c',A,B,C1,D12);    //The plant (continuous-time)
    [K,X]=lqr(P)
    spec(A+B*K)    //check stability
    norm(A'*X+X*A-X*B*inv(R)*B'*X+Q,1)  //Riccati check
    P=syslin('d',A,B,C1,D12);    // Discrete time plant
    [K,X]=lqr(P)     
    spec(A+B*K)   //check stability
    norm(A'*X*A-(A'*X*B)*pinv(B'*X*B+R)*(B'*X*A)+Q-X,1) //Riccati check
  \end{mintednsp}
\end{examples}
%-- see also
\begin{manseealso}
  \manlink{lqe}{lqe} \manlink{gcare}{gcare} \manlink{leqr}{leqr}  
\end{manseealso}
%-- Author
\begin{authors}
  F.D.;   
\end{authors}
