% -*- mode: latex -*-
\mansection{syslin}
\begin{mandesc}
  \short{syslin}{linear system definition} \\ % 
\end{mandesc}
%\index{syslin}\label{syslin}
%-- Calling sequence section
\begin{calling_sequence}
\begin{verbatim}
  [sl]=syslin(dom,A,B,C [,D [,x0] ])  
  [sl]=syslin(dom,N,D)  
  [sl]=syslin(dom,H)  
\end{verbatim}
\end{calling_sequence}
%-- Parameters
\begin{parameters}
  \begin{varlist}
    \vname{dom}: character string (\verb!'c'!, \verb!'d'!), or \verb![]! or a scalar.
    \vname{A,B,C,D}: matrices of the state-space representation (\verb!D! optional with default value zero matrix). 
    For improper systems \verb!D! is a polynomial matrix.
    \vname{x0}: vector (initial state; default value is \verb!0!)
    \vname{N, D}: polynomial matrices
    \vname{H}: rational matrix or linear state space representation
    \vname{sl}: tlist ("\verb!syslin!" list) representing the linear system
  \end{varlist}
\end{parameters}
\begin{mandescription}
  \verb!syslin!  defines a linear system as a list and checks consistency of
  data.\verb!dom! specifies the time domain of the system and can have the
  following values:\verb!dom='c'! for a continuous time system, \verb!dom='d'!
  for a discrete time system, \verb!n! for a sampled system with sampling period
  \verb!n! (in seconds).\verb!dom=[]! if the time domain is undefined
  State-space representation:
\begin{verbatim}
  sl=syslin(dom,A,B,C [,D [,x0] ])
\end{verbatim}
represents the system:
\begin{verbatim}
  s x  = A*x + B*u
  y  = C*x + D*u
  x(0) = x0
\end{verbatim}
The output of \verb!syslin! is a list of the following form: 
\verb!sl=tlist(['lss','A','B','C','D','X0','dt'],A,B,C,D,x0,dom)!
Note that \verb!D! is allowed to be a polynomial matrix (improper systems).
Transfer matrix representation:
\begin{verbatim}
  sl=syslin(dom,N,D) 
  sl=syslin(dom,H)
\end{verbatim}
The output of \verb!syslin! is a list of the following form:  
\verb!sl=tlist(['r','num','den','dt'],N,D,dom)! or \verb!sl=tlist(['r','num','den','dt'],H(2),H(3),dom)!.
Linear systems defined as \verb!syslin! can be manipulated as
usual matrices (concatenation, extraction, transpose, multiplication, etc)
both in state-space or transfer representation.
Most of state-space control functions receive a \verb!syslin! list as input instead
of the four matrices defining the system.
\end{mandescription}
%--example 
\begin{examples}
  \begin{mintednsp}{nsp}
    A=[0,1;0,0];B=[1;1];C=[1,1];
    S1=syslin('c',A,B,C)   //Linear system definition
    S1.A    //Display of A-matrix
    S1.X0, S1.dom, S1.dt // Display of X0 and time domain
    s=poly(0,'s');
    D=s;
    S2=syslin('c',A,B,C,D)
    H1=(1+2*s)/s^2, S1bis=syslin('c',H1)
    H2=(1+2*s+s^3)/s^2, S2bis=syslin('c',H2)
    S1+S2
    [S1,S2]
    ss2tf(S1)-S1bis
    S1bis+S2bis
    S1*S2bis
    size(S1)
  \end{mintednsp}
\end{examples}
%-- see also
\begin{manseealso}
  \manlink{tlist}{tlist} \manlink{lsslist}{lsslist} \manlink{rlist}{rlist} \manlink{ssrand}{ssrand} \manlink{ss2tf}{ss2tf} \manlink{tf2ss}{tf2ss} \manlink{dscr}{dscr} \manlink{abcd}{abcd}  
\end{manseealso}
