% -*- mode: latex -*-
\mansection{frep2tf}
\begin{mandesc}
  \short{frep2tf}{transfer function realization from frequency response} \\ % 
\end{mandesc}
%\index{frep2tf}\label{frep2tf}
%-- Calling sequence section
\begin{calling_sequence}
\begin{verbatim}
  [h [,err]]=frep2tf(frq,repf,dg [,dom,tols,weight])   
\end{verbatim}
\end{calling_sequence}
%-- Parameters
\begin{parameters}
  \begin{varlist}
    \vname{frq}: vector of frequencies in Hz.
    \vname{repf}: vector of frequency response
    \vname{dg}: degree of linear system
    \vname{dom}: time domain (\verb!'c'! or \verb!'d'! or \verb!dt!)
    \vname{tols}: a vector of size 3 giving the relative and absolute tolerance
    and the maximum number of iterations (default values are
    \verb!rtol=1.e-2; atol=1.e-4, N=10!).
    \vname{weight}: vector of weights on frequencies
    \vname{h}: SISO transfer function
    \vname{err}: error (for example if \verb!dom='c'! error is given by 
    \verb!sum(abs(h(2i*pi*frq) - rep)^2)/size(frq,*)!)
  \end{varlist}
\end{parameters}
\begin{mandescription}
  Frequency response to  transfer function conversion. The order of \verb!h!
  is a priori given in \verb!dg! which must be provided.
  The following linear system is solved in the least square sense.
\begin{verbatim}
  weight(k)*(n( phi_k) - d(phi_k)*rep_k)=0, k=1,..,n
\end{verbatim}
  where \verb!phi_k= 2*%i*%pi*frq! when \verb!dom='c'! and
  \verb!phi_k=exp(2*%i*%pi*dom*frq! if not. If the \verb!weight! vector is not
  given a default
  penalization is used (when \verb!dom='c'!).  A stable and minimum phase system
  can be obtained by using function \verb!factors!.
\end{mandescription}
%--example 
\begin{examples}
  \begin{Verbatim}
    s=poly(0,'s');
    h=syslin('c',(s-1)/(s^3+5*s+20))
    frq=0:0.05:3;repf=repfreq(h,frq=frq);
    clean(frep2tf(frq,repf,3))
    Sys=ssrand(1,1,10); 
    frq=logspace(-3,2,200);
    [frq,rep]=repfreq(Sys,frq=frq);  //Frequency response of Sys
    [Sys2,err]=frep2tf(frq,rep,10);Sys2=clean(Sys2)//Sys2 obtained from freq. resp of Sys
    [frq,rep2]=repfreq(Sys2,frq=frq); //Frequency response of Sys2
    xbasc();bode(frq,[rep;rep2])   //Responses of Sys and Sys2
    // 
    sp1 = spec(Sys.A);
    [vsp,isp]=sort(real(sp1));sp1=sp1(isp);
    sp2 = roots(Sys2.den);
    [vsp,isp]=sort(real(sp2));sp2=sp2(isp);
    [sp1,sp2] //poles
    dom=1/1000; // Sampling time 
    z=poly(0,'z');
    h=syslin(dom,(z^2+0.5)/(z^3+0.1*z^2-0.5*z+0.08))
    frq=(0:0.01:0.5)/dom;repf=repfreq(h,frq=frq);
    [Sys2,err]=frep2tf(frq,repf,3,dom);
    [frq,rep2]=repfreq(Sys2,frq=frq); //Frequency response of Sys2
    xbasc();plot2d1(frq',abs([repf;rep2])');
  \end{Verbatim}
\end{examples}
%-- see also
\begin{manseealso}
  \manlink{imrep2ss}{imrep2ss} \manlink{arl2}{arl2} \manlink{time\_id}{time-id} \manlink{armax}{armax} \manlink{frfit}{frfit}  
\end{manseealso}
