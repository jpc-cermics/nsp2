% -*- mode: latex -*-
\mansection{sorder}
\begin{mandesc}
  \short{sorder}{computing the order of a discrete-time system} \\ % 
\end{mandesc}
%\index{sorder}\label{sorder}
%-- Calling sequence section
\begin{calling_sequence}
\begin{verbatim}
  [Ro,n,sval,rcnd] = sorder(meth,alg,jobd,batch,conct,s,Y [,U,tol,printw,ldwork,Ri])  
\end{verbatim}
\end{calling_sequence}
%-- Parameters
\begin{parameters}
  \begin{varlist}
    \vname{meth}: integer option to determine the method to use:
    \begin{varlist}
      \vname{1}: MOESP method with past inputs and outputs;
      \vname{2}: N4SID method.
    \end{varlist}
    \vname{alg}: integer option to determine the algorithm for computing the
    triangular factor of the concatenated block-Hankel matrices built from the
    input-output data:
    \begin{varlist}
      \vname{1}: Cholesky algorithm on the correlation matrix;
      \vname{2}: fast QR algorithm;
      \vname{3}: standard QR algorithm.
    \end{varlist}
    \vname{jobd}: integer option to specify if the matrices B and D should later be computed using the MOESP approach:
    \begin{varlist}
      \vname{1}: the matrices B and D should later be computed using the MOESP approach;
      \vname{2}: the matrices B and D should not be computed using the MOESP approach.
      This parameter is not relevant for meth = 2.
    \end{varlist}
    \vname{batch}: integer option to specify whether or not sequential data
    processing is to be used, and, for sequential processing, whether or not the
    current data block is the first block, an intermediate block, or the last
    block, as follows:
    \begin{varlist}
      \vname{1}: the first block in sequential data processing;
      \vname{2}: an intermediate block in sequential data processing;
      \vname{3}: the last block in sequential data processing;
      \vname{4}: one block only (non-sequential data processing).
    \end{varlist}
    \vname{conct}: integer option to specify whether or not the successive data
    blocks in sequential data processing belong to a single experiment, as
    follows:
    \begin{varlist}
      \vname{1}: the current data block is a continuation of the previous data
      block and/or it will be continued by the next data block;

      \vname{2}: there is no connection between the current data block and the
      previous and/or the next ones.  This parameter is not used if batch = 4.
    \end{varlist}
    \vname{s}: the number of block rows in the input and output block Hankel
    matrices to be processed.  s $>$ 0

    \vname{Y}: the t-by-l output-data sequence matrix.  Column j of Y contains
    the t values of the j-th output component for consecutive time increments.

    \vname{U}: (optional) the t-by-m input-data sequence matrix.  Column j of U
    contains the t values of the j-th input component for consecutive time
    increments. Default: U = [].

    \vname{tol}: (optional) vector of length 2 containing tolerances: tol(1) -
    tolerance used for estimating the rank of matrices. If tol(1) $>$ 0, then
    the given value of tol(1) is used as a lower bound for the reciprocal
    condition number; an m-by-n matrix whose estimated condition number is less
    than 1/tol(1) is considered to be of full rank. If tol(1) $<$= 0, then a
    default value m*n*epsilon\_machine is used, where epsilon\_machine is the
    relative machine precision.  tol(2) - tolerance used for determining an
    estimate of the system order. If tol(2) $>$= 0, the estimate is indicated by
    the index of the last singular value greater than or equal to tol(2).
    (Singular values less than tol(2) are considered as zero.) When tol(2) = 0,
    an internally computed default value, tol(2) = s*epsilon\_machine*sval(1),
    is used, where sval(1) is the maximal singular value, and epsilon\_machine
    the relative machine precision. When tol(2) $<$ 0, the estimate is indicated
    by the index of the singular value that has the largest logarithmic gap to
    its successor. Default value: \verb!tol(1:2) = [0,-1]\verb!.

    \vname{printw}:(optional) switch for printing the warning messages  (Default value is 0).
    \begin{varlist}
      \vname{1}:  print warning messages;
      \vname{0}:  do not print warning messages.
    \end{varlist}
    \vname{ldwork}: (optional) the workspace size. Default: computed by the formulas
\begin{verbatim}
  nr = 2*( m + l )*s
  LDWORK = ( t - 2*s + 3 + 64 )*nr
  if ( CSIZE $>$ MAX( nr*nr + t*( m + l ) + 16, 2*nr ) ) then
  LDWORK = MIN( LDWORK, CSIZE - nr*nr - t*( m + l ) - 16 )
  else
  LDWORK = MIN( LDWORK, MAX( 2*nr, CSIZE/2 ) )
  end if
\end{verbatim}
    LDWORK = MAX( minimum workspace size needed, LDWORK ) where CSIZE is the cache size in double precision words.
    If LDWORK is specified less than the minimum workspace size  needed, that minimum value is used instead.
    
    \vname{Ri}: (optional) if batch = 2 or 3, the 2*(m+l)*s-by-2*(m+l)*s (upper
    triangular, if alg $<$$>$
    2) part of R must contain the (upper triangular) matrix R computed at the
    previous call of this mexfile in sequential data processing. If conct = 1, R
    has an additional column, also set at the previous call.  If alg = 2, R has
    m+l+1 additional columns, set at the previous call.  This parameter is not
    used for batch = 1 or batch = 4.

    \vname{Ro}: if batch = 3 or 4, the 2*(m+l)*s-by-2*(m+l)*s part of R contains
    the processed upper triangular factor R from the QR factorization of the
    concatenated block-Hankel matrices, and further details needed for computing
    system matrices. If batch = 1 or 2, then R contains intermediate results
    needed at the next call of this mexfile. If batch = 1 or 2 and conct = 1, R
    has an additional column, also set before return. If batch = 1 or 2 and alg
    = 2, R has m+l+1 additional columns, set before return.

    \vname{n}: the order of the system.

    \vname{sval}: (optional) the singular values used for estimating the order of the system.

    \vname{rcnd}: (optional) if meth = 2, vector of length 2 containing the
    reciprocal condition numbers of the matrices involved in rank decisions or
    least squares solutions.

\end{varlist}
\end{parameters}
\begin{mandescription}
  sorder computes the order of a discrete-time system using SLICOT routine
  IB01AD.  For one block (data sequences Y, U): [R,n,sval,rcnd] =
  sorder(meth,alg,jobd,4,conct,s,Y,U); For f blocks (data sequences \verb!Yj! and \verb!Uj! for \verb!j = 1: f!):
\begin{verbatim}
  R = sorder(meth,alg,jobd,1,conct,s,Y1,U1);
  for j = 2: f - 1
  R = sorder(meth,alg,jobd,2,conct,s,Yj,Uj,tol,printw,ldwork,R)
  end
  [R,n,sval,rcnd] = sorder(meth,alg,jobd,3,conct,s,Yf,Uf,tol);
\end{verbatim}
  sorder preprocesses the input-output data for estimating the matrices of a
  linear time-invariant dynamical system, using Cholesky or (fast) QR
  factorization and subspace identification techniques (MOESP and N4SID), and
  then estimates the order of a discrete-time realization.  The model structure
  is:
\begin{verbatim}
  x(k+1) = Ax(k) + Bu(k) + w(k),   k $>$= 1,
  y(k)   = Cx(k) + Du(k) + e(k),
\end{verbatim}
  where  
\begin{itemize}
\item x(k)  is the  n-dimensional state vector (at time k),
\item u(k)  is the  m-dimensional input vector,
\item y(k)  is the  l-dimensional output vector,
\item w(k)  is the  n-dimensional state disturbance vector,
\item e(k)  is the  l-dimensional output disturbance vector,
  and  A, B, C, and D  are real matrices of appropriate dimensions.
\end{itemize}
\end{mandescription}
%-- section-Comments
\paragraph{Comments}
1. The Cholesy or fast QR algorithms can be much faster (for large data blocks)
than QR algorithm, but they cannot be used if the correlation matrix, H'*H, is
not positive definite. In such a case, the code automatically switches to the QR
algorithm, if sufficient workspace is provided and batch = 4.

2. If ldwork is specified, but it is less than the minimum workspace size
needed, that minimum value is used instead.
%-- see also
\begin{manseealso}
  \manlink{findBD}{findBD} \manlink{sident}{sident}  
\end{manseealso}
%-- Author
\begin{authors}
  V. Sima, Research Institute for Informatics, Bucharest, Oct. 1999.; ;  Revisions:;    V. Sima, May 2000, July 2000.  
\end{authors}
