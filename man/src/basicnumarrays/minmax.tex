\mansection{min, max and minmax}
\begin{mandesc}
  \short{min}{minimum of matrix elements or minimum of 2 or more matrices}\\ 
  \short{max}{maximum of matrix elements or maximum of 2 or more matrices}\\ 
  \short{minmax}{both minimum and maximum of matrix elements}
\end{mandesc}
%-- Calling sequence section
\begin{calling_sequence}
\begin{verbatim}
  // first form of calling sequences
  m = min(A,dim=mode)
  [m,km] = min(A,dim=mode)
  M = max(A,dim=mode)
  [M,kM] = max(A,dim=mode)
  [m,M] = minmax(A,dim=mode)
  [m,M,km,kM] = minmax(A,dim=mode)

  // second form
  m = min(A1,A2,...)  
  [m,km] = min(A1,A2,...) 
  M = max(A1,A2,...)  
  [M,kM] = max(A1,A2,...)  
\end{verbatim}
\end{calling_sequence}
%-- Parameters
\begin{parameters}
  \begin{varlist}
    \vname{A,A1,A2,...}: numerical matrices (real Mat, real SpColMat or IMat)
    \vname{mode}: a string chosen among \verb+'M'+, \verb+'m'+, \verb+'full'+, \verb+'FULL'+, \verb+'row'+,
    \verb+'ROW'+, \verb+'col'+, \verb+'COL'+ or an non ambiguous abbreviation or an integer. 
    This argument is optional and if omitted 'full' is assumed.
    \vname{m,M}: numerical scalars or vectors or matrices
    \vname{km, kM}: integer scalars or vectors (or matrices) of indices.
  \end{varlist}
\end{parameters}

\begin{mandescription}
These functions have 2 differents forms. In the first form the min, max operators are 
applied to the entries of a unique matrix, in the second form the  min, max operators are 
applied to entries from a set of matrices. More precisely:
\begin{description}
\item[form 1]: is used when only one $m \times n$ sized matrix is given as argument. Depending on the value of the 
           optional argument \verb+dim=mode+ (with 'full' as default value) they compute the $min$ or $max$ element(s):
  \begin{itemize}
    \item among all the matrix elements if dim='full' or 0:
     $$
           m = \min_{i,j} A_{i,j}\,; \; M = \max_{i,j} A_{i,j}\, ;
     $$
    \item along the dimension 1 when dim=1 or 'row' (we get the min or max of each column);
     $$
           m_{1,j} = \min_i A_{i,j}\,; \; M_{1,j} = \max_i A_{i,j}\, ;
     $$
     $m$ and $M$ are then row vectors ($1 \times n$).
    \item along the dimension 2 when dim=2 or 'col' (we get the min or max of each row);
     $$
           m_{i,1} = \min_j A_{i,j}\,; \; M_{i,1} = \max_j A_{i,j}\,; 
     $$
     $m$ and $M$ are then column vectors ($m \times 1$).
    \item along the first non singleton dimension of the given matrix if dim='m' (matlab compatibility).
  \end{itemize}

Additionnaly the first index of the corresponding min or max is provided in the returned values (when dim=0, it is 
given in the ``one way indexing'' meaning of the matrix). The \verb+minmax+ function behave exactly
as described. It is practical since very often one needs both the min and max of a vector or matrix.  

\item[form2]: is used when the argument list contains more than one matrix. It applies on any number of matrices of 
             the {\bf same size}, says $m \times n$ (with the usual short cut that you can use any scalar in place of such a matrix). 
  The result is a $m \times n$ sized matrix with entries given by:
    $$
        m_{i,j} = \min_k Ak_{i,j}\,,  M_{i,j} = \max_k Ak_{i,j}\,,
  $$    
  and $km_{i,j}$ or $kM_{i,j}$ gives the (first) matrix number where the min or max is got.
  Note that this second form is not implemented for the minmax function.
\end{description}
\itemdesc{Remarks}
\begin{itemize} 
\item Nan values are not taking into account in these functions except when the min or max is to 
  be taken among only Nan values. 
\item In the first form you can use the syntax \verb+min(A,mode)+ (in place of \verb+min(A,dim=mode)+)
  if you use a string ('row', 'col',..) as dim descriptor (if you use an integer the interpretor will 
  recognize a call of the second form). This can be useful for scilab compatibility.
\item for sparse matrices: 
  \begin{itemize}
  \item \verb+minmax+ is not defined
  \item the second form is (currently) limited to two matrices. Use \verb+max(A1,max(A2,A3))+ to
    simulate \verb+max(A1,A2,A3)+. 
  \item for the second form \verb+[m,km]=min(A1,A2)+, \verb+[M,kM]=max(A1,A2)+ the matrices 
    of indices $km$ or $kM$ are sparse matrices as  no index (1 or 2) is provided when 
    both $A1_{i,j}$ and  $A2_{i,j}$ are null.
  \end{itemize}
\end{itemize}
\end{mandescription}

%--example 
\begin{examples}
\paragraph{first form}
\begin{mintednsp}{nsp}
A = rand(4,5) 

// min and max of all elements
m = min(A) 
M = max(A) 
[m,M] = minmax(A)
[m,km] = min(A)
[M,kM] = max(A)
[m,M,km,kM] = minmax(A)

// min and max of each column
[m,km] = min(A,dim=1)
[M,kM] = max(A,dim=1)
[m,M,km,kM] = minmax(A,dim=1)

// min and max of each row
[m,km] = min(A,dim=2)
[M,kM] = max(A,dim=2)
[m,M,km,kM] = minmax(A,dim=2)

// behavior with Nan
x = [1, %nan, 2, %nan]
[m,km] = min(x)
[M,kM] = max(x)
[m,M,km,kM] = minmax(A)
\end{mintednsp}

\paragraph{second form}
\begin{mintednsp}{nsp}
A=rand(2,2) 
B = rand(2,2) 
C = rand(2,2) 
m = min(A,B,C) 
[m,km] = min(A,B,C) 

// one (or more) of the matrix can be a scalar 
[m,km] = min(A,0.3,C) 

// using scalar is practical to impose a min and max thresholds on a matrix 
// in this example we want to limit a random matrix between 0.2 and 0.8 
A = rand(2,4) 
B = min(0.8,max(0.2,A))
\end{mintednsp} 

\end{examples}

%-- see also
\begin{manseealso}
  \manlink{find}{find}
\end{manseealso}

