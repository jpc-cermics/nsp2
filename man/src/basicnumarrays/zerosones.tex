\mansection{zeros, ones}
\begin{mandesc}
  \short{zeros}{matrix of zeros}\\ 
  \short{ones}{matrix of ones}
\end{mandesc}
%-- Calling sequence section
\begin{calling_sequence}
\begin{verbatim}
A = zeros(n)
A = zeros(m,n)
A = zeros(dims)
A = ones(n)
A = ones(m,n)
A = ones(dims)
\end{verbatim}
\end{calling_sequence}
%-- Parameters
\begin{parameters}
  \begin{varlist}
    \vname{m,n}: integers
    \vname{dims}: vector of integers of length 1 or 2.
    \vname{A}: matrix of zeros or ones.
  \end{varlist}
\end{parameters}

\begin{mandescription}
\verb+zeros(n)+ is the same than \verb+zeros(n,n)+. If \verb+dims+ is a vector of length 2, 
with $dims_1 = m$ and $dims_2 = n$ then \verb+zeros(m,n)+ and  \verb+zeros(dims)+ are equivalent 
and output a $m \times n$ matrix of zeros. The second form could be useful to get a matrix of 
zeros of same size than another matrix $M$ using:
\begin{verbatim}
    A = zeros(size(M))
\end{verbatim}

The \verb+ones+ function works similarly but output a matrix of ones.

\end{mandescription}


%--example 
\begin{examples}
\begin{program}\HCode{A = ones(2,3) \Hnewline
B = zeros(1,4)\Hnewline
C = zeros(3,0)\Hnewline
D = ones(0,2)\Hnewline
C*D}
\end{program} 

\end{examples}

%-- see also
\begin{manseealso}
\manlink{eye}{eye}, \manlink{repmat}{repmat}
\end{manseealso}

