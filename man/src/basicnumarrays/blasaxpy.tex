\mansection{blas\_axpy}
\begin{mandesc}
  \shortunder{blas\_axpy}{blas_axpy}{add to a vector another vector times a scalar}\\ 
\end{mandesc}
%-- Calling sequence section
\begin{calling_sequence}
\begin{verbatim}
y.blas_axpy[alpha, x]
y.blas_axpy[alpha, x, i1, i2]
y.blas_axpy[alpha, x, i1, i2, j1, j2]
\end{verbatim}
\end{calling_sequence}
%-- Parameters
\begin{parameters}
  \begin{varlist}
    \vname{y,x}: numerical vectors or matrices (both real or both complex)
    \vname{alpha}: real scalar (when \verb+x+ and \verb+y+ are real) or complex scalar (when \verb+x+ and \verb+y+ 
                   are complex).
    \vname{i1,i2,j1,j2}: part of \verb+y+ where the operation applies.
  \end{varlist}
\end{parameters}

\begin{mandescription}

This method modify in place the vector or matrix on which it is applied:
\begin{itemize}
\item \verb+y.blas_axpy[alpha, x]+ do the operation $y \leftarrow y + \alpha x$.
      $x$ should have the same dimensions than $y$.
\item \verb+y.blas_axpy[alpha, x, i1, i2]+ is the same but acts on the sub part of 
      $y$ between components number $i1$ and $i2$ (included) in the one-indexing
      meaning  (see \manlink{indexing arrays}{indexing arrays}). $x$ may have
      any dimensions but should have a total of $i2-i1+1$ components.
\item \verb+y.blas_axpy[alpha, x, i1, i2, j1, j2]+ acts on the submatrix $y(i1:i2,j1:j2)$
      and $x$ should be a matrix with dimensions $(i2-i1+1)\times(j2-j1+1)$.  
\end{itemize}
Remarks:
\begin{itemize} 
\item $y$, $x$ and $\alpha$ should be either all real or all complex.
\item these operations should give the same result than respectively:
      \begin{itemize}
      \item \verb-y = y + alpha*x-
      \item \verb-y(i1:i2) = y(i1:i2) + alpha*x-
      \item \verb-y(i1:i2,j1:j2)) = y(i1:i2,j1:j2) + alpha*x-
      \end{itemize}
      but using the \verb+blas_axpy+ method should be faster.
\end{itemize}
\end{mandescription}

%--example 
\begin{examples}
\begin{Verbatim}
z = -2:2;
y = z
x = [-1,1,-1,1,-1]
y.blas_axpy[2,x]
y 

// same operation but on a subpart of y
y = z
y.blas_axpy[2,x(2:4),2,4]
y

// now axpy operation on a submatrix
X = rand(6,6)
Y = ones(4,4);
X.blas_axpy[10,Y,2,5,2,5]
X
\end{Verbatim}

\end{examples}

%-- see also
\begin{manseealso}
 \manlink{blas\_ger}{blas_ger},\manlink{scale\_rows}{scale_rows},\manlink{scale\_cols}{scale_cols}
\end{manseealso}

