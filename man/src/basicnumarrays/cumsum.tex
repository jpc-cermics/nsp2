\mansection{cumsum and cumprod}
\begin{mandesc}
  \short{cumsum}{cumulative sum of matrix elements}\\ % @mandesc@
  \short{cumprod}{cumulative product of matrix elements}\\ % @mandesc@
\end{mandesc}
%-- Calling sequence section
\begin{calling_sequence}
\begin{verbatim}
  B=cumsum(A,mode)  
  B=cumprod(A,mode)  
  B=cumsum(A,dim=mode) 
  B=cumprod(A,dim=mode)  
\end{verbatim}
\end{calling_sequence}
%-- Parameters
\begin{parameters}
  \begin{varlist}
    \vname{A,B}: matrices.
    \vname{mode}: A string chosen among \verb+'M'+, \verb+'m'+, \verb+'full'+, \verb+'FULL'+, \verb+'row'+,
    \verb+'ROW'+, \verb+'col'+, \verb+'COL'+ or an non ambiguous abbreviation or an integer. 
    This argument is optional and if omitted 'full' is assumed.
  \end{varlist}
\end{parameters}
\begin{mandescription}
  \verb+cumsum+ (resp. \verb+cumprod+) gives the cumulative sum (resp. product) of the 
  elements of the given matrix argument \verb+A+. 
  The second argument gives the dimension to be used for performing the cumsum (resp product) of entries.       
  \begin{itemize}
    \item 'full' or 0 \verb+B+ and \verb+A+ have the same size and 
     $$B_{i,j} = \sum_{k+(l-1)m \le i+(j-1)m } A_{k,l} \quad \left( \text{resp. } \prod_{k+(l-1)m \le i+(j-1)m } A_{k,l} \right)$$ 
     where $m$ is the number of rows of \verb+A+
    \item 'row' or 1  \verb+B+ and \verb+A+ have the same size and 
     $$B_{i,j} = \sum_{k\le i} A_{k,j}\quad \left(\text{resp. }  \prod_{k\le i} A_{k,j}\right) \,. $$     
    \item 'col' or 2  \verb+B+ and \verb+A+ have the same size and 
         $$B_{i,j} = \sum_{l\le j} A_{i,l} \quad \left(\text{resp. }  \prod_{l\le j} A_{i,l}  \right) \,. $$     
    \item 'm' is the cumulative sum (resp. product) along the first non singleton dimension of the given matrix 
      for Matlab compatibility. 
  \end{itemize}
\end{mandescription}
%--example 
\begin{examples}
\begin{Verbatim}
A=rand(4,5) 
cumsum(A,dim='c'); 
// boolean matrix 
cumsum(A>=0.5,dim='c');
cumprod(1:5,dim='m') 
// sparse matrix 
cumsum(sparse(rand(4,6)),dim='r')
\end{Verbatim}
\end{examples}
%-- see also
\begin{manseealso}
  \manlink{sum}{sum}  \manlink{prod}{prod} 
\end{manseealso}

