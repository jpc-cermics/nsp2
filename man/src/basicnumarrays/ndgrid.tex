% -*- mode: latex -*-
\mansection{ndgrid}
\begin{mandesc}
\short{ndgrid}{arrays for multidimensional function evaluation on grid}
\end{mandesc}

%-- Calling sequence section
\begin{calling_sequence}
\begin{verbatim}
[X, Y] = ndgrid(x,y)
[X, Y] = ndgrid(x)
[X, Y, Z] = ndgrid(x,y,z)
[X, Y, Z] = ndgrid(x)
[X, Y, Z, T] = ndgrid(x,y,z,t)
[X1, X2, ..., Xm] = ndgrid(x1,x2,...,xm)
\end{verbatim}
\end{calling_sequence}

%-- Parameters
\begin{parameters}
  \begin{varlist}
   \vname{x, y, z, ...} : vectors
   \vname{X, Y, Z, ...} : matrices
\end{varlist}
\end{parameters}

\begin{mandescription}
 This is an utility routine useful to create arrays for function evaluation on 2, 3, ..., n 
dimensional grids. 

\paragraph{2d case}
For instance in 2d, a grid is defined by two vectors, \verb!x! and
\verb!y! of length nx and ny, and you want to evaluate a function (says {\em f}) 
on all the grid points, that is on all the points of coordinates  $(x_i, y_j)$ 
with  $i=1,..,nx$ and $j=1,..,ny$. In this case, this function can
compute the two matrices \verb!X,Y! of size {\em nx x ny} such that:
$$\left.
  \begin{array}{r} X(i,j) = x_i \\
                   Y(i,j) = y_j
  \end{array} 
  \right\} \forall (i,j) \in  [1,nx] \times [1,ny]
$$
and the evaluation may be done with \verb!Z=f(X,Y)! if the function \verb!f! have
been coded  for evaluation on vectors/matrices arguments (which is obtained (in general) 
by using the element-wise operators \verb!.*!, \verb!./! and \verb!.^! in place of the
usual \verb!*!, \verb!/! and \verb!^!). The form \verb+[X, Y] = ndgrid(x)+ is equivalent 
to \verb+[X, Y] = ndgrid(x,y)+ when y=x.    

\paragraph{3d (or more) cases}
In the 3d case, considering 3 vectors \verb!x,y,z! of length nx, ny and nz,  \verb!X,Y,Z!
should be matrices with 3 dimensions of size {\em nx x ny x nz} such that :
$$\left.
\begin{array}{r}
      X(i,j,k) = x_i  \\
      Y(i,j,k) = y_j  \\
      Z(i,j,k) = z_k  
\end{array} \right\} \forall (i,j) \in  [1,nx] \times [1,ny] \times [1,nz]
$$     
but currently nsp don't support such matrices (that is matrices with more than 2 dimensions). 
Nevertheless you could use \verb+ndgrid+ in this case with the convention that an element
(of one of the 3 output arrays) which should be in position $(i,j,k)$ will be in position:
$$
i + nx(j-1 + ny(k-1))
$$
of the array seen as one dimensional array, or at position:
$$
(i , j + ny(k-1))
$$
of the same array seen as a 2 dimensional array.
\end{mandescription}

%--example 

\begin{examples}
\paragraph{example 1} plot a function...
\begin{program}\HCode{// the function (we use element wize op)\Hnewline
function z=f(x,y)\Hnewline
z = x.^2 + y.^3\Hnewline
endfunction\Hnewline
\Hnewline
// create a simple 2d grid\Hnewline
nx = 40; ny = 40;\Hnewline
x = linspace(-1,1,nx);\Hnewline
y = linspace(-1,1,ny);\Hnewline
[X,Y] = ndgrid(x,y);\Hnewline
\Hnewline
// compute the function on the grid\Hnewline
Z = f(X,Y);\Hnewline
\Hnewline
// 3d plot of the function\Hnewline
xbasc()\Hnewline
plot3d(x,y,Z, flag=[2 6 4]);\Hnewline 
xselect()}
\end{program}

\end{examples}

%-- see also
%\begin{manseealso}
%\end{manseealso}

%-- Author
\begin{authors}
B. Pincon
\end{authors}

