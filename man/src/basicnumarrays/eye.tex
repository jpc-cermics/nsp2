\mansection{eye}
\begin{mandesc}
  \short{eye}{identity matrix or matrix of canonical application between 2 vector spaces} 
\end{mandesc}
%-- Calling sequence section
\begin{calling_sequence}
\begin{verbatim}
I = eye(n)
I = eye(m,n)
I = eye(dims)
\end{verbatim}
\end{calling_sequence}
%-- Parameters
\begin{parameters}
  \begin{varlist}
    \vname{m,n}: integers
    \vname{dims}: vector of integers of length 1 or 2.
    \vname{A}: resulting matrix.
  \end{varlist}
\end{parameters}

\begin{mandescription}
\verb+eye(n)+ is the same than \verb+eye(n,n)+ and output the identity matrix of the $n$ dimensional
vector space. More generally \verb+eye(m,n)+ output the matrix of the canonical application $J$ from
the $n$ dimensional vector space $K^n$ to the  $m$ dimensional vector space $K^m$: 
$$
x = \left[ x_1 ; x_2 ; \dots ; x_n \right]  \longmapsto  J x = 
\left\{ 
\begin{array}{l} 
     \left[ x_1 ; x_2 ; \dots ; x_m \right] \mbox{ if } m \le n \\ \left[ x_1 ; x_2 ; \dots ; x_n ; 0 ; \dots ; 0 \right] \mbox{ otherwise }
\end{array} 
\right.
$$
With $m=3$ and $n=4$ we get:
$$
    I = \left( \begin{array}{cccc} 1 & 0 & 0 & 0 \\ 0 & 1 & 0 & 0 \\ 0 & 0 & 1 & 0
      \end{array} \right)
$$
\verb+eye(dims)+ with $dims_1 = m$ and $dims_2 = n$ is the same than  \verb+eye(m,n)+. If you want
such a matrix of same size than another matrix $M$ use: 
\begin{verbatim}
   I = eye(size(M))
\end{verbatim}

\end{mandescription}


%--example 
\begin{examples}
\begin{Verbatim}
I = eye(2,3) 
I = eye(0,3)
\end{Verbatim} 

\end{examples}

%-- see also
\begin{manseealso}
\manlink{zeros}{zeros}, \manlink{ones}{ones}
\end{manseealso}

