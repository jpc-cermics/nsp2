\mansection{scale\_rows, scale\_cols}
\begin{mandesc}
  \shortunder{scale\_rows}{scale_rows}{multiply each row of a matrix by a different scalar}\\ 
  \shortunder{scale\_cols}{scale_cols}{multiply each column of a matrix by a different scalar}
\end{mandesc}
%-- Calling sequence section
\begin{calling_sequence}
\begin{verbatim}
A.scale_rows[scale_factors]      // method form: A is modified in place
B = scale_rows(A,scale_factors)  // function form: a new matrix B is created

A.scale_cols[scale_factors]      // method form: A is modified in place
B = scale_cols(A,scale_factors)  // function form: a new matrix B is created
\end{verbatim}
\end{calling_sequence}
%-- Parameters
\begin{parameters}
  \begin{varlist}
    \vname{A, B}: numerical matrices (full or sparse).
    \vname{scale\_factors}: numerical vector (full).
  \end{varlist}
\end{parameters}

\begin{mandescription}

The methods \verb+scale_rows+ and \verb+scale_cols+ let to modify in place a matrix A 
(says of size $m \times n$) by scaling (e.g. multiplying by a scalar) either its rows 
or its columns. When scaling the rows  \verb+scale_factors+ should be a vector with $m$ 
elements and when scaling the columns \verb+scale_factors+ should be a vector with 
$n$ elements.

The function form  creates a new matrix by these scaling operations without modifying 
the original matrix.
\end{mandescription}


%--example 
\begin{examples}
\begin{Verbatim}
A=rand(2,3)

B = scale_rows(A,[0.5,2])
A.scale_rows[[0.5,2]]; A

B = scale_cols(A,[0.5,2,4])
A.scale_cols[[0.5,2,4]]; A
\end{Verbatim}

\end{examples}

%-- see also
\begin{manseealso}
% \manlink{find}{find}
\end{manseealso}

