\mansection{dot}
\begin{mandesc}
  \short{dot}{scalar product of 2 vectors (or matrices)}
\end{mandesc}
%-- Calling sequence section
\begin{calling_sequence}
\begin{verbatim}
  C=dot(A,B)  
  C=dot(A,B,mode)  
  C=dot(A,B,dim=mode)  
\end{verbatim}
\end{calling_sequence}
%-- Parameters
\begin{parameters}
  \begin{varlist}
    \vname{A,B}: numerical vectors (or numerical matrices) of same size.
    \vname{mode}: A string chosen among \verb+'M'+, \verb+'m'+, \verb+'full'+, \verb+'FULL'+, \verb+'row'+,
    \verb+'ROW'+, \verb+'col'+, \verb+'COL'+ or an non ambiguous abbreviation or an integer. 
    This argument is optional and if omitted 'full' is assumed.
  \end{varlist}
\end{parameters}
\begin{mandescription}
  \verb+dot+ computes the scalar product between the vectors or matrices \verb+A+ and \verb+B+ :
$$
dot(A,B) = \left\{
\begin{array}{l}
     \sum_{i} \bar{A}_i B_i, \mbox{ for vectors } \\
     \sum_{i,j} \bar{A}_{i,j} B_{i,j}, \mbox{ for matrices } \\
\end{array} \right.
$$ 
 Using the third argument you could compute the dot product between the rows or columns of
 \verb+A+ and \verb+B+, precisely :
  \begin{itemize}
    \item 'full' or 0 (the default case) $$C= \sum_{i,j} \bar{A}_{i,j} B_{i,j}$$
    \item 'row' or 1  computes the scalar product between the corresponding column vectors 
          of the 2 matrices: $$C_j = \sum_{i} \bar{A}_{i,j} B_{i,j}$$ and thus gives a row vector.
    \item 'col' or 2  computes the scalar product between the corresponding row vectors
          of the 2 matrices: $$C_j = \sum_{j} \bar{A}_{i,j} B_{i,j}$$ and thus gives a column vector.
    \item 'm' is the dot product along the first non singleton dimension of the given matrices 
          (for Matlab compatibility). 
  \end{itemize}
\end{mandescription}

%--example 
\begin{examples}
\paragraph{example 1} a simple example (we form 2 orthonormal vectors) :
\begin{Verbatim}
x = randn(5,1);
y = randn(5,1);
// step 0
dot(x,y)
// step 1 normalise x:
x = x/norm(x)
// step 2 remove the component of y in the direction of x
y = y - dot(y,x)*x
// step 3 normalise y
y = y/norm(y)
// test 
norm(x)  // should be 1
norm(y)  // should be 1
dot(x,y) // should be near 0
\end{Verbatim}

\paragraph{example 2} same example with complex vectors
\begin{Verbatim}
x = randn(5,1) + %i*randn(5,1);
y = randn(5,1) + %i*randn(5,1);
// step 0
dot(x,y)
// step 1 normalise x:
x = x/norm(x)
// step 2 remove the component of y in the direction of x
// (be careful for the order in dot in this case)
y = y - dot(x,y)*x
// step 3 normalise y
y = y/norm(y)
// test 
norm(x)  // should be 1
norm(y)  // should be 1
dot(x,y) // should be near 0
\end{Verbatim}

\paragraph{example 3} fastest way to compute the 2-norm of the rows or
 columns vectors of a matrix : in this case you cannot use norm because
 norm computes a matrix norm in this case.
\begin{Verbatim}
A = randn(5,7);
norm_cols_A = sqrt(dot(A,A,dim=1))
// test
norm(A(:,3)) - norm_cols_A(3)
// the same for the rows
norm_rows_A = sqrt(dot(A,A,dim=2))
// test
norm(A(2,:)) - norm_rows_A(2)
// comparizon with another method for a big matrix
A = randn(800,800);
tic(); norm_cols_A = sqrt(dot(A,A,dim=1)); toc()
// the other method (should be slower)
tic(); norm_cols_A = sqrt(sum(A.^2,dim=1)); toc()
// same thing with a complex matrix
A = randn(800,800) + %i*rand(800,800);
tic(); norm_cols_A = sqrt(dot(A,A,dim=1)); toc()
// the other method (should be slower)
tic(); norm_cols_A = sqrt(sum(conj(A).*A,dim=1)); toc()
\end{Verbatim}
\end{examples}

% -- see also
\begin{manseealso}
  \manlink{norm}{norm}  \manlink{sum}{sum} 
\end{manseealso}

