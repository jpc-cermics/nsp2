% -*- mode: latex -*-

\mansection{grand}
\begin{mandesc}
  \short{f}{short description for f} \\
  \short{g}{short description for g} \\
\end{mandesc}

% -- Calling sequence section
\begin{calling_sequence}
\begin{verbatim}
Y=f(m, n, dist_type [,p1,...,pk])  
Y=f(X, dist_type [,p1,...,pk])  
Y=g(n, dist_type [,p1,...,pk])  
S=g(action [,q1,....,ql])  
\end{verbatim}
\end{calling_sequence}
% -- Parameters
\begin{parameters}
  \begin{varlist}
    \vname{m, n}: integers, size of the wanted matrix \verb!Y!
    \vname{X}: a matrix whom only the dimensions (say \verb!m x n!) are used
    \vname{dist\_type}: a string given the distribution which (independants) variates are to be 
    generated (\verb+'bin'+, \verb+'nor'+, \verb+'poi'+, etc ...)
    \vname{p1, ..., pk}: the parameters (reals or integers) required to define completly the distribution 
    \verb!dist_type!
    \vname{Y}: the resulting \verb!m x n! random matrix
  \end{varlist}
\end{parameters}

\begin{mandescription}
  This function may be used to generate random numbers from various
  distributions. In this case you must apply one of the \verb!three first forms!
  of the possible calling sequences to get an \verb!m x n! matrix.  The two
  firsts are equivalent if \verb!X! is a \verb!m x n! matrix, and the third form
  corresponds to 'multivalued' distributions (e.g. multinomial, multivariate
  gaussian, etc...) where a sample is a column vector (says of dim \verb!m!)
  and you get then \verb!n! such random vectors (as an \verb! m x n! matrix).
  \verb!The last form! is used to undertake various manipulations onto the base
  generators like changing the base generator (since v 2.7 you may choose
  between several base generators), changing or retrieving its internal state
  (seeds), etc ... These base generators give random integers following a
  uniform distribution on a large integer interval (lgi), all the others
  distributions being gotten from it (in general via a scheme lgi -$>$ U([0,1))
  -$>$ wanted distribution).

  \begin{itemize}
  \item first item 
  \item second item 
  \end{itemize}

  \begin{description}
  \item[first] desciption item 
  \item[second] description item 
  \end{description}

  \begin{varlist}
    \vname{m}: first varlist item 
    \vname{n}: second varlist item
  \end{varlist}

% itemize without bullets 

  \begin{nbitemize}
    \item  first itemize without bullets 
    \item  second itemize without bullets
  \end{nbitemize}

  \begin{nbitemize}
    \coloritem{p}: first dlist item 
    \coloritem{q}: second dlist item
  \end{nbitemize}

  Using web links in manual 
  \href{http://en.wikipedia.org/wiki/Hankel_matrix}{Hankel Matrix}

\end{mandescription}

\begin{examples}

a first example with $\int_0^\infty f(u)du$.

\[
\int_0^\infty f(u)du
\]

\begin{Verbatim}
  \verb+// backslash in program \\+
  \HCode{\Hnewline
    u=prbs_a(50,10);\Hnewline
    plot2d2("onn",(1:50)',u',1,"151",' ',[0,-1.5,50,1.5]);\Hnewline
    \%z ;// a percent un verbatim 
  }
\end{Verbatim}

\noindent a second example 

\begin{mintednsp}{nspxxx}
  // l'appel
  u=prbs_a(50,10);
  plot2d2("onn",(1:50)',u',1,"151",' ',[0,-1.5,50,1.5]);
\end{mintednsp}

\begin{mintednsp}{nspxxx}
  [a,b]=regexp('pipopopopa','p(i[po]+)(pa)')
  regsub('pipopopopa','p(i[po]+)(pa)','poo\\1')
\end{mintednsp}

\end{examples}

\begin{manseealso}
  \manlink{rand}{rand}  
\end{manseealso}

% -- Authors
\begin{authors}
  \paragraph{randlib} 
  The codes to generate sequences following other distributions than def, unf, lgi,  uin and geom are
  from "Library of Fortran Routines for Random Number  Generation", by Barry W. Brown 
  and James Lovato, Department of Biomathematics, The University of Texas, Houston.  
  \paragraph{mt} 
  The code is the mt19937int.c by M. Matsumoto and  T. Nishimura, "Mersenne Twister: 
  A 623-dimensionally equidistributed  uniform pseudorandom number generator", 
  ACM Trans. on Modeling and  Computer Simulation Vol. 8, No. 1, January, pp.3-30 1998.
  \paragraph{kiss} 
  The code was given by G. Marsaglia at the end of a thread concerning RNG in C in several 
  newsgroups (whom sci.math.num-analysis) "My offer of  RNG's for C was an invitation 
  to dance..." only kiss have been included in Scilab (kiss is made of a combinaison of 
  severals others which are not visible at the scilab level).
  \paragraph{clcg2} 
  The method is from P. L'Ecuyer but the C code is provided at the Luc  Devroye home page 
  (http://cgm.cs.mcgill.ca/~luc/rng.html).
  l'�cole est �t� � o�
\end{authors}


\paragraph{Using VerbatimInput}

\VerbatimInput{\mansrc model/model-verb.txt}

% we can call macros in Verbatim using the commandchars arguments 
% \def\redtext#1{{\textcolor{red}{#1}}}
%\begin{mintednsp}{nsp}[commandchars=@\[\]]
% @redtext[pooo] poo 
%\end{mintednsp}

\paragraph{Minted Nsp}

\begin{mintednsp}{nsp}
//	Verbatim code 
//	to be inserted 
//	in html
[a,b]=regexp("pipopopopa","p(i[po]+)(pa)")
regsub("pipopopopa","p(i[po]+)(pa)","poo\\1")
a < b 
a \ b  
a / b 
z.*x

%inf + %nan 
[poo # foo ]
a & b 
a | b 
x = 'poo' // wrong conversion

function y=f(x)
  for i=1:10
    c=567
  end
endfunction
\end{mintednsp}


\begin{mintednsp}{nsp}
  function [a,u]=hermit(a)
  //[A,U]=hermit(A)
  //Hermite form: U is an unimodular matrix such that A*U is
  //triangular. The output value of A is A*U.
  [m,n]=size(a);if m<>n then error('square matrix only!'),end
  [a,u]=htrianr(a)
  for l=n-1: -1: 1 do
    dl(l:n)=a(l,l:n).degree[];
    for k=l+1:n do
      if dl(k)>=dl(l) then
        all=a(l,l);
        if norm(coeff(all),1)>1.d-10 then
          [r,q]=pdiv(a(l,k),a(l,l))
          if l>1 then a(1:l-1,k)=a(1:l-1,k)-a(1:l-1,l)*q;end
          a(l,k)=r
          u(:,k)=u(:,k)-u(:,l)*q
        end
      end
    end
  end
endfunction
\end{mintednsp}

\end{document}


%\input{model/intepsilon_c.tex}

