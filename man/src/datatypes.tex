\chapter*{Data types}\addcontentsline{toc}{chapter}{data types}


\begin{quote}
\noindent
\hyperlink{Object}{Object} - Nsp general object  \\
\hyperlink{Mat}{Mat} - matrix of real or complex numbers \\
\hyperlink{SpColMat}{SpColMat} - sparse matrix (real or complex numbers) \\
\hyperlink{Smat}{Smat} - matrix of strings  \\
\hyperlink{IMat}{IMat} - matrix of integer numbers \\
\hyperlink{Bmat}{Bmat} - boolean matrix  \\
\hyperlink{Pmat}{Pmat} - polynomial matrix  \\
\hyperlink{list}{List} - list  of nsp objects  \\
\hyperlink{Cells}{Cells} - matrix of nsp objects  \\
\hyperlink{hash}{hash} - hash tables  \\
\hyperlink{MaxMat}{MaxMat} - Max plus matrix  \\
\hyperlink{File}{File} - file data type  \\
\hyperlink{Serial}{Serial} - Container for nsp serialized object \\
\hyperlink{SMio}{SMio} - String container for input ouput \\
\hyperlink{SndFile}{SndFile} - sound file data type \\
\hyperlink{gdate}{gdate} - days representation \\
\hyperlink{Umfpack}{Umfpack} - datatype for LU factorization of sparse matrices \\
\hyperlink{Cholmod}{Cholmod} - datatype for Cholesky or LDLt factorization of sparse matrices \\
\end{quote}


\input datatypes/object.tex 
\input datatypes/list.tex 
\input datatypes/hash.tex 
\input datatypes/file.tex 
\input datatypes/mat.tex 
\input datatypes/spcolmat.tex 
\input datatypes/bmat.tex 
\input datatypes/smat.tex 
\input datatypes/imat.tex 
\input datatypes/pmat.tex 
\input datatypes/maxpmat.tex 
\input datatypes/cells.tex 
\input datatypes/serial.tex 
\input datatypes/sndfile.tex 
\input datatypes/smio.tex 
\input datatypes/gdate.tex 
\input datatypes/umfpack.tex 
\input datatypes/cholmod.tex 

