% -*- mode: latex -*-
%% Scilab ( http://www.scilab.org/ ) - This file is part of Scilab
%% Copyright (C) 1987-2016 - F. Delebecque
%%
%% This program is free software; you can redistribute it and/or modify
%% it under the terms of the GNU General Public License as published by
%% the Free Software Foundation; either version 2 of the License, or
%% (at your option) any later version.
%%
%% This program is distributed in the hope that it will be useful,
%% but WITHOUT ANY WARRANTY; without even the implied warranty of
%% MERCHANTABILITY or FITNESS FOR A PARTICULAR PURPOSE.  See the
%% GNU General Public License for more details.
%%
%% You should have received a copy of the GNU General Public License
%% along with this program; if not, write to the Free Software
%% Foundation, Inc., 59 Temple Place, Suite 330, Boston, MA  02111-1307  USA
%%                                                                                                


\mansection{expm}
\begin{mandesc}
  \short{expm}{exponential of a square matrix}
\end{mandesc}

% -- Calling sequence section
\begin{calling_sequence}
\begin{verbatim}
B = expm(A)
\end{verbatim}
\end{calling_sequence}
% -- Parameters
\begin{parameters}
  \begin{varlist}
    \vname{A}:  square real or complex matrix.
    \vname{B}:  the exponential of \verb+A+
  \end{varlist}
\end{parameters}

\begin{mandescription}
This function computes (an approximation of) the exponential of a (full) square matrix $A$ :
$$
   B = e^A = \sum_{k=0}^{+\infty} \frac{A^k}{k !} = I + A + \frac{A^2}{2} + \frac{A^3}{6} + \dots
$$
using routines from R. B. Sidje 's expokit package.
\end{mandescription}

\begin{examples}

\paragraph{simple example}
\begin{Verbatim}
M = randn(4,4);
lambda = [-5; -0.2; 3; 10];
A = M*diag(lambda)*inv(M);
// compute exp(A) using expm 
B = expm(A)
// compute exp(A) using the fact we know the spectral factorization
C =  M*diag(exp(lambda))*inv(M);
// compare
abs(B-C)
err = norm(B-C)/norm(C)
\end{Verbatim}

\paragraph{Using expm to solve an ode.} The solution of the linear ode, $\dot{u}(t) = A u(t)$, $u(0)=u_0$,
is $u(t) = e^{tA} u(0)$. The solution at time instants $t_k = k \Delta t$ could be
obtained by computing the exponential of the matrix $\Delta t A$ and the recurrence 
$u(t_k) = \exp(\Delta t A) u(t_{k-1})$:  
\begin{Verbatim}
A = [-0.1, -1  ,  0;...
      1  , -0.1,  1;...
      0  , -1  , -0.1];
dt = 0.05;
n = 1000;
t = 0:dt:n*dt;
u0 = [0;0;1];
expAdt = expm(dt*A);
u = zeros(3,n+1);
u(:,1) = u0;
for k=2:n+1
  u(:,k) = expAdt*u(:,k-1);
end
// display the trajectory
xbasc()
param3d(u(1,:)',u(2,:)',u(3,:)', style=2, flag = [4,4])

col = xget("color")
xset("color",5) // red
param3d(u0(1),u0(2),u0(3),style=-3,flag=[0,0]) 
xset("color",col)
\end{Verbatim}
\end{examples}

\begin{manseealso}
  \manlink{sqrtm}{sqrtm}, \manlink{logm}{logm}, \manlink{ode}{ode}  
\end{manseealso}

% -- Authors
\begin{authors}
  This function uses Roger B. Sidje 's routines dgpadm.f, dspadm.f, zgpadm.f and zhpadm.f.
\end{authors}
