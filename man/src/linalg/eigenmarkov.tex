\mansection{eigenmarkov}
\begin{mandesc}
  \short{eigenmarkov}{normalized left and right Markov eigenvectors} \\ % 
\end{mandesc}
%\index{eigenmarkov}\label{eigenmarkov}
%-- Calling sequence section
\begin{calling_sequence}
\begin{verbatim}
  [M,Q]=eigenmarkov(P)  
\end{verbatim}
\end{calling_sequence}
%-- Parameters
\begin{parameters}
  \begin{varlist}
    \vname{P}: real N x N Markov matrix. Sum of entries in each row should add to one.
    \vname{M}: real matrix with N columns.
    \vname{Q}: real matrix with N rows.
  \end{varlist}
\end{parameters}
\begin{mandescription}
  Returns normalized left and right eigenvectors associated with the eigenvalue
  1 of the Markov transition matrix \verb!P!.  If the multiplicity of this
  eigenvalue is \verb!m! and \verb!P!  is \verb!N x N!, then \verb!M! is a
  \verb!m x N! matrix and \verb!Q! a \verb!N x m! matrix.  The row vector
  \verb!M(k,:)! is the probability distribution vector associated with the
  \verb!k!-th ergodic set (recurrent class). The element \verb!M(k,x)! is zero
  if \verb!x! is not in the \verb!k!-th recurrent class.  The element
  \verb!Q(x,k)! is the probability to end in the \verb!k!-th recurrent class
  starting from \verb!x!. If \verb!P^k! converges for large \verb!k! (no
  eigenvalues on the unit circle except 1), then the limit is \verb!Q*M!
  (eigenprojection).
\end{mandescription}
% --example 
\begin{examples}
  \paragraph{Example with two recurrent classes (with 2 and 1 states) and 2 transient states}
  \begin{mintednsp}{nsp}
    P=genmarkov([2,1],2) 
    [M,Q]=eigenmarkov(P);
    if norm(P*Q-Q) > 100*%eps then pause;end
    if Q*M-P^20 > 100*%eps then pause;end
  \end{mintednsp}
\end{examples}
% -- see also
\begin{manseealso}
  \manlink{genmarkov}{genmarkov} \manlink{classmarkov}{classmarkov}  
\end{manseealso}


