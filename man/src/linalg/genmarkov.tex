% -*- mode: latex -*-
%% Scilab ( http://www.scilab.org/ ) - This file is part of Scilab
%% Copyright (C) 1987-2016 - F. Delebecque
%%
%% This program is free software; you can redistribute it and/or modify
%% it under the terms of the GNU General Public License as published by
%% the Free Software Foundation; either version 2 of the License, or
%% (at your option) any later version.
%%
%% This program is distributed in the hope that it will be useful,
%% but WITHOUT ANY WARRANTY; without even the implied warranty of
%% MERCHANTABILITY or FITNESS FOR A PARTICULAR PURPOSE.  See the
%% GNU General Public License for more details.
%%
%% You should have received a copy of the GNU General Public License
%% along with this program; if not, write to the Free Software
%% Foundation, Inc., 59 Temple Place, Suite 330, Boston, MA  02111-1307  USA
%%                                                                                                

\mansection{genmarkov}
\begin{mandesc}
  \short{genmarkov}{generates random markov matrix with recurrent and transient classes} \\ % 
\end{mandesc}
%\index{genmarkov}\label{genmarkov}
%-- Calling sequence section
\begin{calling_sequence}
\begin{verbatim}
  M=genmarkov(rec,tr)  
  M=genmarkov(rec,tr,flag)  
\end{verbatim}
\end{calling_sequence}
%-- Parameters
\begin{parameters}
  \begin{varlist}
    \vname{rec}: integer row vector (its dimension is the number of recurrent classes).
    \vname{tr}: integer (number of transient states)
    \vname{M}: real Markov matrix. Sum of entries in each row should add to one.
    \vname{flag}: string \verb!'perm'!. If given, a random permutation of the states is done.
  \end{varlist}
\end{parameters}
\begin{mandescription}
  returns a random Markov transition probability matrix
  with \verb!n=size(rec,'*')! recurrent classes with \verb!rec(1)!, \ldots, \verb!rec(n)! 
  entries respectively and \verb!tr! transient states.
\end{mandescription}
% --example 
\begin{examples}
  \paragraph{Example with two recurrent classes (with 2 and 1 states) and 2 transient states}
  \begin{Verbatim}
    P=genmarkov([2,1],2,'perm')
    [perm,L]=classmarkov(P);
    P(perm,perm)
  \end{Verbatim}
\end{examples}
%-- see also
\begin{manseealso}
  \manlink{classmarkov}{classmarkov} \manlink{eigenmarkov}{eigenmarkov}  
\end{manseealso}
