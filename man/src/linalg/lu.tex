% -*- mode: latex -*-
%% Nsp
%% Copyright (C) 1998-2015 Jean-Philippe Chancelier Enpc/Cermics, Bruno Pin�on Esial/Iecn
%%
%% This library is free software; you can redistribute it and/or
%% modify it under the terms of the GNU General Public
%% License as published by the Free Software Foundation; either
%% version 2 of the License, or (at your option) any later version.
%%
%% This library is distributed in the hope that it will be useful,
%% but WITHOUT ANY WARRANTY; without even the implied warranty of
%% MERCHANTABILITY or FITNESS FOR A PARTICULAR PURPOSE.  See the GNU
%% General Public License for more details.
%%
%% You should have received a copy of the GNU General Public
%% License along with this library; if not, write to the
%% Free Software Foundation, Inc., 59 Temple Place - Suite 330,
%% Boston, MA 02111-1307, USA.
%%
 

\mansection{lu}
\begin{mandesc}
  \short{lu}{LU factorization}
\end{mandesc}

% -- Calling sequence section
\begin{calling_sequence}
\begin{verbatim}
[L, U[, p [, rc]]] = lu(A)
\end{verbatim}
\end{calling_sequence}
% -- Parameters
\begin{parameters}
  \begin{varlist}
    \vname{A}:  numerical matrix
    \vname{L}: lower triangular matrix 
    \vname{U}: upper triangular matrix
    \vname{p}: permutation vector
    \vname{rc}: real scalar
  \end{varlist}
\end{parameters}

\begin{mandescription}
This function computes a $P A = L U$ factorization of the (full) matrix $A$, 
(say $m \times n$) with (noting $p = min(m,n)$) :
\begin{itemize}
\item $L$ a lower  ($m \times p$) triangular matrix
\item $U$ an upper  ($p \times n$) triangular matrix
\item $P$ a permutation  ($m \times m$) matrix.
      (note that the output argument \verb+p+ is not a permutation matrix
       but a permutation vector (see hereafter)). 
\end{itemize}
Such a factorization is essentially useful to solve a square linear
system $A x = b$ and in particular, to solve several square linear
systems with the same matrix $A$ : $Ax_1=b_1$, $Ax_2 = b_2$, ... with
differents rights hand sides $b_1$, $b_2$,... not known at the same
time (for instance with $b_2$ depending on $x_1$). For LU factorization
of sparse matrices see \manlink{umfpack\_create}{umfpack_create}.

{\bf Caution :} when only 2 output arguments are required, then the first argument is
equal to :
$$
P^{-1}L
$$



\itemdesc{reciprocal condition number} 
In the case of a square matrix, an estimation of the reciprocal
condition number (in 1-norm) could be provided as the fourth output argument.
$$
rc \simeq \frac{1}{\kappa_1(A)} = \frac{1}{||A||_1 ||A^{-1}||_1}
$$
If $y$ is an approximate solution and $x$ the exact solution,
denoting $r = b - A y$ the residual we have the inequality :
$$
\frac{|| y - x||_1}{||x||_1} \le \kappa_1(A) \frac{||r||_1}{||b||_1}
$$
%The reciprocal condition number is a useful information for the
%numerical inversibility of a matrix : with a value less than the 
%epsilon machine (aka as unit relative rounding error) 
 
Finally note that the result must be the same than the one got 
with the function \verb+rcond+ applied on the matrix $A$.
 
\itemdesc{details on the permutation vector} 
Strictly speaking \verb+p+ is the inverse permutation associated
to the permutation matrix $P$ (\verb+p+ is such that $PA$ is equal 
to \verb+A(p,:)+).

\itemdesc{matrix factorization} 
Let \verb!A! be a $m \times n$ matrix with rank $r$ and let \verb![L,U,p]] = lu(A)! its LU factorization. 
Define \verb!P!, the row permutation matrix by \verb!I=eye(size(A,1));P=I(p,:)!. Then \verb!L*U = P*A! and 
\verb!A! and \verb!U! have same rank but the diagonal entries of the \verb!U! matrix are not necessarily 
ordered in such a way that the last rows of \verb!U! are zeroed.
A reliable rank determination can be obtained using the \verb!qr! or \verb!svd! factorization, 
where an additional transformation is performed on the columns of \verb!A!.
\end{mandescription}

\begin{examples}
\begin{Verbatim}
A=rand(5,5);
[L,U,p,rc]=lu(A);

// solving A x1 = b1:
b1=rand(5,1);
x1 = U\(L\b1(p));   // see comment next line 
// you can also use  x1 = solve(U, solve(L,b1(p,:),mode="lo"), mode="up"); see solve help page
norm(A*x1-b1)
    
// solving A x2 = b2:
b2=rand(5,1);
x2 = U\(L\b2(p));
norm(A*x2-b2)
    
// handling multiple right hand side:
b3=rand(5,4);
x3 = U\(L\b3(p,:));
A*x3 - b3
\end{Verbatim}
\end{examples}

\begin{manseealso}
  \manlink{solve}{solve}, \manlink{qr}{qr},\manlink{chol}{chol},\manlink{umfpack\_create}{umfpack_create}
\end{manseealso}

% -- Authors
\begin{authors}
   interface : Bruno Pincon and Jean-Philippe Chancelier. Internally lapack routines dgetrf in the
   real case and zgetrf in the complex case are used (and dgecon or zgecon if
   rcond is required).
\end{authors}
