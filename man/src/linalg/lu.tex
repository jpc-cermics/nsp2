% -*- mode: latex -*-

\mansection{lu}
\begin{mandesc}
  \short{lu}{LU factorization}
\end{mandesc}

% -- Calling sequence section
\begin{calling_sequence}
\begin{verbatim}
[L, U[, p [, rc]]] = lu(A)
\end{verbatim}
\end{calling_sequence}
% -- Parameters
\begin{parameters}
  \begin{varlist}
    \vname{A} :  numerical matrix
    \vname{L} : lower triangular matrix 
    \vname{U} : upper triangular matrix
    \vname{p} : permutation vector
    \vname{rc} : real scalar
  \end{varlist}
\end{parameters}

\begin{mandescription}
This function computes a $P A = L U$ factorization of the matrix $A$, 
(says $m \times n$) with (noting $p = min(m,n)$) :
\begin{itemize}
\item $L$ a lower  ($m \times p$) triangular matrix
\item $U$ an upper  ($p \times n$) triangular matrix
\item $P$ a permutation  ($m \times m$) matrix.
      (note that the output argument \verb+p+ is not a permutation matrix
       but a permutation vector (see hereafter)). 
\end{itemize}
Such a factorization is essentially useful to solve a square linear
system $A x = b$ and in particular, to solve several square linear
systems with the same matrix $A$ : $Ax_1=b_1$, $Ax_2 = b_2$, ... with
differents rights hand sides $b_1$, $b_2$,... not known at the same
time (for instance with $b_2$ depending on $x_1$). 

{\bf Caution :} when only 2 output arguments are required, then the first argument is
equal to :
$$
P^{-1}L
$$



\itemdesc{reciprocal condition number} 
In the case of a square matrix, an estimation of the reciprocal
condition number (in 1-norm) could be provided as the fourth output argument.
$$
rc \simeq \frac{1}{\kappa_1(A)} = \frac{1}{||A||_1 ||A^{-1}||_1}
$$
If $x'$ is an approximate solution and $x$ the exact solution,
denoting $r = b - Ax'$ the residual we have the inequality :
$$
\frac{||x' - x||_1}{||x||_1} \le \kappa_1(A) \frac{||r||_1}{||b||_1}
$$
%The reciprocal condition number is a useful information for the
%numerical inversibility of a matrix : with a value less than the 
%epsilon machine (aka as unit relative rounding error) 
 
Finally note that the result must be the same than the one got 
with the function \verb+rcond+ applied on the matrix $A$.
 
\itemdesc{details on the permutation vector} 
Strictly speaking \verb+p+ is the inverse permutation associated
to the permutation matrix $P$ (\verb+p+ is such that $PA$ is equal 
to \verb+A(p,:)+).

\end{mandescription}

\begin{examples}
  \begin{program}\HCode{A=rand(5,5);\Hnewline
    [L,U,p,rc]=lu(A);\Hnewline
    // solving A x1 = b1:\Hnewline
    b1=rand(5,1);\Hnewline
    x1 = U}\verb+\+\HCode{(L}\verb+\+\HCode{b1(p));\Hnewline
    norm(A*x1-b1)\Hnewline
    // solving A x2 = b2:\Hnewline
    b2=rand(5,1);\Hnewline
    x2 = U}\verb+\+\HCode{(L}\verb+\+\HCode{b2(p));\Hnewline
    norm(A*x2-b2)\Hnewline
    // handling multiple right hand side:\Hnewline
    b3=rand(5,4);\Hnewline
    x3 = U}\verb+\+\HCode{(L}\verb+\+\HCode{b3(p,:));\Hnewline
    A*x3 - b3}
  \end{program}
\end{examples}

\begin{manseealso}
  \manlink{qr}{qr}  
\end{manseealso}

% -- Authors
\begin{authors}
   interface : Bruno Pin�on and Jean-Philippe Chancelier. Internally lapack routines dgetrf in the
   real case and zgetrf in the complex case are used (and dgecon or zgecon if
   rcond is required).
\end{authors}
