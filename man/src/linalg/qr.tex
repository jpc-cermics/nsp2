% -*- mode: latex -*-
%% Nsp
%% Copyright (C) 1998-2015 Jean-Philippe Chancelier Enpc/Cermics, Bruno Pin�on Esial/Iecn
%%
%% This library is free software; you can redistribute it and/or
%% modify it under the terms of the GNU General Public
%% License as published by the Free Software Foundation; either
%% version 2 of the License, or (at your option) any later version.
%%
%% This library is distributed in the hope that it will be useful,
%% but WITHOUT ANY WARRANTY; without even the implied warranty of
%% MERCHANTABILITY or FITNESS FOR A PARTICULAR PURPOSE.  See the GNU
%% General Public License for more details.
%%
%% You should have received a copy of the GNU General Public
%% License along with this library; if not, write to the
%% Free Software Foundation, Inc., 59 Temple Place - Suite 330,
%% Boston, MA 02111-1307, USA.
%%
 

\mansection{qr}
\begin{mandesc}
  \short{qr}{QR factorization}
\end{mandesc}

% -- Calling sequence section
\begin{calling_sequence}
\begin{verbatim}
[Q, R] = qr(A)
[Q, R [,p [,rk [,sval]]]] = qr(A, mode=str, tol=rscal)
\end{verbatim}
\end{calling_sequence}
% -- Parameters
\begin{parameters}
  \begin{varlist}
    \vname{A}:  numerical matrix (say $m \times n$)
    \vname{mode=str}: optional named argument, \verb+mode="x"+(default) or \verb+mode="e"+
    \vname{tol=rscal}: optional named argument, \verb+rscal+ must be a real scalar in (0,1) (default \verb+max(m,n)*%eps+)
    \vname{Q}: unitary or orthogonal matrix
    \vname{R}: upper triangular matrix
    \vname{p}: permutation vector
    \vname{rk}: rank estimation
    \vname{sval}: a few approximate singular values
  \end{varlist}
\end{parameters}

\begin{mandescription}
This function computes a $A = Q R$ or $A P = QR$ factorization of 
the matrix $A$, with:
\begin{itemize}
\item $Q$ an orthogonal matrix (unitary in the complex case) 
\item $R$ an upper triangular matrix (with decreasing diagonal 
      elements in the case $A P = QR$) 
\item $P$ a permutation matrix. The $A P = QR$ factorization is 
      computed when at least a third output argument is needed
      (note that the output argument \verb+p+ is not a permutation matrix
       but a permutation vector (see hereafter)). 
\end{itemize}

When the function is used with a fourth output argument 
a rank detection is done by incremental estimation of max and min
singular values and vectors of the triangular matrice $R$ (same method
than the one used in the lapack routine dgelsy.f). In that case these
estimations can be returned in the output argument \verb+sval+ with:
\begin{itemize}
\item $sval_1$ an estimation of the max singular value,
\item $sval_2$ an estimation of the min singular value,
\item and $sval_3 < sval_2$ when $sval_3$ is under the tolerance 
      level compared to $sval_1$ ($sval_3 < tol sval_1$). The matrix
      is then considered not as a full rank matrix
      ($rank(A)<min(n,m)$) for the choosen tolerance level. 
      Otherwise $sval_3 = sval_2$ and $rank(A)=min(n,m)$.
\end{itemize}

\itemdesc{mode parameter}
\begin{itemize}
   \item When \verb+mode="x"+ $Q$ is an $m \times m$ (orthogonal) matrix and
         $R$ an  $m \times n$ (upper triangular) matrix. The first
         $rk$ columns of $Q$ form an orthogonal basis of the column
         space of $A$.
   \item  \verb+mode="e"+ noting $p=min(n,m)$, then $Q$ is an $m \times p$ matrix and $R$ is
         an $p \times n$ matrix, so we get (in case $n<m$) an economic
         sized factorization.
\end{itemize}
  
\itemdesc{tol parameter} 
By default $tol = max(m,n) \epsilon_m$ is used with $\epsilon_m$ the epsilon
machine (aka the relative unit roundoff error $u$) ($\epsilon_m =$ \verb+2^(-53+)).

\itemdesc{details on the permutation vector} 
The output argument \verb+p+ is the permutation associated
to the permutation matrix $P$ (\verb+p+ is such that $AP$ is equal to \verb+A(:,p)+).

\end{mandescription}

\begin{examples}
\paragraph{example 1} solving a least square problem:
\begin{mintednsp}{nsp}
A=rand(10,5);
b=rand(10,1);
// solving min ||Ax - b|| with a simple economic sized A = QR factorization
[Q,R]=qr(A,mode="e");
x = R\(Q'*b);
// solving min ||Ax - b|| with a AP = QR factorization
[Q,R,p]=qr(A,mode="e");
x = R\(Q'*b);
x(p) = x; // apply the permutation
\end{mintednsp}

\paragraph{example 2} rank detection
\begin{mintednsp}{nsp}
// build a random matrix and make one column a linear combinaison of some others
A=rand(5,4);
A(:,2) = 0.1*A(:,1) - 3*A(:,4); // A must be now of rank 3
[Q,R,p,rk,sval]=qr(A)
// compare sval with the "real" singular values
sv = svd(A);
// sval(1) and sval(2) are rough estimations of sv(1) and sv(3)
esvmax = abs((sval(1)-sv(1))/sv(1))  // relative error for the max sv estimation
esvmin = abs((sval(2)-sv(3))/sv(3))  // relative error for the min sv estimation
\end{mintednsp}

\end{examples}

\begin{manseealso}
  \manlink{qr}{qr}  
\end{manseealso}

% -- Authors
\begin{authors}
   interface : Bruno Pincon and Jean-Philippe Chancelier. Internally lapack routines dgeqrpf (or
   dgeqp3), dorgqr are used in real case and zgeqrpf (or zgeqp3) and
   zungqr are used in the complex case.
\end{authors}
