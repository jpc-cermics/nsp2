% -*- mode: latex -*-
\mansection{toeplitz}
\begin{mandesc}
  \short{toeplitz}{Toeplitz matrix}\\ % @mandesc@
  \short{hankel}{Hankel matrix}
\end{mandesc}
% -- Calling sequence section
\begin{calling_sequence}
\begin{verbatim}
   A=toeplitz(c [,r])   
   A=hankel(c [,r])   
\end{verbatim}
\end{calling_sequence}
\begin{parameters}
  \begin{varlist}
    \vname{c,r}: two matrices of the same type. 
  \end{varlist}
\end{parameters}
\begin{mandescription}
  builds a \href{http://en.wikipedia.org/wiki/Toeplitz_matrix}{Toeplitz}
  (resp. \href{http://en.wikipedia.org/wiki/Hankel_matrix}{Hankel})
  matrix whose first column is described 
  by \verb+c+ and first (resp. last) row is described by \verb+r+. 
  When \verb+r+ is omitted it is assumed to be equal to \verb+c+ for Toeplitz
  matrix and assumed to be zero for Hankel matrix. 
  The two arguments should have the same type which also gives the type of 
  the result. Most matrix types are authorized. 
  When \verb+c+ and \verb+r+ are both given, a warning is issued if they are 
  not comptatible and incompatibility is solved by using values from \verb+c+. 
\end{mandescription}
% --example 
\begin{examples}
  \begin{Verbatim}
    A=toeplitz(m2i(1:5,'uint32'));
    B=hankel(string(1:3));
  \end{Verbatim}
\end{examples}
% -- see also
\begin{authors}
  Jean-Philippe Chancelier. 
\end{authors}

