% -*- mode: latex -*-
%% Scilab ( http://www.scilab.org/ ) - This file is part of Scilab
%% Copyright (C) 1987-2016 - F. Delebecque
%%
%% This program is free software; you can redistribute it and/or modify
%% it under the terms of the GNU General Public License as published by
%% the Free Software Foundation; either version 2 of the License, or
%% (at your option) any later version.
%%
%% This program is distributed in the hope that it will be useful,
%% but WITHOUT ANY WARRANTY; without even the implied warranty of
%% MERCHANTABILITY or FITNESS FOR A PARTICULAR PURPOSE.  See the
%% GNU General Public License for more details.
%%
%% You should have received a copy of the GNU General Public License
%% along with this program; if not, write to the Free Software
%% Foundation, Inc., 59 Temple Place, Suite 330, Boston, MA  02111-1307  USA
%%                                                                                                

\mansection{toeplitz}
\begin{mandesc}
  \short{toeplitz}{Toeplitz matrix}\\ % @mandesc@
  \short{hankel}{Hankel matrix}
\end{mandesc}
% -- Calling sequence section
\begin{calling_sequence}
\begin{verbatim}
   A=toeplitz(c [,r])   
   A=hankel(c [,r])   
\end{verbatim}
\end{calling_sequence}
\begin{parameters}
  \begin{varlist}
    \vname{c,r}: two matrices of the same type. 
  \end{varlist}
\end{parameters}
\begin{mandescription}
  builds a \href{http://en.wikipedia.org/wiki/Toeplitz_matrix}{Toeplitz}
  (resp. \href{http://en.wikipedia.org/wiki/Hankel_matrix}{Hankel})
  matrix whose first column is described 
  by \verb+c+ and first (resp. last) row is described by \verb+r+. 
  When \verb+r+ is omitted it is assumed to be equal to \verb+c+ for Toeplitz
  matrix and assumed to be zero for Hankel matrix. 
  The two arguments should have the same type which also gives the type of 
  the result. Most matrix types are authorized. 
  When \verb+c+ and \verb+r+ are both given, a warning is issued if they are 
  not comptatible and incompatibility is solved by using values from \verb+c+. 
\end{mandescription}
% --example 
\begin{examples}
  \begin{Verbatim}
    A=toeplitz(m2i(1:5,'uint32'));
    B=hankel(sparse(1:3));
  \end{Verbatim}
\end{examples}
% -- see also
\begin{authors}
  Jean-Philippe Chancelier. 
\end{authors}

