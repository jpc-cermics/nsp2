% -*- mode: latex -*-
%% Scilab ( http://www.scilab.org/ ) - This file is part of Scilab
%% Copyright (C) 1987-2016 - F. Delebecque
%%
%% This program is free software; you can redistribute it and/or modify
%% it under the terms of the GNU General Public License as published by
%% the Free Software Foundation; either version 2 of the License, or
%% (at your option) any later version.
%%
%% This program is distributed in the hope that it will be useful,
%% but WITHOUT ANY WARRANTY; without even the implied warranty of
%% MERCHANTABILITY or FITNESS FOR A PARTICULAR PURPOSE.  See the
%% GNU General Public License for more details.
%%
%% You should have received a copy of the GNU General Public License
%% along with this program; if not, write to the Free Software
%% Foundation, Inc., 59 Temple Place, Suite 330, Boston, MA  02111-1307  USA
%%                                                                                                

\mansection{randpencil}
\begin{mandesc}
  \short{randpencil}{random pencil} \\ % 
\end{mandesc}
%\index{randpencil}\label{randpencil}
%-- Calling sequence section
\begin{calling_sequence}
\begin{verbatim}
  F=randpencil(eps,infi,fin,eta)  
\end{verbatim}
\end{calling_sequence}
%-- Parameters
\begin{parameters}
  \begin{varlist}
    \vname{eps}: vector of integers
    \vname{infi}: vector of integers
    \vname{fin}: real vector, or monic polynomial, or vector of monic polynomial
    \vname{eta}: vector of integers
    \vname{F}: real matrix pencil \verb!F=s*E-A!  (\verb!s=poly(0,'s')!)
  \end{varlist}
\end{parameters}
\begin{mandescription}
  Utility function.
  \verb!F=randpencil(eps,infi,fin,eta)! returns a random pencil \verb!F!
  with given Kronecker structure. The structure is given by:
  \verb!eps=[eps1,...,epsk]!: structure of epsilon blocks (size eps1x(eps1+1),....)
  \verb!fin=[l1,...,ln]!  set of finite eigenvalues (assumed real)  (possibly [])
  \verb!infi=[k1,...,kp]! size of J-blocks at infinity
  \verb!ki$>$=1!  (infi=[] if no J blocks).
  \verb!eta=[eta1,...,etap]!: structure ofeta blocks (size eta1+1)xeta1,...)\verb!epsi!'s should be $>$=0, \verb!etai!'s should be $>$=0, \verb!infi!'s should 
  be $>$=1.
  If \verb!fin! is a (monic) polynomial, the finite block admits the roots of 
  \verb!fin! as eigenvalues.
  If \verb!fin! is a vector of polynomial, they are the finite elementary
  divisors of \verb!F! i.e. the roots of \verb!p(i)! are finite
  eigenvalues of \verb!F!.
\end{mandescription}
%--example 
\begin{examples}
  \begin{Verbatim}
    F=randpencil([0,1],[2],[-1,0,1],[3]);
    [Q,Z,Qd,Zd,numbeps,numbeta]=kroneck(F);
    Qd, Zd
    s=poly(0,'s');
    F=randpencil([],[1,2],s^3-2,[]); //regular pencil
    det(F)
  \end{Verbatim}
\end{examples}
%-- see also
\begin{manseealso}
  \manlink{kroneck}{kroneck} \manlink{pencan}{pencan} \manlink{penlaur}{penlaur}  
\end{manseealso}
\begin{authors}
  Fran�ois Delebecque
\end{authors}
