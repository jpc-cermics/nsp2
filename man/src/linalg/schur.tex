% -*- mode: latex -*-
%% Scilab ( http://www.scilab.org/ ) - This file is part of Scilab
%% Copyright (C) 1987-2016 - F. Delebecque
%%
%% This program is free software; you can redistribute it and/or modify
%% it under the terms of the GNU General Public License as published by
%% the Free Software Foundation; either version 2 of the License, or
%% (at your option) any later version.
%%
%% This program is distributed in the hope that it will be useful,
%% but WITHOUT ANY WARRANTY; without even the implied warranty of
%% MERCHANTABILITY or FITNESS FOR A PARTICULAR PURPOSE.  See the
%% GNU General Public License for more details.
%%
%% You should have received a copy of the GNU General Public License
%% along with this program; if not, write to the Free Software
%% Foundation, Inc., 59 Temple Place, Suite 330, Boston, MA  02111-1307  USA
%%                                                                                                


\mansection{schur}
\begin{mandesc}
  \short{schur}{Schur decomposition of a matrix} \\ % 
\end{mandesc}
%-- Calling sequence section
\begin{calling_sequence}
\begin{verbatim}
  [T]= schur(A);
  [U,T] = schur(A)   
  [U,dim [,T] ]=schur(A, sort= sel, args= obj, complex=%t|%f)
\end{verbatim}
\end{calling_sequence}
%-- Parameters
\begin{parameters}
  \begin{varlist}
    \vname{A}: a real or complex square matrix.
    \vname{sel}: a string (\verb!'c'! or \verb!'d'!) or a nsp function.
    \vname{U}: an orthogonal or unitary square matrix
    \vname{T}: an upper triangular or quasi-triangular square matrix
    \vname{dim}: an integer.
    \vname{obj}: a list.
  \end{varlist}
\end{parameters}
\begin{mandescription}
   the function returns a quasi-triangular Schur matrix \verb!T! and 
   a unitary matrix \verb!U! so that \verb!A = U*T*U'! and 
   \verb!U'*U = eye(size(U))!. Depending on the type of the 
   given matrix \verb!A! a real or complex Schur form is returned. 
   The complex Schur form is  upper triangular with eigenvalues
   of matrix \verb!A! on the diagonal of matrix \verb!T!. The real 
   Schur form is quasi-triangular with real eigenvalues of matrix \verb!A! 
   on the diagonal matrix of \verb!T! and complex eigenvalues of 
    \verb!A! in 2-by-2 blocks on the diagonal of \verb!T!.

    It is possible to obtain the complex Schur form of a real matrix 
    by adding an optional argument \verb!complex=%t! when calling function
    \verb!schur!.

    if and optional argument \verb!sort! is given, then an ordered 
    Schur transformation is performed in such a way that eigenvalues 
    selected by the sort function appear in the leading (upper left) 
    diagonal blocks of the quasitriangular Schur matrix. The corresponding 
    invariant subspace has dimension \verb!dim! and is given by the 
    leading \verb!dim! columns of \verb!U!. Using:
    \begin{itemize}
      \item \verb!sort='c'!: the eigenspace associated with
      eigenvalues with negative real parts (stable continuous time eigenspace) is selected.
      \item \verb!sort='d'!: The eigenspace associated with
      eigenvalues with magnitude lower than 1 (stable discrete time eigenspace) is selected.
      \item \verb!sort=sel'!: The eigenspace associated with
      eigenvalues for which the function \verb!sel(r,c,obj)! returns true 
      is selected. Extra arguments to the function \verb!sel! are given 
      by the optional argument \verb!obj! which is a list transmited with \verb!args=obj!.
    \end{itemize}
\end{mandescription}
\begin{examples}
  \begin{mintednsp}{nsp}
    A=diag([-0.9,-2,2,0.9]);X=rand(A);A=inv(X)*A*X;
    [U,T]=schur(A);
    T
    [U,dim,T]=schur(A,sort='c');
    diag(T(1:dim,1:dim))      //stable cont. eigenvalues
    function t=sel(Ev), t=abs(Ev) < 0.95,endfunction
    [U,dim,T]=schur(A,sort = sel);
    and(diag(T(1:dim,1:dim)) < 0.95) // should return true
    
    function t=sel(Ev,value), t=abs(Ev) < value(1) ,endfunction
    [U,dim,T1]=schur(A,sort = sel,args=list(0.95));
    T.equal[T1]
  \end{mintednsp}
\end{examples}
%-- see also
\begin{manseealso}
  \manlink{spec}{spec} \manlink{qz}{qz}
\end{manseealso}
