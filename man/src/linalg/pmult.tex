% -*- mode: latex -*-

\mansection{pmult}
\begin{mandesc}
  \short{pmult}{special matrix multiplication}
\end{mandesc}

% -- Calling sequence section
\begin{calling_sequence}
\begin{verbatim}
 y = pmult(A,x) 
 y = pmult(A,x,flag)
\end{verbatim}
\end{calling_sequence}
% -- Parameters
\begin{parameters}
  \begin{varlist}
    \vname{A}:  sparse or full numerical matrix.
    \vname{x}: full numerical vector or matrix
    \vname{flag}: optional argument (only in case when both A and x are full), integer among 0,1,2 or 3 (default 1).
    \vname{y}: full numerical numerical vector or matrix
  \end{varlist}
\end{parameters}

\begin{mandescription}
This function provides the operation $A'*x$ without computing first the transposition of $A$
(this brings some gain in speed and memory). $A$ could be sparse or full. When $A$ is a full
matrix the optional argument flag provides the additional features:
\begin{description}
\item[flag=0] then $y = A*x$ (the usual matrix product)
\item[flag=1] then $y = A'*x$ (the default case)
\item[flag=2] then $y = A*x'$
\item[flag=3] then $y = A'*x'$
\end{description} 
\end{mandescription}

\begin{examples}
\begin{Verbatim}
// show speed improvments in the sparse case
A=sprand(20000,20000,1/2000,'nor');
x=randn(20000,1); 
tic(); y1 = A'*x; toc()  
tic(); y2 = pmult(A,x); toc() 
norm(y1-y2)/norm(y1)
// compute normal equations of a linear least square pb
A=randn(700,200); b = randn(700,1);
tic(); At=A'; AA=At*A; bb=At*b; toc()
tic(); AA=pmult(A,A); bb=pmult(A,b); toc()
\end{Verbatim}
\end{examples}

%\begin{manseealso}
%  \manlink{qr}{qr} ,  \manlink{kernel}{kernel} 
%\end{manseealso}

% -- Authors
\begin{authors}
   Bruno Pincon
\end{authors}
