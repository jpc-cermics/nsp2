% -*- mode: latex -*-
\mansection{logm}
\begin{mandesc}
  \short{logm}{logarithm of a square matrix} \\ % 
  \short{sqrtm}{square root of a square matrix} \\ % 
\end{mandesc}
%-- Calling sequence section
\begin{calling_sequence}
\begin{verbatim}
  B=logm(A)  
  B=sqrtm(A)  
\end{verbatim}
\end{calling_sequence}
%-- Parameters
\begin{parameters}
  \begin{varlist}
    \vname{A,B}: square numerical matrices.
  \end{varlist}
\end{parameters}
\begin{mandescription}
  \verb!logm(A)! returns the matrix logarithm of matrix \verb!A! and 
  \verb!sqrtm(A)! returns the square root of matrix \verb!A!. 
  If matrix \verb!A! is a symmetric (hermitian) matrix, the 
  computation is performed using the schur decomposition of matrix \verb!A!.
  Otherwise, \verb!A! is turned to a complex matrix, the function 
  \verb!spec! is used to diagonalize the matrix and if possible 
  the result is turned back to a real matrix. Thus, the result will be 
  complex in both cases  if \verb!A! is not positive-semidefinite.
\end{mandescription}
%--example 
\begin{examples}
\begin{itemize}
  \item \verb!logm!
  \begin{Verbatim}
    A=rand(4,4); 
    A=A*diag([1:4])*inv(A);
    B=logm(A)
    isreal(B,%t)
    norm(expm(B)-A)
    A=rand(4,4);
    A=A*A';
    norm(expm(logm(A))-A)
    A=rand(4,4)+%i*rand(4,4);
    A=A*A';
    norm(expm(logm(A))-A)
  \end{Verbatim}
  \item \verb!sqrtm!
  \begin{Verbatim}
    A=rand(4,4); 
    A=A*diag([1:4])*inv(A);
    B=sqrtm(A)
    isreal(B,%t)
    norm(B*B-A)
    A=rand(4,4); A=A*A';
    B=sqrtm(A);
    norm(B*B-A)
  \end{Verbatim}
\end{itemize}
\end{examples}
%-- see also
\begin{manseealso}
  \manlink{expm}{expm} \manlink{schur}{schur}  \manlink{spec}{spec}  
\end{manseealso}
