% -*- mode: latex -*-
%% Scilab ( http://www.scilab.org/ ) - This file is part of Scilab
%% Copyright (C) 1987-2016 - F. Delebecque
%%
%% This program is free software; you can redistribute it and/or modify
%% it under the terms of the GNU General Public License as published by
%% the Free Software Foundation; either version 2 of the License, or
%% (at your option) any later version.
%%
%% This program is distributed in the hope that it will be useful,
%% but WITHOUT ANY WARRANTY; without even the implied warranty of
%% MERCHANTABILITY or FITNESS FOR A PARTICULAR PURPOSE.  See the
%% GNU General Public License for more details.
%%
%% You should have received a copy of the GNU General Public License
%% along with this program; if not, write to the Free Software
%% Foundation, Inc., 59 Temple Place, Suite 330, Boston, MA  02111-1307  USA
%%                                                                                                

\mansection{nlev}
\begin{mandesc}
  \short{nlev}{Leverrier's algorithm} \\ % 
\end{mandesc}
%\index{nlev}\label{nlev}
%-- Calling sequence section
\begin{calling_sequence}
\begin{verbatim}
  [num,den]=nlev(A,z [,rmax])    
\end{verbatim}
\end{calling_sequence}
%-- Parameters
\begin{parameters}
  \begin{varlist}
    \vname{A}: real square matrix
    \vname{z}: character string
    \vname{rmax}: optional parameter (see \verb!bdiag!)
  \end{varlist}
\end{parameters}
\begin{mandescription}
  \verb![num,den]=nlev(A,z [,rmax])! computes
  \verb!(z*eye()-A)^(-1)!
  by block diagonalization of A followed by Leverrier's algorithm
  on each block.
  This algorithm is better than the usual leverrier algorithm but
  still not perfect!
\end{mandescription}
%--example 
\begin{examples}
  \begin{mintednsp}{nsp}
    A=rand(3,3);x=poly(0,'x');
    [n,d]=nlev(A,'x')
    clean(d + poly(A,'x'))
    clean( n*(x*eye(size(A))-A) -d*eye(size(A)),1.e-10)
  \end{mintednsp}
\end{examples}
%-- see also
\begin{manseealso}
  \manlink{coff}{coff} \manlink{coffg}{coffg} \manlink{glever}{glever} \manlink{ss2tf}{ss2tf}  
\end{manseealso}
%-- Author
\begin{authors}
  Fran�ois Delebecque, Serge Steer.     
\end{authors}
