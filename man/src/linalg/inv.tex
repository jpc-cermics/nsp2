% -*- mode: latex -*-
\mansection{inv}
\begin{mandesc}
  \short{inv}{inverse of a square matrix} \\ % 
\end{mandesc}
%-- Calling sequence section
\begin{calling_sequence}
\begin{verbatim}
  [B,rcond]= inv(A)  
\end{verbatim}
\end{calling_sequence}
%-- Parameters
\begin{parameters}
  \begin{varlist}
    \vname{A}: real or complex square matrix (n x n).
    \vname{rcond}: a real scalar.
  \end{varlist}
\end{parameters}
\begin{mandescription}
  Compute the inverse of the square matrix \verb!A!. If requested 
  the estimation of the reciprocal condition number is returned. 
\end{mandescription}
%--example 
\begin{examples}
  \begin{Verbatim}
    A=rand(4,4);
    norm(inv(A)*A)
    T = triu(rand(600,600));  // A badly conditioned matrix 
    [U,rcond] = inv(T);
  \end{Verbatim}
\end{examples}
%-- see also
\begin{manseealso}
  \manlink{pinv}{pinv} 
\end{manseealso}
% -- Authors
\begin{authors}
   interface : Bruno Pincon and Jean-Philippe Chancelier. 
   Internally, if the matrice is triangular and real (resp. complex) the lapack routine dtrtri (resp. ztrtri) is used and otherwise dgetri (resp. zgetri) is used.
\end{authors}
