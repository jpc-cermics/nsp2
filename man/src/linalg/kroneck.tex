% -*- mode: latex -*-
%% Scilab ( http://www.scilab.org/ ) - This file is part of Scilab
%% Copyright (C) 1987-2016 - F. Delebecque
%%
%% This program is free software; you can redistribute it and/or modify
%% it under the terms of the GNU General Public License as published by
%% the Free Software Foundation; either version 2 of the License, or
%% (at your option) any later version.
%%
%% This program is distributed in the hope that it will be useful,
%% but WITHOUT ANY WARRANTY; without even the implied warranty of
%% MERCHANTABILITY or FITNESS FOR A PARTICULAR PURPOSE.  See the
%% GNU General Public License for more details.
%%
%% You should have received a copy of the GNU General Public License
%% along with this program; if not, write to the Free Software
%% Foundation, Inc., 59 Temple Place, Suite 330, Boston, MA  02111-1307  USA
%%                                                                                                

\mansection{kroneck}
\begin{mandesc}
  \short{kroneck}{Kronecker form of matrix pencil} \\ % 
\end{mandesc}
%\index{kroneck}\label{kroneck}
%-- Calling sequence section
\begin{calling_sequence}
\begin{verbatim}
  [Q,Z,Qd,Zd,numbeps,numbeta]=kroneck(F)  
  [Q,Z,Qd,Zd,numbeps,numbeta]=kroneck(E,A)  
\end{verbatim}
\end{calling_sequence}
%-- Parameters
\begin{parameters}
  \begin{varlist}
    \vname{F}: real matrix pencil \verb!F=s*E-A!
    \vname{E,A}: two real matrices of same dimensions
    \vname{Q,Z}: two square orthogonal matrices
    \vname{Qd,Zd}: two vectors of integers
    \vname{numbeps,numeta}: two vectors of integers
  \end{varlist}
\end{parameters}
\begin{mandescription}
  Kronecker form of matrix pencil: \verb!kroneck! computes two
  orthogonal matrices \verb!Q, Z! which put the pencil \verb!F=s*E -A! into
  upper-triangular form:
  \begin{equation*}
    Q(s E -A)Z = \left(
    \begin{array}{cccc} 
      sE(eps)-A(eps)  & \cdot & \cdot & \cdot  \\
      0               & sE(inf)-A(inf) & \cdot & \cdot  \\
      0               & 0              & sE(f)-A(f) & \cdot  \\
      0               & 0              & 0              & sE(eta)-A(eta)
    \end{array}
    \right) 
  \end{equation*}
% \begin{verbatim}
%              | sE(eps)-A(eps) |        X       |      X     |      X        |
%              |----------------|----------------|------------|---------------|
%              |        O       | sE(inf)-A(inf) |      X     |      X        |
%   Q(sE-A)Z = |---------------------------------|----------------------------|
%              |                |                |            |               |
%              |        0       |       0        | sE(f)-A(f) |      X        |
%              |--------------------------------------------------------------|
%              |                |                |            |               |
%              |        0       |       0        |      0     | sE(eta)-A(eta)|
% \end{verbatim}
The dimensions of the four blocks are given by:\verb!eps=Qd(1) x Zd(1)!, \verb!inf=Qd(2) x Zd(2)!,
\verb!f = Qd(3) x Zd(3)!, \verb!eta=Qd(4)xZd(4)!
The \verb!inf! block contains the infinite modes of
the pencil.
The \verb!f! block contains the finite modes of
the pencil
The structure of epsilon and eta blocks are given by:\verb!numbeps(1)! = \verb!#! of eps blocks of size 0 x 1\verb!numbeps(2)! = \verb!#! of eps blocks of size 1 x 2\verb!numbeps(3)! = \verb!#! of eps blocks of size 2 x 3     etc...\verb!numbeta(1)! = \verb!#! of eta blocks of size 1 x 0\verb!numbeta(2)! = \verb!#! of eta blocks of size 2 x 1\verb!numbeta(3)! = \verb!#! of eta blocks of size 3 x 2     etc...
The code is taken from T. Beelen (Slicot-WGS group).
\end{mandescription}
%--example 
\begin{examples}
  \begin{mintednsp}{nsp}
    F=randpencil([1,1,2],[2,3],[-1,3,1],[0,3]);
    Q=rand(17,17);Z=rand(18,18);F=Q*F*Z;
    //random pencil with eps1=1,eps2=1,eps3=1; 2 J-blocks @ infty 
    //with dimensions 2 and 3
    //3 finite eigenvalues at -1,3,1 and eta1=0,eta2=3
    [Q,Z,Qd,Zd,numbeps,numbeta]=kroneck(F);
    [Qd(1),Zd(1)]    //eps. part is sum(epsi) x (sum(epsi) + number of epsi) 
    [Qd(2),Zd(2)]    //infinity part
    [Qd(3),Zd(3)]    //finite part
    [Qd(4),Zd(4)]    //eta part is (sum(etai) + number(eta1)) x sum(etai)
    numbeps
    numbeta
  \end{mintednsp}
\end{examples}
%-- see also
\begin{manseealso}
  \manlink{gschur}{gschur} \manlink{gspec}{gspec} \manlink{systmat}{systmat}
  \manlink{pencan}{pencan} \manlink{randpencil}{randpencil}
  \manlink{trzeros}{trzeros}
\end{manseealso}
\begin{authors}
  Fran�ois Delebecque
\end{authors}
