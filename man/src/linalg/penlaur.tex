% -*- mode: latex -*-
%% Scilab ( http://www.scilab.org/ ) - This file is part of Scilab
%% Copyright (C) 1987-2016 - F. Delebecque
%%
%% This program is free software; you can redistribute it and/or modify
%% it under the terms of the GNU General Public License as published by
%% the Free Software Foundation; either version 2 of the License, or
%% (at your option) any later version.
%%
%% This program is distributed in the hope that it will be useful,
%% but WITHOUT ANY WARRANTY; without even the implied warranty of
%% MERCHANTABILITY or FITNESS FOR A PARTICULAR PURPOSE.  See the
%% GNU General Public License for more details.
%%
%% You should have received a copy of the GNU General Public License
%% along with this program; if not, write to the Free Software
%% Foundation, Inc., 59 Temple Place, Suite 330, Boston, MA  02111-1307  USA
%%                                                                                                

\mansection{penlaur}
\begin{mandesc}
  \short{penlaur}{Laurent coefficients of matrix pencil} \\ % 
\end{mandesc}
%\index{penlaur}\label{penlaur}
%-- Calling sequence section
\begin{calling_sequence}
\begin{verbatim}
  [Si,Pi,Di,order]=penlaur(Fs)  
  [Si,Pi,Di,order]=penlaur(E,A)  
\end{verbatim}
\end{calling_sequence}
%-- Parameters
\begin{parameters}
  \begin{varlist}
    \vname{Fs}: a regular pencil \verb!s*E-A!
    \vname{E, A}: two real square matrices
    \vname{Si,Pi,Di}: three real square matrices
    \vname{order}: integer
  \end{varlist}
\end{parameters}
\begin{mandescription}
  \verb!penlaur! computes the first Laurent coefficients of \verb!(s*E-A)^-1! at
  infinity.\verb!(s*E-A)^-1 = ... + Si/s - Pi - s*Di + ...! at \verb!s! = infinity.\verb!order! = order of the singularity (order=index-1).
  The matrix pencil \verb!Fs=s*E-A! should be invertible.
  For a index-zero pencil, \verb!Pi, Di,...! are zero and \verb!Si=inv(E)!.
  For a index-one pencil (order=0),\verb!Di! =0.
  For higher-index pencils, the terms \verb! -s^2 Di(2), -s^3    Di(3),... ! are given by:\verb! Di(2)=Di*A*Di!, \verb! Di(3)=Di*A*Di*A*Di! (up
  to \verb!Di(order)!).
\end{mandescription}
%-- section-Remark
\paragraph{Remark}
Experimental version: troubles when bad conditioning of \verb!so*E-A!
%--example 
\begin{examples}
  \begin{Verbatim}
    F=randpencil([],[1,2],[1,2,3],[]);
    F=rand(6,6)*F*rand(6,6);[E,A]=pen2ea(F);
    [Si,Pi,Di]=penlaur(F);
    [Bfs,Bis,chis]=glever(F);
    norm(coeff(Bis,1)-Di,1)
  \end{Verbatim}
\end{examples}
%-- see also
\begin{manseealso}
  \manlink{glever}{glever} \manlink{pencan}{pencan} \manlink{rowshuff}{rowshuff}  
\end{manseealso}
%-- Author
\begin{authors}
  Fran�ois Delebecque
\end{authors}
