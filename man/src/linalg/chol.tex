% -*- mode: latex -*-

\mansection{chol}
\begin{mandesc}
  \short{chol}{Cholesky factorization}
\end{mandesc}

% -- Calling sequence section
\begin{calling_sequence}
\begin{verbatim}
[R]=chol(A)
[R,p]=chol(A)
\end{verbatim}
\end{calling_sequence}
% -- Parameters
\begin{parameters}
  \begin{varlist}
    \vname{A}: a symmetric positive definite real or complex matrix. 
    \vname{R}: an upper triangular matrix \verb+R+ such that \verb+R'*R = X+.
    \vname{p}: an integer (0 or the order of the leading minor which is 
not positive definite).
  \end{varlist}
\end{parameters}

\begin{mandescription}
This function computes the Cholesky factorization of a symmetric positive 
definite real or complex matrix, or a partial factorization if the matrix is not 
positive definite.

When called as \verb+R = chol(A)+, if \verb+A+ is positive definite, an upper triangular 
matrix \verb+R+ such that \verb+R'*R = X+ is returned, if not an error exception is 
raised.

When called as \verb+[R,p]= chol(A)+ if \verb+A+ is positive definite an upper triangular 
matrix \verb+R+ such that \verb+R'*R = X+ is returned and \verb+p+ is set to zero. 
If \verb+A+ is not positive definite the order of the leading minor which is 
not positive definite is returned in \verb+p+ and the returned upper triangular matrix \verb+R+ is 
of size \verb+qxq+ where \verb+q=p-1+ and is such that \verb+R'*R=A(1:q,1:q)+.

Note that \verb+chol(A)+ only uses the the diagonal and upper triangle of $A$. 
Cholesky decomposition is based on the Lapack routines \verb+dpotrf+ for real 
matrices and \verb+zpotrf+ for the complex case.
\end{mandescription}

\begin{examples}
\paragraph{example 1} with positive definite matrices 
  \begin{program}\HCode{A=rand(5,5);A=A*A';\Hnewline
      R = chol(A);\Hnewline
      norm(R'*R-A)\Hnewline
      A=rand(5,5)+\%i*rand(5,5);\Hnewline
      A=A*A';\Hnewline
      R=chol(X);\Hnewline
      norm(R'*R-X)}
  \end{program}
  \paragraph{example 2} Example with a not positive definite matrix
  \begin{program}\HCode{A=rand(5,3);A=A*A';\Hnewline
      [R,p] = chol(A);\Hnewline
      q=p-1;norm(R'*R-A(1:q,1:q))}
  \end{program}
\end{examples}

\begin{manseealso}
  \manlink{qr}{qr} 
\end{manseealso}

% -- Authors
\begin{authors}
   interface : Jean-Philippe Chancelier and Bruno Pincon. Internally lapack routines \verb+dpotrf+ for real matrices 
   and \verb+zpotrf+ for the complex case
\end{authors}


 
 
  
