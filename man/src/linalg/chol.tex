% -*- mode: latex -*-
%% Nsp
%% Copyright (C) 1998-2015 Jean-Philippe Chancelier Enpc/Cermics, Bruno Pin�on Esial/Iecn
%%
%% This library is free software; you can redistribute it and/or
%% modify it under the terms of the GNU General Public
%% License as published by the Free Software Foundation; either
%% version 2 of the License, or (at your option) any later version.
%%
%% This library is distributed in the hope that it will be useful,
%% but WITHOUT ANY WARRANTY; without even the implied warranty of
%% MERCHANTABILITY or FITNESS FOR A PARTICULAR PURPOSE.  See the GNU
%% General Public License for more details.
%%
%% You should have received a copy of the GNU General Public
%% License along with this library; if not, write to the
%% Free Software Foundation, Inc., 59 Temple Place - Suite 330,
%% Boston, MA 02111-1307, USA.
%%
 

\mansection{chol}
\begin{mandesc}
  \short{chol}{Cholesky factorization}
\end{mandesc}

% -- Calling sequence section
\begin{calling_sequence}
\begin{verbatim}
[R]=chol(A)
[R,p]=chol(A)
\end{verbatim}
\end{calling_sequence}
% -- Parameters
\begin{parameters}
  \begin{varlist}
    \vname{A}: a symmetric positive definite real or complex matrix. 
    \vname{R}: an upper triangular matrix \verb+R+ such that \verb+R'*R = X+.
    \vname{p}: an integer (0 or the order of the leading minor which is 
not positive definite).
  \end{varlist}
\end{parameters}

\begin{mandescription}
This function computes the Cholesky factorization of a (full) symmetric positive 
definite real or complex matrix, or a partial factorization if the matrix is not 
positive definite (for Cholesky factorization of sparse matrices see  
\manlink{cholmod\_create}{chomod_create}).

When called as \verb+R = chol(A)+, if \verb+A+ is positive definite, an upper triangular 
matrix \verb+R+ such that \verb+R'*R = X+ is returned, if not an error exception is 
raised.

When called as \verb+[R,p]= chol(A)+ if \verb+A+ is positive definite an upper triangular 
matrix \verb+R+ such that \verb+R'*R = X+ is returned and \verb+p+ is set to zero. 
If \verb+A+ is not positive definite the order of the leading minor which is 
not positive definite is returned in \verb+p+ and the returned upper triangular matrix \verb+R+ is 
of size \verb+qxq+ where \verb+q=p-1+ and is such that \verb+R'*R=A(1:q,1:q)+.

Note that \verb+chol(A)+ only uses the the diagonal and upper triangle of $A$. 
Cholesky decomposition is based on the Lapack routines \verb+dpotrf+ for real 
matrices and \verb+zpotrf+ for the complex case.
\end{mandescription}

\begin{examples}
\paragraph{example 1} with positive definite matrices 
\begin{mintednsp}{nsp}
A=rand(5,5);A=A*A';
R = chol(A);
norm(R'*R-A)
A=rand(5,5)+%i*rand(5,5);
A=A*A';
R=chol(A);
norm(R'*R-A)
\end{mintednsp}

\paragraph{example 2} Example with a not positive definite matrix
\begin{mintednsp}{nsp}
A=rand(5,3);A=A*A';
[R,p] = chol(A);
q=p-1;norm(R'*R-A(1:q,1:q))
\end{mintednsp}
\end{examples}

\begin{manseealso}
  \manlink{cholmod\_create}{chomod_create}, \manlink{lu}{lu},\manlink{qr}{qr},  \manlink{solve}{solve} 
\end{manseealso}

% -- Authors
\begin{authors}
   interface: Jean-Philippe Chancelier and Bruno Pincon. Internally lapack routines \verb+dpotrf+ for real matrices 
   and \verb+zpotrf+ for the complex case.
\end{authors}


 
 
  
