% -*- mode: latex -*-
%% Scilab ( http://www.scilab.org/ ) - This file is part of Scilab
%% Copyright (C) 1987-2016 - F. Delebecque
%%
%% This program is free software; you can redistribute it and/or modify
%% it under the terms of the GNU General Public License as published by
%% the Free Software Foundation; either version 2 of the License, or
%% (at your option) any later version.
%%
%% This program is distributed in the hope that it will be useful,
%% but WITHOUT ANY WARRANTY; without even the implied warranty of
%% MERCHANTABILITY or FITNESS FOR A PARTICULAR PURPOSE.  See the
%% GNU General Public License for more details.
%%
%% You should have received a copy of the GNU General Public License
%% along with this program; if not, write to the Free Software
%% Foundation, Inc., 59 Temple Place, Suite 330, Boston, MA  02111-1307  USA
%%                                                                                                

\mansection{quaskro}
\begin{mandesc}
  \short{quaskro}{quasi-Kronecker form} \\ % 
\end{mandesc}
%\index{quaskro}\label{quaskro}
%-- Calling sequence section
\begin{calling_sequence}
\begin{verbatim}
  [Q,Z,Qd,Zd,numbeps,numbeta]=quaskro(F)  
  [Q,Z,Qd,Zd,numbeps,numbeta]=quaskro(E,A)  
  [Q,Z,Qd,Zd,numbeps,numbeta]=quaskro(F,tol)  
  [Q,Z,Qd,Zd,numbeps,numbeta]=quaskro(E,A,tol)  
\end{verbatim}
\end{calling_sequence}
%-- Parameters
\begin{parameters}
  \begin{varlist}
    \vname{F}: real matrix pencil \verb!F=s*E-A!  (\verb!s=poly(0,'s')!)
    \vname{E,A}: two real matrices of same dimensions
    \vname{tol}: a real number (tolerance, default value=1.d-10)
    \vname{Q,Z}: two square orthogonal matrices
    \vname{Qd,Zd}: two vectors of integers
    \vname{numbeps}: vector of integers
  \end{varlist}
\end{parameters}
\begin{mandescription}
  Quasi-Kronecker form of matrix pencil: \verb!quaskro! computes two
  orthogonal matrices \verb!Q, Z! which put the pencil \verb!F=s*E -A! into
  upper-triangular form:

  \begin{equation*}
    Q(s E -A)Z = \left(
    \begin{array}{cccc} 
      sE(eps)-A(eps)  & \cdot & \cdot   \\
      0               & sE(inf)-A(inf) & \cdot  \\
      0               & 0              & sE(r)-A(r) 
    \end{array}
    \right) 
  \end{equation*}
% \begin{verbatim}
%   | sE(eps)-A(eps) |        X       |      X     |
%   |----------------|----------------|------------|
%   |        O       | sE(inf)-A(inf) |      X     |
%   Q(sE-A)Z = |=================================|============|
%   |                                 |            |
%   |                O                | sE(r)-A(r) |
% \end{verbatim}
The dimensions of the blocks are given by:\verb!eps=Qd(1) x Zd(1)!, \verb!inf=Qd(2) x Zd(2)!,
\verb!r = Qd(3) x Zd(3)!
The \verb!inf! block contains the infinite modes of
the pencil.
The \verb!f! block contains the finite modes of
the pencil
The structure of epsilon blocks are given by:\verb!numbeps(1)! = \verb!#! of eps blocks of size 0 x 1\verb!numbeps(2)! = \verb!#! of eps blocks of size 1 x 2\verb!numbeps(3)! = \verb!#! of eps blocks of size 2 x 3     etc...
The complete (four blocks) Kronecker form is given by
the function \verb!kroneck! which calls \verb!quaskro! on
the (pertransposed) pencil \verb!sE(r)-A(r)!.
The code is taken from T. Beelen
\end{mandescription}
%-- see also
\begin{manseealso}
  \manlink{kroneck}{kroneck} \manlink{gschur}{gschur} \manlink{gspec}{gspec}  
\end{manseealso}
\begin{authors}
  Fran�ois Delebecque
\end{authors}
