% -*- mode: latex -*-
%% Scilab ( http://www.scilab.org/ ) - This file is part of Scilab
%% Copyright (C) 1987-2016 - F. Delebecque
%%
%% This program is free software; you can redistribute it and/or modify
%% it under the terms of the GNU General Public License as published by
%% the Free Software Foundation; either version 2 of the License, or
%% (at your option) any later version.
%%
%% This program is distributed in the hope that it will be useful,
%% but WITHOUT ANY WARRANTY; without even the implied warranty of
%% MERCHANTABILITY or FITNESS FOR A PARTICULAR PURPOSE.  See the
%% GNU General Public License for more details.
%%
%% You should have received a copy of the GNU General Public License
%% along with this program; if not, write to the Free Software
%% Foundation, Inc., 59 Temple Place, Suite 330, Boston, MA  02111-1307  USA
%%                                                                                                

\mansection{householder}
\begin{mandesc}
  \short{householder}{Householder orthogonal reflexion matrix} \\ % 
\end{mandesc}
%\index{householder}\label{householder}
%-- Calling sequence section
\begin{calling_sequence}
\begin{verbatim}
  u=householder(v [,w])  
\end{verbatim}
\end{calling_sequence}
%-- Parameters
\begin{parameters}
  \begin{varlist}
    \vname{v}: real or complex column vector
    \vname{w}: real or complex column vector with same size as \verb!v!. Default value is \verb!eye(v)!
    \vname{u}: real or complex column vector
  \end{varlist}
\end{parameters}
\begin{mandescription}
  given 2 column vectors \verb!v!, \verb! w! of same size, \verb!householder(v,w)! returns a unitary 
  column vector \verb!u!, such that \verb! (eye()-2*u*u')*v! is proportional to \verb!w!.
  \verb!(eye()-2*u*u')! is the orthogonal Householder reflexion matrix .\verb!w! default value is \verb! eye(v)!. In this case vector \verb! (eye()-2*u*u')*v! is the 
  vector  \verb! eye(v)*norm(v)!.
\end{mandescription}
%-- see also
\begin{manseealso}
  \manlink{qr}{qr} \manlink{givens}{givens}  
\end{manseealso}


\begin{authors}
  Fran�ois Delebecque
\end{authors}
