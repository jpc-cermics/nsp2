% -*- mode: latex -*-
%% Nsp
%% Copyright (C) 2010-2015 Bruno Pin�on Esial/Iecn
%%
%% This library is free software; you can redistribute it and/or
%% modify it under the terms of the GNU General Public
%% License as published by the Free Software Foundation; either
%% version 2 of the License, or (at your option) any later version.
%%
%% This library is distributed in the hope that it will be useful,
%% but WITHOUT ANY WARRANTY; without even the implied warranty of
%% MERCHANTABILITY or FITNESS FOR A PARTICULAR PURPOSE.  See the GNU
%% General Public License for more details.
%%
%% You should have received a copy of the GNU General Public
%% License along with this library; if not, write to the
%% Free Software Foundation, Inc., 59 Temple Place - Suite 330,
%% Boston, MA 02111-1307, USA.
%%
 

\mansection{lsmr}
\begin{mandesc}
  \short{lsmr}{iterative solver for least-squares problems}
\end{mandesc}
% -- Calling sequence section
\begin{calling_sequence}
\begin{verbatim}
 x = lsmr(A,b)
[x, istop, itn, normr, normAr, normA, condA, normx]...
   = lsmr(A, b, lambda=, atol=, btol=, conlim=, itnlim=, localSize=, show=)
\end{verbatim}
\end{calling_sequence}
% -- Parameters
\begin{parameters}
  \begin{varlist}
    \vname{A}: real sparse (or full) matrix (with dim $m \times n$) or
               function (see explanation here after)
    \vname{b}: real vector (dim $m \times 1$) 
    \vname{lambda=}: optional (default is 0), real scalar.
    \vname{atol=}: optional (default is 1e-6), positive real scalar.
    \vname{btol=}: optional (default is 1e-6), positive real scalar.
    \vname{conlim=}: optional (default is 1e8), positive real scalar.
    \vname{itnlim=}: optional, max number of iterations (default is $min(m,n)$).
    \vname{localSize=}: optional, non negative integer scalar (default is 0).
    \vname{show=}: optional, boolean scalar (default is \verb+%f+). When \verb+%t+
                   prints an iteration log.
    \vname{x}: solution vector (dim $n \times 1$)
    \vname{istop}: flag giving the reason for termination
    \vname{itn}: number of iterations
    \vname{normr}: estimate of the residual norm ($\|Ax-b\|_2$)
    \vname{normAr}: estimate of the residual for the normal equation ($\|A^{\top}(Ax-b)\|_2$)
    \vname{normA}: estimate of $\| A \|_F$ the Frobenius norm of $A$
    \vname{condA}: estimate of  $\| A \|_F \| A^+ \|_F$ the condition number of $A$ ($A^+$
                   is the pseudo-inverse of $A$).
    \vname{normx}: euclidian norm of $x$ ($\|x\|_2$).
  \end{varlist}
\end{parameters}

\begin{mandescription}
  This function solves the linear system $Ax = b$ or solves 
  the least-squares problem:
  $$
  \min \| A x - b \|^2_2. 
  $$
  if the system appears to be inconsistent (which means that there is no 
  $x$ such that $A x=b$ ($b \notin Range(A)$)). $A$ is a rectangular matrix of 
  dimension $m \times n$, where all cases are allowed: $m=n$, $m>n$, or $m<n$. 
  The matrix $A$ may be dense or sparse (usually sparse). Note that in case
  $A$ is not full rank (there is an infinity of solutions to the least square
  problem) lsmr should provide the minimum norm solution.
  
  In place of $A$ you can provide a function (says $Afun$) which takes two 
  arguments such that $Afun(x,1)$ returns $A x$ and   $Afun(x,2)$ 
  returns $A^{\top} x$.
  
  When $\lambda$ is not $0$ lsmr solves the regularized least-squares problem:
  $$
  \min \| A x - b \|^2_2 + \lambda^2 \| x \|^2_2 
  $$
  The lsmr method is described in this \href{http://arxiv.org/abs/1006.0758}{paper}.
  
  \itemdesc{convergence and stopping criteria, istop output paramter}

  \begin{itemize}
  \item lsmr continues iterations until a certain backward error estimate is smaller 
    than some quantity depending on atol and btol. In the following we denote $r = A x - b$ 
    the residual. If $A x = b$ seems to be consistent, lsmr terminates when:
    $$
    \| r \|_2  \le  atol \| A \|_F \| x \|_2 + btol \| b \|_2
    $$
    otherwise, lsmr terminates when;
    $$
    \| A^{\top}r \|_2  \le  atol \| A \|_F \| r \|_2
    $$
    (note that these 2 tests use the estimates of the quantities $\|r\|_2$, $\| A \|_F$ 
    and  $\| A^{\top}r \|_2$).

    If both tolerances are 1.0e-6 (say), the final residual norm should be
    accurate to about 6 digits. (The final $x$ will usually have fewer
    correct digits, depending on $cond(A)$ and the size of $\lambda$.)
    Ideally, $atol$ and $btol$ should be estimates of the relative error in the
    entries of $A$ and $b$ respectively.  For example, if the entries of $A$
    have 7 correct digits, set $atol = 1e-7$. This prevents the algorithm
    from doing unnecessary work beyond the uncertainty of the input data.

  \item lsmr terminates if the estimate of $cond(A)$ exceeds $conlim$.
    For compatible systems $Ax = b$, $conlim$ could be as large as $1.0e+12$ (say).  
    For least-squares problems, $conlim$ should be less than $1.0e+8$. 
    Maximum precision can be obtained by setting $atol = 0$, $btol = 0$, $conlim=0$, 
    but the number of iterations may then be excessive.

  \item lsmr terminates if the number of iterations reaches $itnlim$.
    Note that for ill-conditioned systems, a larger value that the 
    default one ($\min(m,n)$) may be needed.
  \end{itemize}

  \noindent
  istop possible values:
  \begin{description}
  \item[istop=0] means that $x=0$ is a solution.
  \item[istop=1] means $x$ is an approximate solution to $Ax = b$ according to $atol$ and $btol$.
  \item[istop=2] means $x$ approximately solves the least-squares problem according to $atol$.
  \item[istop=3] means $cond(A)$ seems to be greater than $conlim$.
  \item[istop=4] is the same as 1 with $atol$ and $btol$ equal to $\epsilon_m$ (the epsilon machine \verb+%eps+).
  \item[istop=5] is the same as 2 with $atol = \epsilon_m$.
  \item[istop=6] is the same as 3 with $conlim = 1/\epsilon_m$.
  \item[istop=7]  means that $itn$ reached $itnlim$ before the other stopping conditions were satisfied
  \end{description}


  \itemdesc{reorthogonalization (localSize parameter)}

  If the optional localSize parameter is not 0, the last localSize v_k's
  (v-vectors generated by Golub-Kahan bidiagonalization) are reorthogonalized.
  When localSize is Inf lsmr performs reothogonalization on all v_k's. 
  Reorthgonalizing only u_k or both u_k and v_k are not an option here. 
  See the \href{http://arxiv.org/abs/1006.0758}{paper} for details.

\end{mandescription}

\begin{examples}
  \paragraph{example 1} A first basic test
  \begin{Verbatim}
    A = rand(40,20);
    b = ones(40,1);
    [x, istop, itn, normr, normAr, normA, condA, normx] = lsmr(A,b,show=%f);
    // compute the ``exact solution'' with backslash (QR factorization here)
    xx = A\b;
    // compute relative difference 
    norm(x-xx)/norm(x)
    
    // compute solution again with lsmr but with a more stringent atol parameter
    // (rmk: the max number of iterations should be increased too)
    [x, istop, itn, normr, normAr, normA, condA, normx] = lsmr(A,b,show=%f,atol=1e-12,itnlim=30);
    // now the relative difference should be smaller
    norm(x-xx)/norm(x)
    
    // now we modify A such that it becomes rank deficient
    A(:,15:20) = A(:,1:6);
    // lsmr should find the minimum norm solution
    [x, istop, itn, normr, normAr, normA, condA, normx] = lsmr(A,b,show=%f,atol=1e-12,itnlim=30);
    // compute the minimum norm solution with backslash
    xx = A\b;
    // compute relative difference
    norm(x-xx)/norm(x)
  \end{Verbatim}
  
  \paragraph{example 2} Another test, solve a square sparse linear system
  \begin{Verbatim}
    A = 2*speye(1e4,1e4) + sprand(1e4,1e4,1e-3);
    b = ones(1e4,1);
    [x, istop, itn, normr, normAr, normA, condA, normx] = lsmr(A,b,show=%f);
    // compute relative residual
    norm(A*x-b)/norm(b)
    
    // use more stringent atol and btol
    [x, istop, itn, normr, normAr, normA, condA, normx] = lsmr(A,b,show=%f,atol=1e-12,btol=1e-12);
    // compute relative residual
    norm(A*x-b)/norm(b)
\end{Verbatim}

\end{examples}

\begin{manseealso}
  \manlink{Cholmod}{Cholmod}, \manlink{svd}{svd}, \manlink{solve}{solve}.
\end{manseealso}

% -- Authors
\begin{authors}
  David Chin-lung Fong, Michael Saunders (adapted by Bruno Pincon for nsp).
\end{authors}
