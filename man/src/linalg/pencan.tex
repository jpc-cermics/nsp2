% -*- mode: latex -*-
%% Scilab ( http://www.scilab.org/ ) - This file is part of Scilab
%% Copyright (C) 1987-2016 - F. Delebecque
%%
%% This program is free software; you can redistribute it and/or modify
%% it under the terms of the GNU General Public License as published by
%% the Free Software Foundation; either version 2 of the License, or
%% (at your option) any later version.
%%
%% This program is distributed in the hope that it will be useful,
%% but WITHOUT ANY WARRANTY; without even the implied warranty of
%% MERCHANTABILITY or FITNESS FOR A PARTICULAR PURPOSE.  See the
%% GNU General Public License for more details.
%%
%% You should have received a copy of the GNU General Public License
%% along with this program; if not, write to the Free Software
%% Foundation, Inc., 59 Temple Place, Suite 330, Boston, MA  02111-1307  USA
%%                                                                                                

\mansection{pencan}
\begin{mandesc}
  \short{pencan}{canonical form of matrix pencil} \\ % 
\end{mandesc}
%\index{pencan}\label{pencan}
%-- Calling sequence section
\begin{calling_sequence}
\begin{verbatim}
  [Q,M,i1]=pencan(Fs)  
  [Q,M,i1]=pencan(E,A)  
\end{verbatim}
\end{calling_sequence}
%-- Parameters
\begin{parameters}
  \begin{varlist}
    \vname{Fs}: a regular pencil \verb!s*E-A!
    \vname{E,A}: two real square matrices
    \vname{Q,M}: two non-singular real matrices
    \vname{i1}: integer
  \end{varlist}
\end{parameters}
\begin{mandescription}
  Given the regular pencil \verb!Fs=s*E-A!, \verb!pencan! returns matrices \verb!Q! 
  and \verb!M!
  such than \verb!M*(s*E-A)*Q! is in "canonical" form.\verb!M*E*Q! is a block matrix
\begin{verbatim}
  [I,0;
    0,N]
\end{verbatim}
with \verb!N! nilpotent and \verb!i1! = size of the \verb!I! matrix above.\verb!M*A*Q! is a block matrix:
\begin{verbatim}
  [Ar,0;
    0,I]
\end{verbatim}
\end{mandescription}
%--example 
\begin{examples}
  \begin{Verbatim}
    F=randpencil([],[1,2],[1,2,3],[]);
    F=rand(6,6)*F*rand(6,6);
    [Q,M,i1]=pencan(F);
    W=clean(M*F*Q)
    roots(det(W(1:i1,1:i1)))
    det(W($-2:$,$-2:$))
  \end{Verbatim}
\end{examples}
%-- see also
\begin{manseealso}
  \manlink{glever}{glever} \manlink{penlaur}{penlaur} \manlink{rowshuff}{rowshuff}  
\end{manseealso}
%-- Author
\begin{authors}
  Fran�ois Delebecque
\end{authors}
