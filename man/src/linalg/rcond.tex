% -*- mode: latex -*-
%% Nsp
%% Copyright (C) 1998-2015 Jean-Philippe Chancelier Enpc/Cermics, Bruno Pin�on Esial/Iecn
%%
%% This library is free software; you can redistribute it and/or
%% modify it under the terms of the GNU General Public
%% License as published by the Free Software Foundation; either
%% version 2 of the License, or (at your option) any later version.
%%
%% This library is distributed in the hope that it will be useful,
%% but WITHOUT ANY WARRANTY; without even the implied warranty of
%% MERCHANTABILITY or FITNESS FOR A PARTICULAR PURPOSE.  See the GNU
%% General Public License for more details.
%%
%% You should have received a copy of the GNU General Public
%% License along with this library; if not, write to the
%% Free Software Foundation, Inc., 59 Temple Place - Suite 330,
%% Boston, MA 02111-1307, USA.
%%
 

\mansection{rcond}
\begin{mandesc}
  \short{rcond}{estimation of the reciprocal condition in 1-norm}
\end{mandesc}

% -- Calling sequence section
\begin{calling_sequence}
\begin{verbatim}
rc = rcond(A)
\end{verbatim}
\end{calling_sequence}
% -- Parameters
\begin{parameters}
  \begin{varlist}
    \vname{A}:  square numerical matrix
    \vname{rc}: real scalar (should be in [0,1])
  \end{varlist}
\end{parameters}

\begin{mandescription}
This function computes an estimation of:
$$
     \frac{1}{\kappa_1(A)} =  \frac{1}{||A||_1 ||A^{-1}||_1}
$$
(see \manlink{norm}{norm} for definition of matrix 1-norm).
This can be useful for linear system solving $A x = b$. For instance
if $x'$ is an approximate solution of the linear system ($x$ being 
the exact solution), denoting $r = b - Ax'$ the residual we have 
the inequality:
$$
\frac{||x' - x||_1}{||x||_1} \le \kappa_1(A) \frac{||r||_1}{||b||_1}
$$

A matrix with a reciprocal condition number of the order of
epsilon machine $\epsilon_m$ (aka as the unit relative rounding 
error, \verb+%eps+ in nsp) is considered as badly conditionned 
(for the underlying floating point arithmetic). 

When using \verb+x = A\b+ for solving a linear system $Ax=b$, this
estimation of the reciprocal condition number is computed. When
$rc < n \epsilon_m$ ($n$ being the matrix order) the matrix is
considered numerically singular and a \manlink{qr}{qr}
factorisation with rank estimation is used to get a minimum norm
solution. Nevertheless {\bf this test is not done for a triangular 
matrix} so using \verb+rcond+ could be interesting in this case
to detect potential numerical problem. See example 2.
\end{mandescription}

\begin{examples}
\paragraph{example 1}
\begin{Verbatim}
A=rand(51,51);
rc=rcond(A)
// make a singular system:
A(:,5) = 0.3*A(:,1) + 0.9*A(:,21) - 2*A(:,41);
rc=rcond(A)
// solve a Ax=b linear system (the bad conditionning is detected)
b = rand(51,1);
x = A\b;
\end{Verbatim}

\paragraph{example 2} a badly conditionned triangular matrix
\begin{Verbatim}
// make such a triangular matrix
T = triu(rand(500,500));
b = rand(500,1);
// solve T x = b with backslash: no warning occurs
x = T\b;
// computing rcond
rc = rcond(T)
// computing the relative residual
norm(T*x-b)/norm(b)
// solve in least square meaning with rank detection
x = lsq(T,b);
// the relative residual is now better
norm(T*x-b)/norm(b)
\end{Verbatim}
\end{examples}

\begin{manseealso}
  \manlink{lu}{lu}, \manlink{qr}{qr}, \manlink{norm}{norm}  
\end{manseealso}

% -- Authors
\begin{authors}
   interface : Bruno Pincon and Jean-Philippe Chancelier. Internally
   lapack routines dgecon, zgecon, dtrcon, ztrcon could be used.
\end{authors}
