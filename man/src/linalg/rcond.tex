% -*- mode: latex -*-

\mansection{rcond}
\begin{mandesc}
  \short{rcond}{estimation of the reciprocal condition in 1-norm}
\end{mandesc}

% -- Calling sequence section
\begin{calling_sequence}
\begin{verbatim}
rc = rcond(A)
\end{verbatim}
\end{calling_sequence}
% -- Parameters
\begin{parameters}
  \begin{varlist}
    \vname{A}:  square numerical matrix
    \vname{rc}: real scalar (should be in [0,1])
  \end{varlist}
\end{parameters}

\begin{mandescription}
This function computes an estimation of:
$$
     \frac{1}{\kappa_1(A)} =  \frac{1}{||A||_1 ||A^{-1}||_1}
$$
(see \manlink{norm}{norm} for definition of matrix 1-norm).
This can be useful for linear system solving $A x = b$. For instance
if $x'$ is an approximate solution of the linear system ($x$ being 
the exact solution), denoting $r = b - Ax'$ the residual we have 
the inequality:
$$
\frac{||x' - x||_1}{||x||_1} \le \kappa_1(A) \frac{||r||_1}{||b||_1}
$$

A matrix with a reciprocal condition number of the order of
epsilon machine $\epsilon_m$ (aka as the unit relative rounding 
error, \verb+%eps+ in nsp) is considered as badly conditionned 
(for the underlying floating point arithmetic). 

When using \verb+x = A\b+ for solving a linear system $Ax=b$, this
estimation of the reciprocal condition number is computed. When
$rc < n \epsilon_m$ ($n$ being the matrix order) the matrix is
considered numerically singular and a \manlink{qr}{qr}
factorisation with rank estimation is used to get a minimum norm
solution. Nevertheless {\bf this test is not done for a triangular 
matrix} so using \verb+rcond+ could be interesting in this case
to detect potential numerical problem. See example 2.
\end{mandescription}

\begin{examples}
\paragraph{example 1}
  \begin{program}\HCode{A=rand(51,51);\Hnewline
    rc=rcond(A)\Hnewline
    // make a singular system:\Hnewline
    A(:,5) = 0.3*A(:,1) + 0.9*A(:,21) - 2*A(:,41);\Hnewline
    rc=rcond(A)\Hnewline
    // solve a Ax=b linear system (the bad conditionning is detected)\Hnewline
    b = rand(51,1);\Hnewline
    x = A}\verb+\+\HCode{b;}
  \end{program}

\paragraph{example 2} a badly conditionned triangular matrix
  \begin{program}\HCode{// make such a triangular matrix\Hnewline
    T = triu(rand(500,500));\Hnewline
    b = rand(500,1);\Hnewline
    // solve T x = b with backslash: no warning occurs\Hnewline
    x = T}\verb+\+\HCode{b;\Hnewline
    // computing rcond\Hnewline
    rc = rcond(T)\Hnewline
    // computing the relative residual\Hnewline
    norm(T*x-b)/norm(b)\Hnewline
    // solve in least square meaning with rank detection\Hnewline
    x = lsq(T,b);\Hnewline
    // the relative residual is now better\Hnewline
    norm(T*x-b)/norm(b)}
  \end{program}
\end{examples}

\begin{manseealso}
  \manlink{lu}{lu}, \manlink{qr}{qr}, \manlink{norm}{norm}  
\end{manseealso}

% -- Authors
\begin{authors}
   interface : Bruno Pincon and Jean-Philippe Chancelier. Internally
   lapack routines dgecon, zgecon, dtrcon, ztrcon could be used.
\end{authors}
