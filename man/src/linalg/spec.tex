% -*- mode: latex -*-

\mansection{spec}
\begin{mandesc}
  \short{spec}{matrix spectrum, eigenvalues and eigenvectors}
\end{mandesc}

% -- Calling sequence section
\begin{calling_sequence}
\begin{verbatim}
[lambda] = spec(A)
[lambda, V] = spec(A)
\end{verbatim}
\end{calling_sequence}
% -- Parameters
\begin{parameters}
  \begin{varlist}
    \vname{A}:  numerical square matrix
    \vname{lambda}: column vector (eigenvalues)
    \vname{V}: matrix (eigenvectors)
  \end{varlist}
\end{parameters}

\begin{mandescription}
This function computes the eigenvalues of a matrix $A$, and if
needed the associated eigenvectors. The jth column of $V$ 
(\verb+V(:,j)+) is the eigenvector associated to the 
$\lambda_j$ eigenvalue (\verb+lambda(j)+) : $A v^j = \lambda_j v^j$.

\itemdesc{remarks}
\begin{itemize}
\item In case of a defective eigenvalue (an eigenvalue of
      algebraic multiplicity says $k (>1)$ and geometric multiplicity
      $k'$ ($1 \le k' < k$) the associated eigenvectors are not
      linearly independant (the dimension of the associated 
      eigenvectors space must be $k'$ at least numerically for
      a not too small tolerance).
\item When the matrix $A$ is symmetric (real case) or hermitian
      (complex case) the eigenvalues are real and the eigenvectors 
      form a orthonormal basis. This case is simple and computations
      of eigenvalues and eigenvectors are faster. To detect the
      symmetric case the test  $a_{ij} = a_{ji}, \forall 1\le i < j
      \le n$ is done ($a_{ij} = \bar{a}_{ji}$ in the complex case).
      If your matrix $A$ must be symmetric but lose this property because
      of floating point errors you can re-symmetrise it with 
      \verb-A = 0.5*(A+A')-
\end{itemize}

\end{mandescription}

\begin{examples}
\paragraph{example 1} a simple example (we form a matrix with
      eigenvalues 1,2,3,4,5):
  \begin{program}\HCode{X=rand(5,5);\Hnewline
    vp = [1;2;3;4;5];\Hnewline
    A = X*diag(vp)*inv(X);\Hnewline
    // remark: span(X(:,j)) is the eigenspace associated to the eigenvalue j\Hnewline
    [lambda,V]=spec(A)\Hnewline
    A*V - V*diag(lambda)\Hnewline
    // reorder V such that V(:,j) is the eigenvector associated to the eigenvalue j\Hnewline
    V(:,round(lambda))=V;\Hnewline
    V./X  // each colomn must have the same coefficients\Hnewline}
  \end{program}

\paragraph{example 2} hyper-plane symetry:
  \begin{program}\HCode{u = rand(4,1); u = u/norm(u);\Hnewline
    // A is the matrix of the symetry around the hyper-plane orthogonal to u\Hnewline
    // (aka as an hyper-plane reflection or an Houselholder transformation)\Hnewline
    A = eye(4,4) - 2*u*u';\Hnewline
    // -1 must be an eigenvalue of multiplity 1\Hnewline
    //  1 must be an eigenvalue of multiplity 3\Hnewline
    [lambda,V]=spec(A)\Hnewline
    // the eigenspace associated to -1 must be span(u):\Hnewline
    V(:,1)./u}  
  \end{program}

\paragraph{example 3} a rotation matrix in the plane (here eigenvalues
    are necessary complex):
  \begin{program}\HCode{theta =}\verb+%pi+\HCode{/3; ct=cos(theta); st=sin(theta);\Hnewline
    A=[ct,-st;st,ct];\Hnewline
    [lambda,V]=spec(A)}
  \end{program}

\paragraph{example 4} a defective case (1 is an eigenvalue of
    multiplicity 3 and the associated eigenspace must be span([1;0;0]):
  \begin{program}\HCode{A=[1,1,0;0,1,1;0,0,1];\Hnewline
    [lambda,V]=spec(A)\Hnewline
    orth(V)}
  \end{program}
\end{examples}

\begin{manseealso}
  \manlink{qr}{qr}  
\end{manseealso}

% -- Authors
\begin{authors}
   interface : Bruno Pincon and Jean-Philippe Chancelier. Internally lapack routines dsyevr and
   zheevr are used in the real symetric/complex hermitian cases, 
   and dgeev and zgeev otherwise.
\end{authors}
