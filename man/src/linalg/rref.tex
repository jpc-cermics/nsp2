% -*- mode: latex -*-
%% Scilab ( http://www.scilab.org/ ) - This file is part of Scilab
%% Copyright (C) 1987-2016 - F. Delebecque
%%
%% This program is free software; you can redistribute it and/or modify
%% it under the terms of the GNU General Public License as published by
%% the Free Software Foundation; either version 2 of the License, or
%% (at your option) any later version.
%%
%% This program is distributed in the hope that it will be useful,
%% but WITHOUT ANY WARRANTY; without even the implied warranty of
%% MERCHANTABILITY or FITNESS FOR A PARTICULAR PURPOSE.  See the
%% GNU General Public License for more details.
%%
%% You should have received a copy of the GNU General Public License
%% along with this program; if not, write to the Free Software
%% Foundation, Inc., 59 Temple Place, Suite 330, Boston, MA  02111-1307  USA
%%                                                                                                

\mansection{rref}
\begin{mandesc}
  \short{rref}{computes  matrix row echelon form by lu transformations} \\ % 
\end{mandesc}
%\index{rref}\label{rref}
%-- Calling sequence section
\begin{calling_sequence}
\begin{verbatim}
  R=rref(A)  
\end{verbatim}
\end{calling_sequence}
%-- Parameters
\begin{parameters}
  \begin{varlist}
    \vname{A}:  m x n matrix with scalar entries
    \vname{R}:  m x n matrix,row echelon form of a
  \end{varlist}
\end{parameters}
\begin{mandescription}
  \verb!rref! computes the row echelon form of the given matrix by left lu
  decomposition. If ones need the transformation used just call
  \verb!X=rref([A,eye(m,m)])! the row echelon form \verb!R! is \verb!X(:,1:n)! and
  the left transformation \verb!L! is given by \verb!X(:,n+1:n+m)! such as \verb!L*A=R!
\end{mandescription}
%--example 
\begin{examples}
  \begin{Verbatim}
    A=[1 2;3 4;5 6];
    X=rref([A,eye(3,3)]);
    R=X(:,1:2)
    L=X(:,3:5);L*A
  \end{Verbatim}
\end{examples}
%-- see also
\begin{manseealso}
  \manlink{lu}{lu} \manlink{qr}{qr}  
\end{manseealso}
\begin{authors}
  Fran�ois Delebecque
\end{authors}
