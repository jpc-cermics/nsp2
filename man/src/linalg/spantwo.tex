% -*- mode: latex -*-
%% Scilab ( http://www.scilab.org/ ) - This file is part of Scilab
%% Copyright (C) 1987-2016 - F. Delebecque
%%
%% This program is free software; you can redistribute it and/or modify
%% it under the terms of the GNU General Public License as published by
%% the Free Software Foundation; either version 2 of the License, or
%% (at your option) any later version.
%%
%% This program is distributed in the hope that it will be useful,
%% but WITHOUT ANY WARRANTY; without even the implied warranty of
%% MERCHANTABILITY or FITNESS FOR A PARTICULAR PURPOSE.  See the
%% GNU General Public License for more details.
%%
%% You should have received a copy of the GNU General Public License
%% along with this program; if not, write to the Free Software
%% Foundation, Inc., 59 Temple Place, Suite 330, Boston, MA  02111-1307  USA
%%                                                                                                

\mansection{spantwo}
\begin{mandesc}
  \short{spantwo}{sum and intersection of subspaces} \\ % 
\end{mandesc}
%\index{spantwo}\label{spantwo}
%-- Calling sequence section
\begin{calling_sequence}
\begin{verbatim}
  [Xp,dima,dimb,dim]=spantwo(A,B, [tol])  
\end{verbatim}
\end{calling_sequence}
%-- Parameters
\begin{parameters}
  \begin{varlist}
    \vname{A, B}: two real or complex matrices with equal number of rows
    \vname{Xp}: square non-singular matrix
    \vname{dima, dimb, dim}: integers, dimension of subspaces
    \vname{tol}: nonnegative real number
  \end{varlist}
\end{parameters}
\begin{mandescription}
  Given two matrices \verb!A! and \verb!B! with same number of rows,
  returns a square matrix \verb!Xp! (non singular but not necessarily orthogonal) 
  such that:
\begin{verbatim}
  [A1, 0]    (dim-dimb rows)
  Xp*[A,B]=[A2,B2]    (dima+dimb-dim rows)
  [0, B3]    (dim-dima rows)
  [0 , 0]
\end{verbatim}
The  first \verb!dima!  columns of \verb!inv(Xp)! span range(\verb!A!).
Columns \verb!dim-dimb+1! to \verb!dima! of \verb!inv(Xp)! span the 
intersection of range(A) and range(B).
The \verb!dim! first columns of \verb!inv(Xp)! span 
range(\verb!A!)+range(\verb!B!).
Columns \verb!dim-dimb+1! to \verb!dim! of \verb!inv(Xp)! span 
range(\verb!B!).
Matrix \verb![A1;A2]! has full row rank (=rank(A)). Matrix \verb![B2;B3]! has
full row rank (=rank(B)). Matrix \verb![A2,B2]! has full row rank (=rank(A inter B)). Matrix \verb![A1,0;A2,B2;0,B3]! has full row rank (=rank(A+B)).
\end{mandescription}
%--example 
\begin{examples}
  \begin{Verbatim}
    A=[1,0,0,4;
      5,6,7,8;
      0,0,11,12;
      0,0,0,16];
    B=[1,2,0,0]';C=[4,0,0,1]; 
    Sl=ss2ss(syslin('c',A,B,C),rand(A));
    [no,X]=contr(Sl('A'),Sl('B'));CO=X(:,1:no);  //Controllable part
    [uo,Y]=unobs(Sl('A'),Sl('C'));UO=Y(:,1:uo);  //Unobservable part
    [Xp,dimc,dimu,dim]=spantwo(CO,UO);    //Kalman decomposition
    Slcan=ss2ss(Sl,inv(Xp));
  \end{Verbatim}
\end{examples}
%-- see also
\begin{manseealso}
  \manlink{spanplus}{spanplus} \manlink{spaninter}{spaninter}  
\end{manseealso}
%-- Author
\begin{authors}
  Fran�ois Delebecque
\end{authors}
