% -*- mode: latex -*-
%% Scilab ( http://www.scilab.org/ ) - This file is part of Scilab
%% Copyright (C) 1987-2016 - F. Delebecque
%%
%% This program is free software; you can redistribute it and/or modify
%% it under the terms of the GNU General Public License as published by
%% the Free Software Foundation; either version 2 of the License, or
%% (at your option) any later version.
%%
%% This program is distributed in the hope that it will be useful,
%% but WITHOUT ANY WARRANTY; without even the implied warranty of
%% MERCHANTABILITY or FITNESS FOR A PARTICULAR PURPOSE.  See the
%% GNU General Public License for more details.
%%
%% You should have received a copy of the GNU General Public License
%% along with this program; if not, write to the Free Software
%% Foundation, Inc., 59 Temple Place, Suite 330, Boston, MA  02111-1307  USA
%%                                                                                                

\mansection{qz}
\begin{mandesc}
  \short{qz}{Schur decomposition of pencils} \\ % 
\end{mandesc}
%\index{qz}\label{qz}
%-- Calling sequence section
\begin{calling_sequence}
\begin{verbatim}
  [As,Es,Q,Z,dim] = qz(A,E,  sort= sel, args= obj, complex=%t|%f)
\end{verbatim}
\end{calling_sequence}
%-- Parameters
\begin{parameters}
  \begin{varlist}
    \vname{A,E}: two real or complex square matrix of same dimensions.
    \vname{Q,Z}: orthogonal or unitary square matrices.
    \vname{As}: upper triangular or quasi-triangular square matrix
    \vname{Es}: upper triangular  square matrix
    \vname{dim}: integer
    \vname{obj}: a list.
  \end{varlist}
\end{parameters}
\begin{mandescription}
  returns a quasi triangular matrix \verb!As! and a triangular matrix
  \verb!Es! which are the generalized Schur form of the pair 
  \verb!(A, E)! (i.e the Schur form of the pencil \verb!s*E-A!). 
  Moreover, \verb!Q! and \verb!Z! are two unitary matrices such that~:
\begin{verbatim}
      As=Q'*A*Z   and  Es=Q'*E*Z 
\end{verbatim}

    It is possible to obtain the complex Schur form of a real matrix 
    by adding an optional argument \verb!complex=%t! when calling function
    \verb!schur!.

    if and optional argument \verb!sort! is given, then an ordered 
    qz transformation is performed in such a way that generalized eigenvalues 
    selected by the sort function appear in the leading (upper left) 
    diagonal blocks of the quasi-triangular matrices \verb!As! and 
    \verb!Es!. The corresponding right eigenspace 
    has dimension \verb!dim! and is given by the 
    leading \verb!dim! columns of \verb!Z!. Using:
    \begin{itemize}
      \item \verb!sort='c'!: the right  eigenspace associated with
      generalized eigenvalues with negative real parts (stable continuous time eigenspace) is selected.
      \item \verb!sort='d'!: The right eigenspace associated with
      generalized eigenvalues with magnitude lower than 1 (stable discrete time eigenspace) is selected.
      \item \verb!sort=sel'!: The right  eigenspace associated with
      generalized eigenvalues for which the function \verb!sel(r,c,obj)! returns true 
      is selected. Extra arguments to the function \verb!sel! are given 
      by the optional argument \verb!obj! which is a list transmited with \verb!args=obj!.
    \end{itemize}
  \end{mandescription}
  % --example 
\begin{examples}
  \begin{mintednsp}{nsp}
 A = [ -1.,  0., 0.,  1. ; 0., -1., 5.,  0. 
        0.,  0., 2.,  0. ; 1.,  0., 0., -2.];
 E = [  0.,  1.,  0., 0. ; 0.,  0., -1., 0. 
        0.,  0.,  1., 0. ; 0.,  0.,  0., 1. ];
 [As,Es,Q,Z]=qz(A,E);
 Q'*A*Z 
 Q'*E*Z 
 [As,Es,Q,Z,dim] = qz(A,E,sort='c')
  \end{mintednsp}
\end{examples}
%-- see also
\begin{manseealso}
  \manlink{spec}{spec} \manlink{schur}{schur} 
\end{manseealso}
