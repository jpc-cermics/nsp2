% -*- mode: latex -*-
\mansection{norm}
\begin{mandesc}
  \short{norm}{vector and matrix norm}
\end{mandesc}
% -- Calling sequence section
\begin{calling_sequence}
\begin{verbatim}
nx = norm(x)
nx = norm(x,type_norm)
nA = norm(A)
nA = norm(A,type_norm)
\end{verbatim}
\end{calling_sequence}
% -- Parameters
\begin{parameters}
  \begin{varlist}
    \vname{x}:  numerical vector.
    \vname{A}:  numerical matrix.
    \vname{type_norm}: optional scalar or string chosen among
    \verb+1+, \verb+2+, \verb+%inf+ (in fact any real $p \ge 1$ for vectors),
    \verb+'1'+, \verb+'2'+, \verb+'inf'+. The default value is 2.
    \vname{nx,nA}: real positive scalar.
  \end{varlist}
\end{parameters}

\begin{mandescription}
This function computes vector or matrix norm according to the size, $m
\times n$, of the given first argument. When $m$ or $n$ is equal to 1 a vector
norm is computed and returned otherwise a matrix norm is computed and returned. 
The second argument could be a scalar and any real $p \ge 1$ could be used in case of a vector
$x$ to compute its $p-norm$ :
$x$ to compute its $p-norm$ :
$$
    ||x||_p = \left( \sum_{i=1}^n |x_i|^p \right)^{1/p}
$$
with the limit case of the inf-norm got using \verb+p=%inf+ :
$$
    ||x||_{\infty} = \max_i |x_i|
$$
For matrices, the \verb+1+, \verb+2+ and \verb+%inf+ ``induced matrix'' norms 
and the Frobenius norm are defined as follows:
$$
    ||A||_1 = \max_{||x||_1 = 1} ||Ax||_1 = \max_j \sum_i |A_{i,j}|  
$$
$$
    ||A||_2 = \max_{||x||_2 = 1} ||Ax||_2 = \sqrt{\max \mathop{\mathrm{spec}}(A^* A)}
$$
$$
    ||A||_{\infty} = \max_{||x||{\infty} = 1} ||Ax||_{\infty} = \max_i \sum_j |A_{i,j}|  
$$
$$
   ||A||_F =  \left( \sum_{i,j} A_{i,j}^2 \right)^{1/2}
$$
The Frobenius norm is computed by using  \verb+type_norm='fro' or 'Fro'+. 
Finally note that with \verb+type_norm='m' or 'M'+ the max 
of $|A_{i,j}|$ is computed but it is not a matrix norm.
\itemdesc{remark}
The matrix 2-norm is not available for sparse matrix.
\end{mandescription}
\begin{examples}
\paragraph{example 1} a simple example
  \begin{program}\HCode{x=[1,1];\Hnewline
    // 1-norm\Hnewline
    norm(x,1)  // or norm(x,'1')\Hnewline
    // 2-norm\Hnewline
    norm(x,2)  // or norm(x,'2') or simply norm(x)\Hnewline
    // Inf-norm\Hnewline
    norm(x,\%inf)  // or norm(x,'Inf')\Hnewline}
  \end{program}
\paragraph{example 2} invariance of a 2-norm by an orthonormal tranformation
  \begin{program}\HCode{// make an orthonormal matrix\Hnewline
    Q=qr(rand(5,5));\Hnewline
    // a random vector\Hnewline
    x = randn(5,1);\Hnewline
    // its 2-norm\Hnewline
    nx = norm(x,2)\Hnewline
    // the 2-norm Q*x\Hnewline
    ny = norm(Q*x, 2)\Hnewline
    // nx and ny should be near equal\Hnewline
    abs((nx-ny)/nx)}
  \end{program}
\end{examples}

%\begin{manseealso}
%  \manlink{qr}{qr}  
%\end{manseealso}

% -- Authors
\begin{authors}
   interface and some codes by Bruno Pincon and Jean-Philippe
   Chancelier. lapack routines dlange and zlange are used for
   1 and Inf matrix norms and dgesdd zgesdd for matrix 2-norm.
\end{authors}
