% -*- mode: latex -*-
%% Scilab ( http://www.scilab.org/ ) - This file is part of Scilab
%% Copyright (C) 1987-2016 - F. Delebecque
%%
%% This program is free software; you can redistribute it and/or modify
%% it under the terms of the GNU General Public License as published by
%% the Free Software Foundation; either version 2 of the License, or
%% (at your option) any later version.
%%
%% This program is distributed in the hope that it will be useful,
%% but WITHOUT ANY WARRANTY; without even the implied warranty of
%% MERCHANTABILITY or FITNESS FOR A PARTICULAR PURPOSE.  See the
%% GNU General Public License for more details.
%%
%% You should have received a copy of the GNU General Public License
%% along with this program; if not, write to the Free Software
%% Foundation, Inc., 59 Temple Place, Suite 330, Boston, MA  02111-1307  USA
%%                                                                                                

\mansection{classmarkov}
\begin{mandesc}
  \short{classmarkov}{recurrent and transient classes of Markov matrix} \\ % 
\end{mandesc}
%\index{classmarkov}\label{classmarkov}
%-- Calling sequence section
\begin{calling_sequence}
\begin{verbatim}
  [perm,L]=classmarkov(M)  
\end{verbatim}
\end{calling_sequence}
%-- Parameters
\begin{parameters}
  \begin{varlist}
    \vname{M}: real \verb!N x N! Markov matrix. Sum of entries in each row should add to one.
    \vname{perm}: a row vector giving a permutation.
    \vname{L}: a cell, each element of \verb!L! is a vector giving the states of the corresponding class.
  \end{varlist}
\end{parameters}
\begin{mandescription}
  Returns a permutation vector \verb!perm! such that
% \begin{verbatim}
%   M(perm,perm) = [M11 0 0 0 0   0]
%                  [0 M22 0 0     0]
%                  [0 0 M33       0]
%                  [      ...      ]
%                  [0 0       Mrr 0]
%                  [* *        *  Q]
% \end{verbatim}
  \begin{equation*}
    \verb!M(perm,perm)! = 
    \left( 
      \begin{array}{cccccc} 
        M_{1,1} & 0 & 0 & 0 & 0 & 0 \\ 
        0 & M_{2,2} & 0 & 0 & 0 & 0 \\ 
        0 & 0 & M_{3,3} & 0 & 0 & 0 \\ 
        0 & 0 & 0 & \ddots & 0 & 0 \\ 
        0 & 0 & 0 & 0 & M_{r,r} & 0 \\ 
        \cdot & \cdot & \cdot & \cdot & \cdot & Q 
      \end{array} 
    \right)
  \end{equation*}

  Each \verb!Mii! is a Markov matrix of dimension \verb!size(L{i},'*')!. The matrix 
  \verb!Q! is a sub-Markov matrix of dimension \verb!size(L{$},'*')!. The cell
  \verb!L! gives the states classification. Each element of \verb!L!  is a
  vector giving the states of the corresponding class. If \verb!n=size(L,'*')!,
  then the classes from \verb!1! to \verb!n-1! are recurrent classes and the
  class \verb!n! is transient. The vector \verb!perm! is obtained by the
  concatenation of the elements of \verb!L!.
\end{mandescription}
%--example 
\begin{examples}
  \begin{mintednsp}{nsp}
    P=genmarkov([2,1],2,'perm');
    [perm,L]=classmarkov(P);
    // check the result 
    M=P(perm,perm);
    M1=[];n=0;nc = size(L,'*');
    for i=1:nc-1, n=n+size(L{i},'*'); M1=[M1 # P(L{i},L{i})];end 
    if norm(M(1:n,1:n)-M1) > 100*%eps then pause;end
    if norm(M(n+1:$,n+1:$) - P(L{nc},L{nc}))  > 100*%eps then pause;end

  \end{mintednsp}
\end{examples}
%-- see also
\begin{manseealso}
  \manlink{genmarkov}{genmarkov} \manlink{eigenmarkov}{eigenmarkov}  
\end{manseealso}
