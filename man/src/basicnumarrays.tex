\chapter*{basic functions or methods for numerical matrices}\addcontentsline{toc}{chapter}{basic functions or methods for numerical matrices}

\begin{quote}
\noindent
\hyperlink{zeros}{zeros},\hyperlink{ones}{ones} - build matrices of zeros or ones \\
\hyperlink{eye}{eye} - identity matrix or matrix of canonical application between 2 vector spaces \\
\hyperlink{tril}{tril}, \hyperlink{triu}{triu} - extract lower or upper triangle part of a matrix \\
\hyperlink{diag}{diag} - extract a diagonal of a matrix or build a diagonal matrix from a vector \\
\hyperlink{sum}{sum},\hyperlink{prod}{prod} - sum and product of matrix elements \\
\hyperlink{cumsum}{cumsum},\hyperlink{cumprod}{cumprod} - cumulative sum and product of matrix elements \\
\hyperlink{diff}{diff} - difference of a vector or a matrix \\
\hyperlink{min}{min} - minimum of matrix elements or minimum of 2 or more matrices \\
\hyperlink{max}{max} - maximum of matrix elements or maximum of 2 or more matrices \\
\hyperlink{minmax}{minmax} - both minimum and maximum of matrix elements\\
\hyperlink{dot}{dot} - scalar product of 2 vectors (or matrices) \\
\hyperlink{cross}{cross} - cross product of 2 vectors \\
\hyperlink{scale_rows}{scale\_rows} - multiply each row of a matrix by a different scalar \\
\hyperlink{scale_cols}{scale\_cols} - multiply each column of a matrix by a different scalar \\
\hyperlink{ndgrid}{ndgrid} - arrays for multidimensional function evaluation on grid \\
\hyperlink{linspace}{linspace} - uniform mesh/partition of a segment \\
\hyperlink{logspace}{logspace} - logarithmically spaced vector \\
\end{quote}
 
\input basicnumarrays/zerosones.tex 
\input basicnumarrays/eye.tex  
\input basicnumarrays/triul.tex  
\input basicnumarrays/diag.tex  
\input basicnumarrays/sum.tex 
\input basicnumarrays/cumsum.tex 
\input basicnumarrays/diff.tex 
\input basicnumarrays/minmax.tex 
\input basicnumarrays/dot.tex 
\input basicnumarrays/cross.tex 
\input basicnumarrays/scalerowscols.tex
\input basicnumarrays/ndgrid.tex  
\input basicnumarrays/linspace.tex  
\input basicnumarrays/logspace.tex  



