% -*- mode: latex -*-
\mansection{timer, cpuinfo, tic, toc}
\begin{mandesc}
  \short{timer}{elapsed cpu time between timer calls} 
  \short{cpuinfo}{elapsed cpu time since session start} 
  \short{toc}{elapsed time since tic call}
  \short{tic}{set a timer start}
\end{mandesc}
%-- Calling sequence section
\begin{calling_sequence}
\begin{verbatim}
  T=timer();
  T=cpuinfo();
  tic();T=toc();
  Ts=tic();...;T=toc(Ts);
\end{verbatim}
\end{calling_sequence}
\begin{mandescription}
  \verb!timer!  returns the CPU time (in seconds) used by nsp from the 
  preceding call to the \verb!timer! function. 

  \verb!cpuinfo! returns the total CPU time (in seconds) used by nsp 
  from the time it was started. Note that internal representation
  can overflow and wrap around.

  \verb!toc()! returns the elapsed time since last call to \verb!tic()!
  used without lhs. \verb!toc(T)! where \verb!T! was set by a call to 
  \verb!T=tic()! returns the elapsed time since the \verb!T=tic()! 
  instruction was executed. 
\end{mandescription}
\begin{examples}
  \begin{Verbatim}
    A=rand(200,200);timer();inv(A);Tinv=timer();
    n=100;t=zeros(1,n);
    tic();
    for i=1:n 
      tStart = tic();
      inv(A);
      t(i) = toc(tStart);
    end
    avt = toc()/n // average time 
  \end{Verbatim}
\end{examples}
%-- see also

