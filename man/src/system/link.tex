% -*- mode: latex -*-
\mansection{link}
\index{c\_link}\label{c-link}
\index{ulink}\label{ulink}
\begin{mandesc}
  \short{link}{load dynamic library}\\ 
  \short{ulink}{unload dynamic library}\\
  \shortunder{c\_link}{c_link}{check entry points in link table}
\end{mandesc}
%-- Calling sequence section
\begin{calling_sequence}
\begin{verbatim}
 H=link();
 id=link(library,entry-points [,flag]);
 id=link(library);
 link('show');
 ulink(id);
 [rep,id]=c_link(name [, id]);
\end{verbatim}

\end{calling_sequence}
%-- Parameters
\begin{parameters}
  \begin{varlist}
    \vname{library}: It can be an integer giving the id of a previously linked dynamic library or it can be the pathname 
    of a dynamic library (Note that relative pathnames which only contain a library name as for example in \verb!libfoo.so! 
    are not valid, use \verb!./libfoo.so! in that case).
    \vname{entry-points}: a string matrix containing all the symbols to be searched in the dynamic library. 
    \vname{flag}: an optional string which can be \verb!'f'! (naming Fortran) or \verb!'c'! (name C). The default value being 
    \verb!'c'! 
    \vname{id}: an integer which gives the internal id of the shared library. 
    \vname{H}: a hash table. 
    \vname{rep}: a boolean. 
    \vname{name}: a string.
  \end{varlist}
\end{parameters}
\begin{mandescription}
  The \verb!link! function is used to load a dynamic library into nsp. 
  Then, entry points from the dynamic libraries can be added to an internal array of 
  nsp. These entry points can be accessed by a set of nsp functions as for example 
  by the \verb!call! function. The \verb!ulink! function can be used to unload 
  a dynamic library and remove the added entry points. 
  In order to create a dynamic library from a set of file and obtain an link 
  code which does not depend on the underlying operating system the script 
  \verb!ilib_for_link! can be used.
  \begin{itemize} 
  \item \verb!id_libfoo=link('./libfoo.so',['foo','goo'],'f')! load the library 
    \verb!'./libfoo.so'! and add the entries \verb!foo! and \verb!goo! from this 
    library. \verb!'f'! is used to tell that \verb!foo! and \verb!goo! are to be 
    searched as Fortran symbols which on some systems implies that their internal 
    names are to be suffixed and/or prefixed with an underscore. 
  \item \verb!link(id_libfoo,['poo'])! add the entry \verb!poo! which is searched 
    on a previously linked library with id \verb!id_libfoo! 
  \item \verb!id_libfoo=link('./libfoo.so')! just load the library \verb!./libfoo.so! 
    and returns an id which can be used on next calls to \verb!link!. 
    Note that in that case the exprorted symbols from the library \verb!'./libfoo.so'! 
    are global. It means that if you link in a subsequent call an other library 
    which need the symbols from \verb!'./libfoo.so'! to resolve its dependences 
    then \verb!'./libfoo.so'! will be properly used for the dependence resolution. 
    But note that the global propery can also lead to name clashes. If you try 
    to link from an other library a symbol already defined in \verb!'./libfoo.so'! 
    it won't work and the symbol from \verb!'./libfoo.so'! will be used. 
  \item \verb!link('nsp')! or \verb!link('nsp','foo')! can be used to access 
    symbols from the nsp executable.
  \item \verb!H=link()! returns in a hash table all the entry points, the value 
    of each entry point in \verb!H! being the id of the dynamic library it was loaded 
    from. Note that the id information is wrong when a symbol is found in a 
    global (as described in the previous item) library.  
  \item \verb![rep,idr]=c_link(name [,id])! checks if entry point with name \verb!name! 
    is present in the dynamic entry points table. If the answer is positive then \verb!rep!
    is set to true and \verb!idr! is set to the id of the dynamic library containing the 
    symbol \verb!name!. An optional argument can be given to \verb!c_link! to limit the search 
    to the symbols loaded from a dynamic library with id \verb!id!.
  \end{itemize}
\end{mandescription} 
\begin{manseealso}
  \manlink{ulink}{ulink}
  \manlink{c\_link}{c_link}
  \manlink{addinter}{addinter}  
  \manlink{ilib\_for\_link}{ilib_for_link}  
\end{manseealso}

