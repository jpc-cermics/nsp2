% -*- mode: latex -*-
% Scilab ( http://www.scilab.org/ ) - This file is part of Scilab
% Copyright (C) 2008 - INRIA -
%
% This file must be used under the terms of the CeCILL.
% This source file is licensed as described in the file COPYING, which
% you should have received as part of this distribution.  The terms
% are also available at
% http://www.cecill.info/licences/Licence_CeCILL_V2-en.txt
%
\mansection{realtime}
\begin{mandesc}
  \short{realtimeinit}{set up time unit for realtime function} \\
  \short{realtime}{set up dates origin or waits until date} \\
\end{mandesc}
\index{realtimeinit}\label{realtimeinit}
\index{realtime}\label{realtime}
%-- Calling sequence section
\begin{calling_sequence}
\begin{verbatim}
  realtimeinit(time_unit)
  realtime(t)
\end{verbatim}
\end{calling_sequence}
%-- Parameters
\begin{parameters}
  \begin{varlist}
    \vname{time\_unit}: a real number. The number of seconds associated to the \verb!realtime! argument
    \vname{t}: a real number giving a date.
  \end{varlist}
\end{parameters}
\begin{mandescription}
  These two functions can be used to handle real time in nsp.
  The function \verb!realtimeinit(time_unit)! defines the time unit
  associated to the \verb!realtime! argument \verb!t!. A first call to
  \verb!realtime(t0)! sets the current date to (\verb!t0!). Then,
  subsequent calls to \verb!realtime(t)! wait till date \verb!t! is
  reached.

  Note that Gtk events are not managed when function realtime is called.
\end{mandescription}
%--example
\begin{examples}

\noindent A first example
  \begin{mintednsp}{nsp}
    realtimeinit(1/2);//sets time unit to half a second
    realtime(0);//sets current date to 0
    for k=1:10,realtime(k);printf('current time is '+string(k/2)+'sec .\n');end
  \end{mintednsp}

\noindent A second example which prints each 2 seconds the elapsed time since
the call to \verb!tic!
\begin{mintednsp}{nsp}
  function showtime()
    tic();for k=1:10,realtime(k);printf('\r%5.2f',toc());end;printf('\n');
    printf('Elapsed time %5.2f\n',toc());
  endfunction

  realtimeinit(2);realtime(0);showtime();
\end{mintednsp}

\noindent An example with graphic animation

\begin{mintednsp}{nsp}
xsetech(frect=[-1,-1,1,1]*1.5,fixed=%t,clip=%t,iso=%t,axesflag=0);
xset('thickness',2);
xarc(-1,1,2,2,0,360*64,color=3,thickness=2);
x1=[0;cos(90*%pi/180)];
y1=[0;sin(90*%pi/180)];
A=xarrows(x1,y1,arsize=0.2,style=1);
S=xstring(-0.5,1.2," ",fill=%t,w=1,h=0.25);
F=get_current_figure[];
p=5;
realtimeinit(1/p);//sets time unit to 1/p
realtime(0);//sets current date to 0
N=60;
for k=linspace(1,N,5*N);
  realtime(k);
  A.x(2)= cos((90-(p*k*360/60))*%pi/180)
  A.y(2)= sin((90-(p*k*360/60))*%pi/180)
  S.text=sprintf("T=%5.2f s.",k/p);
  F.invalidate[];
  F.process_updates[];
  xpause(0,%t); // gtk events are managed here
end
\end{mintednsp}

\end{examples}
%-- see also
