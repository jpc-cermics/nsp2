% -*- mode: latex -*-
\mansection{registry}
\begin{mandesc}
  \short{registry}{get and set values of the windows registry}
\end{mandesc}
  %-- Calling sequence section
\begin{calling_sequence}
  \begin{verbatim}
    [res,tres]=registry(key,subkey) 
    [res,tres]=registry(key,subkey,name) 
    registry(key,subkey,name, value);
    registry(key,subkey, delete=%t)
  \end{verbatim}
\end{calling_sequence}
%-- Parameters
\begin{parameters}
  \begin{varlist}
    \vname{key}: a root key which is to be chosen among the following 
    possible values: \verb!"HKEY_CLASSES_ROOT"!, 
    \verb!"HKEY_CURRENT_USER"!, \verb!"HKEY_LOCAL_MACHINE"!, 
    \verb!"HKEY_USERS"!, \verb!"HKEY_DYN_DATA"!. 
    \vname{subkey}: a path string giving a node in the registry. 
    \vname{name}: a string 
    \vname{delete}: an optional value given as a boolean. 
    \vname{tres}: the type of the subkey: \verb!'n'! or \verb!'l'! or \verb!'v'!.
    \vname{res}: a string or a scalar. 
  \end{varlist}
\end{parameters}
\begin{mandescription}
  The \verb!registry! command is used to query the registry or to delete keys 
  from the registry. 
  \begin{itemize}
  \item  When called with two arguments and if \verb!key! and \verb!subkey! describe a key present in the registry, the function returns 
    \begin{itemize}
    \item the list of subkeys of the key if the key is a node in the 
      registree tree.  \verb!tres! is then set to  \verb!'n'! and the 
      list is returned in a string matrix. 
    \item the list of the value names of the key if the key is a leaf 
      in the registree tree. The list is returned in a string matrix and 
      \verb!tres! is then set to  \verb!'l'!. 
    \end{itemize}
  \item When called with two arguments and the optional argument 
    \verb!delete=%t! the regsitry key given by  \verb!key! and \verb!subkey!
    is deleted. Note that only leaves of the registry tree can be deleted. 
  \item   When called with three arguments the function returns the value 
    associated to the value-\verb!name! of the key  described by 
    \verb!key! and \verb!subkey!. Note that only strings and integer 
    values are searched. In that case \verb!tres! is set to \verb!v!.
  \item When called with four arguments the function set the
    value-\verb!name! of the key  described by 
    \verb!key! and \verb!subkey! to the value given by \verb!value! 
    Note that the key may preexist or not in the registry tree.
  \end{itemize}
\end{mandescription}
%--example 
\begin{examples}
  \begin{Verbatim} 
    // get the subkeys of a node 
    [n,Tn]=registry("HKEY_LOCAL_MACHINE","Software\\GTK");
    // get the value-names of a leaf 
    [l,Tl]=registry("HKEY_LOCAL_MACHINE","Software\\GTK\\2.0");
    // get a specific value 
    [v1,Tv1]=registry("HKEY_LOCAL_MACHINE","Software\\GTK\\2.0","Path");
    [v2,Tv2]=registry("HKEY_LOCAL_MACHINE","Software\\GTK\\2.0","Version");
    // set a specific value 
    registry("HKEY_LOCAL_MACHINE","Software\\Test","Path","c:\\foo");
    registry("HKEY_LOCAL_MACHINE","Software\\Test","String","nsp");
    registry("HKEY_LOCAL_MACHINE","Software\\Test","K",678);
    [zp,tp]=registry("HKEY_LOCAL_MACHINE","Software\\Test","Path");
    [zs,ts]=registry("HKEY_LOCAL_MACHINE","Software\\Test","String");
    zk=registry("HKEY_LOCAL_MACHINE","Software\\Test","K");
    // delete a leaf 
    registry("HKEY_LOCAL_MACHINE","Software\\Test",delete=%t);
  \end{Verbatim}
\end{examples}
%-- see also


