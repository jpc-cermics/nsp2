% -*- mode: latex -*-
\mansection{file}
\begin{mandesc}
  \short{file}{file and pathnames}
\end{mandesc}
%\index{file}\label{file}

%-- Calling sequence section
\begin{calling_sequence}
  \begin{verbatim}
    file(operation,S);
  \end{verbatim}
\end{calling_sequence}
%-- Parameters
\begin{parameters}
  \begin{varlist}
    \vname{operation}: a string giving an operation on filename
    \vname{S}: a string matrix.
  \end{varlist}
\end{parameters}
\begin{mandescription}
  The \verb!file! command is used to get file attributes, or 
  to get pathname informations, or to build pathnames from 
  components.
\end{mandescription}
%--example 
\begin{examples}
  \begin{nspcode}
    file("atime","doc.txt") ;
    // file("attributes",file [,attrname, attrset]);
    file("attributes","doc.txt");
    //file("copy",["src","dst"])	    dst is a file 
    //or 
    //file("copy",["src1","src2","dst"])  dst is a directory 
    file("mkdir","poo");
    file("copy",["doc.txt","doc1.txt"]) 
    file("copy",["doc.txt","poo"]) 
    // file("delete", "file" or "dir")     dir is deleted if it is empty 
    file("delete","poo/doc.txt");
    file("delete","poo");
    path=file('join',[getcwd(),'doc.txt'])
    file("dirname",path)
    file("executable", path)
    file("exists",path)
    file("extension",path)
    file("isdirectory",path)
    file("isfile",path) 
    file("lstat", path) 
    file("mtime", path) 
    file("nativename",path)
    file("owned", path)
    file("pathtype",path) 
    file("readable",path) 
    system("ln -f -s doc.txt doc.ln")
    if file("readlink","doc.ln")<>"doc.txt" then pause;end
    // file("rename",["src","dst"])
    // or 
    // file("rename",["src1","src2","dst"])
    file("rename",["doc1.txt","doc2.txt"]);
    file("delete","doc2.txt");
    file("rootname", path)
    file("size", path)
    file("split", path)
    file("stat", path)
    file("tail", path)
    file("type", path)
    file("volumes")
    file("writable",path)
  \end{nspcode}
\end{examples}
%-- see also
\begin{manseealso}
  \manlink{File}{File}
\end{manseealso}

