% -*- mode: latex -*-

\mansection{print}

\begin{mandesc}
  \short{print}{print variable value  in the current output stream}\\
  \short{fprint}{print variable value in a file stream}\\
  \short{sprint}{print variable value in a string matrix} 
\end{mandesc}
\index{print}\label{print}
  %-- Calling sequence section
\begin{calling_sequence}
\begin{verbatim}
   print(object [,options])
   fprint(fd,object [,options]) 
   S=sprint(fd,object [,options]) 
   where named options are 
   as_read=boolean, depth=int,indent=int, latex=boolean,name=string,table=boolean) 
\end{verbatim}
\end{calling_sequence}

\begin{parameters}
  \begin{varlist}
    \vname{object}: object to be printed.
    \vname{as_read}: if true, object is to be printed in such a way that the result can be 
    parsed back by nsp.
    \vname{depth}: control, for recursive objects, the depth at which printing
    is done.
    \vname{indent}: indentation level.
    \vname{name}: string used to override the variable name.
    \vname{latex}: if true, prints in a \LaTeX matrix style.
    \vname{table}: if true, table style instead of matrix style is used in \LaTeX printing.
  \end{varlist}
\end{parameters}

\begin{mandescription}
  prints the given variable in the current output stream (\verb+print+), 
  or in a file steam (\verb+fprint+), or in a string matric (\verb+sprint+).
  The file stream \verb+fd+ is an object of type \manlink{File}{File} which 
  is created with \manlink{fopen}{fopen}.
\end{mandescription}
% --example 
\begin{examples}
  \begin{Verbatim}
    l =list(rand(3,3),%t, 67, "scilab",list(5,6));
    print(l,depth=0);
  \end{Verbatim}
\end{examples}
% -- see also
\begin{manseealso}
  \manlink{file}{file}  
  \manlink{fopen}{fopen}  
  \manlink{File}{File}  
\end{manseealso}

