% -*- mode: latex -*-
\mansection{who}
\begin{mandesc}
  \short{who}{get current variables}
\end{mandesc}

\begin{calling_sequence}
\begin{verbatim}
  who
  who tag 
  x=who( hash= %t|%f )
  x=who( tag,hash= %t|%f )
\end{verbatim}
\end{calling_sequence}
\begin{parameters}
  \begin{varlist}
    \vname{x}: a hash table or string matrix depending on the value of 
    the optional named argument \verb+hash+ (default value is \verb+%f+).
    \vname{tag}: a string chosen among \verb+'local'+,  \verb+'global'+,
    \verb+'caller'+, or  \verb+'constants'+.
  \end{varlist}
\end{parameters}
\begin{mandescription}
  \verb!who! returns a hash table filled with the current variables. 
  A first argument chosen among \verb+'local'+,  \verb+'global'+,
  \verb+'caller'+, or  \verb+'constants'+ can be given with the following 
  meaning~:
\begin{description}
  \item['local'] get variables from the local environment. 
  \item['global'] get variables from the global environment \emph{i.e} 
    variables declared as global.
  \item['caller'] get variables from the caller. 
  \item['constants'] get variables from the predefined constants. 
\end{description}
  If the optional 
  named argument \verb+hash+ is set to \verb+%f+ then the function will 
  return a string matrix filled with variable names instead of a hash table.
\end{mandescription} 
\begin{examples}
  \begin{Verbatim} 
    for i=1:100; execstr('x'+string(i)+'='+string(i));end ;
    who 
    H= who(hash=%t);
    editvar('H');
  \end{Verbatim}
\end{examples}



