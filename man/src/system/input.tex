% -*- mode: latex -*-
\mansection{input}
\begin{mandesc}
  \short{input}{wait for user keyboard input}
\end{mandesc}
%\index{input}\label{input}
%-- Calling sequence section
\begin{calling_sequence}
\begin{verbatim}
  [x]=input(str, eval=%f|%t, env = H, accept_empy=%f|%t)     
\end{verbatim}
\end{calling_sequence}
%-- Parameters
\begin{parameters}
  \begin{varlist}
    \vname{str}: a character string giving the prompt.
    \vname{eval}: a boolean (default value is true).
    \vname{env}: a hash table.
    \vname{accept_empty}: a boolean (default value is false).
  \end{varlist}
\end{parameters}
\begin{mandescription}
  displays the string \verb!str! in the console and waits for a sequence 
  of characters (ended by a new line) entered in the main interaction 
  window. If \verb!eval=%f!, then the string entered by the user is 
  returned directly. If \verb!eval=%t!, the the string entered by the 
  user is first evaluated and the value of \verb!ans! is returned by 
  the function. If the flag \verb!accept_empy=%t! then an empty string 
  is returned by the function when the user enter an empty answer. 
  If \verb!env! is given, it must be a hash table which is used as a 
  context when evaluating the user answer.
\end{mandescription}
%--example 
\begin{examples}
  \begin{mintednsp}{nsp}
    // yes or no ? empty answer considered as no 
    x=input("yes or no ? ",accept_empty=%t, env=hash(yes=1,no=0));
    // if x.equal[1] then ... end 
  \end{mintednsp}
\end{examples}
%-- see also

