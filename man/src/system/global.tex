% -*- mode: latex -*-
\mansection{global, clearglobal, isglobal}
\begin{mandesc}
  \short{global}{declare variables as global variables} \\ % 
  \short{clearglobal}{undeclare variables as global variables} \\ % 
  \short{isglobal}{replaced by the exists function} \\ % 
\end{mandesc}
%\index{global}\label{global}
%-- Calling sequence section
\begin{calling_sequence}
\begin{verbatim}
  global('name1',...,'namen',def=value);
  global name1 ... namen
  global name1, ..., namen
\end{verbatim}
\end{calling_sequence}
%-- Parameters
\begin{parameters}
  \begin{varlist}
    \vname{name1,..., namen}: valid ident names
  \end{varlist}
\end{parameters}
\begin{mandescription}
  The \verb!global! keyword or function is used to declare that some idents 
  refers to variables in a global frame and not to the local frame.
  If the variable referred by the ident name does not pre-exists 
  it is created with an empty matrix as default value or with the value 
  given by the optional argument \verb!def! if present. 
  \verb!global! can be used anywhere in nsp code.

  The \verb!clearglobal! keyword or function is used to remove a global 
  declaration for an ident.

  The \verb!isglobal! function does not exists in nsp. Similar result 
  can be obtained with \verb!exists(identname,'global');!.
  
\end{mandescription}
%--example 
\begin{examples}
  \begin{Verbatim}
    global('a', def=1); 
    function y=f(x); global a; y=x+a; endfunction
    function g(x); global('a',def=2);a=x; endfunction
    f(3)
    g(3)
    f(5)
    exists('a','global');
    clearglobal a;
  \end{Verbatim}
\end{examples}
%-- see also
\begin{manseealso}
  \manlink{who}{who} \manlink{isglobal}{isglobal} \manlink{clearglobal}{clearglobal} 
\end{manseealso}
