% -*- mode: latex -*-
\mansection{global, clearglobal, isglobal}
\begin{mandesc}
  \short{global}{declare variables as global variables} \\ % 
  \short{clearglobal}{undeclare variables as global variables} \\ % 
  \short{isglobal}{replaced by the exists function} \\ % 
\end{mandesc}
%\index{global}\label{global}
%-- Calling sequence section
\begin{calling_sequence}
\begin{verbatim}
  global('name1',...,'namen',oname1=def1,...,onamen=defn);
  global name1 ... namen
  global name1, ..., namen
\end{verbatim}
\end{calling_sequence}
%-- Parameters
\begin{parameters}
  \begin{varlist}
    \vname{name1,..., namen}: valid ident names
    \vname{oname1,..., onamen}: valid ident names
  \end{varlist}
\end{parameters}
\begin{mandescription}
  The \verb!global! keyword or function is used to declare that some idents 
  refers to variables in a global frame and not to variables in the local frame.
  If the variable referred by the ident name does not pre-exists 
  it is created with an empty matrix as default value or with the given value 
  if the ident is given as an optional argument \verb!name=value!. 
  \verb!global! can be used anywhere in nsp code.
  
  The \verb!clearglobal! keyword or function is used to remove a global 
  declaration for an ident.
  
  The \verb!isglobal! function does not exists in nsp. Similar result 
  can be obtained with \verb!exists(identname,'global');!.
\end{mandescription}
%--example 
\begin{examples}
  \begin{Verbatim}
    function y=f(x); global(a=5); y=x+a; endfunction;
    f(0);
    clearglobal a;
    global(a=10);
    f(0);
    exists('a','global');
  \end{Verbatim}
\end{examples}
%-- see also
\begin{manseealso}
  \manlink{who}{who} \manlink{isglobal}{isglobal} \manlink{clearglobal}{clearglobal} 
\end{manseealso}
