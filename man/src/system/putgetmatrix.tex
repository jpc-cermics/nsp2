% -*- mode: latex -*-

\mansection{put\_matrix, get\_matrix, fprintfMat, fscanfMat}

\begin{mandesc}
  \short{put_matrix}{(method) print a matrix in a file stream}\\
  \short{fprintfMat}{print a matrix in a file}\\
  \short{get_matrix}{(method) read a matrix from a file stream}\\
  \short{fscanfMat}{read a matrix in a file} 
\end{mandesc}
\index{put\_matrix}\label{put-matrix}
\index{get\_matrix}\label{get-matrix}
  %-- Calling sequence section
\begin{calling_sequence}
\begin{verbatim}
   F.put_matrix[M,format=str1,sep=str2,header=title]
   fprintfMat(filename,M,format=str1,sep=str2,header=title)

   [M,title] = F.get_matrix[]
   [M,title] = fscanfMat(filename,format=str1)
\end{verbatim}
\end{calling_sequence}

\begin{parameters}
  \begin{varlist}
    \vname{M}: numerical matrix (Mat)
    \vname{header=title}: string matrix (SMat)
    \vname{F}: file stream
    \vname{filename}: path
    \vname{format=str1}: a C-like format
    \vname{sep=str2}: string to be used as a separator between 2 numbers (default is \verb+" "+)
  \end{varlist}
\end{parameters}

\begin{mandescription}
 The functions \verb+fprintfMat+ and \verb+fscanfMat+ are shortcuts for respectively:
\begin{Verbatim}
// fprintfMat
F = fopen(filename,mode="w")
F.put_matrix[M,format=str1,sep=str2,header=title]
F.close[]

// fscanfMat
F = fopen(filename,mode="r")
[M,title] = F.get_matrix[]
F.close[]
\end{Verbatim}
so we will only try to explain how \verb+put_matrix+ and \verb+get_matrix+ work.
\end{mandescription}

\paragraph{put\_matrix}

  This method outputs the matrix \verb+M+ in the file stream, using a line for each 
row of the matrix (that is a end line character is printed after the last number of
each row):

\begin{itemize}

\item \itemdesc{header option} if given the string matrix \verb+title+ is output in the
file stream before the matrix \verb+M+ ; 

\item \itemdesc{sep option} if given the string \verb+str2+ will be printed between
2 numbers of the same row in place of a blank character (if you want that your
matrix can be read back into nsp by \verb+fscanfMatrix+ use only blank, tab, coma 
or semicolon).

\item \itemdesc{format option} each number is converted following the format 
given by the string  \verb+str1+ which must be a C format. When format is not
provided a format based on the current output nsp format (see the 
\manlink{format}{format} help page) is used. The three kind of formats are  
\verb+"%f"+,  \verb+"%e"+ and  \verb+"%g"+:
\begin{itemize}
\item \verb+"%f"+ this one converts a binary floating point number into a fixed
point decimal number (with 6 digits after the point by default). This format is not 
adapted for numbers with a small magnitude (which are truncated to 0) and take many 
places for numbers with a large magnitude (use \verb+"%e"+ or   \verb+"%g"+ formats
in this cases). Some variations:
\begin{itemize}
\item  \verb+"%.9f"+ specify 9 digits after the point
\item  \verb+"%.0f"+ in this case the point is not printed (useful for matrix with 
integer values)
\item  \verb+"%13.9f"+ specify 9 digits after the point and a total of 13 characters.
Notes that specifying the total number of characters lets to print
a file which is more human readable (as all columns will be well aligned).
\end{itemize}

\item \verb+"%e"+ this one converts a binary floating point number into a decimal
floating point number, that is using a sign, a significant part and an exponent
part. So it is adapted to large and small magnitudes. Note that  \verb+"%e"+ have 
only 6 decimal digits accuracy (it corresponds to  \verb+"%.6e"+). Some variations: 
\begin{itemize}
\item \verb+"%.17e"+ this one should have the property to be exactly back converted 
for any double floating point number.
\item \verb+"%24.17e"+ same than before but with specifying the total number of characters
(so the columns will be well aligned). Note that when you give the total number of 
characters the converted number can use less and by default it will be right justified 
in the field. If you want the number to be left justified you can use  \verb+"%-24.17e"+
\item if you want only 10 decimal digits accuray you can use  \verb+"%.10e"+ or
\verb+"%17.10e"+
\end{itemize}
\item  \verb+"%g"+ this format could be interesting because it converts a binary
floating point number using the \verb+"%f"+ format if the magnitude number is not too
large or too small (without the fractional part if the number is an integer) and 
otherwise it uses the \verb+"%e"+ format.

\item finally if you want to print positive numbers with the \verb-+- sign (which is
omitted by default) you should add a  \verb-+- just before the format letter (\verb+f+,
\verb+e+ or \verb+g+). Examples: \verb/"%-24.17+e"/, \verb/"%-24.17+g"/
\end{itemize}
\end{itemize}

Note that  if \verb+M+ is a complex matrix, the printed matrix will have twice the number
of columns, each original column giving 2 successive columns with the real part coming first
followed by the imaginary part.


\paragraph{get\_matrix}

 This method tries to read a matrix in the given file stream. The matrix can be preceded by
any number of text lines which can be recovered in a string matrix as second output argument.

The first numeric line establishes the number of columns of the matrix and next lines
are considered as the next matrix rows as long as they are ``numeric'' with the same
number of values (that is of columns).

Among a line, 2 numbers can be separated by
any number of blank or tab characters and/or one coma or semicolon.


% --example 
\begin{examples}
\begin{Verbatim}
x = (1:9)'; pr = pdf("geom",n,0.7);
Mat = [x,pr];
title = ["some probabilities of G(0.7) (the geometric distribution)";
         "X pr(X)"];
fprintfMat("toto.dat",Mat,header=title)

// read  back into nsp
[M,title] = fscanfMat("toto.dat")
max(abs((M-Mat)./Mat))  // should not be 0

// use the nsp long format (it uses 16 decimal digits for the significant part)
format("long")
fprintfMat("toto.dat",Mat,header=comment)
// read  back to nsp
M = fscanfMat("toto.dat")
max(abs((M-Mat)./Mat))  // error should be less than before but not 0

// to recover back exactly one needs 17 decimal digits for the significant part
fprintfMat("toto.dat",Mat,header=comment,format="%.17g")
// read  back to nsp
M = fscanfMat("toto.dat")
max(abs((M-Mat)./Mat))  // error should be 0
\end{Verbatim}
\end{examples}

% -- see also
\begin{manseealso}
  \manlink{fopen}{fopen}, \manlink{File}{File}
\end{manseealso}

