% -*- mode: latex -*-
\mansection{scanf}

\begin{mandesc}
  \short{scanf}{implements the {\texttt scanf} function}\\
  \short{fscanf}{implements the {\texttt fscanf} function}\\
  \short{sscanf}{implements the {\texttt sscanf} function}
\end{mandesc}
\index{scanf}\label{scanf}
  %-- Calling sequence section
\begin{calling_sequence}
\begin{verbatim}
   [v_1,...v_n]=scanf([iter,]format)
   [v_1,...v_n]=fscanf([iter,]fd,format)
   [v_1,...v_n]=sscanf([iter,]S,format)
\end{verbatim}
\end{calling_sequence}

\begin{parameters}
  \begin{varlist}
    \vname{iter}: an integer which when given gives the number of times the given 
    format is to be scanned (if \verb+iter=-1+ the format is reused until end of file or 
    end of string matrix is reached).
    \vname{format}: a string which gives the scan directives and conforming to the 
    standard directives of the C library.
    \vname{fd}: a \manlink{File}{File} object.
    \vname{S}: a string column matrix. 
    The format is used to scan the string matrix \verb+S+ row by row. Note that 
    if \verb+S+ is not a column matrix only the first column will be used. 
  \end{varlist}
\end{parameters}

\begin{mandescription}
  The \verb+scanf+ family of functions scans input according to format which contains 
  conversion specifications. The results from conversions are stored in the 
  given left hand side names in the order of the conversion. If less arguments are 
  given then some converted variables are omitted. If the left hand side argument 
  list exceed the number of conversions a warning is returned. 
  Note that using a list argument in the left hand side, it is possible to 
  obtain any number of stored conversions without giving an explicit extended list. 
  This last case is illustrated in one of the given examples. 
\end{mandescription}
% --example 
\begin{examples}
  \begin{Verbatim}
    [a,b]=sscanf(2,["8 9 10";"11 12 14"],"%d %d");
    S=string(rand(4,3));
    S=catenate(S,col=" "); // change to column matrix
    format=catenate(smat_create(1,3,"%f"),sep=" ");
    [a,b,c]=sscanf(-1,S,format)
  \end{Verbatim}

  \begin{Verbatim}
    n=10;
    A=int(100*rand(4,n));
    S=string(A);
    S=catenate(S,col=" "); // change to column matrix
    format=catenate(smat_create(1,n,"%d"),sep=" ");
    L=list();
    L(1:n)=sscanf(-1,S,format);
    L.compact[];
    A.equal[L(1)]
  \end{Verbatim}

\end{examples}
% -- see also
\begin{manseealso}
  \manlink{printf}{printf}
  \manlink{fopen}{fopen}  
  \manlink{File}{File}  
\end{manseealso}

