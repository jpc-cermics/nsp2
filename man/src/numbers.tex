\chapter*{numbers (properties, display,...)} 
\addcontentsline{toc}{chapter}{numbers (properties, display,...)}

\begin{quote}
\noindent
\hyperlink{format}{format} - number printing and display format\\
\hyperlink{number_properties}{number\_properties} - determine floating-point parameters \\
\hyperlink{frexp}{frexp} - dissect floating-point numbers into base 2 exponent and mantissa\\
\hyperlink{isreal}{isreal} - check if a variable has real or complex entries\\
\hyperlink{complex}{complex} - build complex numbers from real and imaginary parts\\
\hyperlink{real}{real} - extract real part of a complex scalar, vector or matrix\\
\hyperlink{imag}{imag} - extract imaginary part of a complex scalar, vector or matrix\\
\hyperlink{angle, arg}{angle, arg} - compute argument of a complex (or real) scalar, vector or matrix\\
\hyperlink{isfinite}{isfinite} - check for "Finite" entries \\
\hyperlink{isnan}{isnan} - check for "NaN'' entries \\
\hyperlink{isnan}{isinf} - check for "Infinite entries \\
\hyperlink{nearfloat}{nearfloat} - get previous or next floating-point number\\
\hyperlink{primes}{primes} - computes prime numbers\\
\hyperlink{isprime}{isprime} - primality test\\
\hyperlink{factor}{factor} - prime factorization of an integer\\
\hyperlink{rounding}{ceil} - {smallest integral value not less than argument} \\
\hyperlink{rounding}{fix} - {integer part}\\
\hyperlink{rounding}{floor} - {largest integral value not greater than argument}\\
\hyperlink{rounding}{int} - {integer part} \\
\hyperlink{rounding}{round} - {round to nearest integer, away from zero}
\end{quote}

\input numbers/format.tex
\input numbers/number_properties.tex
\input numbers/frexp.tex
\input numbers/isreal.tex 
\input numbers/complex.tex 
\input numbers/angle.tex 
\input numbers/isnan.tex 
\input numbers/nearfloat.tex 
\input numbers/primes.tex 
\input numbers/isprime.tex 
\input numbers/factor.tex 
\input numbers/rounding.tex

