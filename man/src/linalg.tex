\chapter*{linear algebra}\addcontentsline{toc}{chapter}{linear algebra}

\HCode{<hr>}
Linear system solving
\begin{quote}
\noindent
\hyperlink{inv}{inv} -  inverse of a square matrix  \\
\hyperlink{solve}{solve} - solve a linear system \\
\hyperlink{lsmr}{lsmr} - iterative solver for least-squares problems \\
\end{quote}

\HCode{<hr>}
Matrix factorizations

\begin{quote}
\noindent
\hyperlink{fullrk}{fullrf} - full rank factorization \\ 
\hyperlink{fullrfk}{fullrfk} - {full rank factorization of $A^k$} \\
\hyperlink{lu}{lu} - LU factorization (for full matrix) \\
\hyperlink{chol}{chol} - Cholesky factorization (for full matrix) \\
\hyperlink{Cholmod}{Cholmod},\;
\hyperlink{cholmod_create}{cholmod\_create} - Cholesky or LDLt factorization (for sparse matrix) \\
\hyperlink{colcomp}{colcomp} - {column compression, kernel, nullspace} \\ 
\hyperlink{qr}{qr} - QR factorization \\ 
\hyperlink{umfpack_create}{umfpack\_create},\;
\hyperlink{umfpack_solve}{umfpack\_solve} - LU factorization (for sparse matrix) \\
\hyperlink{range}{range} - range (span) of $A^k$ \\ 
\hyperlink{rowcomp}{rowcomp} - {performs a row compression of a matrix}
\end{quote}

\HCode{<hr>}
Eigenvalues, Singular values and associated factorizations

\begin{quote}
\noindent
\hyperlink{svd}{svd} - singular value decomposition \\
\hyperlink{spec}{spec} - matrix spectrum, eigenvalues and eigenvectors\\
\hyperlink{schur}{schur} - Schur decomposition of a matrix \\
\hyperlink{qz}{qz} - Schur decomposition of pencils \\
\end{quote}


\HCode{<hr>}
Matrix functions

\begin{quote}
\noindent
\hyperlink{expm}{expm} - exponential of a square matrix \\ %
\hyperlink{sqrtm}{sqrtm} - square root of a square matrix \\ %
\hyperlink{logm}{logm} - logarithm of a square matrix \\ %
\end{quote}


\HCode{<hr>}
Subspaces, properties, norm,\dots

\begin{quote}
\noindent
\hyperlink{kernel}{kernel} - computes an orthonormal basis of the null space of a matrix\\
\hyperlink{orth}{orth} - computes an orthonormal basis of the range of a matrix\\
\hyperlink{rank}{rank} - Numerical rank calculation \\
\hyperlink{rcond}{rcond} - estimation of the reciprocal condition in 1-norm\\
\hyperlink{norm}{norm} - vector and matrix norm\\
\hyperlink{issymmetric}{issymmetric} - symmetry or hermitian test\\
\hyperlink{istriangular}{istriangular} - upper or lower triangular test\\
\hyperlink{lower_upper_bandwidths}{lower\_upper\_bandwidths} - computes lower and upper bandwidths\\
\hyperlink{pmult}{pmult} - special matrix multiplication\\
\end{quote}

\HCode{<hr>}
Matrices zoology\dots

\begin{quote}
\noindent
\hyperlink{hankel}{hankel} - Hankel matrix \\
\hyperlink{testm}{testm} - Test matrices  \\
\hyperlink{testmatrix}{testmatrix} - Test matrices  \\
\hyperlink{toeplitz}{toeplitz} - Toeplitz matrix \\
\end{quote}

\input linalg/inv.tex 
\input linalg/solve.tex 
\input linalg/lsmr.tex 

\input linalg/lu.tex 
\input linalg/chol.tex 
\input linalg/colcomp.tex 
\input linalg/qr.tex 
\input linalg/rowcomp.tex 
\input linalg/fullrf.tex 
\input linalg/fullrfk.tex 
\input linalg/range.tex 

\input linalg/svd.tex
\input linalg/spec.tex 
\input linalg/schur.tex
\input linalg/qz.tex
\input linalg/logm.tex
\input linalg/expm.tex
\input linalg/kernel.tex
\input linalg/orth.tex
\input linalg/rank.tex
\input linalg/rcond.tex 
\input linalg/norm.tex 
\input linalg/properties.tex 
\input linalg/pmult.tex 
\input linalg/toeplitz.tex 
\input linalg/testm.tex 


\HCode{<hr>}
Advanced functions

\begin{quote}
\noindent
\hyperlink{hankel}{hankel} - Hankel matrix \\
\hyperlink{arl2}{arl2} - SISO model realization by L2 transfer approximation \\ 
\hyperlink{aff2ab}{aff2ab} - {linear (affine) function to A,b conversion} \\  
\hyperlink{coff}{coff} - {resolvent (cofactor method)  } \\  
\hyperlink{cond}{cond} - {condition number} \\  
\hyperlink{givens}{givens} - {Givens transformation} \\  
\hyperlink{glever}{glever} - {inverse of matrix pencil} \\  
\hyperlink{householder}{householder} - {Householder orthogonal reflexion matrix} \\  
\hyperlink{kroneck}{kroneck} - {Kronecker form of matrix pencil} \\  
\hyperlink{linsolve}{linsolve} - {linear equation solver} \\  
\hyperlink{nlev}{nlev} - {Leverrier's algorithm} \\  
\hyperlink{pbig}{pbig} - {eigen-projection} \\  
\hyperlink{pen2ea}{pen2ea} - {pencil to E,A conversion} \\  
\hyperlink{pencan}{pencan} - {canonical form of matrix pencil} \\  
\hyperlink{penlaur}{penlaur} - {Laurent coefficients of matrix pencil} \\  
\hyperlink{polar}{polar} - {polar form} \\  
\hyperlink{projspec}{projspec} - {spectral operators} \\  
\hyperlink{proj}{proj} - {projection} \\  
\hyperlink{psmall}{psmall} - {spectral projection} \\  
\hyperlink{quaskro}{quaskro} - {quasi-Kronecker form} \\  
\hyperlink{randpencil}{randpencil} - {random pencil} \\  
\hyperlink{rowshuff}{rowshuff} - {shuffle algorithm} \\  
\hyperlink{rref}{rref} - {computes  matrix row echelon form by lu transformations} \\  
\hyperlink{spaninter}{spaninter} - {subspace intersection} \\  
\hyperlink{spanplus}{spanplus} - {sum of subspaces} \\  
\hyperlink{spantwo}{spantwo} - {sum and intersection of subspaces} \\  
\hyperlink{sqroot}{sqroot} - {W*W' hermitian factorization} \\  
\hyperlink{sva}{sva} - {singular value approximation} \\  
\hyperlink{trace}{trace}
\end{quote}

% \input linalg-sci/aff2ab.tex
% \input linalg-sci/arl2.tex
% \input linalg-sci/coff.tex
% \input linalg-sci/cond.tex
% \input linalg-sci/givens.tex
% \input linalg-sci/glever.tex
% \input linalg-sci/householder.tex
% \input linalg-sci/kroneck.tex
% \input linalg-sci/linsolve.tex
% \input linalg-sci/nlev.tex
% \input linalg-sci/pbig.tex
% \input linalg-sci/pen2ea.tex
% \input linalg-sci/pencan.tex
% \input linalg-sci/penlaur.tex
% \input linalg-sci/polar.tex
% \input linalg-sci/projspec.tex
% \input linalg-sci/proj.tex
% \input linalg-sci/psmall.tex
% \input linalg-sci/quaskro.tex
% \input linalg-sci/randpencil.tex
% \input linalg-sci/rowshuff.tex
% \input linalg-sci/rref.tex
% \input linalg-sci/spaninter.tex
% \input linalg-sci/spanplus.tex
% \input linalg-sci/spantwo.tex
% \input linalg-sci/sqroot.tex
% \input linalg-sci/sva.tex
% \input linalg-sci/trace.tex
