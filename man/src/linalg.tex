\chapter*{linear algebra}\addcontentsline{toc}{chapter}{linear algebra}

\begin{quote}
\noindent
\hyperlink{qr}{qr} - QR factorization \\
\hyperlink{lu}{lu} - LU factorization (for full matrix) \\
\hyperlink{umfpack_create}{umfpack\_create},\hyperlink{umfpack_solve}{umfpack\_solve}
- LU factorization (for sparse matrix) \\
\hyperlink{chol}{chol} - Cholesky factorization (for full matrix) \\
\hyperlink{Cholmod}{Cholmod}, \hyperlink{cholmod_create}{cholmod\_create} - Cholesky or LDLt factorization (for sparse matrix) \\
\hyperlink{spec}{spec} - matrix spectrum, eigenvalues and eigenvectors\\
\hyperlink{solve}{solve} - solve a linear system\\
\hyperlink{kernel}{kernel} - computes an orthonormal basis of the null space of a matrix\\
\hyperlink{orth}{orth} - computes an orthonormal basis of the range of a matrix\\
\hyperlink{norm}{norm} - vector and matrix norm\\
\hyperlink{rcond}{rcond} - estimation of the reciprocal condition in 1-norm\\
\hyperlink{issymmetric}{issymmetric} - symmetry or hermitian test\\
\hyperlink{istriangular}{istriangular} - upper or lower triangular test\\
\hyperlink{lower_upper_bandwidths}{lower\_upper\_bandwidths} - computes lower and upper bandwidths\\
\hyperlink{pmult}{pmult} - special matrix multiplication\\
\hyperlink{rank}{rank} - Numerical rank calculation \\
\hyperlink{schur}{schur} - Schur decomposition of a matrix \\
\end{quote}

\input linalg/qr.tex 
\input linalg/lu.tex 
\input linalg/chol.tex 
\input linalg/spec.tex 
\input linalg/solve.tex 
\input linalg/kernel.tex
\input linalg/orth.tex
\input linalg/norm.tex 
\input linalg/rcond.tex 
\input linalg/properties.tex 
\input linalg/pmult.tex 
\input linalg/rank.tex
\input linalg/schur.tex

