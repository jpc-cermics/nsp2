\mansection{mean}

\begin{mandesc}
  \short{mean}{arithmetic mean, arithmetic trimmed mean and weighted mean}
\end{mandesc}

%-- Calling sequence section
\begin{calling_sequence}
\begin{verbatim}
  mu = mean(X)
  mu = mean(X, dim=dimarg, skip_nan=b, weights=wgt, trim=s)  
  mu = mean(X, dimarg, skip_nan=b, weights=wgt, trim=s)  
\end{verbatim}
\end{calling_sequence}
%-- Parameters
\begin{parameters}
  \begin{varlist}
    \vname{X}: vector or matrix (complex vectors or matrices are allowed for non trimmed mean (ie for trim=0))
    \vname{dim=dimarg}: A string chosen among \verb+'M'+, \verb+'m'+, \verb+'*'+,\verb+'full'+, \verb+'FULL'+, \verb+'row'+,
    \verb+'ROW'+, \verb+'col'+, \verb+'COL'+ or an non ambiguous abbreviation or an integer. 
    This argument is optional and if omitted 'full' is assumed.
    \vname{skip_nan=b}: boolean scalar (default is \verb+%f+).
    \vname{trim=s}: real scalar in $[0,0.5]$.
    \vname{weights=wgt}: non negative vector or matrix of same dimensions than \verb+X+
  \end{varlist}
\end{parameters}

\begin{mandescription}
  \verb+mean+ computes the arithmetic mean:
$$
    \mu = \frac{1}{n} \sum_{k=1}^n X_k 
$$ 
trimmed or not (for a trimmed mean, a part of extremal values at both ends are discarded) 
or the weighted mean:
$$
    \mu = \sum_{k=1}^n w_k X_k 
$$
of the elements of a vector or matrix (in this last case the weights should be provided
using the weights option). Using the dim option described you can compute either
the mean of all the elements or the mean of each row or each column. 

\itemdesc{dim option} 
  The dim argument (default full) gives the dimension to be used for performing the mean operation:
  \begin{itemize}
    \item 'full' or 0: mean of all the elements 
    \item 'row' or 1: mean of each column (a row vector is returned)
    \item 'col' or 2: mean of each row (a column vector is returned)
    \item 'm' (or -2): (for Matlab compatibility) mean along the first non 
          singleton dimension of the first argument.
  \end{itemize}

\itemdesc{skip\_nan option}
   This option is available for usual, weighted or trimmed mean : when it is \verb+%t+,  Nan values 
  (which stand for ``missing or not available values'') are not taking into account in the
   mean computation.

\itemdesc{trim option}
  When the trim parameter $s$ is not zero, a trimmed mean is computed, that is
  after having removed a fraction of $s$ (extremal) values at both ends of \verb+X+. For
  the maximal value ($s=0.5$) one gets the median. If you want to remove exactly $p$
  extremal values at both ends of a vector having $n$ elements use $s = p/n$.  
   
\itemdesc{weights option}
  This option lets to compute weighted means. The weights should be non negative. Inside
the function they are normalized such that they sum up to $1$.

\end{mandescription}
%--example 
\begin{examples}
\paragraph{example 1} empirical mean got from random samples of known probability distributions 
\begin{Verbatim}
n = 500;
X = randn(n,1);  // a sample from N(0,1)
mean(X)   // should tend to zero when n -> +oo

X = grand(n,1,"bin",100,0.35); // a sample from Bin(100,0.35)
mean(X)   // should tend to 100*0.35 when n -> +oo

X = grand(n,1,"geom",0.2); // a sample from G(0.2)
mean(X)   // should tend to 1/0.2 = 5 when n -> +oo
\end{Verbatim}

\paragraph{example 2} mean of each row and each column
\begin{Verbatim}
X = rand(6,4);
// mean of the columns
mean(X,dim=1)
// mean of the rows
mean(X,dim=2)
\end{Verbatim}

\paragraph{example 3} trimmed mean
\begin{Verbatim}
X = [-7,1,0.0.5,-0.4,0.2,0.9,1.2,12];
// usual mean
mean(X)
// trimmed mean (1 extremal value removed at both ends)
mean(X,trim=1/length(X))
\end{Verbatim}

\paragraph{example 4} weighted mean
\begin{Verbatim}
n = 20; p = 0.4
X = 0:n;
wgt = pdf("bin",X, n,p);
mean(X, weights=wgt)  // should be exactly n*p=0.8
\end{Verbatim}


\end{examples}

%-- see also
\begin{manseealso}
   \manlink{median}{median}, \manlink{var}{var}, \manlink{std}{std}
\end{manseealso}

% -- Authors
\begin{authors}
  Jerome Lelong, Bruno Pincon
\end{authors}
