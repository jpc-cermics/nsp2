% -*- mode: latex -*-
\mansection{cdf, icdf}
\begin{mandesc}
  \short{cdf}{cumulative distribution functions} \\
  \short{icdf}{inverse cumulative distribution (aka quantile) functions} \\
\end{mandesc}
%-- Calling sequence section
\begin{calling_sequence}
\begin{verbatim}
[P,Q]=cdf(dist_type, x [,p1,...,pk])  
x=icdf(dist_type, P, [,p1,...,pk], Q=Q)
\end{verbatim}
\end{calling_sequence}
  %-- Parameters
\begin{parameters}
  \begin{varlist}
   \vname{dist\_type}: a string giving the probability distribution (\verb!"bin"! for binomial, \verb!"nor"!
   for normal, \verb!"poi"! for Poisson, etc ...)
   \vname{p1, ..., pk}: (scalar) parameters (reals or integers) required to define the distribution \verb!dist_type!
    \vname{x}: scalar or vector or matrix with abscissae where the cdf function must be evaluated or
               result for the icdf function.
   \vname{P,Q}: results of the cdf function (in this case \verb+P+ and \verb+Q+ have the same dimensions than \verb+x+)
                or inputs of the icdf function ($Q$ being given through the named optional argument \verb+Q=+)
  \end{varlist}
  \end{parameters}
  
\begin{mandescription}

For a given distribution \verb+dist_type+ and its associated parameter(s) \verb+p1, ..., pk+ :
\begin{itemize}
\item \verb![P,Q]=cdf(dist,x[,p1,...,pk])! computes the cumulative distribution function at \verb+x+, that
is if $X$ is a random variate of the given distribution :
$$
       P = F(x) := Pr ( X \le x ); \quad Q = Pr ( X > x ) = 1 - P
$$

\item and \verb!x=icdf(dist_type, P, [,p1,...,pk], Q=Q)! computes the inverse cumulative distribution function
(aka the quantile function) at \verb+P+ (\verb+Q+ can be also provided, see here after). If $F$ is the associated
cdf, its (generalized) ``inverse'' is defined by:: 
$$
\mbox{ for } P \in (0,1), \qquad   icdf(P) = F^{-1}(P) := inf \{ x \in \R : P \le F(x) \} 
$$
\end{itemize}

\paragraph{Remarks}
\begin{itemize}
\item In exact arithmetic it is equivalent to compute (for cdf) or to provide (for icdf) $P$ or $Q$ as they are
      linked by the relation $P+Q=1$ and the default is to use only $P$ ($Q$ is the second output of cdf and
      an optional input for icdf). {\bf But} as the effective computation use floating point arithmetic it 
      can be useful to compute or provide $Q$ when $P$ is near $1$ to get better accuracy (see examples).
\item When \verb+dist_type = "bet" | "bin" | "nbn" | "chi" | "nch" | "f" | "nf" | "gam" | "nor" | "poi" | "t" | "nt"+ 
      the cdf and icdf functions are just easy drivers onto the corresponding cdfxxx functions and in
      most cases :
      \begin{quote}
         \verb+[P,Q]=cdf("xxx",x,p1,..,pk)+ is equivalent to \verb+[P,Q]=cdfxxx("PQ",x,p1,..,pk)+\\
         \verb+x=icdf("xxx",P,p1,..,pk,Q=Q)+ is equivalent to \verb+x=cdfxxx("X",p1,..,pk,P,Q)+
      \end{quote}
      But note that:
      \begin{enumerate}
      \item \verb+"X"+ is not always the switch key to get the inverse cdf with the cdfxxx functions.
      \item For cdf and icdf functions the parameters p1,..,pk should be scalars while they should have
            the same dimensions than x, P, and Q for the cdfxxx functions.
      \item There are some slight differences for the icdf of discrete distributions (\verb+"bin", "nbn", "poi"+): 
            the associated cdfxxx functions (cdfbin, cdfnbn and cdfpoi) use internally a continuous extension of 
            the discrete cdf which is considered to compute the inverse cdf (but not the cdf as 
            \verb+cdfxxx("PQ",x,...)+ is applied internally on floor(x)). With the icdf function an additionnal 
            step is done to apply the definition of the generalized inverse given before and integer values are returned.
      \item The cdfxxx functions have the additionnal following feature : the value of a distribution parameter 
            can be computed when $x$ and $P$ (and the eventual  other distribution parameters) are known.
      \end{enumerate}
\end{itemize}

\paragraph{Supported distributions}
\begin{itemize}

\item \itemdesc{beta}  
\verb![P,Q]=cdf("bet",x,a,b)! and \verb!x=icdf("bet",P,a,b,Q=Q)! compute the
  cdf and icdf of the \hyperlink{betapdf}{beta distribution} with shape parameters $a > 0$ and $b > 0$.
 
\item \itemdesc{binomial}
\item \verb![P,Q]=cdf("bin",x,n,p)! and \verb!x=icdf("bin",P,n,p,Q=Q)! compute the cdf and icdf
  of the  \hyperlink{binpdf}{binomial distribution} with parameters $p \in [0,1]$ and $n$
  (positive integer). 

\item \itemdesc{negative binomial} \verb![P,Q]=cdf("nbn",x,r,p)! and \verb!x=icdf("bin",P,n,p,Q=Q)! compute
  the cdf and icdf of the \hyperlink{nbnpdf}{negative binomial distribution} with parameters $r > 0$ ($r$ not
  necessarily integer) and $p \in (0,1]$. 


\item \itemdesc{Cauchy} \verb![P,Q]=cdf("cau",x,sigma)! and  \verb!x=icdf("cau",P,sigma,Q=Q)! compute the 
  cdf and icdf of the  \hyperlink{caupdf}{Cauchy distribution} parameter $\sigma > 0$.

\item \itemdesc{chi square} \verb![P,Q]=cdf("chi",x,nu)! and  \verb!x=icdf("chi",P,nu,Q=Q)! compute the
cdf and icdf of the \hyperlink{chipdf}{chisquare distribution} with parameter $\nu > 0$ (number of dof).

\item \itemdesc{non central chi square} \verb![P,Q]=cdf("nch",x,nu,lambda)! and  \verb!x=icdf("nch",P,nu,lambda,Q=Q)! compute the
cdf and icdf of the non central chi square distribution %\hyperlink{chipdf}{chisquare distribution} 
with parameter $\nu > 0$ (number of dof) and  $\lambda > 0$ (non centrality parameter).

\item \itemdesc{exponential} \verb![P,Q]=cdf("exp",x,tau)! and \verb!x=icdf("exp",P,tau,Q=Q)! compute the cdf
and icdf of the  \hyperlink{exppdf}{exponential distribution} with parameter $\tau > 0$.
Caution: the parameter is more frequently $\lambda = 1/\tau$.

\item \itemdesc{F variance ratio} \verb![P,Q]=cdf("f",x,nu1,nu2)! and \verb!x=icdf("f",P,nu1,nu2,Q=Q)! compute
the cdf and icdf of the variance ratio distribution with parameters $\nu_1 > 0$ and $\nu_2 >
  0$ (respectively number of dof in the numerator and denominator). 

\item \itemdesc{non central F variance ratio} \verb![P,Q]=cdf("nf",x,nu1,nu2,lambda)! and \verb!x=icdf("f",P,nu1,nu2,lambda,Q=Q)! compute
the cdf and icdf of the non central F variance ratio distribution with parameters $\nu_1 > 0$, $\nu_2 > 0$ 
(respectively number of dof in the numerator and denominator) and $lambda \ge 0$ (non  centrality parameter).


\item \itemdesc{gamma} \verb![P,Q]=cdf("gam",x,a,b)! and  \verb!x=icdf("gam",P,a,b,Q=Q)! compute the cdf
 and icdf of the \hyperlink{gampdf}{gamma distribution} with parameters $a > 0$ (shape parameter) and $b \ge 0$
  (rate parameter). Caution: second parameter is often a scale parameter, says $s = 1/b$.


\item \itemdesc{Gauss Laplace} \verb![P,Q]=cdf("nor",x,mu,sigma)! and \verb!x=icdf("nor",P,mu,sigma,Q=Q)! compute
the cdf and icdf of the  \hyperlink{norpdf}{normal distribution} with parameters $\mu$ and $\sigma > 0$.


\item \itemdesc{geometric} \verb![P,Q]=cdf("geom",x,p)! and \verb!x=icdf("geom",P,p,Q=Q)! compute
the cdf and icdf of the \hyperlink{geompdf}{geometric distribution} with parameter $p \in (0,1]$.

\item \itemdesc{Kolmogorov} \verb![P,Q]=cdf("k",x,n)! computes the cdf of the Kolmogorov distribution 
with parameter $n$ (positive integer). Note that the icdf is not supported.

\item \itemdesc{limit Kolmogorov} \verb![P,Q]=cdf("klim",x)! and  \verb!x=icdf("klim",P,Q=Q)! compute the cdf 
and icdf of the limit Kolmogorov distribution (Kolmogorov distribution when $n \rightarrow \infty$). 

\item \itemdesc{Laplace} \verb![P,Q]=cdf("lap",x,a)! and \verb!x=icdf("lap",P,a,Q=Q)! compute the cdf
and icdf of the \hyperlink{lappdf}{Laplace distribution} with parameter $a > 0$.

\item \itemdesc{logistic} \verb![P,Q]=cdf("logi",x,a,b)! and \verb!x=icdf("logi",P,a,b,Q=Q)! compute the
cdf and icdf of the \hyperlink{logipdf}{logistic distribution} with parameter $a$ and $b > 0$.

\item \itemdesc{log normal} \verb![P,Q]=cdf("logn",x,mu,sigma)! and  \verb!x=icdf("logn",P,mu,sigma,Q=Q)! compute
the cdf and icdf of the  \hyperlink{lognpdf}{log normal distribution} with parameters $\mu$ and $\sigma > 0$.

\item \itemdesc{Pareto} \verb![P,Q]=cdf("par",x,a,b)! and  \verb!x=icdf("par",P,a,b,Q=Q)! compute the
cdf and icdf of the \hyperlink{parpdf}{Pareto  distribution} with shape parameter $a > 0$ and scale parameter $b > 0$.

\item \itemdesc{Poisson} \verb![P,Q]=cdf("poi",x,mu)! and \verb!x=icdf("poi",P,mu,Q=Q)! compute the
cdf and icdf of the \hyperlink{poipdf}{Poisson distribution} with parameter $\mu \ge 0$.


\item \itemdesc{Rayleigh} \verb![P,Q]=cdf("ray",x,sigma)! and \verb!x=icdf("ray",P,sigma,Q=Q)! compute
the cdf and icdf of the \hyperlink{raypdf}{Rayleigh distribution} with parameter $\sigma > 0$.

\item \itemdesc{tail Rayleigh} \verb![P,Q]=cdf("tray",x,sigma, a)! and \verb!x=icdf("tray",P,sigma,a,Q=Q)! compute
the cdf and icdf of the \hyperlink{traypdf}{tail (from $a$) of the Rayleigh distribution} with parameters $\sigma > 0$ 
and $a \ge 0$.

\item \itemdesc{Student's t} \verb![P,Q]=cdf("t",x,nu)! and \verb!x=icdf("t",P,nu,Q=Q)! compute the
cdf and icdf of the \hyperlink{tpdf}{t distribution} with parameter $\nu > 0$ (number of dof).


\item \itemdesc{non central Student's t} \verb![P,Q]=cdf("t",x,nu,delta)! and \verb!x=icdf("t",P,nu,delta,Q=Q)! compute the
cdf and icdf of the non central t distribution with parameters $\nu > 0$ (number of dof) and
$\delta$ (non centrality parameter).


\item \itemdesc{uniform (uin)} \verb![P,Q]=cdf("uin",x,n1,n2)! and \verb!x=icdf("uin",P,n1,n2,Q=Q)! compute
the cdf and icdf of the \hyperlink{uinpdf}{uniform distribution on integers $n_1,n_1+1,\dots,n_2$}.

\item \itemdesc{uniform (unf)} \verb![P,Q]=cdf("unf",x,a,b)!  and \verb!x=icdf("unf",P,a,b,Q=Q)! compute
the cdf and icdf of the \hyperlink{unfpdf}{uniform distribution on the interval $[a,b]$} ($a < b$).

\item \itemdesc{Weibull} \verb![P,Q]=cdf("wei",x,a,b)! and \verb!x=icdf("wei",P,a,b,Q=Q)! compute the
cdf and icdf of the \hyperlink{weipdf}{Weibull  distribution} with scale parameter $a > 0$ and shape parameter 
$b > 0$.

\end{itemize}

\end{mandescription}


\begin{examples}
  
\paragraph{example 1} an example with the gaussian distribution
\begin{mintednsp}{nsp}
// plot pdf and cdf on [-5,5]
x = linspace(-5,5,200)';
y = pdf("nor",x,0,1);
p = cdf("nor",x,0,1);
xclear(); plot2d(x,[y,p],style=[2,5],leg="pdf@cdf");
xtitle("The N(0,1) density and cumulative distribution functions")

// apply icdf on p should return x
xx = icdf("nor",p,0,1);
max(abs(x-xx)./abs(x))  // max relative error
// if one use q, the error between x and xx should be lower
[p,q] = cdf("nor",x,0,1);
xx = icdf("nor",p,0,1,Q=q);
max(abs(x-xx)./abs(x))
\end{mintednsp}
  
\paragraph{example 2} an example with the binomial distribution
\begin{mintednsp}{nsp}
n = 50; p = 0.45; // we work with Bin(n=50,p=0.45)
X = -1:n+1; Xp = nearfloat("pred",X);
x = redim([Xp;X],-1,1);
[P,Q] = cdf("bin",x,n,p);
// compare with the N(n*p,sqrt(n*p*(1-p))
mu = n*p; sigma = sqrt(n*p*(1-p));
[Pn,Qn] = cdf("nor",x,mu,sigma);
xclear(); plot2d(x,[P,Pn],style=[2,5],leg="bin@nor");
xtitle("The Bin(n,p) and N(n*p,sqrt(n*p*(1-p)) cumulative distribution functions")

// differences between icdf("bin",..) and cdfbin("S",...)
// (cdfbin use a continuous extension) 
m = 2^13+1;
P = linspace(0,1,m)';
x = icdf("bin",P,n,p);
xx = cdfbin("S",n*ones(m,1),p*ones(m,1),(1-p)*ones(m,1),P,1-P);
xclear(); plot2d(P,[x,xx],style=[2,5],leg="icdf(''bin'',..)@cdfbin(''S'',...)");
xtitle("differences between icdf(""bin"",..) and cdfbin(""S"",...)")

// with x=0:n, icdf(cdf(x)) should be equal to x
// 1/ try without using Q for icdf
x = 0:n;
[P,Q] = cdf("bin",x,n,p);
xx = icdf("bin",P,n,p);
// we don't recover x due to numerical floating point computation:
x.equal[xx]
// 2/ now use Q: we get additionnal accuracy and we recover x
xx = icdf("bin",P,n,p,Q=Q);
x.equal[xx]
\end{mintednsp}
 
\end{examples}

\begin{manseealso}
  \manlink{grand}{grand}, \manlink{pdf}{pdf}, \manlink{dist\_stat}{dist_stat}, 
  \manlink{cdfbet}{cdfbet}, \manlink{cdfbin}{cdfbin}, \manlink{cdfnbn}{cdfnbn},
  \manlink{cdfchi}{cdfchi}, \manlink{cdfchn}{cdfchn}, \manlink{cdff}{cdff}, \manlink{cdffnc}{cdffnc}, 
  \manlink{cdfgam}{cdfgam}, \manlink{cdfnor}{cdfnor}, \manlink{cdfpoi}{cdfpoi}, \manlink{cdft}{cdft}, \manlink{cdftnc}{cdftnc}.
\end{manseealso}

%-- Authors

\begin{authors}
  Buno Pincon
\end{authors}

