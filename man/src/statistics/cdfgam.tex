% -*- mode: latex -*-
\mansection{cdfgam}
\begin{mandesc}
  \short{cdfgam}{cumulative distribution function gamma distribution}
\end{mandesc}
\index{cdfgam}\label{cdfgam}
  %-- Calling sequence section
\begin{calling_sequence}
\begin{verbatim}
[P,Q]=cdfgam("PQ",X,Shape,Rate)  
[X]=cdfgam("X",Shape,Rate,P,Q)  
[Shape]=cdfgam("Shape",Rate,P,Q,X)  
[Rate]=cdfgam("Rate",P,Q,X,Shape)  
\end{verbatim}
\end{calling_sequence}
%-- Parameters
\begin{parameters}
  \begin{varlist}
    \vname{P,Q,X,Shape,Rate}: five real vectors of the same size.
    \vname{P,Q (Q=1-P)} the integral from 0 to X of the gamma density. Input range: [0,1].
    \vname{X}:  the upper limit of integration of the gamma density. Input range: [0, +infinity). Search range: [0,1E300]
      \vname{Shape}:  the shape parameter of the gamma density. Input range: (0, +infinity). Search range: [1E-300,1E300]
      \vname{Rate}:  the scale parameter of the gamma density. Input range: (0, +infinity). Search range: (1E-300,1E300]
  \end{varlist}
\end{parameters}
\begin{mandescription}
  Calculates any one parameter of the gamma distribution given values for the others.
  The gamma density is given by for $x>0$ by~:
  \begin{equation}
    x^{\alpha-1} \frac{ \beta^\alpha  e^{- \beta x}}{\Gamma(\alpha)}
  \end{equation}
  where $\alpha = \mbox{\tt Shape}$ and $\beta = \mbox{\tt Rate}$. 
  Note that usually the second parameter of the gamma distribution 
  is a \verb!Scale! parameter equal to the inverse of the \verb!Rate! 
  (so you have to use \verb!Rate = 1/Shape!).

  Cumulative distribution function (P) is calculated directly by
  the code associated with: DiDinato, A. R. and Morris, A. H. 
  Computation of the  incomplete
  gamma function  ratios  and their  inverse.   ACM  Trans.  Math.
  Softw. 12 (1986), 377-393.
  Computation of other parameters involve a seach for a value that
  produces  the desired  value  of P.   The search relies  on  the
  monotinicity of P with the other parameter.
\end{mandescription}

\begin{program}\HCode{x=linspace(0,20,100);\Hnewline
shape =1;\Hnewline
scale =0.5;\Hnewline
y= cdfgam("PQ",x,shape*ones(x),scale*ones(x));\Hnewline
\Hnewline
function y=f(x)\Hnewline
  y = x^(shape-1)*scale^shape*exp(-x*scale)./gamma(shape);\Hnewline
endfunction;\Hnewline
\Hnewline
z=x;\Hnewline
for i=1:length(x)\Hnewline
  z(i) = intg(0,x(i),f);\Hnewline
end\Hnewline
err=max(abs(y-z));}
\end{program}

\begin{authors}
  Nsp interface by Jean-Philippe Chancelier. Code from DCDFLIB: 
  Library of Fortran Routines for Cumulative Distribution
  Functions, Inverses, and Other Parameters (February, 1994)
  Barry W. Brown, James Lovato and Kathy Russell. The University of Texas.
\end{authors}
