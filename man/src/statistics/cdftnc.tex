% -*- mode: latex -*-
\mansection{cdftnc}
\begin{mandesc}
  \short{cdftnc}{cumulative distribution function Student's non-central T distribution}
\end{mandesc}
\index{cdftnc}\label{cdftnc}
%-- Calling sequence section
\begin{calling_sequence}
\begin{verbatim}
  [P,Q]=cdftnc("PQ",T,Df,Pnonc)  
  [T]=cdftnc("T",Df,Pnonc,P,Q)  
  [Df]=cdftnc("Df",Pnonc,P,Q,T)  
\end{verbatim}
\end{calling_sequence}
%-- Parameters
\begin{parameters}
  \begin{varlist}
    \vname{P,Q,T,Df,PNONC}: six real vectors of the same size.
    \vname{P,Q (Q=1-P)}: the integral from -infinity to t of the t-density. Input range: (0,1].
      \vname{T}: upper limit of integration of the t-density. Input range: ( -infinity, +infinity). Search range: [ -1E150, 1E150 ]
      \vname{DF:} degrees of freedom of the t-distribution. Input range: (0 , +infinity). Search range: [1e-300, 1E10]
  \end{varlist}
\end{parameters}
\begin{mandescription}
  Calculates any one parameter of the non-central T distribution given
  values for the others.
  Upper tail    of  the  cumulative  noncentral t is calculated using 
  formulae  from page 532  of Johnson, Kotz,  Balakrishnan, Coninuous 
  Univariate Distributions, Vol 2, 2nd Edition.  Wiley (1995) 
  Computation of other parameters involve a seach for a value that 
  produces  the desired  value  of P.   The search relies  on  the 
  monotinicity of P with the other parameter. 
\end{mandescription}

\begin{authors}
  Nsp interface by Jean-Philippe Chancelier. Code from DCDFLIB: 
  Library of Fortran Routines for Cumulative Distribution
  Functions, Inverses, and Other Parameters (February, 1994)
  Barry W. Brown, James Lovato and Kathy Russell. The University of Texas.
\end{authors}
