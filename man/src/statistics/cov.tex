\mansection{cov}

\begin{mandesc}
  \short{cov}{covariance matrix and weighted covariance matrix} \\
\end{mandesc}

%-- Calling sequence section
\begin{calling_sequence}
\begin{verbatim}
  // first form
  mcov = cov(X)
  [mcov, mcor] = cov(X)
  mcov = cov(X, skip_nan=b, weights=w, unbiaised=b)
  [mcov, mcor] = cov(X, skip_nan=b, weights=w, unbiaised=b)

  // second form
  mcov = cov(X,Y)
  [mcov, mcor] = cov(X,Y)
  mcov = cov(X,Y, skip_nan=b, weights=w, unbiaised=b)
  [mcov, mcor] = cov(X,Y, skip_nan=b, weights=w, unbiaised=b)
\end{verbatim}
\end{calling_sequence}
%-- Parameters
\begin{parameters}
  \begin{varlist}
    \vname{X, Y}: real vectors or matrices
    \vname{skip_nan=b}: boolean scalar (default is \verb+%f+).
    \vname{weights=w}: non negative vector
    \vname{unbiased=b}:  boolean scalar (default is \verb+%t+).
  \end{varlist}
\end{parameters}

\begin{mandescription}
  \verb+cov+ works on an array $X$ of size $m \times n$ where each column corresponds to
 a ``variable'' and each row to an ``observation'' (a $1 \times n$ vector with a value 
for each of the $n$ variables). \verb+cov+ computes an estimation of the covariance
matrix between the underlying random variables using the formula:
$$
    mcov_{i,j} = \frac{1}{den} \sum_{k=1}^m (X_{k,i} - \bar{X}_i) (X_{k,j} - \bar{X}_j) \quad 1 \le
   i,j \le n 
$$ 
with:
$$
   \bar{X}_i = \frac{1}{m} \sum_{k=1}^m X_{k,i}
   \quad \mbox{ and } \quad  den = \left\{ \begin{array}{cl}  m-1 & \mbox{ if unbiased is true}\\
                                                       m   & \mbox{ otherwise} \end{array}\right.
$$
But if the  weights $w$ are provided,  \verb+cov+ computes an estimation of the weighted covariance
matrix using the following formula:
$$
     mcov_{i,j} = \frac{1}{1 - c} \sum_{k=1}^m w_k (X_{k,i} - \bar{X}_i) (X_{k,j} - \bar{X}_j) \quad 1 \le
   i,j \le n  
$$
with:
$$
    \bar{X}_i =  \sum_{k=1}^m  w_k X_{k,i}
   \quad \mbox{ and } \quad  c = \left\{ \begin{array}{cl} \sum_{p=1}^m w_p^2  & \mbox{ if unbiased is true}\\
                                                       0   & \mbox{ otherwise} \end{array}\right.
$$
These matrices are symmetric ($mcov_{i,j} = mcov_{j,i}$) and semi-positive definite.

\noindent{\bf Remarks:}
\begin{itemize}
\item Additionnaly an estimate of the correlation matrix can be provided as second argument.

\item The second form  \verb+cov(X,Y [,options])+ computes the covariance matrix
of two variables, that is $X$ and $Y$ are considered as two big column vectors (so \verb+cov(X,Y)+
is equivalent to \verb+cov([X(:),Y(:)])+. 
\end{itemize}


\itemdesc{skip\_nan option}
   This option is available for both the usual or weighted covariance.
When it is \verb+%t+, all observations (that is all rows) with at least 
one nan values are not taking into account in the computation.
Nan values stand for ``missing or not available values''.
   
\itemdesc{weights option}
  This option lets to compute weighted covariance matrix. The weights should be non negative. Inside
the function they are normalized such that they sum up to $1$.
   
\end{mandescription}

%--example 
\begin{examples}
\begin{Verbatim}
m = 500;
Mu = [-2;5];
Cv = [0.5, 0.3; 0.3, 2];
// generate m random sample vectors of the multinormal distribution
// with mean Mu and covariance matrix Cv 
X = grand(m,"mn",Mu,Cv)'; // transpose to get the good format for cov
estimed_mean = mean(X,dim=1)  // should tend to Mu when m -> + oo
estimed_cov = cov(X)          // should tend to Cv when m -> + oo
\end{Verbatim}
\end{examples}

%-- see also
\begin{manseealso}
  \manlink{var}{var}, \manlink{mean}{mean}, \manlink{median}{median}
\end{manseealso}

% -- Authors
\begin{authors}
  Bruno Pincon
\end{authors}
