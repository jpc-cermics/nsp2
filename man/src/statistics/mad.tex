\mansection{mad}

\begin{mandesc}
  \short{mad}{median of absolute deviations around the median or mean of absolute deviations around the mean}
\end{mandesc}

%-- Calling sequence section
\begin{calling_sequence}
\begin{verbatim}
  md = mad(X)
  md = mad(X, dim=dimarg, skip_nan=b, flag=str)  
  md = mad(X, dimarg, skip_nan=b, flag=str)  
\end{verbatim}
\end{calling_sequence}
%-- Parameters
\begin{parameters}
  \begin{varlist}
    \vname{X}: real vector or matrix
    \vname{dim=dimarg}: a string chosen among \verb+'M'+, \verb+'m'+, \verb+'*'+,\verb+'full'+, \verb+'FULL'+, \verb+'row'+,
    \verb+'ROW'+, \verb+'col'+, \verb+'COL'+ or an non ambiguous abbreviation or an integer. 
    This argument is optional and if omitted 'full' is assumed.
    \vname{skip_nan=b}: boolean scalar (default is \verb+%f+).
    \vname{flag=str}: a string among \verb+"median"+ or \verb+"mean"+ or any non ambiguous abbreviation 
          (default is \verb+"median"+)
  \end{varlist}
\end{parameters}

\begin{mandescription}
 By default or when \verb+flag="median"+ \verb+mad+ computes the median of the absolute deviations
around the median of all the elements of \verb+X+, or of each row or each column of \verb+X+ depending upon
the parameter \verb+dim+.

 When \verb+flag="mean"+  \verb+mad+ computes the mean of the absolute deviations around the
mean.

 Both mad (median based or mean based) like \manlink{std}{std}, compute a measure 
of the statistical dispersion of the sample vector $X$ but are less sensible to 
outliers. 


\itemdesc{dim option} 
  The dim argument (default full) gives the dimension to be used for performing the mad operation:
  \begin{itemize}
    \item 'full' or 0: mad of all the elements 
    \item 'row' or 1: mad of each column (a row vector is returned)
    \item 'col' or 2: mad of each row (a column vector is returned)
    \item 'm' (or -2): (for Matlab compatibility) mad along the first non 
          singleton dimension of the first argument.
  \end{itemize}

\itemdesc{skip\_nan option}
   When this option is \verb+%t+,  Nan values (which stand for ``missing or not 
 available values'') are not taking into account in the computation.
\end{mandescription}
%--example 
\begin{examples}
\begin{Verbatim}
// use mad to estimate the std : if X are gaussian random variates
// then a consistent estimator is mad / cdfnor("X",0,1,3/4,1/4)
sigma= 2 
X = grand(1,40,"nor",2,sigma);
mad(X)/cdfnor("X",0,1,3/4,1/4)
std(X)
\end{Verbatim}
\end{examples}

%-- see also
\begin{manseealso}
   \manlink{mean}{mean}, \manlink{var}{var}, \manlink{std}{std}
\end{manseealso}

% -- Authors
\begin{authors}
 Bruno Pincon
\end{authors}
