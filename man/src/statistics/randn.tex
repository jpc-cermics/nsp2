% -*- mode: latex -*-
\mansection{randn}
\begin{mandesc}
  \short{randn}{generates random variates from the normal distribution}
\end{mandesc}
%-- Calling sequence section
\begin{calling_sequence}
\begin{verbatim}
X = randn()     // output only one random number 
X = randn(m, n)
X = randn(Mat)
\end{verbatim}
\end{calling_sequence}
  %-- Parameters
\begin{parameters}
  \begin{varlist}
    \vname{m, n}: integers, size of the wanted matrix \verb!X!
   \vname{Mat}: a matrix whom only the dimensions (say \verb!m x n!) are used
   \vname{X}: the resulting \verb!m x n! random matrix
  \end{varlist}
  \end{parameters}
  
\begin{mandescription}
  This function generates random variates from the  \hyperlink{norpdf}{normal distribution} with
 mean 0 and standard deviation 1. It is a shortcut for \verb+grand(m,n,"nor",0,1)+.
  The first calling sequence \verb+randn()+ returns only one random number. The
 third calling sequence \verb+randn(Mat)+ is equivalent to the second  
 \verb+randn(m,n)+ with the dimensions $m$ and $n$ from the matrix $Mat$.

  The internal code uses the ziggurat rejection method as explained in "Marsaglia 
G and WW Tsang, The Ziggurat Method for Generating Random Variables, Journal of Statistical 
 Software, vol. 5 (2000), no. 8"
 
\end{mandescription}

%-- see also
%--example 
\begin{examples}
\begin{mintednsp}{nsp}
n = 50000;
X = randn(n,1);  // a sample from N(0,1)

// draw an histogram
xbasc()
histplot(20,X)
// superpose the normal density
x = linspace(-4,4,200)';
y = pdf("nor",x,0,1);
plot2d(x,y,style=5)
\end{mintednsp}
\end{examples}

\begin{manseealso}
  \manlink{grand}{grand}, \manlink{pdf}{pdf}   
\end{manseealso}



