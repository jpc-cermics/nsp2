% -*- mode: latex -*-
\mansection{cdfbin}
\begin{mandesc}
  \short{cdfbin}{cumulative distribution function binomial distribution}
\end{mandesc}
\index{cdfbin}\label{cdfbin}
%-- Calling sequence section
\begin{calling_sequence}
\begin{verbatim}
[P,Q]=cdfbin("PQ",S,Xn,Pr,Ompr)      // compute cdf
[S]=cdfbin("S",Xn,Pr,Ompr,P,Q)       // compute inverse cdf (quantile function)
[Xn]=cdfbin("Xn",Pr,Ompr,P,Q,S)      // compute Xn parameter
[Pr,Ompr]=cdfbin("PrOmpr",P,Q,S,Xn)  // compute Pr and Ompr parameter
\end{verbatim}
\end{calling_sequence}
%-- Parameters
\begin{parameters}
  \begin{varlist}
    \vname{P,Q,S,Xn,Pr,Ompr}: six real vectors of the same size.
    \vname{P,Q (Q=1-P)}: the cumulation from 0 to S of the binomial distribution. (Probability of S or fewer successes in Xn trials each with probability of success Pr.) Input range: [0,1].
    \vname{S}: the number of successes observed. Search range: [0, XN]
    \vname{Xn}: the number of binomial trials. Input range: (0, +infinity). Search range: [1E-300, 1E300]
    \vname{Pr,Ompr (Ompr=1-Pr)}: the probability of success in each binomial trial. Input range: [0,1]. Search range: [1E-300,1]
  \end{varlist}
\end{parameters}

\begin{mandescription}
  Calculates any one parameter of the binomial distribution given values 
  for the others. Formula  26.5.24 of  Abramowitz and Stegun,  Handbook   of
  Mathematical   Functions (1966) is used  to reduce the  binomial
  distribution  to  the  cumulative beta distribution.
  Computation of other parameters involve a search for a value that
  produces  the desired  value  of P.
\end{mandescription}

\begin{examples}
\begin{Verbatim}
// compute then display the cumulative distribution function
Xn = 18; pr = 0.45;
x = 0:Xn+1; xp = nearfloat("pred",x);
x = [xp;x]; x.redim[-1,1]; 
v = ones(size(x));
P=cdfbin("PQ",x,Xn*v,pr*v,(1-pr)*v); // you can use cdf("bin",x,Xn,pr)
xbasc()
plot2d(x,P,style=2,leg="cdf(x)")
\end{Verbatim}
\end{examples}

\begin{authors}
  Nsp interface by Jean-Philippe Chancelier. Fortran code from DCDFLIB
  adapted in C and improved/modified by Bruno Pincon and Jean-Philippe Chancelier.  
  DCDFLIB: Library of Fortran Routines for Cumulative Distribution
  Functions, Inverses, and Other Parameters (February, 1994)
  Barry W. Brown, James Lovato and Kathy Russell. The University of Texas.
\end{authors}

