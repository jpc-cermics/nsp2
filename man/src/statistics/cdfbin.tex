% -*- mode: latex -*-
\mansection{cdfbin}
\begin{mandesc}
  \short{cdfbin}{cumulative distribution function Binomial distribution}
\end{mandesc}
\index{cdfbin}\label{cdfbin}
%-- Calling sequence section
\begin{calling_sequence}
\begin{verbatim}
[P,Q]=cdfbin("PQ",S,Xn,Pr,Ompr)  
[S]=cdfbin("S",Xn,Pr,Ompr,P,Q)  
[Xn]=cdfbin("Xn",Pr,Ompr,P,Q,S)  
[Pr,Ompr]=cdfbin("PrOmpr",P,Q,S,Xn)  
\end{verbatim}
\end{calling_sequence}
%-- Parameters
\begin{parameters}
  \begin{varlist}
    \vname{P,Q,S,Xn,Pr,Ompr}: six real vectors of the same size.
    \vname{P,Q (Q=1-P)}: the cumulation from 0 to S of the binomial distribution. (Probablility of S or fewer successes in XN trials each with probability of success PR.) Input range: [0,1].
    \vname{S}: the number of successes observed. Input range: [0, XN] Search range: [0, XN]
    \vname{Xn}: the number of binomial trials. Input range: (0, +infinity). Search range: [1E-300, 1E300]
    \vname{Pr,Ompr (Ompr=1-Pr)}: the probability of success in each binomial trial. Input range: [0,1]. Search range: [0,1]
  \end{varlist}
\end{parameters}

\begin{mandescription}
  Calculates any one parameter of the binomial distribution given values for the others.
  Formula  26.5.24    of   Abramowitz  and    Stegun,  Handbook   of
  Mathematical   Functions (1966) is   used  to reduce the  binomial
  distribution  to  the  cumulative incomplete    beta distribution.
  Computation of other parameters involve a seach for a value that
  produces  the desired  value  of P.   The search relies  on  the
  monotinicity of P with the other parameter.
\end{mandescription}
\begin{authors}
  Nsp interface by Jean-Philippe Chancelier. Code from DCDFLIB: 
  Library of Fortran Routines for Cumulative Distribution
  Functions, Inverses, and Other Parameters (February, 1994)
  Barry W. Brown, James Lovato and Kathy Russell. The University of Texas.
\end{authors}

