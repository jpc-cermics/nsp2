% -*- mode: latex -*-
\mansection{pdf}
\begin{mandesc}
  \short{pdf}{probability density or mass functions}
\end{mandesc}
%-- Calling sequence section
\begin{calling_sequence}
\begin{verbatim}
y=pdf(dist_type, x [,p1,...,pk])  
\end{verbatim}
\end{calling_sequence}
  %-- Parameters
\begin{parameters}
  \begin{varlist}
   \vname{dist\_type}: a string giving the probability distribution (\verb!"bin"! for binomial, \verb!"nor"!
   for normal, \verb!"poi"! for Poisson, etc ...)
    \vname{x}: scalar or vector or matrix with abscissae where the density function must be evaluated
   \vname{p1, ..., pk}: (scalar) parameters (reals or integers) required to define the distribution \verb!dist_type!
   \vname{y}: values at \verb+x+ of the choosen density.
  \end{varlist}
  \end{parameters}
  
\begin{mandescription}
  Computes various probability density functions (for continuous
  distributions) 
  and various probability mass functions (for discrete distributions). Most of the
  scalar probability distributions of the \manlink{grand}{grand}
  function are supported. 
\begin{itemize}

\item \itemdesc{\hypertarget{betapdf}{beta}}  \verb!y=pdf("bet",x,a,b)! computes the beta
  density with shape parameters $a > 0$ and $b > 0$:
$$
     f(x; a, b) =
        \frac{1}{\beta(a,b)} x^{a-1}(1-x)^{b-1} \;
        \mbox{for } x \in (0,1)
$$

\item \itemdesc{\hypertarget{binpdf}{binomial}} \verb!y=pdf("bin",x,n,p)!. Computes
  the binomial mass function with parameters $p \in [0,1]$ and $n$
  (positive integer):
$$
     f(x; n, p) = \frac{ n! }{ x! (n-x)!} p^x (1-p)^{n-x} \;\mbox{for } x \in \{0,1,\dots,n\}
$$

\item \itemdesc{\hypertarget{nbnpdf}{negative binomial}} \verb!y=pdf("nbn",x,r,p)!. Computes
  the negative binomial mass function with parameters $r > 0$ ($r$ not
  necessarily integer) and $p \in (0,1]$:
$$
     f(x; r, p) = \frac{ \Gamma(r+x) }{ x! \Gamma(r)} p^r (1-p)^x \;\mbox{for } x \in \{0,1,\dots\}
$$
  When $r$ is integer, $f(x; r, p)$ is the probability to get $x$ failures among 
 independant Bernoulli trials (each one with probability $p$ of success) which are
 stopped when $r$ success has occured (so the number of Bernoulli trials is $r+x$).  

\item \itemdesc{\hypertarget{caupdf}{Cauchy}} \verb!y=pdf("cau",x,sigma)! computes the Cauchy
  density with parameter $\sigma > 0$:
$$
     f(x; \sigma) = \frac{ \sigma }{ \pi ( x^2 + \sigma^2 ) }
$$


\item \itemdesc{\hypertarget{chipdf}{chi square}} \verb!y=pdf("chi",x,nu)! computes the chi square
  density with parameter $\nu > 0$ (number of dof):
$$
     f(x; \nu) = \frac{1}{2^{\nu/2} \Gamma(\nu/2)} x^{\nu/2-1} e^{-x/2}\; \mbox{ for } x > 0 
$$

%% \item \itemdesc{\hypertarget{nchipdf}{non central chi square}} \verb!y=pdf("nch",x,nu,xnonc)! computes the non central chi square
%%   density with parameters $\nu > 0$ (number of dof) and $\lambda > 0$ (non centrality parameter):
%% $$
%%      f(x; \nu, \lambda) = \frac{1}{2} e^{-(x+\lambda)/2}\left(\frac{x}{\lambda}\right)^{\nu/4-1/2}
%%      I_{\nu/2-1}(\sqrt{\lambda x}) \; \mbox{ for } x > 0 
%% $$
%% where $I$ is a modified Bessel function of the first kind.


\item \itemdesc{\hypertarget{exppdf}{exponential}} \verb!y=pdf("exp",x,tau)! computes the exponential
  density with parameter $\tau > 0$:
$$
     f(x; \tau) = \frac{1}{\tau} e^{-x/\tau} \; \mbox{for } x
     \ge 0
$$
Caution: the parameter is more frequently $\lambda = 1/\tau$.


\item \itemdesc{\hypertarget{fpdf}{F variance ratio}} \verb!y=pdf("f",x,nu1,nu2)! computes
  the variance ratio density with parameters $\nu_1 > 0$ and $\nu_2 >
  0$ (respectively number of dof in the numerator and denominator).


\item \itemdesc{\hypertarget{gampdf}{gamma}} \verb!y=pdf("gam",x,a,b)! computes the gamma
  density with parameters $a > 0$ (shape parameter) and $b \ge 0$
  (rate parameter):
$$
     f(x; a, b) = \frac{ b^a x^{a-1} e^{-bx} } {\Gamma(a)}\; \mbox{for } x > 0
$$
Caution: second parameter is often a scale parameter, says $s = 1/b$.

\item \itemdesc{\hypertarget{norpdf}{Gauss Laplace}} \verb!y=pdf("nor",x,mu,sigma)! computes the normal
  density with parameters $\mu$ and $\sigma > 0$:
$$
     f(x; \mu, \sigma) = \frac{ 1 }{ \sigma \sqrt{2\pi}}
     e^{-\frac{1}{2} \left( \frac{x-\mu}{\sigma} \right)^2 }
$$


\item \itemdesc{\hypertarget{geompdf}{geometric}} \verb!y=pdf("geom",x,p)!. Computes
  the geometric mass function with parameter $p \in (0,1]$:
$$
     f(x; p) = (1-p)^{x-1} p \;\mbox{for } x \in \{1, 2, \dots\}
$$

\item \itemdesc{hypergeometric} \verb!y=pdf("hyp",x,n,r,b)! computes
  the hypergeometric mass function with parameter $n$ (positive
  integer) $r$ and $b$ non negative integers with $n \le r + b$. 
$$
     f(x; n, r,b) = 
    \frac{\left(\begin{array}{c}r\\x\end{array}\right)
          \left(\begin{array}{c}b\\n-x\end{array}\right)}
         {\left(\begin{array}{c}r+b\\n\end{array}\right)}
         \;\mbox{for } x \in \{0, 1, \dots, n\} \; (\mbox{in fact }x \in \{\max(0,n-b),\dots, \min(n,r)\})
$$
$f(x; n,r,b)$ is the probability to get $x$ red balls
in a sample (without replacement) of $n$ balls in an urn
with $r$ red balls and $b$ blue balls.


\item \itemdesc{\hypertarget{lappdf}{Laplace}} \verb!y=pdf("lap",x,a)! computes the Laplace
  density with parameter $a > 0$:
$$
     f(x; a) = \frac{1}{2a} e^{-\frac{|x|}{a}}
$$

\item \itemdesc{\hypertarget{logipdf}{logistic}} \verb!y=pdf("logi",x,a,b)! computes the logistic
  density with parameter $a$ and $b > 0$:
$$
     f(x; a,b) = \frac{y}{b (1+y)^2}, \; \mbox{ with } y = e^{-{|x-a|}{b}}
$$


\item \itemdesc{\hypertarget{lognpdf}{log normal}} \verb!y=pdf("logn",x,mu,sigma)! computes
  the log normal density with parameters $\mu$ and $\sigma > 0$:
$$
     f(x; \mu, \sigma) = \frac{ 1 }{ \sigma \sqrt{2\pi}}
     e^{-\frac{1}{2} r^2 }, \; \mbox{ with } r = \frac{\log(x)-\mu}{\sigma}
$$


\item \itemdesc{\hypertarget{parpdf}{Pareto}} \verb!y=pdf("par",x,a,b)! computes the Pareto
  density with shape parameter $a > 0$ and scale parameter $b > 0$:
$$
     f(x; a, b) = \frac{ a/b }{ (x/b)^{a+1} }, \; \mbox{ for } x \ge b
$$


\item \itemdesc{\hypertarget{poipdf}{Poisson}} \verb!y=pdf("poi",x,mu)! computes
  the Poisson mass function with parameter $\mu \ge 0$:
$$
     f(x; \mu) = \frac{\mu^x e^{-\mu}}{x!} \;\mbox{for } x \in \{0, 1, 2, \dots\}
$$

\item \itemdesc{\hypertarget{raypdf}{Rayleigh}} \verb!y=pdf("ray",x,sigma)! computes the Rayleigh
  density with parameter $\sigma > 0$:
$$
     f(x; \sigma) = \frac{x}{\sigma^2} e^{-\frac{1}{2} (x/\sigma)^2 }, \; \mbox{ for } x \ge 0
$$

\item \itemdesc{\hypertarget{traypdf}{tail Rayleigh}} \verb!y=pdf("tray",x,sigma, a)! computes
  the tail Rayleigh density with parameters $\sigma > 0$ and $a \ge 0$:
$$
     f(x; \sigma, a) = \frac{x}{\sigma^2} e^{-\frac{1}{2} (x-a)(x+a)/\sigma^2 }, \; \mbox{ for } x \ge a
$$


\item \itemdesc{\hypertarget{tpdf}{Student's t}} \verb!y=pdf("t",x,nu)! computes
  the Student's t distribution density with parameter $\nu > 0$
  (number of dof).
$$
    f(x; \nu) = \frac{\Gamma(\frac{\nu+1}{2})}{\sqrt{\nu \pi}\Gamma(\frac{\nu}{2})} \left( 1 + \frac{t^2}{\nu} \right)^{-(\nu+1)/2} 
$$


\item \itemdesc{\hypertarget{uinpdf}{Uniform (uin)}} \verb!y=pdf("uin",x,n1,n2)! computes
  the  uniform ($U\{n_1,n_1+1,\dots,n_2\}$) mass function with parameters $n_1$ and $n_2$
  (integers such that $n_1 \le n_2$).
$$
     f(x; n_1, n_2) = \frac{ 1 }{ n_2 - n_1 + 1} \;\mbox{for } x \in \{n_1,n_1+1,\dots,n_2\}
$$

\item \itemdesc{\hypertarget{unfpdf}{Uniform (unf)}} \verb!y=pdf("unf",x,a,b)! computes
  the  uniform ($U(a,b)$) density function with parameters $a$ and $b$ ($a < b$).
$$
     f(x; a, b) = \frac{1}{b-a} \;\mbox{for } x \in [a,b]
$$


\item \itemdesc{\hypertarget{weipdf}{Weibull}} \verb!y=pdf("wei",x,a,b)! computes the Weibull
  density with parameters $a > 0$ and $b > 0$:
$$
     f(x; a, b) = \frac{b}{a} e^{ (b-1)\log(x/a) - (x/a)^b }, \; \mbox{ for } x > 0
$$

\end{itemize}

\end{mandescription}


\begin{examples}
  
\paragraph{example 1} draw N(0,1) 
\begin{Verbatim}
x = linspace(-5,5,200)';
y = pdf("nor",x,0,1);
xclear(); plot2d(x,y,style=2);
xtitle("The N(0,1) density function")
\end{Verbatim}
  
\paragraph{example 2} draw Gamma(5,1) empirically (with simulation and
an histogram) and exactly (with pdf):
\begin{Verbatim}
// generate a rather big sample of Gamma(5,1) variates
X = grand(1e7,1,"gam",5,1);
// draw an histogram using 200 classes
xclear();histplot(200,X)
// superpose the exact Gamma (5,1) density
x = linspace(0,20,200)';
y = pdf("gam",x,5,1);
xset("thickness",2);plot2d(x,y,style=3,strf="000");xset("thickness",1)
xtitle("Gamma (5,1) histogram and density")
\end{Verbatim}

\paragraph{example 3} using pdf and \verb+grand(..,"disc",..)+ to simulate
binomial random variates (when many random variates must be generated
this is faster than using directly \verb+grand(..,"bin",..)+):
\begin{Verbatim}
// compute B(500,0.3) probabilities
p = pdf("bin",0:500,500,0.3);
// now simulate B(500,0.3)
tic(); X = grand(1e7,1,"disc",p)-1; toc()
// compare with grand(..,"bin",..)
tic(); Y = grand(1e7,1,"bin",500,0.3); toc()
// display histogram for each sample and superpose exact probabilities
classes = (100:201 - 0.5);
xclear();
subplot(1,2,1); histplot(classes,X)
plot2d2(classes,p(101:201),style=5)
xtitle("B(500,0.3) with grand(..,""disc"",...)")
subplot(1,2,2); histplot(classes,Y)
plot2d2(classes,p(101:201),style=5)
xtitle("B(500,0.3) with grand(..,""bin"",...)")
\end{Verbatim}

\paragraph{example 5} using pdf and \verb+grand(..,"disc",..)+ to simulate
hypergeometric random variates. This is useful as grand doesn't yet
support direct hypergeometric random generation.
\begin{Verbatim}
// compute H(600,700,800) probabilities
p = pdf("hyp",0:600,600,700,800);
// now simulate 
tic(); X = grand(1e7,1,"disc",p)-1; toc()
// display histogram and superpose exact probabilities
classes = (240:321) - 0.5;
xclear();
histplot(classes,X)
plot2d2(classes,p(241:322),style=5)
xtitle("H(600,700,800) with grand(..,""disc"",...)")
\end{Verbatim}

\end{examples}



\begin{manseealso}
  \manlink{grand}{grand}, \manlink{dist\_stat}{dist_stat}, \manlink{cdf}{cdf}, \manlink{icdf}{icdf}  
\end{manseealso}

%-- Authors

\begin{authors}
  beta, binomial, negative binomial, f, t, gamma, hypergeometric and 
  poisson densities uses the Catherine Loader 's dbinom package. Other 
  densities and nsp interface Bruno Pincon.
\end{authors}

