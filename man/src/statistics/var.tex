\mansection{var}

\begin{mandesc}
  \short{var}{variance and weighted variance}
  \short{std}{standard deviation and weighted standard deviation}
\end{mandesc}

%-- Calling sequence section
\begin{calling_sequence}
\begin{verbatim}
  V = var(X)
  V = var(X, dim=dimarg, skip_nan=b, weights=w, unbiaised=b)  
  V = mean(X, dimarg, skip_nan=b, weights=w, unbiaised=b)  
  sd = std(X)
  sd = std(X, dim=dimarg, skip_nan=b, weights=w, unbiaised=b)  
  sd = std(X, dimarg, skip_nan=b, weights=w, unbiaised=b)  
\end{verbatim}
\end{calling_sequence}
%-- Parameters
\begin{parameters}
  \begin{varlist}
    \vname{X}: vector or matrix (complex vectors or matrices are allowed see here after)
    \vname{dim=dimarg}: A string chosen among \verb+'M'+, \verb+'m'+, \verb+'*'+,\verb+'full'+, \verb+'FULL'+, \verb+'row'+,
    \verb+'ROW'+, \verb+'col'+, \verb+'COL'+ or an non ambiguous abbreviation or an integer. 
    This argument is optional and if omitted 'full' is assumed.
    \vname{skip_nan=b}: boolean scalar (default is \verb+%f+).
    \vname{weights=w}: non negative vector or matrix of same dimensions than \verb+X+ 
    \vname{unbiased=b}:  boolean scalar (default is \verb+%t+).
  \end{varlist}
\end{parameters}

\begin{mandescription}
  \verb+var+ computes the variance using the following formulae:
$$
\begin{array}{ll}
    V = \frac{1}{n-1} \sum_{k=1}^n (X_k - \bar{X})^2  &  \mbox{ when unbiased is true} \\ 
    V = \frac{1}{n} \sum_{k=1}^n (X_k - \bar{X})^2  &  \mbox{ otherwise} 
\end{array} \mbox{ with }  \bar{X} = \frac{1}{n} \sum_{k=1}^n X_k
$$ 
or the weighted variance (when the weights $w$ are provided) using:
$$
\begin{array}{ll}
    V = \frac{1}{1- \sum_{j=1}^n w_j^2} \sum_{k=1}^n w_k (X_k - \bar{X})^2  &  \mbox{ when unbiased is true} \\ 
    V =                                \sum_{k=1}^n w_k (X_k - \bar{X})^2  &  \mbox{ otherwise} 
\end{array} \mbox{ with }  \bar{X} = \sum_{k=1}^n w_k X_k
$$ 
Standard deviation is simply computed as the square root of the variance.

Using the dim option you can compute either the variance or standard deviation
of all the elements or of each row or each column. 

When the entry $X$ is complex then the ``variance'' computed with \verb+var+ corresponds 
to the sum of the variance of the real part and the variance of the imaginary part (this 
corresponds to the matlab/octave behavior). If you want to compute the covariance matrix 
of a complex valued signal use \verb+cov(real(X),imag(X))+.

\itemdesc{dim option} 
  The dim argument (default full) gives the dimension to be used for performing the mean operation:
  \begin{itemize}
    \item 'full' or 0: var or std of all the elements 
    \item 'row' or 1: var or std of each column (a row vector is returned)
    \item 'col' or 2: var or std of each row (a column vector is returned)
    \item 'm' (or -2): (for Matlab compatibility) var or std along the first non 
          singleton dimension of the first argument.
  \end{itemize}

\itemdesc{skip\_nan option}
   This option is available for the usual or weighted variance or standard deviation. 
When it is \verb+%t+,  Nan values (which stand for ``missing or not available values'') 
are not taking into account in the computation.
   
\itemdesc{weights option}
  This option lets to compute weighted variance or standard deviation. The weights should be non negative. Inside
the function they are normalized such that they sum up to $1$.
   
\end{mandescription}

%--example 
\begin{examples}
\paragraph{example 1} empirical variance got from random samples of known probability distributions 
\begin{Verbatim}
n = 500;
X = randn(n,1);  // a sample from N(0,1)
var(X)   // should tend to 1 when n -> +oo
std(X)   // should tend to 1 when n -> +oo

X = grand(n,1,"bin",100,0.35); // a sample from Bin(100,0.35)
var(X)   // should tend to 100*0.35*0.65=22.75  when n -> +oo
std(X)   // should tend to sqrt(100*0.35*0.65)  when n -> +oo

X = grand(n,1,"geom",0.2); // a sample from G(0.2)
var(X)   // should tend to (1-0.2)/0.2^2 = 20 when n -> +oo
std(X)   // should tend to sqrt(20) when n -> +oo
\end{Verbatim}

\paragraph{example 2} var of each row and each column
\begin{Verbatim}
X = rand(6,4);
// var of the columns
var(X,dim=1)
// var of the rows
var(X,dim=2)
\end{Verbatim}

\paragraph{example 3} weighted variance
\begin{Verbatim}
n = 20; p = 0.4
X = 0:n;
wgt = pdf("bin",X, n,p);
var(X, weights=wgt, unbiased=%f)  // should be exactly n*p*(1-p)=4.8
\end{Verbatim}


\end{examples}

%-- see also
\begin{manseealso}
  \manlink{mean}{mean}
\end{manseealso}

% -- Authors
\begin{authors}
  Jerome Lelong, Bruno Pincon
\end{authors}
