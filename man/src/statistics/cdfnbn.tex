% -*- mode: latex -*-
\mansection{cdfnbn}
\begin{mandesc}
  \short{cdfnbn}{cumulative distribution function  negative binomial distribution}
\end{mandesc}
\index{cdfnbn}\label{cdfnbn}
%-- Calling sequence section
\begin{calling_sequence}
\begin{verbatim}
  [P,Q]=cdfnbn("PQ",S,Xn,Pr,Ompr)  
  [S]=cdfnbn("S",Xn,Pr,Ompr,P,Q)  
  [Xn]=cdfnbn("Xn",Pr,Ompr,P,Q,S)  
  [Pr,Ompr]=cdfnbn("PrOmpr",P,Q,S,Xn)  
\end{verbatim}
\end{calling_sequence}
%-- Parameters
\begin{parameters}
  \begin{varlist}
    \vname{P,Q,S,Xn,Pr,Ompr}: six real vectors of the same size.
    \vname{P,Q (Q=1-P)}: the cumulation from 0 to S of the  negative binomial distribution. Input range: [0,1].
    \vname{S}: the upper limit of cumulation of the binomial distribution. There are F or fewer failures before the XNth success. Input range: [0, +infinity). Search range: [0, 1E300]
      \vname{Xn}: the number of successes. Input range: [0, +infinity). Search range: [0, 1E300]
        \vname{Pr}:  the probability of success in each binomial trial. Input range: [0,1]. Search range: [0,1].
        \vname{Ompr}:  \verb+1-PR+ Input range: [0,1]. Search range: [0,1]. \verb+PR + OMPR = 1.0+.
  \end{varlist}
\end{parameters}
\begin{mandescription}
  Calculates any one parameter of the negative binomial
  distribution given values for the others.
  The  cumulative  negative   binomial  distribution  returns  the
  probability that there  will be  F or fewer failures before  the
  XNth success in binomial trials each of which has probability of
  success PR.
  The individual term of the negative binomial is the probability of
  S failures before XN successes and is
  Choose \verb!( S, XN+S-1 ) * PR^(XN) * (1-PR)^S!
  Formula   26.5.26   of   Abramowitz  and  Stegun,  Handbook   of
  Mathematical Functions (1966) is used  to  reduce calculation of
  the cumulative distribution  function to that of  an  incomplete
  beta.
  Computation of other parameters involve a seach for a value that
  produces  the desired  value  of P.   The search relies  on  the
  monotinicity of P with the other parameter.
\end{mandescription}

\begin{authors}
  Nsp interface by Jean-Philippe Chancelier. Code from DCDFLIB: 
  Library of Fortran Routines for Cumulative Distribution
  Functions, Inverses, and Other Parameters (February, 1994)
  Barry W. Brown, James Lovato and Kathy Russell. The University of Texas.
\end{authors}
