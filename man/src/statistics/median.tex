\mansection{median}

\begin{mandesc}
  \short{median}{median computation}
\end{mandesc}

%-- Calling sequence section
\begin{calling_sequence}
\begin{verbatim}
  md = median(X)
  md = median(X, dim=dimarg, skip_nan=b)  
  md = mean(X, dimarg, skip_nan=b)  
\end{verbatim}
\end{calling_sequence}
%-- Parameters
\begin{parameters}
  \begin{varlist}
    \vname{X}: real vector or matrix
    \vname{dim=dimarg}: A string chosen among \verb+'M'+, \verb+'m'+, \verb+'*'+,\verb+'full'+, \verb+'FULL'+, \verb+'row'+,
    \verb+'ROW'+, \verb+'col'+, \verb+'COL'+ or an non ambiguous abbreviation or an integer. 
    This argument is optional and if omitted 'full' is assumed.
    \vname{skip_nan=b}: boolean scalar (default is \verb+%f+).
  \end{varlist}
\end{parameters}

\begin{mandescription}
  \verb+median+ computes the median of the all the elements of \verb+X+
or the median of each row or each column of \verb+X+ depending upon
the parameter \verb+dim+. The median of $n$ sorted numbers 
$X(1) \le X(2) \le \dots \le X(n)$ is equal to $X((n+1)/2)$ when $n$
is odd and to  $(X(n/2)+X(n/2+1))/2$ when $n$ is even.

\itemdesc{dim option} 
  The dim argument (default full) gives the dimension to be used for performing the median operation:
  \begin{itemize}
    \item 'full' or 0: median of all the elements 
    \item 'row' or 1: median of each column (a row vector is returned)
    \item 'col' or 2: median of each row (a column vector is returned)
    \item 'm' (or -2): (for Matlab compatibility) median along the first non 
          singleton dimension of the first argument.
  \end{itemize}

\itemdesc{skip\_nan option}
   When this option is \verb+%t+,  Nan values (which stand for ``missing or not 
 available values'') are not taking into account in the computation.
\end{mandescription}
%--example 
\begin{examples}
\begin{Verbatim}
X = rand(1,5)
median(X)
X = [%nan,X,%nan]
median(X)
median(X, skip_nan=%t) 

X = randn(10,7)
median(X, dim=1) // or median(X,1)
median(X, dim=2) // or median(X,2)
\end{Verbatim}
\end{examples}

%-- see also
\begin{manseealso}
   \manlink{mean}{mean}, \manlink{var}{var}, \manlink{std}{std}
\end{manseealso}

% -- Authors
\begin{authors}
 Bruno Pincon
\end{authors}
