% -*- mode: latex -*-
\mansection{cdfchn}
\begin{mandesc}
  \short{cdfchn}{cumulative distribution function  non-central chi-square distribution}
\end{mandesc}
\index{cdfchn}\label{cdfchn}
%-- Calling sequence section
\begin{calling_sequence}
\begin{verbatim}
  [P,Q]=cdfchn("PQ",X,Df,Pnonc)  
  [X]=cdfchn("X",Df,Pnonc,P,Q);  
  [Df]=cdfchn("Df",Pnonc,P,Q,X)  
  [Pnonc]=cdfchn("Pnonc",P,Q,X,Df)  
\end{verbatim}
\end{calling_sequence}
%-- Parameters
\begin{parameters}
  \begin{varlist}
    \vname{P,Q,X,Df,Pnonc}: five real vectors of the same size.
    \vname{P,Q (Q=1-P)}:  the integral from 0 to X of the non-central chi-square distribution. 
    Input range: [0, 1-1E-16).
      \vname{X}: upper limit of integration of the non-central chi-square distribution. Input range: [0, +infinity). Search range: [0,1E300]
        \vname{Df}: degrees of freedom of the non-central chi-square distribution. Input range: (0, +infinity). Search range: [ 1E-300, 1E300]
        \vname{Pnonc}:  non-centrality parameter of the non-central chi-square distribution. Input range: [0, +infinity). Search range: [0,1E4]
  \end{varlist}
\end{parameters}

\begin{mandescription}
  Calculates any one parameter of the non-central chi-square
  distribution given values for the others. Let $\lambda = \mbox{Pnonc}$ and 
  $k = \mbox{Df}$ the probability density function of the noncentral chi-square 
  distribution is given for $x\in [0,\infty)$  by~:
  \begin{equation} 
    \frac{1}{2}e^{-(x+\lambda)/2}\left (\frac{x}{\lambda} \right)^{k/4-1/2} I_{k/2-1}(\sqrt{\lambda x})
  \end{equation}
  where $I_a(x)$ is a modified Bessel function of the first kind given by~:
  \begin{equation} 
      I_a(y) = (y/2)^a \sum_{j=0}^\infty \frac{ (y^2/4)^j}{j! \Gamma(a+j+1)} 
  \end{equation}
  Formula  26.4.25 of Abramowitz and Stegun,
  Handbook  of Mathematical  Functions (1966) is used to compute the cumulative
  distribution function.
  Computation of other parameters involve a seach for a value that
  produces  the desired  value  of P.   The search relies  on  the
  monotinicity of P with the other parameter.
  The computation time  required for this  routine is proportional
  to the noncentrality  parameter  (PNONC).  Very large  values of
  this parameter can consume immense  computer resources.  This is
  why the search range is bounded by $10^4$.
\end{mandescription}

\begin{authors}
  Nsp interface by Jean-Philippe Chancelier. Code from DCDFLIB: 
  Library of Fortran Routines for Cumulative Distribution
  Functions, Inverses, and Other Parameters (February, 1994)
  Barry W. Brown, James Lovato and Kathy Russell. The University of Texas.
\end{authors}
