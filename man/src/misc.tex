\chapter*{misc}\addcontentsline{toc}{chapter}{misc}

Ici c'est un fourre-tout provisoire... 

\HCode{<hr>}
System

\begin{quote}
\noindent
\hyperlink{acquire}{acquire} - search a variable in callers  \\
\hyperlink{clear}{clear},\hyperlink{clearglobal}{clearglobal} - removes variables from current or global environnement  \\
\end{quote}


\HCode{<hr>}
Logical operators
\begin{quote}
\noindent
\hyperlink{and}{and} - logical and operator \\
\end{quote}

\HCode{<hr>}
Logical operator (Matlab compatibility)

\begin{quote}
\noindent
\hyperlink{any}{any} - checks if any element is nonzero or true  \\
\hyperlink{all}{all} - checks if all elements are nonzero or true \\
\end{quote}

\HCode{<hr>}
Basic function operating on numerical arrays
\begin{quote}
\noindent
\hyperlink{sum}{sum},\hyperlink{prod}{prod} - sum and product of matrix elements \\
\hyperlink{cumsum}{cumsum},\hyperlink{cumprod}{cumprod} - cumulative sum and product of matrix elements \\
\end{quote}
 

\HCode{<hr>}
Random number generation and probability/statistic special functions

\begin{quote}
\noindent
\hyperlink{grand}{grand} - random numbers generators \\
\hyperlink{pdf}{pdf} - probability density or mass functions \\
\hyperlink{cdfbet}{cdfbet} - beta cdf and invcdf function \\
\hyperlink{cdfbin}{cdfbin} - binomial cdf and invcdf function \\
\hyperlink{cdfchi}{cdfchi} - chi square cdf and invcdf function \\
\hyperlink{cdfchn}{cdfchn} - non central chi square cdf and invcdf function \\
\hyperlink{cdff}{cdff} - f cdf and invcdf function \\
\hyperlink{cdffnc}{cdffnc} - non central f cdf and invcdf function \\
\hyperlink{cdfgam}{cdfgam} - gamma cdf and invcdf function \\
\hyperlink{cdfnbn}{cdfnbn} - negative binomial cdf and invcdf function \\
\hyperlink{cdfnor}{cdfnor} - normal cdf and invcdf function \\
\hyperlink{cdfpoi}{cdfpoi} - poisson cdf and invcdf function \\
\hyperlink{cdft}{cdft} - student t cdf and invcdf function \\
\hyperlink{cdftnc}{cdftnc} - non central student t cdf and invcdf function \\
\end{quote}

\HCode{<hr>}
A few functions related to integer numbers

\begin{quote}
\noindent
\hyperlink{primes}{primes} - computes prime numbers\\
\hyperlink{isprime}{isprime} - primality test\\
\hyperlink{factor}{factor} - prime factorization of an integer\\
\end{quote}

\HCode{<hr>}
Coordinates transformation functions

\begin{quote}
\noindent
\hyperlink{sph2cart}{sph2cart},\hyperlink{cart2sph}{cart2sph} -
cartesian to spherical coordinates transformations functions \\
\end{quote}


\HCode{<hr>}
Special mathematical functions

\begin{quote}
\noindent
\hyperlink{legendre}{legendre} - associated legendre function\\
\end{quote}


\input misc/acquire.tex
\input misc/and.tex 
\input misc/cdfbet.tex
\input misc/cdfbin.tex
\input misc/cdfchi.tex
\input misc/cdfchn.tex
\input misc/cdff.tex
\input misc/cdffnc.tex
\input misc/cdfgam.tex
\input misc/cdfnbn.tex
\input misc/cdfnor.tex
\input misc/cdfpoi.tex
\input misc/cdft.tex
\input misc/cdftnc.tex
\input misc/cart2sph.tex 
\input misc/clear.tex
\input misc/factor.tex 
\input misc/grand.tex 
\input misc/pdf.tex 
\input misc/isprime.tex 
\input misc/legendre.tex 
\input misc/perms.tex 
\input misc/primes.tex 
\input misc/sum.tex 
\input misc/cumsum.tex 
\input misc/any.tex 
\input misc/all.tex 


