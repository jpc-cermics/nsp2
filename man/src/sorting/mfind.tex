% -*- mode: latex -*-

\mansection{mfind}
\begin{mandesc}
  \short{mfind}{find indices of elements of a numerical matrix satisfying some inequalities}
\end{mandesc}

% -- Calling sequence section
\begin{calling_sequence}
\begin{verbatim}
[ind1, ind2, ..., indn, indnp1] = mfind(x, op1, s1, op2, s2, .., opn, sn)
\end{verbatim}
\end{calling_sequence}
% -- Parameters
\begin{parameters}
  \begin{varlist}
    \vname{x}:  real numerical vector or matrix
    \vname{op1, op2, ...}: strings specifying a comparizon operator (to choose among \verb+"<=", "<", ">", ">=", "==","~="+)
    \vname{s1, s2, ...}:  real scalars
  \end{varlist}
\end{parameters}

\begin{mandescription}
For each element of \verb+x+, this function do the tests:
\begin{verbatim}
     x(k) opj  sj,    j = 1, 2, ...., n
\end{verbatim}
until one is true, in which case the element index (here k) is added in the
corresponding output index vector (indj if tests 1,...,j-1 are false and test j is true).
If all the n tests fail, the element index is added in the last index vector \verb+indnp1+.

This function could be useful to evaluate a real function which has a piecewise definition
(see examples). In this case it could be faster to use \verb+mfind+ in place of several \verb+find+.

Note: mfind is currently limited to a maximum of 6 tests.
\end{mandescription}

\begin{examples}
  
\paragraph{example}
using mfind to evaluate a function with a piecewise definition: 
$$
f(x) = \left\{ \begin{array}{ll}
    1     & \mbox{ if } x \le 0 \\
  \cos(x) & \mbox{ if } 0 \le x \le \pi/2 \\
  x-\pi/2 & \mbox{ if } x \ge  \pi/2 
\end{array}\right.
$$
\begin{Verbatim}
function y=f(x) 
y = zeros(size(x)); 
[ind1, ind2, ind3, indNan] = mfind(x,"<=",0,"<=",%pi/2,"<=",%inf); 
y(ind1) = 1; 
y(ind2) = cos(x(ind2)); 
y(ind3) = x(ind3) - %pi/2; 
y(indNan) = %nan 
endfunction 
x = linspace(-1,%pi,400)'; 
y = f(x); 
plot2d(x,y); 
f(%nan) 
f(%inf) 
f(-%inf) 
\end{Verbatim}

 
\end{examples}

\begin{manseealso}
   \manlink{find}{find}, \manlink{has}{has}, \manlink{bsearch}{bsearch}  
\end{manseealso}

% -- Authors
\begin{authors}
  Bruno Pincon
\end{authors}
