% -*- mode: latex -*-

\mansection{unique}
\begin{mandesc}
  \short{unique}{compute unique components of a vector or of a list or unique rows or columns of a matrix}
\end{mandesc}

% -- Calling sequence section
\begin{calling_sequence}
\begin{verbatim}
y = unique(x)
[y [,ind [,occ]]] = unique(x)
[y [,ind [,occ]]] = unique(x,first_ind=value, ind_type=str, which=str)
\end{verbatim}
\end{calling_sequence}
% -- Parameters
\begin{parameters}
  \begin{varlist}
    \vname{x}: vector (or matrix) of numbers (real float (Mat) or integers
    (IMat) or strings or cells, or a list.

    \vname{y}: vector (or matrix for unique with rows or columns) or list (when
    x is a list) containing unique components of x (or the unique rows or
    columns, see option which)

    \vname{ind}: vector of same length than y, ind(i) is an index such that
    y(i)=x(ind(i)). \verb+ind+ could be stored in an array of floating point
    numbers (Mat) (the default) or in an array of usual C integers (IMat of
    itype int) if you use \verb+ind_type = "int"+

    \vname{occ}: vector of same length than y, occ(i) is the number of
    occurrences of y(i) in the vector x.

    \vname{first\_ind}: optional boolean scalar, useful only when x is a vector
    of numbers or strings. In this case using \verb!first_ind=%t! ind(i) is
    the smaller index such that y(i) = x(ind(i)).
    
    \vname{ind_type = str}: named optional argument, a string among
    \verb+{"double","int"}+ (default is \verb+"double"+) which gives the type
    for the index vector \verb+ind+.

    \vname{which = str}: named optional argument, a string among
    \verb+{"elements","columns","rows"}+ (default is \verb+"elements"+) or any
    non ambiguous abreviation.
  \end{varlist}
\end{parameters}

\begin{mandescription}
  This function computes the unique components of a vector or a matrix of
  numbers, strings or cells.

  Using the option \verb+which = "rows"+ or \verb+which = "columns"+ you can
  get, instead of the unique elements, the unique rows or columns of the entry
  matrix (this feature is only available for real matrix (Mat) or integer matrix
  (IMat)).
  
  As the number of occurrences of each unique component could be returned, this
  function can be also useful for basic statistic computations.
\end{mandescription}
\begin{examples}
  \paragraph{example 1} with a vector of numbers
  \begin{Verbatim}
    x = [0.5, -1, 2, 2, -1, 0.5, 2, -1, 2];
    [y, ind, occ] = unique(x)
  \end{Verbatim}

  \paragraph{example 2} with a vector of strings:
  \begin{Verbatim}
    x = ["toto", "foo", "bar", "toto", "foobar", "bar", "toto", "foo", "bar"]
    [y, ind, occ] = unique(x);
  \end{Verbatim}

  \paragraph{example 3} with a vector of cells:
  \begin{Verbatim}
    x = {["foo", "bar"], [0;1], [0,1], ["foo", "bar"], [0;1], {"toto"}, {"toto"}}
    [y, ind, occ] = unique(x);
\end{Verbatim}

\paragraph{example 4} with a list:
\begin{Verbatim}
  x = list(["foo", "bar"], [0;1], [0,1], ["foo", "bar"], [0;1], "toto", "toto")
  [y, ind, occ] = unique(x)
\end{Verbatim}

\paragraph{example 5} using unique for rows or columns:
\begin{Verbatim}
  // for columns
  A = grand(2,8,"uin",-1,1)
  B = unique(A,which="columns")  // or simply which="c"
  [B,ind,occ] = unique(A,which="col")

  // for rows
  A = grand(8,2,"uin",-1,1)
  B = unique(B,which="rows")     // or simply which="r"
  [B,ind,occ] = unique(A,which="rows")
\end{Verbatim}


\paragraph{example 6} using unique for basic statistics:
\begin{Verbatim}
  // generate a big sample of the binomial distribution B(6,0.4)
  m = 1e5;
  x = grand(m,1,"bin",6,0.4); 
  // computes the empirical probabilities associated to each values
  [y, ind, occ] = unique(x);
  p_emp = occ/m;
  // computes exact probabilities (using the cumulative distribution func)
  v = ones(7,1);
  p_cum = cdfbin("PQ", (0:6)', 6*v, 0.4*v, 0.6*v);
  p_exact = [p_cum(1); p_cum(2:$)-p_cum(1:$-1)];
  // compare both empirical and exact ones
  xbasc()
  e = 0.05;
  plot2d3([y-e,y+e], [p_exact,p_emp], style=[1 2], ...
  rect=[-1,0,7,1.1*max(p_exact)],...
  leg="exact probabilies@empirical probabilities")

  // we want to verify if grand(...,"perm",...) chooses random permutation uniformly
  // generate a big sample of permutations
  prms = grand(24e3,"perm",4);  // there 4!=24 possible permutation 
  [p,ind,occ] = unique(prms,which="col");
  size(p)   // should be 4 x 24
  xbasc()
  plot2d2(1:numel(occ),occ,style=2,rect=[0,0,25,max(occ)+100])
  xtitle("occurence for each 24 permutations should be around 1000")
\end{Verbatim}

\end{examples}

\begin{manseealso}
  \manlink{setdiff}{setdiff}, \manlink{union}{union}, \manlink{intersect}{intersect}, \manlink{setxor}{setxor}  
\end{manseealso}

% -- Authors
\begin{authors}
  Bruno Pincon
\end{authors}
