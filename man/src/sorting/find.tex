% -*- mode: latex -*-

\mansection{find}
\begin{mandesc}
  \short{find}{returns indices of the true or non zeros components of a vector or matrix}
\end{mandesc}

% -- Calling sequence section
\begin{calling_sequence}
\begin{verbatim}
ind = find(M)
ind = find(M, ind_type=str)  // or ind = find(M, str)
[indi,indj]  = find(M)
[indi,indj]  = find(M, ind_type=str)) // or [indi,indj]  = find(M, str))
[indi,indj,val] = find(M) // for numerical matrices M only
\end{verbatim}
\end{calling_sequence}
% -- Parameters
\begin{parameters}
  \begin{varlist}
    \vname{M}:  boolean or  numerical vector or matrix
    \vname{ind, indi, indj}: vectors of indices
    \vname{ind_type = str}: named optional argument, a string given the type for the index vector(s) (default is \verb+"double"+)
    \vname{val}: vector with the non zeros values of M
  \end{varlist}
\end{parameters}

\begin{mandescription}
With one output argument \verb+ind=find(M)+,  $ind$ contains all indices 
corresponding to true components if $M$ is a boolean vector or matrix 
and to non zeros components if it is a numerical matrix. If \verb+M+
is a matrix (that is with more than one row or one
column) the returned indices correspond to the one indexing way (aka
linear indexing)
that is considering the matrix as a big vector build by stacking 
the columns (aka as fortran major order or column major order),
see \manlink{indexing arrays}{indexing arrays}.

With 2 output arguments \verb+[indi,indj]=find(M)+,
you get the  row and column indices of the true or non zeros components.

The type for the index vector(s) is \verb+"double"+ by default but you
can obtain them as vector(s) a given integer type by choosing the optional named argument
\verb+ind_type+ among \verb+"int"+,  \verb+"uint"+, \verb+"int64"+, \verb+"uint64"+,
\verb+"int32"+, \verb+"uint32"+,\verb+"int16"+, \verb+"uint16"+, \verb+"int8"+, or \verb+"uint8"+.
(Note that this feature is currently available only with a boolean entry, that is
when  $M$ is a numerical matrix index vectors could be yet only \verb+"double"+).

The form with 3 output arguments \verb+[indi,indj,val]=find(M)+, is available
for numerical matrix M and the last argument \verb+val+ is filled with
the non zeros values of M.


This function is very important to vectorize tests and even in the 
case of a matrix the form which returns only one 
index vector is generally more powerfull.
\end{mandescription}

\begin{examples}
  
\paragraph{example 1}
\begin{Verbatim}
x= [0, 0.3, 0.11, -0.4, 0.5, 2, 0.6, 0.7]
ind = find( 0.2 < x  &  x < 0.7 )  // ind as double
ind = find( 0.2 < x  &  x < 0.7, ind_type="int")  // ind as int

// reshape the vector as a 2 x 4 matrix
x = matrix(x,2,4)
ind=find( 0.2 < x  &  x < 0.7 ) // must give the same than before 
[i,j]=find( 0.2 < x  &  x < 0.7 )
\end{Verbatim}
  
\paragraph{example 2} using find to evaluate (on a vector) a function with a
piecewise definition: $f(x) = (x-a)/(b-a)$, for  $x \in
[a,b]$, $f(x) = (c-x)/(c-b)$, for $x \in [b,c]$ and otherwise $f(x) = 0$. 
\begin{Verbatim}
a=-1; b=0; c=2;
x = linspace(-2,3,101)';
y = zeros(size(x));
ind_a_b = find( a <= x  &  x <= b, "int");
y(ind_a_b) = (x(ind_a_b)-a)/(b-a);
ind_b_c = find( b < x  &  x <= c, "int");
y(ind_b_c) = (c-x(ind_b_c))/(c-b);
xbasc()
plot2d(x,y,style=2)
\end{Verbatim}
  
\end{examples}

\begin{manseealso}
 \manlink{indexing arrays}{indexing arrays}, \manlink{mfind}{mfind}, \manlink{has}{has}, \manlink{bsearch}{bsearch}  
\end{manseealso}

% -- Authors
\begin{authors}
  Jean-Philippe Chancelier, Bruno Pincon
\end{authors}
