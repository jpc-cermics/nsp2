% -*- mode: latex -*-

\mansection{find}
\begin{mandesc}
  \short{find}{returns indices of the true or non zeros components of a vector or matrix}
\end{mandesc}

% -- Calling sequence section
\begin{calling_sequence}
\begin{verbatim}
i = find(M)
[i,j]  = find(M)
\end{verbatim}
\end{calling_sequence}
% -- Parameters
\begin{parameters}
  \begin{varlist}
    \vname{M}:  boolean or  numerical vector or matrix
    \vname{i, j}: vectors of indices
  \end{varlist}
\end{parameters}

\begin{mandescription}
With one output argument \verb+i=find(M)+,  $i$ contains all indices 
corresponding to true components if $M$ is a boolean vector or matrix 
and to non zeros components if it is a numerical matrix. If \verb+M+
is a matrix (that is with more than one row or one
column) the returned indices correspond to the one indexing way, 
that is considering the matrix as a big vector build by stacking 
the columns (aka as fortran major order or column major order).

With 2 output arguments \verb+[i,j]=find(M)+,
you get the  row and column indices of the true or non zeros components.
 
This function is very important to vectorize tests and even in the 
case of a matrix the form which returns only one 
index vector is generally more powerfull.
\end{mandescription}

\begin{examples}
  
\paragraph{example 1}
\begin{program}\HCode{x=[0, 0.3, 0.11, -0.4, 0.5, 2, 0.6, 0.7]\Hnewline
ind=find( 0.2 }\verb+<+\HCode{ x & x }\verb+<+\HCode{ 0.7 )\Hnewline
\Hnewline
// reshape the vector as a 2 x 4 matrix\Hnewline
x = matrix(x,2,4)\Hnewline
ind=find( 0.2 }\verb+<+\HCode{ x & x }\verb+<+\HCode{ 0.7 ) // must give the same than before \Hnewline
[i,j]=find( 0.2 }\verb+<+\HCode{ x & x }\verb+<+\HCode{ 0.7 )}
\end{program}
  
\paragraph{example 2} using find to evaluate a function with a
piecewise definition: $f(x) = (x-a)/(b-a)$, for  $x \in
[a,b]$, $f(x) = (c-x)/(c-b)$, for $x \in [b,c]$ and otherwise $f(x) = 0$. 
\begin{program}\HCode{a=-1; b=0; c=2;\Hnewline
x = linspace(-2,3,101)';\Hnewline
y = zeros(x);\Hnewline
ind_a_b = find( a }\verb+<+\HCode{= x & x }\verb+<+\HCode{= b);\Hnewline
y(ind_a_b) = (x(ind_a_b)-a)/(b-a);\Hnewline
ind_b_c = find( b }\verb+<+\HCode{ x  & x }\verb+<+\HCode{= c);\Hnewline
y(ind_b_c) = (c-x(ind_b_c))/(c-b);\Hnewline
xbasc()\Hnewline
plot2d(x,y,style=2)}
\end{program}
  
\end{examples}

\begin{manseealso}
  \manlink{bsearch}{bsearch}  
\end{manseealso}

% -- Authors
\begin{authors}
  Jean-Philippe Chancelier
\end{authors}
