% -*- mode: latex -*-

\mansection{issorted}
\begin{mandesc}
  \short{issorted}{test if a vector or matrix is sorted}
\end{mandesc}

% -- Calling sequence section
\begin{calling_sequence}
\begin{verbatim}
b  = issorted(x)
b  = issorted(x, flag=str, strict_order=bool)
b  = issorted(x, flag=str, strict_order=bool, order_nan=bool)
\end{verbatim}
\end{calling_sequence}
% -- Parameters
\begin{parameters}
  \begin{varlist}
    \vname{x}:  matrix or vector of floating point numbers (Mat), integer numbers (IMat) or strings.
    \vname{flag=str}: optional named argument, a string among
    \verb+"g","c","r","lc","lr"+ (default is \verb+"g"+)
    \vname{strict_order=bool}: optional named argument, a scalar boolean (default is \verb+%f+)
    \vname{order_nan=bool}: optional named argument (only for matrix of floating point numbers), a scalar boolean (default is \verb+%f+).
    \vname{b}: output, a boolean scalar or vector.
  \end{varlist}
\end{parameters}

\begin{mandescription}
This function tests if a vector or a matrix is sorted in increasing order 
(in strict increasing order if \verb+strict_order=%t+).Several  kinds of tests are available 
depending on the input argument \verb+flag+:

\itemdesc{{\tt flag = "g"}} In this case the function tests if  the vector $x$  is sorted. 
If $x$ is a matrix it is considered as a big vector using the
 major column order. The output $b$ is a boolean scalar.

\itemdesc{{\tt flag = "c"}} In this case the function tests if each row of $x$ is sorted. The output
$b$ is a boolean column vector (of length the number of rows of $x$).

\itemdesc{{\tt flag = "r"}} In this case the function tests if each column of $x$ is sorted. The output
$b$ is a boolean row vector (of length the number of coluns of $x$).

\itemdesc{{\tt flag = "lc"}} In this case the function tests if the columns of $x$ are sorted using lexical order.
The output $b$ is a boolean scalar.

\itemdesc{{\tt flag = "lr"}} In this case the function tests if the rows of $x$ are sorted using lexical order.
The output $b$ is a boolean scalar.


\paragraph{Remark}

For matrix of floating point numbers, Nan values are considered larger
than Inf (and equal among them) when \verb+order_nan=%t+. This feature
is only available when \verb+strict_order=%f+. With this option, 
\verb+issorted+ should returns \verb+%t+ (or a vector of \verb+%t+) on every
vector or matrix $x$ which has been previously sorted by the \manlink{sort}{sort}
using \verb+x = sort(x, flag, "i")+ even if Nan are present (due to the fact
that sort places Nan value at the end for an increasing order sort).

\end{mandescription}

\begin{examples}

\paragraph{with flag="g"}
\begin{mintednsp}{nsp}
x = [ -1, -1, 2, 3, 6.5 ]
issorted(x)
issorted(x,strict_order=%t)

// behavior with Nan
x = [ x, %nan ]
issorted(x)  // should be false
issorted(x,order_nan=%t) // should be true

// with a string vector
x = ["bar", "foo","foo", "toto"]
issorted(x)
issorted(x, strict_order=%t)
\end{mintednsp}

\paragraph{with flag="c" or  flag="r"}
\begin{mintednsp}{nsp}
x = rand(3,4)
issorted(x,flag="c")
x = sort(x,"c","i")
issorted(x,flag="c")

issorted(x,flag="r")
x = sort(x,"r","i")
issorted(x,flag="r")
\end{mintednsp}

\paragraph{with flag="lc" or  flag="lr"}
\begin{mintednsp}{nsp}
x = [1, 2, 2, %pi;
     1, 2, 2, 4 ];
issorted(x,flag="lr",strict_order=%t)
x(2,4) = %pi
issorted(x,flag="lr",strict_order=%t)
issorted(x,flag="lr",strict_order=%f)

x = [1,   1, 1;
     0.5, 1, 2;
     0.2,-2, 6]
issorted(x,flag="lc")
\end{mintednsp}

\end{examples}

\begin{manseealso}
  \manlink{sort}{sort}, \manlink{bsearch}{bsearch}, \manlink{isnan}{isnan}.   
\end{manseealso}

% -- Authors
\begin{authors}
Bruno Pincon.
\end{authors}
