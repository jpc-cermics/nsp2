% -*- mode: latex -*-

\mansection{intersect}
\begin{mandesc}
  \short{intersect}{set intersection of two matrices, vectors or lists}
\end{mandesc}

% -- Calling sequence section
\begin{calling_sequence}
\begin{verbatim}
C = intersect(A,B)
[C, kA, kB] = intersect(A,B, ind_type=str, which=str)
\end{verbatim}
\end{calling_sequence}
% -- Parameters
\begin{parameters}
  \begin{varlist}
    \vname{A, B}: vectors or matrices of numbers (Mat or IMat), strings or cells , or lists.
    \vname{C}: vector, matrix or list with the intersection of components (elements, rows or columns) of A and B.
    \vname{kA, kB}: index vectors such that $C = A(kA)$ and $C = B(kB)$

    \vname{ind_type = str}: named optional argument, a string among \verb+{"double","int"}+ (default is \verb+"double"+)
    which gives the type for the index vectors  \verb+kA+ and \verb+kB+. 
    \vname{which = str}: named optional argument, a string among \verb+{"elements","columns","rows"}+ (default is
    \verb+"elements"+) or any non ambiguous abreviation. This option is useful only for matrices of floating 
       point numbers (Mat) or integer numbers (IMat) for which the intersection operation can be done also for
       columns or rows.
  \end{varlist}
\end{parameters}

\begin{mandescription}
\begin{itemize}
\item  By default (or when which option is "elements" ) this function computes the set 
  intersection $C = A \cap B$
  between 2 vectors (of numbers, strings or cells) or between 2
  lists, considered as sets.  If A or B are matrices, they  are considered as big 
  column vectors. For vectors of numbers or strings $C$ is sorted in ascending order.
  $C$ is a row vectors if both A and B are row vectors, otherwise it is
  a column vector.

\item  Using the option \verb+which = "rows"+ or \verb+which = "columns"+  the function 
  computes the intersection of the rows or of the columns of the two entry 
  matrices (feature available only for Mat and IMat matrices).
\end{itemize}

\end{mandescription}

\begin{examples}

\paragraph{example 1} with vectors of numbers:
\begin{mintednsp}{nsp}
A = [0.5, -1, 2, 1, 2, -1, 0.5, 2, -1, 2];
B = [2, 0.5, 3, 912];
C = intersect(A,B)
[C, kA, kB] = intersect(A,B)
// we must have C = A(kA) and C = B(kB) up to a permutation
C.equal[sort(A(kA),"g","i")]
C.equal[sort(B(kB),"g","i")]
\end{mintednsp}

\paragraph{example 2} with vectors of strings:
\begin{mintednsp}{nsp}
A = ["toto", "foo", "bar", "tumbi", "foobar", "bar", "toto", "foo", "bar"]
B = [ "foo", "bar", "nsp", "tumbi" ]
C = intersect(A,B)
[C, kA, kB] = intersect(A,B)
// we must have C = A(kA) and C = B(kB) up to a permutation
C.equal[sort(A(kA),"g","i")]
C.equal[sort(B(kB),"g","i")]
\end{mintednsp}

\paragraph{example 3} with vectors of cells:
  \begin{mintednsp}{nsp}
A = {"toto", [0,1;1,0], "foo", 1, 2, 1, "bar"}
B = {  "tumbi", 5 ,"foo", 2,  [0,1;1,0] }
C = intersect(A,B)
[C, kA, kB] = intersect(A,B)
// we must have C = A(kA) and C = B(kB)
C.equal[A(kA)]
C.equal[B(kB)]
  \end{mintednsp}

\paragraph{example 4} with lists:
\begin{mintednsp}{nsp}
A = list("toto", [0,1;1,0], "foo", 1, 2, 1, "bar")
B = list(  "tumbi", 5 ,"foo", 2,  [0,1;1,0] )
C = intersect(A,B)
[C, kA, kB] = intersect(A,B)
// we must have C = A(kA) and C = B(kB)
C.equal[A.sublist[kA]]
C.equal[B.sublist[kB]]
\end{mintednsp}


\paragraph{example 5} intersection of rows or columns:
\begin{mintednsp}{nsp}
// for columns
A = grand(2,6,"uin",-1,1)
B = grand(2,7,"uin",-1,1)
[C,kA,kB] = intersect(A,B,which="columns",ind_type="int")
C.equal[A(:,kA)]
C.equal[B(:,kB)]

// for rows
A = grand(8,2,"uin",-1,1)
B = grand(5,2,"uin",-1,1)
[C,kA,kB] = intersect(A,B,which="rows")  // or simply which="r"
// we must have C = sort([A(kA,:);B(kB,:)],"lr","i")
C.equal[A(kA,:)]
C.equal[B(kB,:)]
\end{mintednsp}

\end{examples}

\begin{manseealso}
  \manlink{unique}{unique}, \manlink{union}{union}, \manlink{setdiff}{setdiff}, \manlink{setxor}{setxor}    
\end{manseealso}

% -- Authors
\begin{authors}
  Bruno Pincon
\end{authors}
