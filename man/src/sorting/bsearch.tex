% -*- mode: latex -*-

\mansection{bsearch}

\begin{mandesc}
  \short{bsearch}{binary search}
\end{mandesc}

% -- Calling sequence section
\begin{calling_sequence}
\begin{verbatim}
[ind [,occ [,info]]]  = bsearch(X, v ,match=str1, interval=str2, assume_sorted=b, ind_type=str3)
\end{verbatim}
\end{calling_sequence}
% -- Parameters
\begin{parameters}
  \begin{varlist}
    \vname{X}:  matrix or vector of real numbers (Mat) or integers (IMat) or strings (SMat).
    \vname{v}: vector of real numbers, or integers, or strings in strict
    increasing order  $v_1 <  v_2 < ... v_n$ with $n \ge 2$ 
    when  \verb+match="i"+
    \vname{match=str1}: optional named argument, \verb+match="i"+
                         (default) for interval matching and
                         \verb+match="v"+ for exact value matching.
    \vname{interval=str2}: optional named argument useful only
                           when  \verb+match="i"+, \verb+interval="[--)"+
                         (default) or  \verb+interval="(--]"+
    \vname{assume_sorted=b}: boolean optional named argument (default
                              is \verb+%f+). Using \verb+assume_sorted=%t+ disable the test
                              on the strictly ordering of
                              \verb+v+. This can be useful if you know
                              that \verb+v+ is strictly ordered or to
                              use bsearch with a \verb+v+ which is
                              only sorted (that is with redundant elements).  
    \vname{ind_type = str3}: named optional argument, a string given the type for the index vector ind 
                             (default is \verb+"double"+)
    \vname{ind}: matrix or vector of indices with same dimensions than X.
    \vname{occ}: vector of same length than v (but with $n-1$ components 
                  when  \verb+match="i"+)
    \vname{info}: integer scalar
  \end{varlist}
\end{parameters}

\begin{mandescription}
This function is useful to search in an ordered table and/or to count the number of components
of a vector or matrix falling in some classes (a class being an interval or a value).

\itemdesc{interval matching case}
Selected by default or with  \verb+match="i"+, the function searches for each
X(i) in which of the $n-1$ intervals it falls, the intervals being defined by:
$$
   I_1 = [v_1, v_2), \;I_2 = [v_2, v_3),\;...,I_{n-2} = [v_{n-2},
   v_{n-1}),\; I_{n-1} = [v_{n-1}, v_n] 
$$
when \verb+interval="[--)"+ and by :
$$
   I_1 = [v_1, v_2], \;I_2 = (v_2, v_3],\;...,I_{n-2} = (v_{n-2},
   v_{n-1}],\; I_{n-1} = (v_{n-1}, v_n] 
$$
when \verb+interval="(--]"+
\begin{itemize}
   \item ind(i) is the interval number of X(i) (0 if X(i) is not in [v(1),v(n)])
   \item occ(k)  is the number of components of X which are in $I_k$
   \item info is the number of components of X which are not in [v(1),v(n)] 
\end{itemize}

\itemdesc{exact value matching case}
Selected with  \verb+match="v"+, in this case, the function searches,
for each X(i) if it is equal to one v(k) then:
\begin{itemize}
   \item ind(i) is equal to the index of the component of v which matches X(i) 
                   (ind(i) = k if X(i)=v(k)) or 0 if X(i) is not in v.
   \item occ(k) is the number of components of X equal to v(k)
   \item info is the number of components of X which are not in the set $\{v_1,...,v_n\}$ 
\end{itemize}

The type for the vector of indices \verb+ind+ is \verb+"double"+ (integers stored as
doubles) by default but you can get \verb+ind+ as a vector of integers (integers stored as
C int) using \verb+ind_type="int"+.

\end{mandescription}

\begin{examples}
  
\paragraph{example 1} numbers and interval matching:
\begin{program}\HCode{x=grand(1000,1,"def");\Hnewline
    v=linspace(0,1,11)';  // 10 intervals \Hnewline
    [ind,occ]=bsearch(x,v);\Hnewline
    xclear()\Hnewline
    plot2d2(v,[occ;occ(10)],style=2)}
\end{program}
  
\paragraph{example 2} playing with interval type:
\begin{program}\HCode{x = [1, 2, 3, 4]\Hnewline
[ind, occ] = bsearch(x,x,interval="[--)")\Hnewline
[ind, occ] = bsearch(x,x,interval="(--]")}
\end{program}
  
\paragraph{example 3} strings and exact matching:
\begin{program}\HCode{x = ["toto","est","un","foo"];\Hnewline
v = ["ab","baba","bar","foo","toto","un"];\Hnewline
[ind, occ,info] = bsearch(x,v,match="v")}
\end{program}
  
\paragraph{example 4} strings and interval matching:
\begin{program}\HCode{m = 5000;\Hnewline
    x = smat_create(m,1);\Hnewline
    for i=1:m, x(i) = ascii(grand(1,4,"uin",97,122)); end\Hnewline
    v = split("a b c d e f g h i j k l m n o p q r s t u v w x y z");\Hnewline
    [ind, occ,info] = bsearch(x,v,match="i");\Hnewline
    occ}
  \end{program}
  
\end{examples}

\begin{manseealso}
  \manlink{find}{find}, \manlink{mfind}{mfind}, \manlink{has}{has}, \manlink{unique}{unique}  
\end{manseealso}

% -- Authors
\begin{authors}
  Bruno Pincon
\end{authors}
