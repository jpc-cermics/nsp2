% -*- mode: latex -*-

\mansection{union}
\begin{mandesc}
  \short{union}{compute the union of two vectors or two lists or the union of the rows or columns of two matrices}
\end{mandesc}

% -- Calling sequence section
\begin{calling_sequence}
\begin{verbatim}
C = union(A,B)
[C [,kA, kB]] = union(A, B, ind_type=str, which=str)
\end{verbatim}
\end{calling_sequence}
% -- Parameters
\begin{parameters}
  \begin{varlist}
    \vname{A, B}: both vectors of numbers, strings or cells , or both lists.
    \vname{C}: vector or list with the union of elements of A and B.
    \vname{kA, kB}: vectors of indices such that $C = A(kA) \cup B(kB)$.
    \vname{ind_type = str}: named optional argument, a string among \verb+{"double","int"}+ (default is \verb+"double"+)
    which gives the type for the index vectors  \verb+kA+ and \verb+kB+. 
    \vname{which = str}: named optional argument, a string among \verb+{"elements","columns","rows"}+ (default is
    \verb+"elements"+) or any non ambiguous abreviation. This option is useful only for matrices of floating 
       point numbers (Mat) or integer numbers (IMat) for which the union operation can be done also for
       columns or rows.
  \end{varlist}
\end{parameters}

\begin{mandescription}
  This function computes the set union $C = A \cup B$
  between 2 vectors (of numbers, strings or cells) or between 2
  lists considered as sets (but uniqueness of each element in A or B is not
  required). If A or B are matrices, they are considered as a big column vectors.
  On output, $C$, $kA$ and $kB$ are row vectors if  $A$ and $B$ are both row vectors, otherwise they
  are column vectors. For numbers or strings $C$ is sorted.
\end{mandescription}

\begin{examples}

\paragraph{example 1} with vectors of numbers:
\begin{Verbatim}
A = [0.5, -1, 2];
B = [2, 2, -2, 4.4];
C = union(A,B)
[C, kA, kB] = union(A,B)
// we must have C = sort([A(kA),B(kB)],"g","i")
C.equal[sort([A(kA),B(kB)],"g","i")]
// index vectors kA and kB can be of IMat type
[C, kA, kB] = union(A,B,ind_type="int"))
\end{Verbatim}

\paragraph{example 2} with vectors of strings:
\begin{Verbatim}
A = ["beer", "red vine"]
B = [ "foo", "bar", "beer"]
C = union(A,B)
[C, kA, kB] = union(A,B)
// we must have C = sort([A(kA),B(kB)],"g","i")
C.equal[sort([A(kA),B(kB)],"g","i")]}
\end{Verbatim}

\paragraph{example 3} with vectors of cells:
\begin{Verbatim}
A = {"toto", [0,1], 1, 2, 1, "bar"}
B = { "foo",  3, [0,1], [2;3]}
C = union(A,B)
\end{Verbatim}

\paragraph{example 4} with lists:
\begin{Verbatim}
A = list("toto", [0,1], 1, 2, 1, "bar")
B = list( "foo", 3, [0,1], [2;3] )
C = union(A,B)
\end{Verbatim}

\paragraph{example 5} unions of rows or columns:
\begin{Verbatim}
// for columns
A = grand(2,6,"uin",-1,1)
B = grand(2,7,"uin",-1,1)
[C,kA,kB] = union(A,B,which="columns")  // or simply which="c"
// we must have C = sort([A(:,kA),B(:,kB)],"lc","i")
C.equal[sort([A(:,kA),B(:,kB)],"lc","i")]

// for rows
A = grand(8,2,"uin",-1,1)
B = grand(5,2,"uin",-1,1)
[C,kA,kB] = union(A,B,which="rows")  // or simply which="r"
// we must have C = sort([A(kA,:);B(kB,:)],"lr","i")
C.equal[sort([A(kA,:);B(kB,:)],"lr","i")]
\end{Verbatim}

\end{examples}

\begin{manseealso}
  \manlink{unique}{unique}, \manlink{setdiff}{setdiff}, \manlink{intersect}{intersect}, \manlink{setxor}{setxor}  
\end{manseealso}

% -- Authors
\begin{authors}
  Bruno Pincon
\end{authors}
