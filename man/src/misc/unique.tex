% -*- mode: latex -*-

\mansection{unique}
\begin{mandesc}
  \short{unique}{compute unique components of a vector (of numbers,
  strings or cells) or of a list} \\
\end{mandesc}

% -- Calling sequence section
\begin{calling_sequence}
\begin{verbatim}
y = unique(x)
[y [,ind [,occ]]] = unique(x)
[y [,ind [,occ]]] = unique(x,first_ind=value)
\end{verbatim}
\end{calling_sequence}
% -- Parameters
\begin{parameters}
  \begin{varlist}
    \vname{x} : vector of numbers or strings or cells, or a list.
    \vname{y} : vector (or list when x is a list) containing unique components of x
    \vname{ind} : vector of same length than y, ind(i) is an index
                  such that y(i)=x(ind(i)).
    \vname{occ} : vector of same length than y, occ(i) is the number
                  of occurrences of y(i) in the x.
    \vname{first\_ind} : optional boolean scalar, useful only when x is a vector
    of numbers or strings. In this case using \verb!first_ind=%t! ind(i) is
    the smaller index such that y(i) = x(ind(i)).
  \end{varlist}
\end{parameters}

\begin{mandescription}
  This function computes the unique components of a vector of numbers,
  strings or cells. As the number of occurrences of each unique
  component may be returned, it can be also useful for basic statistic
  computations.
\end{mandescription}

\begin{examples}

\noindent A first example with a vector of numbers:
  
  \begin{program}
    x = [0.5, -1, 2, 2, -1, 0.5, 2, -1, 2];
    [y, ind, occ] = unique(x)
  \end{program}

\noindent   a second example with a vector of strings:
  
  \begin{program}
    x = ["toto", "foo", "bar", "toto", "foobar", "bar", "toto", "foo", "bar"]
    [y, ind, occ] = unique(x)
  \end{program}

\noindent   a third one with a vector of cells:
  
  \begin{program}
    x = {["foo", "bar"], [0;1], [0,1], ["foo", "bar"], [0;1], {"toto"}, {"toto"}}
    [y, ind, occ] = unique(x)
  \end{program}

\noindent   a fourth one with a list:
  
  \begin{program}
    x = list(["foo", "bar"], [0;1], [0,1], ["foo", "bar"], [0;1], "toto", "toto")
    [y, ind, occ] = unique(x)
  \end{program}

\noindent another example:
  
  \begin{program}
    // generate a big sample of the binomial distribution B(6,0.4)
    m = 1e5;
    x = grand(m,1,"bin",6,0.4); 
    // computes the empirical probabilities associated to each values
    [y, ind, occ] = unique(x);
    p_emp = occ/m;
    // computes exact probabilities (using the cumulative distribution func)
    v = ones(7,1);
    p_cum = cdfbin("PQ", (0:6)', 6*v, 0.4*v, 0.6*v);
    p_exact = [p_cum(1); p_cum(2:$)-p_cum(1:$-1)];
    // compare both empirical and exact ones
    xbasc()
    e = 0.05;
    plot2d3([y-e,y+e], [p_exact,p_emp], style=[1 2], ...
            rect=[-1,0,7,1.1*max(p_exact)],...
            leg="exact probabilies@empirical probabilities")
  \end{program}
\end{examples}

\begin{manseealso}
  \manlink{model}{model}  
\end{manseealso}

% -- Authors
\begin{authors}
   Bruno
\end{authors}
