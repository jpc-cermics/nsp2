% -*- mode: latex -*-

\mansection{unique}
\begin{mandesc}
  \short{unique}{compute unique components of a vector or of a list}
\end{mandesc}

% -- Calling sequence section
\begin{calling_sequence}
\begin{verbatim}
y = unique(x)
[y [,ind [,occ]]] = unique(x)
[y [,ind [,occ]]] = unique(x,first_ind=value)
\end{verbatim}
\end{calling_sequence}
% -- Parameters
\begin{parameters}
  \begin{varlist}
    \vname{x} : vector of numbers or strings or cells, or a list.
    \vname{y} : vector (or list when x is a list) containing unique components of x
    \vname{ind} : vector of same length than y, ind(i) is an index
                  such that y(i)=x(ind(i)).
    \vname{occ} : vector of same length than y, occ(i) is the number
                  of occurrences of y(i) in the x.
    \vname{first\_ind} : optional boolean scalar, useful only when x is a vector
    of numbers or strings. In this case using \verb!first_ind=%t! ind(i) is
    the smaller index such that y(i) = x(ind(i)).
  \end{varlist}
\end{parameters}

\begin{mandescription}
  This function computes the unique components of a vector of numbers,
  strings or cells. As the number of occurrences of each unique
  component may be returned, it can be also useful for basic statistic
  computations.
\end{mandescription}

\begin{examples}

\paragraph{example 1} with a vector of numbers
\begin{program}\HCode{x = [0.5, -1, 2, 2, -1, 0.5, 2, -1, 2];\Hnewline
[y, ind, occ] = unique(x)}
\end{program}

\paragraph{example 2} with a vector of strings:
\begin{program}\HCode{x = ["toto", "foo", "bar", "toto", "foobar", "bar", "toto", "foo", "bar"]\Hnewline
[y, ind, occ] = unique(x)}
\end{program}

\paragraph{example 3} with a vector of cells:
\begin{program}\HCode{x = {["foo", "bar"], [0;1], [0,1], ["foo", "bar"], [0;1], {"toto"}, {"toto"}}\Hnewline
[y, ind, occ] = unique(x)}
\end{program}

\paragraph{example 4} with a list:
\begin{program}\HCode{x = list(["foo", "bar"], [0;1], [0,1], ["foo", "bar"], [0;1], "toto", "toto")\Hnewline
[y, ind, occ] = unique(x)}
\end{program}

\paragraph{example 5} using unique for basic statistic:
\begin{program}\HCode{// generate a big sample of the binomial distribution B(6,0.4)\Hnewline
m = 1e5;\Hnewline
x = grand(m,1,"bin",6,0.4); \Hnewline
// computes the empirical probabilities associated to each values\Hnewline
[y, ind, occ] = unique(x);\Hnewline
p_emp = occ/m;\Hnewline
// computes exact probabilities (using the cumulative distribution func)\Hnewline
v = ones(7,1);\Hnewline
p_cum = cdfbin("PQ", (0:6)', 6*v, 0.4*v, 0.6*v);\Hnewline
p_exact = [p_cum(1); p_cum(2:$)-p_cum(1:$-1)];\Hnewline
// compare both empirical and exact ones\Hnewline
xbasc()\Hnewline
e = 0.05;\Hnewline
plot2d3([y-e,y+e], [p_exact,p_emp], style=[1 2], ...\Hnewline
         rect=[-1,0,7,1.1*max(p_exact)],...\Hnewline
         leg="exact probabilies@empirical probabilities")}
\end{program}
\end{examples}

\begin{manseealso}
  \manlink{setdiff}{setdiff}  
\end{manseealso}

% -- Authors
\begin{authors}
  Bruno Pincon
\end{authors}
