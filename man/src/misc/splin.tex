% -*- mode: latex -*-

\mansection{splin}
\begin{mandesc}
  \short{splin}{cubic spline interpolation}
\end{mandesc}

%-- Calling sequence section
\begin{calling_sequence}
  \begin{verbatim}
   d = splin(x, y [,spline_type [, der]])
  \end{verbatim}
\end{calling_sequence}

  %-- Parameters

\begin{parameters}
  \begin{varlist}
   \vname{x } : a strictly increasing (row or column) vector with at least 2 components
   \vname{y } : a vector of same format than \verb!x!
   \vname{spline\_type } : (optional) a string selecting the kind of spline to compute
   \vname{der } : (optional) a vector with 2 components, with the end points derivatives (to 
               provide when spline\_type="clamped")
   \vname{d  } : vector of the same format than \verb!x!
  \end{varlist}
  \end{parameters}

  \begin{mandescription}
    This function computes a cubic spline or sub-spline $s$ which interpolates the 
    $(x_i,y_i)$ points, ie, we have $s(x_i)=yi$ for $i=1,..,n$. 
    The resulting spline $s$ is completly defined by the triplet $(x,y,d)$ where
    $d$ is the vector with the derivatives at the $x_i$: $s'(x_i)=d_i$.
    (this is called the {\em Hermite} form). 
    The evaluation of the spline at some points must be done by the  \manlink{interp}{interp}  function.
    Several kind of splines may be computed by selecting the appropriate \verb!spline_type!
    parameter: 
    
  \begin{itemize}

       \item \itemdesc{"not\_a\_knot"}  this is the default case, the cubic spline is computed by using the
             following conditions (considering $n$ points
             $x_1,...,x_n$ :
	     $$
    s^{(3)}(x_2-) = s^{(3)}(x_2+), \; s^{(3)}(x_{n-1}-) = s^{(3)}(x_{n-1}+)
         $$

       \item \itemdesc{"clamped"} in this case the cubic spline is computed by using the end points
             derivatives which must be provided as the last argument \verb!der!:
    $$
    s'(x_1) = der_1, \; s'(x_n) = der_2
    $$     

       \item \itemdesc{"natural"} the cubic spline is computed by using the conditions:
    $$
    s''(x_1) = 0, \; s''(x_n) = 0
    $$     

       \item \itemdesc{"periodic"} a periodic cubic spline is computed ($y$ must verify $y_1=y_n$)
              by using the conditions:
    $$
    s'(x_1) = s'(x_n), \;  s''(x_1) = s''(x_n)
    $$     

       \item \itemdesc{"monotone"} in this case a sub-spline ($s$ is  only one continuously differentiable)
              is computed by using a local scheme for the $d_i$ such that
              $s$ is monotone on each interval:
    \begin{quote}
    when $y_i <= y_{i+1}$,  $s$ is increasing on $[x_i, x_{i+1}]$ \newline
    and when $y_i >= y_{i+1}$,  $s$ is decreasing on $[x_i, x_{i+1}]$ 
    \end{quote}

       \item \itemdesc{"fast"} in this case a sub-spline is also computed by using a simple local scheme for 
              the $d_i$ :  $d_i$ is the derivative at $x_i$ of the interpolation polynomial of 
              $(x_{i-1},y_{i-1}), \; (x_i,y_i), \; (x_{i+1},y_{i+1})$, except for the end points ($d_1$ being
              computed from the 3 left most points and $d_n$ from the 3 right most points).
       \item \itemdesc{"fast\_periodic"} same as before but use also a centered formula for $d_1 = s'(x_1) = d_n = s'(x_n)$.
              by using the periodicity of the underlying function ($y$ must verify $y_1=y_n$). 
  \end{itemize}

  \end{mandescription}

  %-- section-Remarks

  \paragraph{Remarks}
  \begin{itemize}

       \item From an accuracy point of view use essentially the  type "clamped" if you know the
       end point derivatives, else use "not\_a\_knot". But if the underlying approximated
       function is periodic use the "periodic" type. Under the good assumptions these
       kind of splines got an $O(h^4)$ asymptotic behavior of the error. Don't use the 
        type "natural" unless the underlying function have zero second end points 
       derivatives.
       \item The "monotone" or "fast" type may be useful in some cases, for
       instance to limit oscillations (these kind of
       sub-splines have an $O(h^3)$ asymptotic behavior of the error).
       \item If $n=2$ (and \verb!spline_type! is not "clamped") linear interpolation
       is used. If {\em n=3} and \verb!spline_type! is  "not\_a\_knot", then
       a  sub-spline type is in fact computed.
    \end{itemize}

  %--example 

  \begin{examples}
\paragraph{example 1} interpolating the Runge function.
\begin{program}\HCode{function y=runge(x), y = 1./(1 + x.^2), endfunction;\Hnewline
a = -5; b = 5; n = 11; m = 400;\Hnewline
x = linspace(a, b, n)';\Hnewline
y = runge(x);\Hnewline
d = splin(x, y);\Hnewline
xx = linspace(a, b, m)';\Hnewline
yyi = interp(xx, x, y, d);\Hnewline
yye = runge(xx);\Hnewline
xbasc()\Hnewline
plot2d(xx, [yyi yye], style=[2 5], leg="interpolation spline@exact function")\Hnewline
plot2d(x, y, style=-9)\Hnewline
xtitle("interpolation of the Runge function")\Hnewline
}
    \end{program}
\paragraph{example 2} behavior of different splines on some random datas.
 \begin{program}\HCode{a = 0; b = 1;        // interval of interpolation\Hnewline
n = 10;              // nb of interpolation  points\Hnewline
m = 800;             // discretisation for evaluation\Hnewline
x = linspace(a,b,n)'; // abscissae of interpolation points\Hnewline
y = rand(x);          // ordinates of interpolation points\Hnewline
xx = linspace(a,b,m)';\Hnewline
yk = interp(xx, x, y, splin(x,y,"not_a_knot"));\Hnewline
yf = interp(xx, x, y, splin(x,y,"fast"));\Hnewline
ym = interp(xx, x, y, splin(x,y,"monotone"));\Hnewline
xbasc()\Hnewline
plot2d(xx, [yf ym yk], style=[5 2 3], strf="121", leg="fast@monotone@not a knot spline")\Hnewline
plot2d(x,y,style=-9, strf="000")  // to show interpolation points\Hnewline
xtitle("Various spline and sub-splines on random datas")\Hnewline
xselect()\Hnewline
 }
\end{program}

  \end{examples}

  %-- see also

  \begin{manseealso}
\manlink{interp}{interp} 
  \end{manseealso}

  %-- Authors

  \begin{authors}
	\paragraph{} Bruno Pin\c{c}on
	\paragraph{} F. N. Fritsch (pchim.f Slatec routine translated
	in C is used for monotone interpolation) 
  \end{authors}

