% -*- mode: latex -*-

\mansection{parse\_dim\_arg}
\begin{mandesc}
 \shortunder{parse\_dim\_arg}{parse_dim_arg}{parse a string or number corresponding to a dimensional argument}
\end{mandesc}
% -- Calling sequence section
\begin{calling_sequence}
\begin{verbatim}
 rep = parse_dim_arg( dim_arg, names=[arg_name,func_name])  
\end{verbatim}
\end{calling_sequence}
% -- Parameters
\begin{parameters}
  \begin{varlist}
    \vname{dim_arg}: a string choosen among  \verb+'M'+, \verb+'m'+, \verb+'*'+, \verb+'full'+, \verb+'FULL'+, \verb+'row'+,
    \verb+'ROW'+, \verb+'col'+, \verb+'COL'+ or an non ambiguous abbreviation or an integer.
    \vname{names=[arg_name,func_name]}: optional named argument, a string matrix with 2 strings.
    \vname{rep}: an integer (0 for full, 1 for row, 2 for column, -2 for matlab compatibility flag).
  \end{varlist}
\end{parameters}
\begin{mandescription}
  This function could be useful to parse a ``dimensional argument'' (like in 
the functions sum, prod, ...) which generally indicates the dimension along which 
an operation must be done. In nsp this kind of argument could be either a string
or an integer and this function returns the corresponding integer:  

\begin{tabular}{|r|l|l|}
\hline
integer & equivalent strings & comment \\
\hline
0       & \verb+"full"+, \verb+"FULL"+, \verb+"*"+  &  operation along all the matrix (considered as a big vector) \\
1       & \verb+"row"+, \verb+"ROW"+                &  operation along all the first dimension \\
2       & \verb+"col"+, \verb+"COL"+                &  operation along all the second dimension \\
-2      & \verb+"m"+, \verb+"M"+                    &  flag for matlab compatibility \\
\hline
\end{tabular}
 
In case of a string, the function tests if it is in the set of
valid strings or if it corresponds to a non ambiguous abbreviation. In case of a number
the function tests if it is an integer larger than or equal to -2. If the test
fails, an error is setted with an error message. To have a more
meaningful message you can provide the optional argument 
\verb+names=[arg_name,func_name]+.
\end{mandescription}

% --example 
\begin{examples}

\begin{mintednsp}{nsp}
function mu = freqmean(x,freq,dim=0)
   dim = parse_dim_arg(dim,names=["dim","freqmean"])
   if dim == -1 || dim > 2 then
      error("Error: unsupported dim flag")
   elseif dim == -2 then
      if isvector(x) then, dim = 0, else, dim = 1, end
   end
   mu = sum( x.*freq, dim )./ sum( freq, dim )
endfunction;
x = [1, 2, 2.5, 4, 9; 5, -1, 0, 1, -5]
f = [1, 9, 2,   1, 2; 3,  2, 1, 1,  4]
mu = freqmean(x,f,dim="m");
mu = freqmean(x,f,dim=2);
mu = freqmean(x,f,dim="c");
// uncorrect calls
if execstr('mu = freqmean(x,f,dim=-3)',errcatch=%t) then pause;end
if execstr('mu = freqmean(x,f,dim=3)',errcatch=%t) then pause;end
lasterror();
\end{mintednsp}
\end{examples}

% -- see also
\begin{manseealso}
\manlink{is_string_in_array}{is_string_in_array}
\end{manseealso}

\begin{authors}
  Bruno Pincon
\end{authors}


