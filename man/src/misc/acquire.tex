% -*- mode: latex -*-

\mansection{acquire}
\begin{mandesc}
  \short{acquire}{acquire value from calling stack frames} \\
\end{mandesc}

% -- Calling sequence section
\begin{calling_sequence}
\begin{verbatim}
 y = acquire(name [,where], def=object );
\end{verbatim}
\end{calling_sequence}
% -- Parameters
\begin{parameters}
  \begin{varlist}
    \vname{name}: a string giving the name of symbol to search.
    \vname{y}: a nsp object.
    \vname{where}: an optional character string which should be chosen 
    among \verb+'all'+, \verb+'caller'+, \verb+'callers'+, \verb+'local'+,
    \verb+'global'+, \verb+'function'+, \verb+'nsp-function'+, or 
    \verb+'callable'+.
    \vname{object}: a nsp object.
  \end{varlist}
\end{parameters}

\begin{mandescription}
  This function is used to search a symbol in the frames defined by where (the default value 
  is \verb+"callers"+) and copy the value of the symbol in the current frame if such a symbol is found. If the symbol is not found an error is returned or a default value is returned if 
  the named optional argument \verb+def+ was given. 
\end{mandescription}

\begin{examples}
\begin{Verbatim}
x=1:3;
function y=f(n)
 if n >= 10 then y=acquire('x'); 
 else y=f(n+1);
 end 
endfunction 
y=f(0);
y=acquire('foo','caller',def=67);
\end{Verbatim}
\end{examples}

\begin{manseealso}
  \manlink{resume}{resume}  
\end{manseealso}

% -- Authors
\begin{authors}
  Jean-Philippe Chancelier
\end{authors}
