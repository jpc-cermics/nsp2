\mansection{sum and prod}
\begin{mandesc}
  \short{sum}{sum of matrix elements}\\ % @mandesc@
  \short{prod}{product of matrix elements}\\ % @mandesc@
\end{mandesc}
%-- Calling sequence section
\begin{calling_sequence}
\begin{verbatim}
  B=sum(A,mode)  
  B=prod(A,mode)  
  B=sum(A,dim=mode) 
  B=prod(A,dim=mode)  
\end{verbatim}
\end{calling_sequence}
%-- Parameters
\begin{parameters}
  \begin{varlist}
    \vname{A,B} : matrices.
    \vname{mode} : A string chosen among \verb+'M'+, \verb+'m'+, \verb+'full'+, \verb+'FULL'+, \verb+'row'+,
    \verb+'ROW'+, \verb+'col'+, \verb+'COL'+ or an non ambiguous abbreviation or an integer. 
    This argument is optional and if omitted 'full' is assumed.
  \end{varlist}
\end{parameters}
\begin{mandescription}
  \verb+sum+ (resp. \verb+prod+) gives the sum (resp. the product) of the elements of the given matrix
  argument \verb+A+. 
  The second argument gives the dimension to be used for performing the sum (resp product) of elements.
  \begin{itemize}
    \item 'full' or 0  $$B= \sum_{i,j} A_{i,j} \quad \left(\text{resp. } \prod_{i,j} A_{i,j} \right)$$ and thus gives a scalar.
    \item 'row' or 1  $$B_j = \sum_{i} A_{i,j}\quad \left(\text{resp. } \prod_{i} A_{i,j} \right)$$ and thus gives a row vector.
    \item 'col' or 2  $$B_i = \sum_{j} A_{i,j}\quad \left(\text{resp. } \prod_{j} A_{i,j}\right)$$ and thus gives a column vector.
    \item 'm' is the sum (resp. product) along the first non singleton dimension of the given matrix 
      for Matlab compatibility. 
  \end{itemize}
\end{mandescription}
%--example 
\begin{examples}
  \begin{program}\HCode{A=rand(4,5) \Hnewline
      sum(A,'c'); \Hnewline
      // boolean matrix \Hnewline
      sum(A>=0.5,'c');\Hnewline
      prod(1:5,'m') \Hnewline
      // sparse matrix 
      sum(sparse(rand(4,6)),'r')  }
  \end{program}
\end{examples}
%-- see also
\begin{manseealso}
  \manlink{cumsum}{cumsum}  \manlink{cumprod}{cumprod} 
\end{manseealso}

