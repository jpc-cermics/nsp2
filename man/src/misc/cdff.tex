% -*- mode: latex -*-

\mansection{cdff}
\begin{mandesc}
  \short{cdff}{cumulative distribution function F distribution}
\end{mandesc}
\index{cdff}\label{cdff}
%-- Calling sequence section
\begin{calling_sequence}
\begin{verbatim}
  [P,Q]=cdff("PQ",F,Dfn,Dfd)  
  [F]=cdff("F",Dfn,Dfd,P,Q);  
  [Dfn]=cdff("Dfn",Dfd,P,Q,F);  
  [Dfd]=cdff("Dfd",P,Q,F,Dfn)  
\end{verbatim}
\end{calling_sequence}
%-- Parameters
\begin{parameters}
  \begin{varlist}
    \vname{P,Q,F,Dfn,Dfd} : five real vectors of the same size.
    \vname{P,Q (Q=1-P)  } :  The integral from 0 to F of the f-density. Input range: [0,1].
    \vname{F} : Upper limit of integration of the f-density. Input range: [0, +infinity). Search range: [0,1E300]
      \vname{Dfn} : Degrees of freedom of the numerator sum of squares. Input range: (0, +infinity). Search range: [ 1E-300, 1E300]
      \vname{Dfd} : Degrees of freedom of the denominator sum of squares. Input range: (0, +infinity). Search range: [ 1E-300, 1E300]
  \end{varlist}
\end{parameters}
\begin{mandescription}
  Calculates any one parameter of the F distribution
  given values for the others.
  Formula   26.6.2   of   Abramowitz   and   Stegun,  Handbook  of
  Mathematical  Functions (1966) is used to reduce the computation
  of the  cumulative  distribution function for the  F  variate to
  that of an incomplete beta.
  Computation of other parameters involve a seach for a value that
  produces  the desired  value  of P.   The search relies  on  the
  monotinicity of P with the other parameter.
  The value of the  cumulative  F distribution is  not necessarily
  monotone in  either degrees of freedom.  There  thus may  be two
  values  that  provide a given CDF  value.   This routine assumes
  monotonicity and will find an arbitrary one of the two values.
\end{mandescription}

\begin{authors}
  Nsp interface by Jean-Philippe Chancelier. Code from DCDFLIB: 
  Library of Fortran Routines for Cumulative Distribution
  Functions, Inverses, and Other Parameters (February, 1994)
  Barry W. Brown, James Lovato and Kathy Russell. The University of Texas.
\end{authors}
