\mansection{linspace}
\begin{mandesc}
  \short{linspace}{regular mesh/partition of a segment}
\end{mandesc}
%-- Calling sequence section
\begin{calling_sequence}
\begin{verbatim}
  x = linspace(a, b, n)
\end{verbatim}
\end{calling_sequence}

%-- Parameters
\begin{parameters}
  \begin{varlist}
    \vname{a,b}: real scalars or column vectors of same length
    \vname{n}: a non negative integer
  \end{varlist}
\end{parameters}
\begin{mandescription}
  \verb+linspace+ computes a row vector of $n$ components equally spaced from $a$ to $b$ included. This
  corresponds to mesh the segment $[a,b]$ with $n-1$ intervals of same length $(b-a)/(n-1)$.

   When $a$ and $b$ are column vectors, the $i$ th row of $x$ is such a vector starting from $a(i)$ and ended
  at $b(i)$.

  {\em Special cases: } when $n=1$ then $x$ is equal to $b$ and when $n=0$ an empty matrix is returned.
\end{mandescription}

%--example 
\begin{examples}
\begin{Verbatim}
// cut [0,1] in 4 intervals   
x = linspace(0,1,5)

// cut [0,1] in 5 intervals   
x = linspace(0,1,6)

// using linspace to plot a function   
x = linspace(0,2*%pi,61)';
y1 = cos(x); y2 = sin(x);
xbasc(); plot2d(x,[y1,y2],style=[2,5],leg="cos@sin")
\end{Verbatim}

\end{examples}

% -- see also
\begin{manseealso}
  \manlink{logspace}{logspace}
\end{manseealso}

