% -*- mode: latex -*-

\mansection{intg}
\begin{mandesc}
  \short{intg}{one dimensional integration over a finite interval}
\end{mandesc}

% -- Calling sequence section
\begin{calling_sequence}
\begin{verbatim}
[Ia, [,ea_estim, [,ier]]]  = intg(a, b, f)
[Ia, [,ea_estim, [,ier]]]  = intg(a, b, f, atol=ea, rtol=er, args=L, limit=maxsub)
\end{verbatim}
\end{calling_sequence}
% -- Parameters
\begin{parameters}
  \begin{varlist}
    \vname{a,b}:  real scalars defining the integration interval
    \vname{f}: nsp function defining the function to integrate
    \vname{atol=ea}: optional named argument, requested absolute error
                      on the result (default is 1e-14)
    \vname{rtol=er}: optional named argument, requested relative error
                      on the result (default is 1e-8).
    \vname{args=L}: optional named argument, useful to pass
    additional parameters needed the function f.
    \vname{limit=maxsub}: optional named argument, max number of
    subdivisions allowed (default 750).
    \vname{Ia}: computed approximate result
    \vname{ea_estim}: estimated error
    \vname{ier}: scalar given some information about the computation.
  \end{varlist}
\end{parameters}

\begin{mandescription}
This function computes an approximation $I_a$ of:
$$
   I = \int_a^b f(x) dx
$$
such that:
$$
   | I - I_a | \le  max(atol, rtol*|I|)
$$
Additional parameters for the function $f$ could be passed using the
named optional argument \verb+args+ which must be a
list. \verb+ea_estim+ is a computed estimation of $| I - I_a |$. The
\verb+ier+ variable could take values from 0 to 5: 
\begin{itemize}
\item ier = 0,  normal and reliable termination of the routine. It is assumed that the
      requested  accuracy has been achieved.
\item other values correspond to an abnormal termination of the routine. The estimates
      for integral and error are less reliable. It is assumed that the  requested accuracy
      has not been achieved:
      \begin{itemize}
      \item ier = 1: maximum number of subdivisions (the limit
            parameter) allowed has been achieved. You can try to
            increase limit but generally it is advised to have a close look at the
            integrand, in order to determine the integration
            difficulties. If the position of a local difficulty can
            be determined (i.e.  singularity, discontinuity within
            the interval) one will probably gain from  splitting up
            the interval at this point and calling the integrator
            on the sub-ranges. If possible, an appropriate
            special-purpose integrator should be used which is
            designed for handling the type  of difficulty involved.
      \item ier = 2: the occurrence of roundoff error is detected
            which prevents the requested tolerance from being
            achieved. The error may be under-estimated.
      \item ier = 3: extremely bad integrand behaviour occurs at 
            some interior points of the integration interval.
      \item ier = 4: it is presumed that the requested tolerance 
            cannot be achieved, and that the returned result is the
            best which can be obtained.
      \item ier = 5: the integral is probably divergent, or slowly
            convergent. It must be noted that divergency can occur with
            any other value of ier.
      \end{itemize}
\end{itemize}

\end{mandescription}

\begin{examples}
  
\paragraph{example 1} a simple example 
\begin{program}\HCode{function y=f1(x);y=exp(x);endfunction\Hnewline
[I,ea] = intg(0,1,f1)\Hnewline
abs(I-\%e-1) //near exact absolute error}
\end{program}
  
\paragraph{example 2} comparizon with the error function
\begin{program}\HCode{function y=f2(x);y=2*exp(-x^2)/sqrt(\%pi);endfunction\Hnewline
[Ia,ea] = intg(0,0.5,f2)\Hnewline
I = erf(0.5)\Hnewline
abs(I - Ia)}
\end{program}
  
\paragraph{example 3} passing extra parameters
\begin{program}\HCode{function y=g(x,L);p1=L(1);p2=L(2);y=p1*x + sin(p2*x);endfunction\Hnewline
[Ia,ea] = intg(0,2*\%pi,g,args=list(1,1))\Hnewline
// here the exact result should be 2*pi^2\Hnewline
abs(Ia - 2*\%pi^2)}
\end{program}
  
\end{examples}

\begin{manseealso}
  \manlink{ode}{ode}  
\end{manseealso}

% -- Authors
\begin{authors}
  uses the routine dqags (Robert Piessens and Elise de Doncker) from
  quadpack. Nsp interface: Bruno Pincon.
\end{authors}
