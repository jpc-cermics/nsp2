\mansection{cross}
\begin{mandesc}
  \short{cross}{cross product of 2 vectors}
\end{mandesc}
%-- Calling sequence section
\begin{calling_sequence}
\begin{verbatim}
  C=cross(A,B)  
  C=cross(A,B,dim)  
\end{verbatim}
\end{calling_sequence}
%-- Parameters
\begin{parameters}
  \begin{varlist}
    \vname{A,B} : numerical vectors or matrices of same size $3 \times n$ or $m \times 3$. 
    \vname{dim} : an integer to choose among 1 or 2
  \end{varlist}
\end{parameters}
\begin{mandescription}
  \verb+cross+ computes the cross product between the vectors \verb+A+ and \verb+B+ of length 3:
$$
\begin{array}{l}
     C_1 = A_2 B_3 - A_3 B_2 \\
     C_2 = B_1 A_3 - B_3 A_1 \\
     C_3 = A_1 B_2 - A_2 B_1
\end{array}
$$
 If  $A$ and $B$ are rows vectors then $C$ is a row vector, generally:
 \begin{itemize}
 \item if $A$ and $B$ are $3 \times n$, then $C$ is $3 \times n$ and its columns are the 
 cross product  of the corresponding column vectors of  $A$ and $B$.
 \item if $A$ and $B$ are $m \times 3$, then $C$ is $m \times 3$ and its rows are the 
 cross product  of the corresponding row vectors of  $A$ and $B$.
 \item {\bf so} the third argument is only mandatory when  $A$ and $B$ have size $3 \times 3$
 to choose between the cross product of the respective rows or columns of $A$ and $B$.
\end{itemize}
When $A$ and $B$ are both real vectors then $|| C ||_2$ (\verb+norm(C)+) is equal to the 
array of the parallelogram defined by $A$ and $B$.   
\end{mandescription}

%--example 
\begin{examples}
\paragraph{example 1} simple example
  \begin{program}\HCode{x = randn(3,1);\Hnewline
    y = randn(3,1);\Hnewline
    z = cross(x,y)\Hnewline
    // verify orthonogonality:\Hnewline
    dot(x,z)\Hnewline
    dot(y,z)}
  \end{program}
\end{examples}

% -- see also
\begin{manseealso}
  \manlink{dot}{dot}
\end{manseealso}

