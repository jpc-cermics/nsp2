% -*- mode: latex -*-
\mansection{linear\_interpn}
\begin{mandesc}
  \shortunder{linear\_interpn}{linear_interp}{n dimensional linear interpolation}\\ % @mandesc@
\end{mandesc}
\index{linear\_interpn}\label{linear-interpn}
%-- Calling sequence section
\begin{calling_sequence}
\begin{verbatim}
vp = linear_interpn(xp1,xp2,..,xpn, x1, ..., xn, v [,out_mode])
\end{verbatim}
\end{calling_sequence}
%-- Parameters
\begin{parameters}
  \begin{varlist}
    \vname{xp1, xp2, .., xpn} : real vectors (or matrices) of same size
    \vname{x1 ,x2, ..., xn} :  strictly increasing row vectors (with at least 2 components)
    defining the n dimensional interpolation grid
    \vname{v} :  vector (case n=1), matrix (case n=2) or hypermatrix (case n $>$ 2) with the
    values of the underlying interpolated function at the grid points.
   \vname{out\_mode} : (optionnal) string defining the evaluation outside the grid (extrapolation)
   \vname{vp} :  vector or matrix of same size than \verb!xp1, ..., xpn!
  \end{varlist}
\end{parameters}
\begin{mandescription}
  Given a n dimensional grid defined by the n vectors \verb!x1 ,x2, ..., xn! and the
  values \verb!v! of a function (says {\em f}) at the grid points :
\begin{verbatim}
      v(i1,i2,...,in) = f(x1(i1),x2(i2), ..., xn(in))
\end{verbatim}
  This function computes the linear interpolant of {\em f} from the grid (called {\em s}
  in the following) at the points which coordinates are defined by the vectors (or matrices) 
  \verb!xp1, xp2, ..., xpn!:
\begin{verbatim}
   vp(i) = s(xp1(i),xp2(i), ..., xpn(i))
   or vp(i,j) = s(xp1(i,j),xp2(i,j), ..., xpn(i,j)) in case the xpk are matrices
\end{verbatim}
    The \verb!out_mode! parameter set the evaluation rule for extrapolation: if we note 
    {\em Pi=(xp1(i),xp2(i),...,xpn(i))} then \verb!out_mode! defines the
    evaluation rule when:
\begin{verbatim}
   P(i) not in  [x1(1) x1($)] x [x2(1) x2($)] x ... x [xn(1) xn($)]
\end{verbatim}
    The different choices are:
  \begin{itemize}
  \item \itemdesc{"by\_zero"} : an extrapolation by zero is done
  \item \itemdesc{"by\_nan"} : extrapolation by Nan
  \item \itemdesc{"C0"} : the extrapolation is defined as follows:
\begin{verbatim}
     s(P) = s(proj(P)) where proj(P) is nearest point from P 
                       located on the grid boundary.
\end{verbatim}
\item \itemdesc{"natural"} : the extrapolation is done by using the nearest n-linear patch from the point.
\item \itemdesc{"periodic"} : \verb!s! is extended by periodicity. 
  \end{itemize}
\end{mandescription}
%--example 
\begin{examples}
  \paragraph{example 1 : 1d linear interpolation}
  \begin{program}\HCode{x = linspace(0,2*%pi,11);\Hnewline
y = sin(x);\Hnewline
xx = linspace(-2*%pi,4*%pi,400)';\Hnewline
yy = linear_interpn(xx, x, y, "periodic");\Hnewline
xbasc()\Hnewline
plot2d(xx,yy,style=2)\Hnewline
plot2d(x,y,style=-9, strf="000")\Hnewline
xtitle("linear interpolation of sin(x) with 11 interpolation points")} 
    \end{program}
  \paragraph{example 2 : bilinear interpolation}
  \begin{program}\HCode{n = 8;\Hnewline
x = linspace(0,2*%pi,n); y = x;\Hnewline
z = 2*sin(x')*sin(y);\Hnewline
xx = linspace(0,2*%pi, 40);\Hnewline
[xp,yp] = ndgrid(xx,xx);\Hnewline
zp = linear_interpn(xp,yp, x, y, z);\Hnewline
xbasc()\Hnewline
plot3d(xx, xx, zp, flag=[2 6 4])\Hnewline
[xg,yg] = ndgrid(x,x);\Hnewline
param3d1(xg,yg, list(z,-9*ones(1,n)), flag=[0 0])\Hnewline
xtitle("Bilinear interpolation of 2sin(x)sin(y)")\Hnewline
legends("interpolation points",-9,1)\Hnewline
xselect()} 
  \end{program}
  \paragraph{example 3 : bilinear interpolation and experimentation 
    with all the outmode features}
  \begin{program}\HCode{nx = 20; ny = 30;\Hnewline
x = linspace(0,1,nx);\Hnewline
y = linspace(0,2, ny);\Hnewline
[X,Y] = ndgrid(x,y);\Hnewline
z = 0.4*cos(2*%pi*X).*cos(%pi*Y);\Hnewline
nxp = 60 ; nyp = 120;\Hnewline
xp = linspace(-0.5,1.5, nxp);\Hnewline
yp = linspace(-0.5,2.5, nyp);\Hnewline
[XP,YP] = ndgrid(xp,yp);\Hnewline
zp1 = linear_interpn(XP, YP, x, y, z, "natural");\Hnewline
zp2 = linear_interpn(XP, YP, x, y, z, "periodic");\Hnewline
zp3 = linear_interpn(XP, YP, x, y, z, "C0");\Hnewline
zp4 = linear_interpn(XP, YP, x, y, z, "by_zero");\Hnewline
zp5 = linear_interpn(XP, YP, x, y, z, "by_nan");\Hnewline
xbasc()\Hnewline
subplot(2,3,1)\Hnewline
   plot3d(x, y, z, leg="x@y@z", flag = [2 4 4])\Hnewline
   xtitle("initial function 0.4 cos(2 pi x) cos(pi y)")\Hnewline
subplot(2,3,2)\Hnewline
   plot3d(xp, yp, zp1, leg="x@y@z", flag = [2 4 4])\Hnewline
   xtitle("Natural")\Hnewline
subplot(2,3,3)\Hnewline
   plot3d(xp, yp, zp2, leg="x@y@z", flag = [2 4 4])\Hnewline
   xtitle("Periodic")\Hnewline
subplot(2,3,4)\Hnewline
   plot3d(xp, yp, zp3, leg="x@y@z", flag = [2 4 4])\Hnewline
   xtitle("C0")\Hnewline
subplot(2,3,5)\Hnewline
   plot3d(xp, yp, zp4, leg="x@y@z", flag = [2 4 4])\Hnewline
   xtitle("by_zero")\Hnewline
subplot(2,3,6)\Hnewline
   plot3d(xp, yp, zp5, leg="x@y@z", flag = [2 4 4])\Hnewline
   xtitle("by_nan")\Hnewline
xselect()}
\end{program}

\paragraph{example 4 : trilinear interpolation}
(see splin3d help page which have the same example with 
tricubic spline interpolation)
\begin{program}\HCode{getf("SCI/demos/interp/interp_demo.sci") \Hnewline
func =  "v=(x-0.5).^2 + (y-0.5).^3 + (z-0.5).^2";\Hnewline
deff("v=f(x,y,z)",func);\Hnewline
n = 5; \Hnewline
x = linspace(0,1,n); y=x; z=x;\Hnewline
[X,Y,Z] = ndgrid(x,y,z);\Hnewline
V = f(X,Y,Z);\Hnewline
// compute (and display) the linear interpolant on some slices\Hnewline
m = 41;\Hnewline
dir = ["z="  "z="  "z="  "x="  "y="];\Hnewline
val = [ 0.1   0.5   0.9   0.5   0.5];\Hnewline
ebox = [0 1 0 1 0 1];\Hnewline
\Hnewline
XF=[]; YF=[]; ZF=[]; VF=[];\Hnewline
for i = 1:length(val)\Hnewline
   [Xm,Xp,Ym,Yp,Zm,Zp] = slice_parallelepiped(dir(i), val(i), ebox, m, m, m);\Hnewline
   Vm = linear_interpn(Xm,Ym,Zm, x, y, z, V);\Hnewline
   [xf,yf,zf,vf] = nf3dq(Xm,Ym,Zm,Vm,1);\Hnewline
   XF = [XF xf]; YF = [YF yf]; ZF = [ZF zf]; VF = [VF vf]; \Hnewline
   Vp =  linear_interpn(Xp,Yp,Zp, x, y, z, V);\Hnewline
   [xf,yf,zf,vf] = nf3dq(Xp,Yp,Zp,Vp,1);\Hnewline
   XF = [XF xf]; YF = [YF yf]; ZF = [ZF zf]; VF = [VF vf]; \Hnewline
end\Hnewline
nb_col = 128;\Hnewline
vmin = min(VF); vmax = max(VF);\Hnewline
color = dsearch(VF,linspace(vmin,vmax,nb_col+1));\Hnewline
xset("colormap",jetcolormap(nb_col));\Hnewline
xbasc()\Hnewline
xset("hidden3d",xget("background"))\Hnewline
colorbar(vmin,vmax)\Hnewline
plot3d(XF, YF, list(ZF,color), flag=[-1 6 4])\Hnewline
xtitle("tri-linear interpolation of "+func)\Hnewline
xselect()}
  \end{program}
\end{examples}
%-- see also
\begin{manseealso}
  \manlink{interpln}{interpln} \manlink{splin}{splin} \manlink{splin2d}{splin2d} \manlink{splin3d}{splin3d}  
\end{manseealso}
%-- Author
\begin{authors}
  B. Pincon
\end{authors}

