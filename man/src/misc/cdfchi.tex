% -*- mode: latex -*-
\mansection{cdfchi}
\begin{mandesc}
  \short{cdfchi}{cumulative distribution function chi-square distribution}
\end{mandesc}
\index{cdfchi}\label{cdfchi}
%-- Calling sequence section
\begin{calling_sequence}
\begin{verbatim}
  [P,Q]=cdfchi("PQ",X,Df)  
  [X]=cdfchi("X",Df,P,Q);  
  [Df]=cdfchi("Df",P,Q,X)  
\end{verbatim}
\end{calling_sequence}
%-- Parameters
\begin{parameters}
  \begin{varlist}
    \vname{P,Q,Xn,Df}: four real vectors of the same size.
    \vname{P,Q (Q=1-P)}:  the integral from 0 to X of the chi-square distribution. Input range: [0, 1]. 
    \vname{X}: upper limit of integration of the non-central chi-square distribution. Input range: [0, +infinity). Search range: [0,1E300]  
      \vname{Df}: degrees of freedom of the chi-square distribution. Input range: (0, +infinity). Search range: [ 1E-300, 1E300]
  \end{varlist}
\end{parameters}
\begin{mandescription}
  Calculates any one parameter of the chi-square 
  distribution given values for the others.
  The chi-square distribution is given by:
  \begin{equation}
    \frac{(1/2)^{(Df/2)}}{\Gamma(Df/2)} x^{(Df/2 -1)} * e^{-x/2} 
  \end{equation}
  with domain $(0,\infty)$.
  Formula    26.4.19   of Abramowitz  and     Stegun, Handbook  of
  Mathematical Functions   (1966) is used   to reduce the chi-square
  distribution to the incomplete distribution.
  Computation of other parameters involve a seach for a value that
  produces  the desired  value  of P.   The search relies  on  the
  monotinicity of P with the other parameter.
\end{mandescription}

\begin{authors}
  Nsp interface by Jean-Philippe Chancelier. Code from DCDFLIB: 
  Library of Fortran Routines for Cumulative Distribution
  Functions, Inverses, and Other Parameters (February, 1994)
  Barry W. Brown, James Lovato and Kathy Russell. The University of Texas.
\end{authors}
