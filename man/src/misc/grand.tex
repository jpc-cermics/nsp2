% -*- mode: latex -*-
\mansection{grand}
\begin{mandesc}
  \short{grand}{random numbers generation}
\end{mandesc}
%-- Calling sequence section
\begin{calling_sequence}
\begin{verbatim}
X=grand(m, n, dist_type [,p1,...,pk])  
X=grand(Mat, dist_type [,p1,...,pk])  
X=grand(n, dist_type [,p1,...,pk])  
S=grand(action [,q1,....,ql])  
\end{verbatim}
\end{calling_sequence}
  %-- Parameters
\begin{parameters}
  \begin{varlist}
    \vname{m, n}: integers, size of the wanted matrix \verb!Y!
   \vname{Mat}: a matrix whom only the dimensions (say \verb!m x n!) are used
   \vname{dist\_type}: a string given the distribution which (independants) variates are to be 
     generated (\verb!"bin"!, \verb!"nor"!, \verb!"poi"!, etc ...)
   \vname{p1, ..., pk}: the parameters (reals or integers) required to define the distribution 
    \verb!dist_type!
   \vname{X}: the resulting \verb!m x n! random matrix
   \vname{action}: a string given the action onto the base generator(s) (\verb!"setgen"! to change the current base 
     generator,  \verb!"getgen"! to get the current base generator name, \verb!"getsd"! to get the 
     state (seeds) of the current base generator, etc ...)
   \vname{q1, ..., ql} : the parameters (generally one string) needed to define the action
   \vname{S} : output of the action (generally a string or a real column vector)
  \end{varlist}
  \end{parameters}
  
\begin{mandescription}
  This function may be used to generate random numbers from various distributions. In this 
  case you must apply one of the three first forms of the possible
  calling sequences to get an \verb!m x n! matrix. 
  The two firsts are equivalent if \verb!Mat! is a \verb!m x n! matrix, 
  and the third one corresponds to multivariate distributions (e.g. multinomial, multivariate
  gaussian, etc...) where a sample is a column vector (says of dim \verb!m!)
  and you get then \verb!n! such random vectors (as an \verb!m x n! matrix). 
  The last form is used to undertake various manipulations onto the base generators
  like changing the base generator or changing or retrieving its internal state (seeds), 
  etc ... These base generators give random integers following a
  uniform distribution on a large integer interval of the form $[0,n)$
  (with $n=2^{32}$ in most cases). Others distributions are got from
  this one through various transformations.
\end{mandescription}

%-- section-Getting random numbers from a given distribution
\paragraph{Getting random numbers from a given distribution}

\begin{description}
\item[univariate distributions] Rmk: 
\begin{enumerate}
\item distribution parameters could be either scalars or vectors or
  matrices of same dimensions than the wanted output matrix
\item densities or mass functions for most these distributions are
  explicited in the \manlink{pdf}{pdf} help page.
\end{enumerate}

\begin{itemize}
\item \itemdesc{beta} \verb!X=grand(m,n,"bet",a,b)! generates random variates from 
  the beta distribution with parameters $a$ and $b$ (positive reals).
%  Related function(s) : \manlink{cdfbet}{cdfbet} .

\item \itemdesc{binomial} 
  \verb!X=grand(m,n,"bin",n,p)!  generates random variates from the binomial 
  distribution with parameters $n$ (positive integer) and $p$
  (real in [0,1]).
%  Related function(s) : \manlink{binomial}{binomial} , \manlink{cdfbin}{cdfbin} .

\item \itemdesc{negative binomial} 
  \verb!X=grand(m,n,"nbn",r,p)! generates random variates from the negative binomial 
  distribution with parameters $r$ (positive real) and $p$ (real 
  in $(0,1]$).
%  Related function(s) : \manlink{cdfnbn}{cdfnbn} .
  
\item \itemdesc{Cauchy} 
  \verb!X=grand(m,n,"cauchy", sigma)! generates random variates from the Cauchy distribution 
  with (scale) parameter $sigma$ (positive real).
  
\item \itemdesc{chi square} 
  \verb!X=grand(m,n,"chi", nu)! generates random variates from the chisquare distribution 
  with $nu$ (positive real) degrees of freedom. 
%  Related function(s) : \manlink{cdfchi}{cdfchi} .  
  
\item \itemdesc{non central chisquare} 
  \verb!X=grand(m,n,"nch",nu,xnonc)! generates random variates from the non central chisquare
  distribution with $nu$ degrees of freedom (real $\ge 1$) and
  noncentrality parameter $xnonc$ (non negative real).
%  Related function(s) : \manlink{cdfchn}{cdfchn} .
  
\item \itemdesc{exponential} \verb!X=grand(m,n,"exp",tau)! generates random variates from the exponential
  distribution with mean $\tau$ (non negative real). Caution : usually
  the exponential distribution 's parameter is $\lambda= 1/\tau$.  
  
\item \itemdesc{F variance ratio} 
  \verb!X=grand(m,n,"f",nu1,nu2)! generates random variates from the F 
  (variance ratio) distribution with $nu1$ (positive real)
  degrees of freedom in the numerator and $nu2$ (positive real) 
  degrees of freedom in the denominator. 
% Related function(s) : \manlink{cdff}{cdff} .
  
\item \itemdesc{non central F variance ratio} 
  \verb!X=grand(m,n,"nf",nu1,nu2,xnonc)! generates random variates from the noncentral 
  F (variance ratio)  distribution with $nu1$ (real $\ge 1$) degrees of freedom 
  in the numerator, and $nu2$ (positive real) degrees of freedom in the denominator, 
  and noncentrality parameter $xnonc$ (non negative real). 
%  Related function(s) : \manlink{cdffnc}{cdffnc} .
  
\item \itemdesc{gamma} \verb!X=grand(m,n,"gam",a,b)! generates random variates from the gamma 
  distribution with shape parameter $a$ (positive real) and rate
  parameter $b$ (positive real). Caution: often the second parameter
  is the ``scale'', the inverse of the ``rate''.  
%  Related function(s) : \manlink{gamma}{gamma} , \manlink{cdfgam}{cdfgam} .
  
\item \itemdesc{Gauss Laplace (normal)} 
  \verb!X=grand(m,n,"nor",mu,sigma)! generates random variates from the normal 
  distribution with mean $\mu$ (real)  and standard deviation $\sigma$
  (non negative real). 
% Related function(s) : \manlink{cdfnor}{cdfnor} .
  
\item \itemdesc{geometric} \verb!X=grand(m,n,"geom", p)! generates random variates from the geometric
  distribution with parameter $p$ (real in $(0,1]$): number of Bernoulli trials (with 
  probability succes of $p$) until a succes is met.

\item \itemdesc{finite discrete distributions} 
  \verb!X=grand(m,n,"disc",P)! could generates random integers from any
  finite discrete probability distribution given by the probability
  vector $P$ ($P(i)$ being the probability to get $X=i$). Rmk: if the 
  support of the distribution is not $\{1,\dots,length(P)\}$ you have
  an additionnal step for generate ramdom variates, for example to
  generate random variate with support  $\{0,1,\dots,length(P)-1\}$
  simply use \verb!X=grand(m,n,"disc",P)-1!. More generally if $P(i)$ is
  the probability associated to $val(i)$ then you can use 
  \verb!ind=grand(m,n,"disc",P); X = val(ind); X.redim[m,n]!. 
  
\item \itemdesc{Laplace} 
  \verb!X=grand(m,n,"lap",a)! generates random variates from the Laplace 
  distribution with (scale) parameter $a$ (positive real).
  
\item \itemdesc{logistic} 
  \verb!X=grand(m,n,"logi",a)! generates random variates from the logistic 
  distribution with (location) parameter $a$ (real) and (scale) parameter $b$ (positive real).
  
\item \itemdesc{log normal} 
  \verb!X=grand(m,n,"logn",mu,sigma)! generates random variates from the log normal
  distribution with parameter $\mu$ (real) and  $\sigma$ (non negative real).
  
\item \itemdesc{Pareto} 
  \verb!X=grand(m,n,"par",a,b)! generates random variates from the Pareto 
  distribution with (shape) parameter $a$ (positive real) and (location) parameter $b$ (positive real).

\item \itemdesc{Poisson} \verb!X=grand(m,n,"poi",mu)! generates random
  variates from the Poisson distribution with parameter $\mu$. 
  
\item \itemdesc{Rayleigh} 
  \verb!X=grand(m,n,"ray",sigma)! generates random variates from the Rayleigh 
  distribution with (scale) parameter $\sigma$ (positive real).
  
\item \itemdesc{tail Rayleigh} 
  \verb!X=grand(m,n,"tray",a, sigma)! generates random variates from the
  tail (from $a$) of the Rayleigh distribution with (scale) parameter 
  $\sigma$ (positive real). The (tail) parameter $a$ must be a non negative
  real (for $a=0$ we have the original Rayleigh distribution).

\item \itemdesc{uniform (def)} \verb!X=grand(m,n,"def")! generates random variates from the uniform 
  distribution over $[0,1)$ (1 is never returned).

\item \itemdesc{uniform (unf)} \verb!X=grand(m,n,"unf",low,high)! generates random reals uniformly distributed 
    in $[low, high]$.

\item \itemdesc{uniform (uin)} \verb!X=grand(m,n,"uin",low,high)! generates random integers uniformly 
      distributed between \verb!low! and \verb!high! (included). \verb!high!
      and \verb!low! must be integers such that \verb!high-low+1 > 2,147,483,561!.

\item \itemdesc{uniform (lgi)} \verb!X=grand(m,n,"lgi")! returns the basic output of the current generator : random integers  
      following a uniform distribution over : 
      \begin{itemize}
      \item \verb![0, 2^32 - 1]! for mt, kiss, fsultra and well1024a
      \item \verb![0, 2^31 - 2]! for clcg4
      \item \verb![0, 2147483561]! for clcg2
      \end{itemize}

\item \itemdesc{uniform (8bits)} \verb!X=grand(m,n,"8bits")! generates random integers uniformly 
      distributed over $[0,255]$.
  
\item \itemdesc{Weibull} 
  \verb!X=grand(m,n,"wei",a,b)! generates random variates from the Weibull 
  distribution with (scale) parameter $a$ (positive real) and (shape) parameter $b$ (positive real).
\end{itemize}

\item[multivariate distributions]
\begin{itemize}
\item \itemdesc{multivariate gaussian (multivariate normal)}
  \verb!X=grand(n,"mn",Mean,Cov)! generates  $n$ multivariate normal random variates ; 
  $Mean$ must be a $m \times 1$ matrix and $Cov$ a  $m \times m$ 
  symetric positive definite matrix  ($X$ is then a  $m \times n$
  matrix).

\item \itemdesc{multinomial} 
   \verb!X=grand(n,"mul",nb,P)! generates \verb!n! observations from the Multinomial 
  distribution :  class \verb!nb! events in \verb!m! categories (put \verb!nb!
  "balls" in \verb!m! "boxes"). \verb!P(i)! is the probability 
  that an event will be classified into category i. \verb!P! the vector of probabilities
  is of size  \verb!m-1! (the probability of category \verb!m! being \verb!1-sum(P)!).
  \verb!X! is of size \verb!m x n!, each column \verb!X(:,j)! being an observation 
  from multinomial distribution and \verb!X(i,j)! the number of events falling in category 
  \verb!i! (for the \verb!j! th observation) (\verb!sum(X(:,j)) = nb!).
  
\item \itemdesc{uniform on the unit sphere}
  \verb!X=grand(n,"sph",d)! generate ($n$) random points uniformly
  distributed on the unit sphere of \verb!R^d!. For a sphere of radius
  $r$ and center $C$ (a column vector), you can use: 
  \verb!X = C*ones(1,n) + r*grand(n,"sph",d)!.
  
\item \itemdesc{uniform inside the unit sphere}
  \verb!X=grand(n,"insph",d)! generate ($n$) random points uniformly
  distributed inside the unit sphere of \verb!R^d!. For a sphere of radius
  $r$ and center $C$ (a column vector), you can use: 
  \verb!X = C*ones(1,n) + r*grand(n,"insph",d)!.
  
\item \itemdesc{uniform on simplex}
  \verb!X=grand(n,"simp",dim)! generate ($n$) random points uniformly
  distributed on the simplex $\{ (x_1,...,x_{dim}): x_i >=0, \sum
  x_i = 1 \}$. To generate on the simplex with vertices (columnwise) 
  stacked in the matrix \verb!P = [P_1 | .... | P_(dim+1)]! (the jth column of $P$ 
  correspond to the j th  vertex), you can use:
  \verb!coord = grand(n,"simp",dim); X = P*[coord; 1-sum(coord,1)];!.
 
\item \itemdesc{markov} 
  \verb!X=grand(n,"markov",P,x0)! generates \verb!n! successive states of a Markov chain 
  described  by the transition matrix \verb!P!. Initial state is  given by 
  \verb!x0!. If \verb!x0! is a matrix of size \verb!m=size(x0,"*")! 
  then \verb!X! is a matrix of size \verb!m x n!. \verb!X(i,:)! is the sample 
  path  obtained from initial state \verb!x0(i)!.

\item \itemdesc{random permutations} \verb!X=grand(n,"prm",vect)! generate $n$ random permutations of the
  column vector $vect$ (in a uniform way over all possible permutations).

\item \itemdesc{random permutations of $(1,..,m)$}
  \verb!X=grand(n,"perm",m)! generate $n$ random permutations of the
  vector $[1;2;...;m]$ (in a uniform way over the all $m!$ possible permutations).

\item \itemdesc{random samples}
  \verb!X=grand(n,"smpl",p,m)! generate $n$ random samples of size $p$
  of the vector $[1;2;...;m]$ (that is select $p$ different elements
  of $\{1,2,...,m\}$) in a uniform way. If you need ordered samples
  sort the resulting matrix with \verb!X=sort(X,"r","i")!.

\end{itemize}

\end{description}


%-- section-Set/get the current generator and its state
\paragraph{Set/get the current generator and its state}
You have the possibility to choose between different base 
generators (which give random integers following the \verb!"lgi"!
distribution) :
\begin{itemize}
\item \itemdesc{mt}  the Mersenne-Twister of M. Matsumoto and T. Nishimura, period about \verb!2^19937!, 
  state given by an array of \verb!624! integers (plus an index onto this array); this  
  is the default generator.
\item \itemdesc{kiss}  The Keep It Simple Stupid of G. Marsaglia,  period about \verb!2^123!,
  state given by \verb!4! integers.
\item \itemdesc{clcg2}  a Combined 2 Linear Congruential Generator of P. L'Ecuyer,
  period about \verb!2^61!, state given by \verb!2! integers.
\item \itemdesc{clcg4}  a Combined 4 Linear Congruential Generator of P. L'Ecuyer,
  period about \verb!2^121!, state given by 4 integers ; this one is 
  splitted in \verb!101! different virtual (non over-lapping) generators 
  which may be useful for different tasks (see 'Actions specific to clcg4' and
  'Test example for clcg4').
\item \itemdesc{well1024a} from Francois Panneton, Pierre L'Ecuyer and
  Makoto Matsumoto. Period of about \verb!2^1024!, state given by an array of $32$
  integers (plus an index onto this array);
\item \itemdesc{fsultra} a Subtract-with-Borrow generator mixing with a congruential
  generator of Arif Zaman and George Marsaglia, period more than \verb!10^356!,
  state given by an array of 37 integers (plus an index onto this array, a flag (0 or 1)
  and another integer). 
\end{itemize}
The differents actions common to all the generators, are:
\begin{itemize}

\item \itemdesc{action= "getgen"}  \verb!S=grand("getgen")! returns the current base generator ( \verb!S! is
  a string among "mt", "kiss", "clcg2", "clcg4", "well1024a", "fsultra").
\item \itemdesc{action= "setgen"} \verb!grand("setgen",gen)! sets the current base generator to be \verb!gen!
  a string among "mt", "kiss", "clcg2", "clcg4", "well1024a", "fsultra" (notes that this call 
  returns the new current generator, ie gen).
\item \itemdesc{action= "getsd"} \verb!S=grand("getsd")! gets the current state of the current base
  generator ; \verb!S! is given as a column vector (of integers) of dimension \verb!625! 
  for mt (the first being an index in \verb![1,624]!), \verb!4! for kiss, \verb!2! 
  for clcg2,  \verb!40! for fsultra, \verb!4! for clcg4 (for this one
  you get the current state of the current virtual generator), \verb!33! for
  well1024a (the first being an index in \verb![1,32]!).
\item \itemdesc{action= "setsd"} \verb!grand("setsd",S)! sets the state of the current 
  base generator (the new seeds). $S$ is either a vector describing a
  complete state or only one integer (in this case a complete state for each
  generator is got from another random generator). 
\item \itemdesc{action= "phr2sd"} \verb!Sd=grand("phr2sd", phrase)! given a \verb!phrase! (character string) generates 
  a \verb!1 x 2! vector \verb!Sd! which may be used as seeds to change the state of a 
  base generator (initialy suited for clcg2). 
\end{itemize}

%-- section-Options specific to clcg4
\paragraph{Options specific to clcg4}
The clcg4 generator may be used as the others generators but it offers the advantage 
to be splitted in several (\verb!101!) virtual generators with non over-lapping 
sequences (when you use a classic generator you may change the initial state (seeds) 
in order to get another sequence but you are not warranty to get a complete  different one). 
Each virtual generator corresponds to a sequence of \verb!2^72! values which is 
further split into \verb!V=2^31! segments (or blocks) of length \verb!W=2^41!.
For a given virtual generator you have the possibility to return at the beginning of the 
sequence or at the beginning of the current segment or to go directly at the next segment. 
You may also change the initial state (seed) of the generator \verb!0! with the 
"setall" option which then change also the initial state of the other virtual generators 
so as to get synchronisation (ie in function of the new initial state of gen \verb!0! 
the initial state of gen \verb!1..100! are recomputed so as to get \verb!101! 
non over-lapping sequences.   

\begin{itemize}
\item \itemdesc{action= "setcgn"} \verb!grand("setcgn",G)! sets the current virtual generator for clcg4 (when clcg4
  is set, this is the virtual (clcg4) generator number \verb!G! which is used);  the virtual clcg4 
  generators are numbered from \verb!0,1,..,100! (and so \verb!G! must be an integer 
  in  \verb![0,100]!) ; by default the current virtual generator is \verb!0!.
\item \itemdesc{action= "getcgn"} \verb!S=grand("getcgn")! returns the number of the current virtual clcg4 generator.
\item \itemdesc{action= "initgn"} \verb!grand("initgn",I)!
  reinitializes the state of the current virtual generator :
      \begin{itemize}
      \item \verb!I = -1! :sets the state to the initial seed
      \item \verb!I = 0! :sets the state to the beginning of the current segment
      \item \verb!I = 1! :sets the state to the beginning of the next segment
      \end{itemize}
\item \itemdesc{action= "setall"} \verb!grand("setall",s1,s2,s3,s4)! sets the initial state of generator \verb!0! 
  to \verb!s1,s2,s3,s4!. The initial seeds of the other generators are set accordingly 
  to have synchronisation. For constraints on \verb!s1, s2, s3, s4! see the "setsd" action.
\item \itemdesc{action= "advnst"} \verb!grand("advnst",K)! advances the state of the current generator by \verb!2^K! values 
  and  resets the initial seed to that value. 
\end{itemize}


%-- section-Test example for clcg4

\paragraph{Test example for clcg4}

An example of  the  need of the splitting capabilities of clcg4 is as  follows. 
Two statistical techniques are being compared on  data of  different sizes. The first 
technique uses   bootstrapping  and is   thought to   be  as accurate using less data   
than the second method   which  employs only brute force.  For the first method, a data
set of size uniformly distributed between 25 and 50 will be generated.  Then the data set  
of the specified size will be generated and analyzed.  The second method will  choose a 
data set size between 100 and 200, generate the data  and analyze it.  This process will 
be repeated 1000 times.  For  variance reduction, we  want the  random numbers  used in the 
two methods to be the  same for each of  the 1000 comparisons.  But method two will  use more
random  numbers than   method one and  without this package, synchronization might be difficult.  
With clcg4, it is a snap.  Use generator 0 to obtain  the sample size for  method one and 
generator 1  to obtain the  data.  Then reset the state to the beginning  of the current  block
and do the same  for the second method.  This assures that the initial data  for method two is 
that used by  method  one.  When both  have concluded,  advance the block for both generators.

%-- see also

\begin{manseealso}
  \manlink{rand}{rand}  
\end{manseealso}

%-- Authors

\begin{authors}

  Nsp interface by Jean-Philippe Chancelier and Bruno Pincon, codes for
  various distributions Bruno Pincon, codes for base generators :
\begin{itemize}
  \item \itemdesc{mt} 
  M. Matsumoto and  T. Nishimura.
  
  \item \itemdesc{kiss} 
  The code was given by G. Marsaglia at the end of a thread concerning RNG in C in several 
  newsgroups (whom sci.math.num-analysis) "My offer of  RNG's for C was an invitation 
  to dance..." only kiss have been included in Nsp (kiss is made of a combinaison of 
  severals others which are not visible at the nsp level).
  
  \item \itemdesc{clcg2} 
  The method is from P. L'Ecuyer but the C code is provided at the Luc  Devroye home page 
  (http://cgm.cs.mcgill.ca/~luc/rng.html).
  
  \item \itemdesc{clcg4} 
  Pierre L'Ecuyer, Terry H.Andres
  
  \item \itemdesc{well1024a} 
  Francois Panneton, Pierre L'Ecuyer, Makoto Matsumoto.
   
  \item \itemdesc{fsultra} 
  Arif Zaman (\verb+arif@stat.fsu.edu+) and George Marsaglia (\verb+geo@stat.fsu.edu+)
\end{itemize}
\end{authors}

