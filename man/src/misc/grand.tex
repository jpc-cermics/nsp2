% -*- mode: latex -*-
\mansection{grand}
\begin{mandesc}
  \short{grand}{Random number generator(s)}
\end{mandesc}
%-- Calling sequence section
  \begin{calling_sequence}
    \begin{verbatim}
Y=grand(m, n, dist_type [,p1,...,pk])  
Y=grand(X, dist_type [,p1,...,pk])  
Y=grand(n, dist_type [,p1,...,pk])  
S=grand(action [,q1,....,ql])  
    \end{verbatim}
  \end{calling_sequence}
  %-- Parameters
  \begin{parameters}
  \begin{varlist}
    \vname{m, n}: integers, size of the wanted matrix \verb!Y!
   \vname{X}: a matrix whom only the dimensions (say \verb!m x n!) are used
   \vname{dist\_type}: a string given the distribution which (independants) variates are to be 
     generated ('bin', 'nor', 'poi', etc ...)
   \vname{p1, ..., pk}: the parameters (reals or integers) required to define completly the distribution 
    \verb!dist_type!
   \vname{Y}: the resulting \verb!m x n! random matrix
   \vname{action}: a string given the action onto the base generator(s) ('setgen' to change the current base 
     generator,  'getgen' to retrieve the current base generator name, 'getsd' to retrieve the 
     state (seeds) of the current base generator, etc ...)
   \vname{q1, ..., ql} : the parameters (generally one string) needed to define the action
   \vname{S} : output of the action (generaly a string or a real column vector)
  \end{varlist}
  \end{parameters}
  
\begin{mandescription}
  This function may be used to generate random numbers from various distributions. In this 
  case you must apply one of the 
  \verb!three first forms! of the possible calling sequences to get an \verb!m x n! matrix. 
  The two firsts are equivalent if \verb!X! is a \verb!m x n! matrix, 
  and the third form corresponds to 'multivalued' distributions (e.g. multinomial, multivariate
  gaussian, etc...) where a sample is a column vector (says of dim \verb!m!)
  and you get then \verb!n! such random vectors (as an \verb! m x n! matrix). 
  \verb!The last form! is used to undertake various manipulations onto the base generators
  like changing the base generator (since v 2.7 you may choose between several base generators), 
  changing or retrieving its internal state (seeds), etc ... These base generators give random
  integers following a uniform distribution on a large integer interval (lgi), all the others 
  distributions being gotten from it (in general via a scheme lgi -$>$ U([0,1)) -$>$ wanted distribution).
\end{mandescription}

%-- section-Getting random numbers from a given distribution
\paragraph{Getting random numbers from a given distribution}
\begin{itemize}
\item \itemdesc{beta} : \verb!Y=grand(m,n,'bet',A,B)! generates random variates from 
  the beta distribution with parameters \verb!A! and \verb!B!. 
  The density of the beta distribution is given by  (\verb!0 $<$ x $<$ 1!) :
  \[
  x^{A-1}(1-x)^{B-1}/ \mbox{beta}(A,B) I_{(0,1)}(x)
  \]
  \verb!A! and \verb!B! must be reals $>$\verb!10^(-37)!.
  Related function(s) : \manlink{cdfbet}{cdfbet} .
\item \itemdesc{binomial} 
  : \verb!Y=grand(m,n,'bin',N,p)!  generates random variates from the binomial 
  distribution with parameters \verb!N! (positive integer) and \verb!p!
  (real in [0,1]) : number of successes in \verb!N! independant 
  Bernouilli trials with probability \verb!p! of success. 
  Related function(s) : \manlink{binomial}{binomial} , \manlink{cdfbin}{cdfbin} .
\item \itemdesc{negative binomial} 
  : \verb!Y=grand(m,n,'nbn',N,p)! generates random variates from the negative binomial 
  distribution with parameters \verb!N! (positive integer) and \verb!p! (real 
  in (0,1)) : number of failures occurring before \verb!N! successes 
  in independant Bernouilli trials with probability \verb!p! of success.
  Related function(s) : \manlink{cdfnbn}{cdfnbn} .
  
\item \itemdesc{chisquare} 
  : \verb!Y=grand(m,n,'chi', Df)! generates random variates from the chisquare distribution 
  with \verb!Df! (real $>$ 0.0) degrees of freedom. 
  Related function(s) : \manlink{cdfchi}{cdfchi} .  
  
\item \itemdesc{non central chisquare} 
  : \verb!Y=grand(m,n,'nch',Df,Xnon)! generates random variates from the non central chisquare
  distribution with \verb!Df! degrees of freedom (real $>$= 1.0) 
  and noncentrality parameter \verb!Xnonc! (real $>$= 0.0).
  Related function(s) : \manlink{cdfchn}{cdfchn} .
  
\item \itemdesc{exponential} : \verb!Y=grand(m,n,'exp',Av)! generates random variates from the exponential
  distribution with mean \verb!Av! (real $>$= 0.0).
\item \itemdesc{F variance ratio} 
  : \verb!Y=grand(m,n,'f',Dfn,Dfd)! generates random variates from the F 
  (variance ratio) distribution with \verb!Dfn! (real $>$ 0.0)
  degrees of freedom in the numerator and \verb!Dfd! (real $>$ 0.0) 
  degrees of freedom in the denominator. Related function(s) : \manlink{cdff}{cdff} .
  
\item \itemdesc{non central F variance ratio} 
  : \verb!Y=grand(m,n,'nf',Dfn,Dfd,Xnon)! generates random variates from the noncentral 
  F (variance ratio)  distribution with \verb!Dfn! (real $>$= 1) degrees of freedom 
  in the numerator, and \verb!Dfd! (real $>$ 0) degrees of freedom in the denominator, 
  and noncentrality parameter \verb!Xnonc! (real $>$= 0). 
  Related function(s) : \manlink{cdffnc}{cdffnc} .
  
\item \itemdesc{gamma} : \verb!Y=grand(m,n,'gam',shape,scale)! generates random variates from the gamma 
  distribution with parameters \verb!shape! (real $>$ 0) and \verb!scale! 
  (real $>$ 0). The density of the gamma distribution is :
  \[
  \frac{\lambda}{\Gamma(\alpha)} e^{-\lambda x} (\lambda x)^{\alpha-1} I_{{R}^+}(x)
  \]
  where \verb+shape=+$\alpha$ and \verb+scale=+$\lambda$.
  Related function(s) : \manlink{gamma}{gamma} , \manlink{cdfgam}{cdfgam} .
  
\item \itemdesc{Gauss Laplace (normal)} 
  : \verb!Y=grand(m,n,'nor',Av,Sd)! generates random variates from the normal 
  distribution with mean \verb!Av! (real)  and standard deviation \verb!Sd!
  (real $>$= 0). Related function(s) : \manlink{cdfnor}{cdfnor} .
  
\item \itemdesc{multivariate gaussian (multivariate normal)} : \verb!Y=grand(n,'mn',Mean,Cov)! generates  \verb!n! multivariate normal random variates ; 
  \verb!Mean! must be a \verb!m x 1! matrix and \verb!Cov! a  \verb!m x m! 
  symetric positive definite matrix  (\verb!Y! is then a  \verb!m x n! matrix).
\item \itemdesc{geometric} : \verb!Y=grand(m,n,'geom', p)! generates random variates from the geometric
  distribution with parameter \verb!p! : number of Bernouilli trials (with 
  probability succes of \verb!p!) until a succes is met. \verb!p! must 
  be in \verb![pmin,1]! (with \verb!pmin = 1.3 10^(-307)!).
\item \itemdesc{markov} 
  : \verb!Y=grand(n,'markov',P,x0)! generate \verb!n! successive states of a Markov chain 
  described  by the transition matrix \verb!P!. Initial state is  given by 
  \verb!x0!. If \verb!x0! is a matrix of size \verb!m=size(x0,'*')! 
  then \verb!Y! is a matrix of size \verb!m x n!. \verb!Y(i,:)! is the sample 
  path  obtained from initial state \verb!x0(i)!.
\item \itemdesc{multinomial} 
  : \verb!Y=grand(n,'mul',nb,P)! generates \verb!n! observations from the Multinomial 
  distribution :  class \verb!nb! events in \verb!m! categories (put \verb!nb!
  "balls" in \verb!m! "boxes"). \verb!P(i)! is the probability 
  that an event will be classified into category i. \verb!P! the vector of probabilities
  is of size  \verb!m-1! (the probability of category \verb!m! being \verb!1-sum(P)!).
  \verb!Y! is of size \verb!m x n!, each column \verb!Y(:,j)! being an observation 
  from multinomial distribution and \verb!Y(i,j)! the number of events falling in category 
  \verb!i! (for the \verb!j! th observation) (\verb!sum(Y(:,j)) = nb!).
  
\item \itemdesc{Poisson} : \verb!Y=grand(m,n,'poi',mu)! generates random variates from the Poisson 
  distribution with mean \verb!mu (real $>$= 0.0)!.
\item \itemdesc{random permutations} : \verb!Y=grand(n,'prm',vect)! generate \verb!n! random permutations of the
  column vector (\verb!m x 1!) \verb!vect!.
\item \itemdesc{uniform (def)} : \verb!Y=grand(m,n,'def')! generates random variates from the uniform 
  distribution over \verb![0,1)! (1 is never return).
  \item \itemdesc{uniform (unf)} : \verb!Y=grand(m,n,'unf',Low,High)! generates random reals uniformly distributed 
    in \verb![Low, High)!.
    \item \itemdesc{uniform (uin)} : \verb!Y=grand(m,n,'uin',Low,High)! generates random integers uniformly 
      distributed between \verb!Low! and \verb!High! (included). \verb!High!
      and \verb!Low! must be integers such that \verb!(High-Low+1) $>$ 2,147,483,561!.
    \item \itemdesc{uniform (lgi)} : \verb!Y=grand(m,n,'lgi')! returns the basic output of the current generator : random integers  
      following a uniform distribution over : 
\end{itemize}


%-- section-Set/get the current generator and its state
\paragraph{Set/get the current generator and its state}
You have the possibility to choose between different base 
generators (which give random integers following the 'lgi' distribution, the others 
being gotten from it) :
\begin{itemize}
\item \itemdesc{mt} : the Mersenne-Twister of M. Matsumoto and T. Nishimura, period about \verb!2^19937!, 
  state given by an array of \verb!624! integers (plus an index onto this array); this  
  is the default generator.
\item \itemdesc{kiss} : The Keep It Simple Stupid of G. Marsaglia,  period about \verb!2^123!,
  state given by \verb!4! integers.
\item \itemdesc{clcg2} : a Combined 2 Linear Congruential Generator of P. L'Ecuyer,
  period about \verb!2^61!, state given by \verb!2! integers ; this was 
  the only generator previously used by grand (but slightly modified)
\item \itemdesc{clcg4} : a Combined 4 Linear Congruential Generator of P. L'Ecuyer,
  period about \verb!2^121!, state given by 4 integers ; this one is 
  splitted in \verb!101! different virtual (non over-lapping) generators 
  which may be useful for different tasks (see 'Actions specific to clcg4' and
  'Test example for clcg4').
\item \itemdesc{urand} : the generator used by the scilab function \manlink{rand}{rand} , state
  given by \verb!1! integer, period of \verb!2^31! (based  on  theory  
  and suggestions  given  in  d.e. knuth (1969),  vol  2. State). This
  is the faster of this list but a little outdated (don't use it for
  serious simulations).
\item \itemdesc{fsultra} : a Subtract-with-Borrow generator mixing with a congruential
  generator of Arif Zaman and George Marsaglia, period more than \verb!10^356!,
  state given by an array of 37 integers (plus an index onto this array, a flag (0 or 1)
  and another integer). 
\end{itemize}
The differents actions common to all the generators, are:
\begin{itemize}

\item \itemdesc{action= 'getgen'} : \verb!S=grand('getgen')! returns the current base generator ( \verb!S! is
  a string among 'mt', 'kiss', 'clcg2', 'clcg4', 'urand', 'fsultra'.
\item \itemdesc{action= 'setgen'} : \verb!grand('setgen',gen)! sets the current base generator to be \verb!gen!
  a string among 'mt', 'kiss', 'clcg2', 'clcg4', 'urand', 'fsultra' (notes that this call 
  returns the new current generator, ie gen).
\item \itemdesc{action= 'getsd'} : \verb!S=grand('getsd')! gets the current state (the current seeds) of the current base
  generator ; \verb!S! is given as a column vector (of integers) of dimension \verb!625! 
  for mt (the first being an index in \verb![1,624]!), \verb!4! for kiss, \verb!2! 
  for clcg2,  \verb!40! for fsultra, \verb!4! for clcg4 
  (for this last one you get the current state of the current virtual generator) and \verb!1! 
  for urand.
\item \itemdesc{action= 'setsd'} : \verb!grand('setsd',S), grand('setsd',s1[,s2,s3,s4])! sets the state of the current 
  base generator (the new seeds) :
\item \itemdesc{action= 'phr2sd'} : \verb!Sd=grand('phr2sd', phrase)! given a \verb!phrase! (character string) generates 
  a \verb!1 x 2! vector \verb!Sd! which may be used as seeds to change the state of a 
  base generator (initialy suited for clcg2). 
\end{itemize}

%-- section-Options specific to clcg4
\paragraph{Options specific to clcg4}
The clcg4 generator may be used as the others generators but it offers the advantage 
to be splitted in several (\verb!101!) virtual generators with non over-lapping 
sequences (when you use a classic generator you may change the initial state (seeds) 
in order to get another sequence but you are not warranty to get a complete  different one). 
Each virtual generator corresponds to a sequence of \verb!2^72! values which is 
further split into \verb!V=2^31! segments (or blocks) of length \verb!W=2^41!.
For a given virtual generator you have the possibility to return at the beginning of the 
sequence or at the beginning of the current segment or to go directly at the next segment. 
You may also change the initial state (seed) of the generator \verb!0! with the 
'setall' option which then change also the initial state of the other virtual generators 
so as to get synchronisation (ie in function of the new initial state of gen \verb!0! 
the initial state of gen \verb!1..100! are recomputed so as to get \verb!101! 
non over-lapping sequences.   

\begin{itemize}
\item \itemdesc{action= 'setcgn'} : \verb!grand('setcgn',G)! sets the current virtual generator for clcg4 (when clcg4
  is set, this is the virtual (clcg4) generator number \verb!G! which is used);  the virtual clcg4 
  generators are numbered from \verb!0,1,..,100! (and so \verb!G! must be an integer 
  in  \verb![0,100]!) ; by default the current virtual generator is \verb!0!.
\item \itemdesc{action= 'getcgn'} : \verb!S=grand('getcgn')! returns the number of the current virtual clcg4 generator.
\item \itemdesc{action= 'initgn'} : \verb!grand('initgn',I)! reinitializes the state of the current virtual generator
\item \itemdesc{action= 'setall'} : \verb!grand('setall',s1,s2,s3,s4)! sets the initial state of generator \verb!0! 
  to \verb!s1,s2,s3,s4!. The initial seeds of the other generators are set accordingly 
  to have synchronisation. For constraints on \verb!s1, s2, s3, s4! see the 'setsd' action.
\item \itemdesc{action= 'advnst'} : \verb!grand('advnst',K)! advances the state of the current generator by \verb!2^K! values 
  and  resets the initial seed to that value. 
\end{itemize}


%-- section-Test example for clcg4

\paragraph{Test example for clcg4}

An example of  the  need of the splitting capabilities of clcg4 is as  follows. 
Two statistical techniques are being compared on  data of  different sizes. The first 
technique uses   bootstrapping  and is   thought to   be  as accurate using less data   
than the second method   which  employs only brute force.  For the first method, a data
set of size uniformly distributed between 25 and 50 will be generated.  Then the data set  
of the specified size will be generated and analyzed.  The second method will  choose a 
data set size between 100 and 200, generate the data  and analyze it.  This process will 
be repeated 1000 times.  For  variance reduction, we  want the  random numbers  used in the 
two methods to be the  same for each of  the 1000 comparisons.  But method two will  use more
random  numbers than   method one and  without this package, synchronization might be difficult.  
With clcg4, it is a snap.  Use generator 0 to obtain  the sample size for  method one and 
generator 1  to obtain the  data.  Then reset the state to the beginning  of the current  block
and do the same  for the second method.  This assures that the initial data  for method two is 
that used by  method  one.  When both  have concluded,  advance the block for both generators.

%-- see also

\begin{manseealso}
  \manlink{rand}{rand}  
\end{manseealso}

%-- Authors

\begin{authors}

  Nsp interface by Jean-Philippe Chancelier and Bruno Pincon. 
\begin{itemize}
  \item \itemdesc{randlib} 
  The codes to generate sequences following other distributions than def, unf, lgi,  uin and geom are
  from "Library of Fortran Routines for Random Number  Generation", by Barry W. Brown 
  and James Lovato, Department of Biomathematics, The University of Texas, Houston.  
  
  \item \itemdesc{mt} 
  The code is the mt19937int.c by M. Matsumoto and  T. Nishimura, "Mersenne Twister: 
  A 623-dimensionally equidistributed  uniform pseudorandom number generator", 
  ACM Trans. on Modeling and  Computer Simulation Vol. 8, No. 1, January, pp.3-30 1998.
  
  \item \itemdesc{kiss} 
  The code was given by G. Marsaglia at the end of a thread concerning RNG in C in several 
  newsgroups (whom sci.math.num-analysis) "My offer of  RNG's for C was an invitation 
  to dance..." only kiss have been included in Nsp (kiss is made of a combinaison of 
  severals others which are not visible at the nsp level).
  
  \item \itemdesc{clcg2} 
  The method is from P. L'Ecuyer but the C code is provided at the Luc  Devroye home page 
  (http://cgm.cs.mcgill.ca/~luc/rng.html).
  
  \item \itemdesc{clcg4} 
  The code is from P. L'Ecuyer and Terry H.Andres and provided at the P. L'Ecuyer
  home page ( \verb+http://www.iro.umontreal.ca/~lecuyer/papers.html+) A paper is also provided 
  and this new package is the logical successor of an old 's one from : P.  L'Ecuyer
  and S. Cote.   Implementing a Random   Number Package with Splitting Facilities.  ACM Transactions 
  on Mathematical  Software 17:1,pp 98-111.
  
  \item \itemdesc{fsultra} 
  code from Arif Zaman (\verb+arif@stat.fsu.edu+) and George Marsaglia (\verb+geo@stat.fsu.edu+)
\end{itemize}

\end{authors}

