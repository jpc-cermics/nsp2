% -*- mode: latex -*-
\mansection{cdffnc}
\begin{mandesc}
  \short{cdffnc}{cumulative distribution function non-central f-distribution}
\end{mandesc}
\index{cdffnc}\label{cdffnc}
%-- Calling sequence section
\begin{calling_sequence}
\begin{verbatim}
  [P,Q]=cdffnc("PQ",F,Dfn,Dfd,Pnonc)  
  [F]=cdffnc("F",Dfn,Dfd,Pnonc,P,Q);  
  [Dfn]=cdffnc("Dfn",Dfd,Pnonc,P,Q,F);  
  [Dfd]=cdffnc("Dfd",Pnonc,P,Q,F,Dfn)  
  [Pnonc]=cdffnc("Pnonc",P,Q,F,Dfn,Dfd);  
\end{verbatim}
\end{calling_sequence}
%-- Parameters
\begin{parameters}
  \begin{varlist}
    \vname{P,Q,F,Dfn,Dfd,Pnonc} : six real vectors of the same size.
    \vname{P,Q (Q=1-P)  } The integral from 0 to F of the non-central f-density. Input range: [0,1-1E-16).
      \vname{F} : Upper limit of integration of the non-central f-density. Input range: [0, +infinity). Search range: [0,1E300]
        \vname{Dfn} : Degrees of freedom of the numerator sum of squares. Input range: (0, +infinity). Search range: [ 1E-300, 1E300]
        \vname{Dfd} : Degrees of freedom of the denominator sum of squares. Must be in range: (0, +infinity). Input range: (0, +infinity). Search range: [ 1E-300, 1E300]
        \vname{Pnonc} : The non-centrality parameter Input range: [0,infinity) Search range: [0,1E4]
  \end{varlist}
\end{parameters}
\begin{mandescription}
  Calculates any one parameter of the Non-central F distribution given values for the others.

  Formula  26.6.20   of   Abramowitz   and   Stegun,  Handbook  of
  Mathematical  Functions (1966) is used to compute the cumulative
  distribution function.
  Computation of other parameters involve a seach for a value that
  produces  the desired  value  of P.   The search relies  on  the
  monotinicity of P with the other parameter.
  The computation time  required for this  routine is proportional
  to the noncentrality  parameter  (PNONC).  Very large  values of
  this parameter can consume immense  computer resources.  This is
  why the search range is bounded by 10,000.
  The  value  of the  cumulative  noncentral F distribution is not
  necessarily monotone in either degrees  of freedom.  There  thus
  may be two values that provide a given  CDF value.  This routine
  assumes monotonicity  and will find  an arbitrary one of the two
  values.
\end{mandescription}

\begin{authors}
  Nsp interface by Jean-Philippe Chancelier. Code from DCDFLIB: 
  Library of Fortran Routines for Cumulative Distribution
  Functions, Inverses, and Other Parameters (February, 1994)
  Barry W. Brown, James Lovato and Kathy Russell. The University of Texas.
\end{authors}
