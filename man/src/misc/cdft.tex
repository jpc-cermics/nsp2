% -*- mode: latex -*-
\mansection{cdft}
\begin{mandesc}
  \short{cdft}{cumulative distribution function Student's T distribution}
\end{mandesc}
\index{cdft}\label{cdft}
%-- Calling sequence section
\begin{calling_sequence}
\begin{verbatim}
  [P,Q]=cdft("PQ",T,Df)  
  [T]=cdft("T",Df,P,Q)  
  [Df]=cdft("Df",P,Q,T)  
\end{verbatim}
\end{calling_sequence}
%-- Parameters
\begin{parameters}
  \begin{varlist}
    \vname{P,Q,T,Df} : six real vectors of the same size.
    \vname{P,Q (Q=1-P)  } : The integral from -infinity to t of the t-density. Input range: (0,1].
      \vname{T} : Upper limit of integration of the t-density. Input range: ( -infinity, +infinity). Search range: [ -1E150, 1E150 ]
      \vname{DF:  } Degrees of freedom of the t-distribution. Input range: (0 , +infinity). Search range: [1e-300, 1E10]
  \end{varlist}
\end{parameters}
\begin{mandescription}
  Calculates any one parameter of the T distribution given
  values for the others. The T distribution is given by~:
  \begin{equation}
    \frac{\Gamma(\frac{\nu+1}{2})} {\sqrt{\nu\pi}\,\Gamma(\frac{\nu}{2})} \left(1+\frac{t^2}{\nu} \right)^{-(\frac{\nu+1}{2})} 
  \end{equation}
  with $\nu = \mbox{\tt Df}$  and where $\Gamma$ is the Gamma function.
  Formula  26.5.27  of   Abramowitz   and  Stegun,   Handbook   of
  Mathematical Functions  (1966) is used to reduce the computation
  of the cumulative distribution function to that of an incomplete
  beta.
  Computation of other parameters involve a seach for a value that
  produces  the desired  value  of P.   The search relies  on  the
  monotinicity of P with the other parameter.
\end{mandescription}

\begin{program}\HCode{x=linspace(-10,10,100);\Hnewline
df = 3;\Hnewline
y= cdft("PQ",x,df*ones(x));\Hnewline
function y=f(x)\Hnewline
  y = gamma((df+1)/2)/( sqrt(df*\%pi) *gamma(df/2))*(1+x^2/df)^(-(df+1)/2);\Hnewline
endfunction;\Hnewline
\Hnewline
z=x;\Hnewline
for i=1:length(x)\Hnewline
  z(i) = intg(-1000,x(i),f);\Hnewline
end\Hnewline
err=max(abs(y-z));}
\end{program}


\begin{authors}
  Nsp interface by Jean-Philippe Chancelier. Code from DCDFLIB: 
  Library of Fortran Routines for Cumulative Distribution
  Functions, Inverses, and Other Parameters (February, 1994)
  Barry W. Brown, James Lovato and Kathy Russell. The University of Texas.
\end{authors}
