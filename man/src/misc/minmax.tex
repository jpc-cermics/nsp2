\mansection{min, max and minmax}
\begin{mandesc}
  \short{min}{minimum of matrix elements or minimum of 2 or more matrices}\\ 
  \short{max}{maximum of matrix elements or maximum of 2 or more matrices}\\ 
  \short{minmax}{both minimum and maximum of matrix elements}
\end{mandesc}
%-- Calling sequence section
\begin{calling_sequence}
\begin{verbatim}
  // first form of calling sequences
  m = min(A,dim=mode)
  [m,km] = min(A,dim=mode)
  M = max(A,dim=mode)
  [M,kM] = max(A,dim=mode)
  [m,M] = minmax(A,dim=mode)
  [m,M,km,kM] = minmax(A,dim=mode)

  // second form
  m = min(A1,A2,...)  
  [m,km] = min(A1,A2,...) 
  M = max(A1,A2,...)  
  [M,kM] = max(A1,A2,...)  
\end{verbatim}
\end{calling_sequence}
%-- Parameters
\begin{parameters}
  \begin{varlist}
    \vname{A,A1,A2,...}: numerical real matrices.
    \vname{mode}: a string chosen among \verb+'M'+, \verb+'m'+, \verb+'full'+, \verb+'FULL'+, \verb+'row'+,
    \verb+'ROW'+, \verb+'col'+, \verb+'COL'+ or an non ambiguous abbreviation or an integer. 
    This argument is optional and if omitted 'full' is assumed.
    \vname{m,M}: numerical scalars or vectors or matrices
    \vname{km, kM}: integer scalars or vectors (or matrix) of indices.
  \end{varlist}
\end{parameters}

\begin{mandescription}

  These functions have 2 differents forms.
\begin{itemize}
\item \itemdesc{form 1}
Applied on only one matrix (of size $m \times n$) with the optional argument \verb+dim=mode+ 
(if not provided 'full' is assumed) they compute the $min$ or $max$ element(s):
  \begin{itemize}
    \item among all the matrix elements if dim='full' or 0:
     $$
           m = \min_{i,j} A_{i,j}; \; M = \max_{i,j} A_{i,j}
     $$
    \item along the dimension 1 when dim=1 or 'row' (we get the min or max of each column);
     $$
           m_j = \min_i A_{i,j}; \; M_j = \max_i A_{i,j}; 
     $$
     $m$ and $M$ are then row vectors ($1 \times n$).
    \item along the dimension 2 when dim=2 or 'col' (we get the min or max of each row);
     $$
           m_i = \min_j A_{i,j}; \; M_i = \max_j A_{i,j}; 
     $$
     $m$ and $M$ are then column vectors ($m \times 1$).
    \item along the first non singleton dimension of the given matrix if dim='m' (matlab compatibility).
  \end{itemize}
Additionnaly the first index of the corresponding min or max could be provided (when dim=0, it is 
given in the ``one way indexing'' meaning of the matrix). The \verb+minmax+ function behave exactly
as described. It is practical since very often one needs both the min and max of a vector or matrix.  
     
\item \itemdesc{form2}
  This form (which don't have the optional argument dim) applies on any number of matrices of {\bf same
  sizes} says $m \times n$ (with the usual short cut that you can use any scalar in place of such a matrix). 
  The result is a matrix of size  $m \times n$ with:
  $$
      m_{i,j} = \min_k Ak_{i,j}, \;  M_{i,j} = \max_k Ak_{i,j}
  $$
  and $km_{i,j}$ or $kM_{i,j}$ gives the (first) matrix number where the min or max is got.
  There is no second form for the minmax function.

\item \itemdesc{Remarks}
\begin{itemize} 
\item Nan values are not taking into account in these functions except
when the min or max is to be taken among only Nan values. 
\item For the first form you can use the syntax \verb+min(A,mode)+ (in place of \verb+min(A,dim=mode)+)
if you use a string ('row', 'col',..) as dim descriptor (if you use an integer the interpretor will 
recognize a call of the second form). This could be useful for scilab compatibility.
\item for sparse matrices: 
\begin{itemize}
\item \verb+minmax+ is not defined
\item the second form is (currently) limited to two matrices. Use \verb+max(A1,max(A2,A3))+ to
simulate \verb+max(A1,A2,A3)+. 
\end{itemize}
\end{itemize}

\end{itemize}

\end{mandescription}


%--example 
\begin{examples}
\paragraph{first form}
\begin{program}\HCode{A = rand(4,5) \Hnewline
// min and max of all elements\Hnewline
m = min(A) \Hnewline
M = max(A) \Hnewline
[m,M] = minmax(A)\Hnewline
[m,km] = min(A)\Hnewline
[M,kM] = max(A)\Hnewline
[m,M,km,kM] = minmax(A)\Hnewline
// min and max of each column\Hnewline
[m,km] = min(A,dim=1)\Hnewline
[M,kM] = max(A,dim=1)\Hnewline
[m,M,km,kM] = minmax(A,dim=1)\Hnewline
// min and max of each row\Hnewline
[m,km] = min(A,dim=2)\Hnewline
[M,kM] = max(A,dim=2)\Hnewline
[m,M,km,kM] = minmax(A,dim=2)\Hnewline
// behavior with Nan\Hnewline
x = [1, \%nan, 2, \%nan]\Hnewline
[m,km] = min(x)\Hnewline
[M,kM] = max(x)\Hnewline
[m,M,km,kM] = minmax(A)}
\end{program}

\paragraph{second form}
\begin{program}\HCode{A=rand(2,2) \Hnewline
B = rand(2,2) \Hnewline
C = rand(2,2) \Hnewline
m = min(A,B,C) \Hnewline
[m,km] = min(A,B,C) \Hnewline
// one (or more) of the matrix can be a scalar \Hnewline
[m,km] = min(A,0.3,C) \Hnewline
// using scalar is practical to impose a min and max thresholds on a matrix \Hnewline
// in this example we want to limit a random matrix between 0.2 and 0.8 \Hnewline
A = rand(2,4) \Hnewline
B = min(0.8,max(0.2,A))}
\end{program} 

\end{examples}

%-- see also
\begin{manseealso}
  \manlink{find}{find}
\end{manseealso}

