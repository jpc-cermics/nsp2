% -*- mode: latex -*-
\mansection{cdfbet}
\begin{mandesc}
  \short{cdfbet}{cumulative distribution function Beta distribution} 
\end{mandesc}
\index{cdfbet}\label{cdfbet}
%-- Calling sequence section
\begin{calling_sequence}
\begin{verbatim}
[P,Q]=cdfbet("PQ",X,Y,A,B)  
[X,Y]=cdfbet("XY",A,B,P,Q)  
[A]=cdfbet("A",B,P,Q,X,Y)  
[B]=cdfbet("B",P,Q,X,Y,A)  
\end{verbatim}
\end{calling_sequence}
%-- Parameters
\begin{parameters}
  \begin{varlist}
    \vname{P,Q,X,Y,A,B} : five real vectors of the same size.
    \vname{P,Q (Q=1-P)  } : The integral from 0 to X of the beta distribution (Input range: [0, 1].)
    \vname{Q} : 1-P
    \vname{X,Y (Y=1-X)  } : Upper limit of integration of beta density (Input range: [0,1],  Search range: [0,1]) A,B : The two parameters of the beta density (input range: (0, +infinity), Search range: [1D-300,1D300] )
  \end{varlist}
\end{parameters}
\begin{mandescription}
  Calculates any one parameter of the beta distribution given
  values for the others. The beta density is given by
  \begin{equation}
    \frac{\Gamma(A+B)}{\Gamma(A)\Gamma(B)} x^{(A-1)} * (1-x)^{(B-1)}
  \end{equation}
  with domain $(0,1)$.
  The cumulative distribution function is calculated directly by
  code associated with the following reference: 
  DiDinato, A. R. and Morris,  A.   H.  Algorithm 708: Significant
  Digit Computation of the Incomplete  Beta  Function Ratios.  ACM
  Trans. Math.  Softw. 18 (1993), 360-373.
  Computation of other parameters involve a seach for a value that
  produces  the desired  value  of P.   The search relies  on  the
  monotinicity of P with the other parameter.
\end{mandescription}

\begin{program}\HCode{x=linspace(0,1,100);\Hnewline
A =2;\Hnewline
B =3;\Hnewline
y= cdfbet("PQ",x,1-x,A*ones(x),B*ones(x));\Hnewline
\Hnewline
function y=f(x)\Hnewline
  y = gamma(A+B)/(gamma(A)*gamma(B))*x^(A-1)*(1-x)^(B-1)\Hnewline
endfunction;\Hnewline
\Hnewline
z=x;\Hnewline
for i=1:length(x)\Hnewline
  z(i) = intg(0,x(i),f);\Hnewline
end\Hnewline
err=max(abs(y-z));}
\end{program}

\begin{authors}
  Nsp interface by Jean-Philippe Chancelier. Code from DCDFLIB: 
  Library of Fortran Routines for Cumulative Distribution
  Functions, Inverses, and Other Parameters (February, 1994)
  Barry W. Brown, James Lovato and Kathy Russell. The University of Texas.
\end{authors}

