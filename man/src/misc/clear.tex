% -*- mode: latex -*-

\mansection{clear}

\begin{mandesc}
  \short{clear}{remove variables from current environment} \\
  \short{clearglobal}{remove variables from global environment} 
\end{mandesc}
\index{clear}\label{clear}
\index{clearglobal}\label{clearglobal}
%-- Calling sequence section
\begin{calling_sequence}
\begin{verbatim}
  clear name1,...,namen;
  clear ;
  clear('name1',...,'namen');
  clear();
  clearglobal name1,...,namen;
  clearglobal ;
  clearglobal('name1',...,'namen');
  clearglobal();
\end{verbatim}

\end{calling_sequence}

\begin{mandescription}
  The \verb+clear+ (resp. \verb+clearglobal+) is used to remove (unprotected) variables given by their names 
  form the current (resp. the global) environment. If no argument is given to the command then all the 
  variables current (resp global) are removed. 
\end{mandescription}
%-- see also
\begin{manseealso}
  \manlink{who}{who}  
\end{manseealso}

