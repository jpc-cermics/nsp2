\mansection{dot}
\begin{mandesc}
  \short{dot}{scalar product of 2 vectors (or matrices)}
\end{mandesc}
%-- Calling sequence section
\begin{calling_sequence}
\begin{verbatim}
  C=dot(A,B)  
  C=dot(A,B,mode)  
  C=dot(A,B,dim=mode)  
\end{verbatim}
\end{calling_sequence}
%-- Parameters
\begin{parameters}
  \begin{varlist}
    \vname{A,B}: numerical vectors (or numerical matrices) of same size.
    \vname{mode}: A string chosen among \verb+'M'+, \verb+'m'+, \verb+'full'+, \verb+'FULL'+, \verb+'row'+,
    \verb+'ROW'+, \verb+'col'+, \verb+'COL'+ or an non ambiguous abbreviation or an integer. 
    This argument is optional and if omitted 'full' is assumed.
  \end{varlist}
\end{parameters}
\begin{mandescription}
  \verb+dot+ computes the scalar product between the vectors or matrices \verb+A+ and \verb+B+ :
$$
dot(A,B) = \left\{
\begin{array}{l}
     \sum_{i} \bar{A}_i B_i, \mbox{ for vectors } \\
     \sum_{i,j} \bar{A}_{i,j} B_{i,j}, \mbox{ for matrices } \\
\end{array} \right.
$$ 
 Using the third argument you could compute the dot product between the rows or columns of
 \verb+A+ and \verb+B+, precisely :
  \begin{itemize}
    \item 'full' or 0 (the default case) $$C= \sum_{i,j} \bar{A}_{i,j} B_{i,j}$$
    \item 'row' or 1  computes the scalar product between the corresponding column vectors 
          of the 2 matrices: $$C_j = \sum_{i} \bar{A}_{i,j} B_{i,j}$$ and thus gives a row vector.
    \item 'col' or 2  computes the scalar product between the corresponding row vectors
          of the 2 matrices: $$C_j = \sum_{j} \bar{A}_{i,j} B_{i,j}$$ and thus gives a column vector.
    \item 'm' is the dot product along the first non singleton dimension of the given matrices 
          (for Matlab compatibility). 
  \end{itemize}
\end{mandescription}

%--example 
\begin{examples}
\paragraph{example 1} a simple example (we form 2 orthonormal vectors) :
  \begin{program}\HCode{x = randn(5,1);\Hnewline
    y = randn(5,1);\Hnewline
    // step 0\Hnewline
    dot(x,y)\Hnewline
    // step 1 normalise x:\Hnewline
    x = x/norm(x)\Hnewline
    // step 2 remove the component of y in the direction of x\Hnewline
    y = y - dot(y,x)*x\Hnewline
    // step 3 normalise y\Hnewline
    y = y/norm(y)\Hnewline
    // test \Hnewline
    norm(x)  // should be 1\Hnewline
    norm(y)  // should be 1\Hnewline
    dot(x,y) // should be near 0}
  \end{program}

\paragraph{example 2} same example with complex vectors
  \begin{program}\HCode{x = randn(5,1) + \%i*randn(5,1);\Hnewline
    y = randn(5,1) + \%i*randn(5,1);\Hnewline
    // step 0\Hnewline
    dot(x,y)\Hnewline
    // step 1 normalise x:\Hnewline
    x = x/norm(x)\Hnewline
    // step 2 remove the component of y in the direction of x\Hnewline
    // (be careful for the order in dot in this case)\Hnewline
    y = y - dot(x,y)*x\Hnewline
    // step 3 normalise y\Hnewline
    y = y/norm(y)\Hnewline
    // test \Hnewline
    norm(x)  // should be 1\Hnewline
    norm(y)  // should be 1\Hnewline
    dot(x,y) // should be near 0}
  \end{program}

\paragraph{example 3} fastest way to compute the 2-norm of the rows or
 columns vectors of a matrix : in this case you cannot use norm because
 norm computes a matrix norm in this case.
  \begin{program}\HCode{A = randn(5,7);\Hnewline
    norm_cols_A = sqrt(dot(A,A,dim=1))\Hnewline
    // test\Hnewline
    norm(A(:,3)) - norm_cols_A(3)\Hnewline
    // the same for the rows\Hnewline
    norm_rows_A = sqrt(dot(A,A,dim=2))\Hnewline
    // test\Hnewline
    norm(A(2,:)) - norm_rows_A(2)\Hnewline
    // comparizon with another method for a big matrix\Hnewline
    A = randn(800,800);\Hnewline
    tic(); norm_cols_A = sqrt(dot(A,A,dim=1)); toc()\Hnewline
    // the other method (should be slower)\Hnewline
    tic(); norm_cols_A = sqrt(sum(A.^2,dim=1)); toc()\Hnewline
    // same thing with a complex matrix\Hnewline
    A = randn(800,800) + \%i*rand(800,800);\Hnewline
    tic(); norm_cols_A = sqrt(dot(A,A,dim=1)); toc()\Hnewline
    // the other method (should be slower)\Hnewline
    tic(); norm_cols_A = sqrt(sum(conj(A).*A,dim=1)); toc()}
\end{program}
\end{examples}

% -- see also
\begin{manseealso}
  \manlink{norm}{norm}  \manlink{sum}{sum} 
\end{manseealso}

