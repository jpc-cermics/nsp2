% -*- mode: latex -*-

\mansection{derivative}
\begin{mandesc}
  \short{derivative}{approximation of first and second derivatives of a function}
\end{mandesc}

% -- Calling sequence section
\begin{calling_sequence}
\begin{verbatim}
Ja = derivative(F,x)
[Ja,Ha] = derivative(F,x)
[Ja,Ha] = derivative(F,x, h=stepsize, order=1|2|4, args=F_param, Q=dir)
\end{verbatim}
\end{calling_sequence}

% -- Parameters
\begin{parameters}
  \begin{varlist}
    \vname{F}: nsp function
    \vname{x}: column vector
    \vname{h}: optional named argument, stepsize of finite difference approximation
    \vname{order}: optional named argument, order of the finite difference formula (default is 2)
    \vname{args}: optional named argument, useful to pass additional parameters needed by the function F.
    \vname{Q}: optional named argument, an orthonormal matrix (default is the identity matrix)
    \vname{Ja}: approximate Jacobian of F at x
    \vname{Ha}: approximate Hessean of F at x
  \end{varlist}
\end{parameters}

\begin{mandescription}

This function computes an approximation of the Jacobian matrix of a numerical
function $F$ and optionaly its Hessean at a point \verb+x+. \verb+x+
can be a simple scalar or a column vector and \verb+F+ should return a 
scalar or a column vector, that is: 
$$
    F: \left\{\begin{array}{l} R^m \rightarrow R^n  \\ 
                               x \longmapsto F(x) 
              \end{array} \right.
$$
And:
$$
    Ja \simeq J_F(x) = \left[ \frac{\partial F_i}{\partial x_j}(x)  
                      \right]_{\begin{array}{l}1 \le i \le n\\1\le j
                          \le m \end{array}}
$$
The approximate Hessian is given as a $(n \times m) \times m$ array
with the $m \times m$ Hessian matrices of each component of $F$
stacked by row:
$$
   Ha  \simeq H_F(x) = \left[ \begin{array}{c} H_{F_1}(x) \\ \hline \\ \dots
     \\ \hline \\  H_{F_n}(x) \end{array} \right] \;\; \mbox{ with }
   H_{F_k}(x) =  \left[  \frac{\partial^2 F_k}{\partial x_i \partial x_j}(x)   
                \right]_{\begin{array}{l}1 \le i \le m\\1\le j
                          \le m \end{array}}
$$

\itemdesc{Q argument}
By default the derivatives are computed using directionnal derivatives
with the canonical base vectors of $R^m$ as directions
but any orthonormal base could be used instead. When \verb+Q+ is
provided then the direction used are its columns.

\itemdesc{usefulness}
This function could be helpful within numerical algorithms involving
first or second derivatives of a function (like solving $f(x)=0$
or minimizing $f(x)$). First, instead of providing exact derivatives
you can approximate its with this function. Second if you write a
nsp function to compute exact derivatives you can verify it
by comparing with the output of \verb+derivative+ on several points.

\itemdesc{remark}
Numerical approximation of derivatives is generally an unstable process.  
The step size $h$ must be small to get a low error but if it is too small floating  
point errors will dominate by cancellation. The default step size
(which depends on the formula order) is adapted in most usual cases 
but could not optimal in others, for instance if some component of $x$
are big (in this case try something like \verb+h=sqrt(%eps)*norm(x)+). 
You can also play with the formula order and/or
using random orthogonal matrices Q (\verb+Q = qr(rand(m,m))+)

\itemdesc{passing supplementary arguments to F}
An additional parameter for the function $f$ could be passed using the
named optional argument \verb+args+ which can be any nsp object, see
example 3 (see also \manlink{intg}{intg} example section).

\end{mandescription}

\begin{examples}
  
\paragraph{example 1} a simple example with $f(x)=cos(x)exp(-x)$
\begin{program}\HCode{function y=f(x);y=cos(x)*exp(-x);endfunction\Hnewline
function y=df(x);y=-(sin(x)+cos(x))*exp(-x);endfunction\Hnewline
function y=d2f(x);y=2*sin(x)*exp(-x);endfunction\Hnewline
x = 0.5\Hnewline 
// compute approximation of f'(x) and f''(x)\Hnewline 
[ypa, yppa] = derivative(f,x)\Hnewline 
// compute exact values\Hnewline 
yp = df(x)\Hnewline 
ypp = d2f(x)\Hnewline 
// compute errors\Hnewline 
abs((ypa-yp)/yp)\Hnewline 
abs((yppa-ypp)/ypp)\Hnewline 
// the same using formula of order 1\Hnewline 
[ypa, yppa] = derivative(f,x,order=1)\Hnewline 
// compute errors\Hnewline 
abs((ypa-yp)/yp)\Hnewline 
abs((yppa-ypp)/ypp)\Hnewline 
// the same using formula of order 4\Hnewline 
[ypa, yppa] = derivative(f,x,order=4)\Hnewline 
// compute errors\Hnewline 
abs((ypa-yp)/yp)\Hnewline 
abs((yppa-ypp)/ypp)}
\end{program}

\paragraph{example 2} a more complicated function:
$$f(x,y,z) = \frac{\cos(x) z}{1+y^2} + e^{-x^2} y^3 (2z^2-1)$$.
\begin{program}\HCode{function v = f(x)\Hnewline
v = cos(x(1))/(1+x(2)^2)*x(3) + exp(-x(1)^2)*x(2)^3*(2*x(3)^2-1)\Hnewline
endfunction\Hnewline
// a function providing the exact gradient and hessian:\Hnewline 
function [J,H] = df(X)\Hnewline
x = X(1); y = X(2); z = X(3)\Hnewline
tx = exp(-x^2); ty = 1+y^2; tz = 2*z^2-1\Hnewline
J = [ -2*x*tx*y^3*tz - sin(x)*z/ty,...\Hnewline
       3*tx*y^2*tz - 2*cos(x)*y*z/ty^2,...\Hnewline
       4*tx*y^3*z + cos(x)/ty ]\Hnewline
dfxx = 2*tx*y^3*tz*(2*x^2-1)-cos(x)*z/ty\Hnewline
dfxy = 2*sin(x)*y*z/ty^2 - 6*x*tx*y^2*tz\Hnewline
dfxz = -8*x*tx*y^3*z - sin(x)/ty\Hnewline
dfyy = 6*tx*y*tz -2*cos(x)*z/ty^2*(1-4*y^2/ty)\Hnewline
dfyz = 12*tx*y^2*z - 2*cos(x)*y/ty^2\Hnewline
dfzz = 4*tx*y^3\Hnewline
H = [ dfxx ,dfxy, dfxz;...\Hnewline
      dfxy, dfyy, dfyz;...\Hnewline
      dfxz, dfyz, dfzz ]\Hnewline
endfunction\Hnewline
// try an a random point\Hnewline
X = rand(3,1)\Hnewline
//  exact gradient and Jacobian at X\Hnewline
[J,H] = df(X)\Hnewline
[Ja,Ha] = derivative(f,X)\Hnewline
norm(J-Ja)/norm(J)\Hnewline
norm(H-Ha)/norm(Ha)\Hnewline
// using formula of order 4\Hnewline
[Ja,Ha] = derivative(f,X,order=4)\Hnewline
norm(J-Ja)/norm(J)\Hnewline
norm(H-Ha)/norm(Ha)}
\end{program}
  
\paragraph{example 3} an easy example which shows how to pass
supplementary arguments:
\begin{program}\HCode{function y=f(x,p);A=p{1};b=p{2};y=0.5*x'*A*x-b'*x;endfunction\Hnewline
A = [1,2;2,-0.5]; b=[1;-1];\Hnewline 
x = randn(2,1);\Hnewline 
[J,H] = derivative(f,x,args={A,b})\Hnewline 
// as the function is quadratic errors should be null with formula of\Hnewline 
// order 2 and 4  but are not due to numerical noise by cancellation\Hnewline 
J - (A*x-b)'\Hnewline 
H - A}\Hnewline 
\end{program}

\end{examples}


\begin{manseealso}
  \manlink{intg}{intg}  
\end{manseealso}

% -- Authors
\begin{authors}
  Rainer von Seggern, Bruno Pincon
\end{authors}
