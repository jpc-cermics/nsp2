\mansection{diff}
\begin{mandesc}
  \short{diff}{difference of a vector or a matrix}
\end{mandesc}
%-- Calling sequence section
\begin{calling_sequence}
\begin{verbatim}
  d=diff(x)  
  d=diff(x,dim=dimval, order=orderval)  
  d=diff(x,dimval,orderval)
\end{verbatim}
\end{calling_sequence}
%-- Parameters
\begin{parameters}
  \begin{varlist}
    \vname{x}: numerical vector or matrix
    \vname{dimval}: A string chosen among \verb+'M'+, \verb+'m'+, \verb+'full'+, \verb+'FULL'+, \verb+'row'+,
    \verb+'ROW'+, \verb+'col'+, \verb+'COL'+ or an non ambiguous abbreviation or an integer. 
    This argument is optional and if omitted 'full' is assumed.
    \vname{orderval}: a non negative integer (default is 1)   
  \end{varlist}
\end{parameters}
\begin{mandescription}
  \verb+diff+ computes the difference of a vector \verb+x+:
$$
d_1 = x_1 - x_2, \, d_2 = x_2 - x_3, \dots, d_{n-1} = x_{n-1} - x_n
$$
 so you get a vector (row if $x$ is row or column if $x$ is column) with 
 $n-1$ components (if $x$ has $n$ components).

 The {\em order} parameter lets to compute second, third, etc... differences of the vector
 with the following meaning: \verb+diff(x, order=k)+ correspond to successively apply 
 $k$ time the diff operator, for instance  \verb+dd=diff(x, order=2)+ gives a vector of
 length $n-2$ with:
$$
dd_1 = x_1 - 2x_2 + x_3, \, dd_2 = x_2 - 2x_3 + x_4, \dots, dd_{n-2} = x_{n-2} - 2x_{n-1} + x_n
$$
Using the {\em dim} argument you could compute the difference of the rows or columns vectors
of a matrix \verb+A+, precisely ($A$ being $m \times n$):
  \begin{itemize}
    \item 'row' or 1  computes the difference of the column vectors (the output is then a $(m-order) \times n$
          sized matrix).
    \item 'col' or 2  computes the difference of the rows vectors (the output is then a $m \times (n-order)$
          sized matrix).
    \item 'full' or 0 (the default case) if the matrix is not a row vector it is considered as 
           a big column vector (using the column major order), so the output is of size $(mn - order) \times 1$
    \item 'm' is the difference along the first non singleton dimension of the given matrice 
          (for Matlab compatibility). 
  \end{itemize}
\end{mandescription}

%--example 
\begin{examples}
\paragraph{example 1} basic examples
  \begin{program}\HCode{x = 0:6;\Hnewline
    diff(x)\Hnewline
    x = (0:6).^2;\Hnewline
    diff(x,order=1)\Hnewline
    diff(x,order=2)\Hnewline
    diff(x,order=3)}
  \end{program}

\paragraph{example 2} approximate derivatives of a function
  \begin{program}\HCode{n = 12;\Hnewline
    x = linspace(0,2*\%pi,n);\Hnewline
    y = sin(x);\Hnewline
    // diff(y)./diff(x) gives a good approximation\Hnewline
    // of the derivatives at the middle point of the mesh\Hnewline
    // Here as the mesh is uniform diff(x) could be replaced with\Hnewline
    // the (uniform) step size 2*\%pi/n\Hnewline
    d = diff(y)/(2*\%pi/n)\Hnewline
    // compute middle points:\Hnewline
    xm = 0.5*(x(1:$-1) + x(2:$));\Hnewline
    // now compute the exact derivative\Hnewline
    d_exact = cos(xm)\Hnewline
    abs(d - d_exact)}
  \end{program}

\end{examples}

% -- see also
\begin{manseealso}
  \manlink{sum}{sum},\manlink{derivative}{derivative} 
\end{manseealso}

