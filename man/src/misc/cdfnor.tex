% -*- mode: latex -*-
\mansection{cdfnor}
\begin{mandesc}
  \short{cdfnor}{cumulative distribution function normal distribution}
\end{mandesc}
\index{cdfnor}\label{cdfnor}
%-- Calling sequence section
\begin{calling_sequence}
\begin{verbatim}
  [P,Q]=cdfnor("PQ",X,Mean,Std)  
  [X]=cdfnor("X",Mean,Std,P,Q)  
  [Mean]=cdfnor("Mean",Std,P,Q,X)  
  [Std]=cdfnor("Std",P,Q,X,Mean)  
\end{verbatim}
\end{calling_sequence}
%-- Parameters
\begin{parameters}
  \begin{varlist}
    \vname{P,Q,X,Mean,Std}: six real vectors of the same size.
    \vname{P,Q (Q=1-P)}: the integral from -infinity to X of the normal density. Input range: (0,1].
      \vname{X}: upper limit of integration of the normal-density. Input range: ( -infinity, +infinity)
      \vname{Mean}: the mean of the normal density. Input range: (-infinity, +infinity)
      \vname{Sd}: standard Deviation of the normal density. Input range: (0, +infinity).
  \end{varlist}
\end{parameters}
\begin{mandescription}
  Calculates any one parameter of the normal distribution given values for the others.
  The normal density is given by~:
  \begin{equation}
    \frac{1}{\sigma\sqrt{2\pi}} \; \exp\left(-\frac{\left(x-\mu\right)^2}{2\sigma^2} \right)
  \end{equation} 
  with $\sigma = \mbox{\tt Std}$ and $\mu = \mbox{\tt Mean}$.
  The  cumulative standard normal distribution is computed with a C rewriten version of 
  anorm from W.D Cody (1993) ("ALGORITHM 715: SPECFUN - A Portable FORTRAN
  Package of Special Function Routines and Test Drivers" ACM Transactions on Mathematical Software. 19, 22-32) 

  The rational functions from pages  90-95  of Kennedy and Gentle,
  Statistical  Computing,  Marcel  Dekker, NY,  1980 are  used  as
  starting values to Newton's Iterations which compute the inverse
  standard normal.  Therefore no  searches  are necessary for  any
  parameter. 
  For X $<$ -15, the asymptotic expansion for the normal is used  as
  the starting value in finding the inverse standard normal.
  This is formula 26.2.12 of Abramowitz and Stegun.
\end{mandescription}

\begin{program}\HCode{x=linspace(-5,5,100);\Hnewline
mu =1;\Hnewline
sig =0.5;\Hnewline
y= cdfnor("PQ",x,mu*ones(x),sig*ones(x));\Hnewline
\Hnewline
function y=f(x)\Hnewline
  y= (1/2)*(1 + erf((x-mu)/(sig*sqrt(2))));\Hnewline
endfunction;\Hnewline
\Hnewline
z=f(x);\Hnewline
err=max(abs(y-z));}
\end{program}

\begin{authors}
  Nsp interface by Jean-Philippe Chancelier. Code from DCDFLIB: 
  Library of Fortran Routines for Cumulative Distribution
  Functions, Inverses, and Other Parameters (February, 1994)
  Barry W. Brown, James Lovato and Kathy Russell. The University of Texas.
\end{authors}
