% -*- mode: latex -*-
\mansection{cdfpoi}
\begin{mandesc}
  \short{cdfpoi}{cumulative distribution function poisson distribution}
\end{mandesc}
\index{cdfpoi}\label{cdfpoi}
%-- Calling sequence section
\begin{calling_sequence}
\begin{verbatim}
  [P,Q]=cdfpoi("PQ",S,Xlam)  
  [S]=cdfpoi("S",Xlam,P,Q)  
  [Xlam]=cdfpoi("Xlam",P,Q,S);  
\end{verbatim}
\end{calling_sequence}
%-- Parameters
\begin{parameters}
  \begin{varlist}
    \vname{P,Q,S,Xlam} : four real vectors of the same size.
    \vname{P,Q (Q=1-P)  } :  The cumulation from 0 to S of the poisson density. Input range: [0,1].
    \vname{S} :Upper limit of cumulation of the Poisson. Input range: [0, +infinity). Search range: [0,1E300]
      \vname{Xlam} :  Mean of the Poisson distribution. Input range: [0, +infinity). Search range: [0,1E300]
  \end{varlist}
\end{parameters}
\begin{mandescription}
  Calculates any one parameter of the Poisson
  distribution given values for the others.
  Formula   26.4.21  of   Abramowitz  and   Stegun,   Handbook  of
  Mathematical Functions (1966) is used  to reduce the computation
  of  the cumulative distribution function to that  of computing a
  chi-square, hence an incomplete gamma function.
  Cumulative  distribution function  (P) is  calculated  directly.
  Computation of other parameters involve a seach for a value that
  produces  the desired value of  P.   The  search relies  on  the
  monotinicity of P with the other parameter.
\end{mandescription}

\begin{authors}
  Nsp interface by Jean-Philippe Chancelier. Code from DCDFLIB: 
  Library of Fortran Routines for Cumulative Distribution
  Functions, Inverses, and Other Parameters (February, 1994)
  Barry W. Brown, James Lovato and Kathy Russell. The University of Texas.
\end{authors}
