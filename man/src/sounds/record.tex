% -*- mode: latex -*-
\mansection{record}
\begin{mandesc}
  \short{record}{record audio samples with portaudio} 
\end{mandesc}
%-- Calling sequence section
\begin{calling_sequence}
\begin{verbatim}
  y = record(device=int,channels=int,seconds=num,samplerate=num,o_device=num)
\end{verbatim}
\end{calling_sequence}
%-- Parameters
\begin{parameters}
  \begin{varlist}
   \vname{device}: device to be used for input (default value \verb+-1+). The default
   input device is used if \verb+device+ is equal to \verb+-1+.
   \vname{channels}: number of channels to record.
   \vname{seconds}: record for \verb+seconds+. If \verb+seconds+ is negative the 
   record length is unbounded and recording is to be stopped by the user using the 
   Stop Audio submenu of the Control menu. 
   \vname{o_device}: if set to a value greater or equal to \verb+-1+ the recorded 
   samples are also played in the output device given by \verb+o_device+.
   \vname{samplerate} give the audio sample rate (default value is 44100).
  \end{varlist}
\end{parameters}

\begin{mandescription}
  record sample sounds and return the samples in a matrix (the number of 
  rows gives the number of channels, the number of columns gives the number of samples).
\end{mandescription}
%--example 
\begin{examples}
  \begin{Verbatim}
    y = record(seconds=3,rate=44100);
    play(y,sync=%t);
  \end{Verbatim}
\end{examples}
\begin{manseealso}
  \manlink{play}{play}  
  \manlink{playfile}{playfile}  
  \manlink{player\_create}{player-create}  
\end{manseealso}

