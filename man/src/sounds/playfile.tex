% -*- mode: latex -*-

\mansection{playfile}
\begin{mandesc}
  \short{playfile}{play a sound file using libsndile and portaudio} 
\end{mandesc}
%-- Calling sequence section
\begin{calling_sequence}
\begin{verbatim}
  playfile(y, sync=bool,device=int)
\end{verbatim}
\end{calling_sequence}
%-- Parameters
\begin{parameters}
  \begin{varlist}
    \vname{file}: a path name to a file containing sample sounds. 
    \vname{sync}: optional boolean argument (\verb+%f+ by default). 
    If true the call to \verb+playfile+ 
    will block until the end of the play. The waiting time is spent
    in a gtk loop. The audio play may be interupted by 
    Ctrl-C, or by using the Stop Audio submenu of the Control menu. 
    If \verb+sync+ is false, the \verb+playfile+ is run in a non-blocking way, the 
    play may be stopped by the  Stop Audio submenu or by an other call 
    to \verb+playfile+. 
   \vname{device}: device to be used for output (default value \verb+-1+). The default output 
   device is used if \verb+device+ is equal to \verb+-1+.
  \end{varlist}
\end{parameters}

\begin{mandescription}
  play sample sounds contained in a file. The sound file is read with 
  the help of the \verb+libsndfile+ library and should have a format compatible with 
  \verb+libsndfile+.
\end{mandescription}
%--example 
\begin{examples}
  \begin{Verbatim}
    rate=44100;
    t = linspace(0,3,3*rate);
    y = sin(2*%pi*440*t); // mono 
    F=sndfile('TMPDIR/test.wav',mode='w',samplerate=rate,channels=1);
    F.write[x];
    F.close[];
    // now play the file 
    playfile('TMPDIR/test.wav')
  \end{Verbatim}
\end{examples}
\begin{manseealso}
  \manlink{record}{record }
  \manlink{play}{play}  
  \manlink{player\_create}{player_create}  
\end{manseealso}

