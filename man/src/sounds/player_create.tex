% -*- mode: latex -*-

\mansection{play}
\begin{mandesc}
  \shortunder{player\_create}{player_create}{play audio samples using portaudio} 
\end{mandesc}
%-- Calling sequence section
\begin{calling_sequence}
\begin{verbatim}
  P=player_create(y, device=int,samplerate=num,channels=int)  
  P.write[y];
  P.stop[];
\end{verbatim}
\end{calling_sequence}
%-- Parameters
\begin{parameters}
  \begin{varlist}
    \vname{y}: audio sample given as a real matrix. The number of 
    rows must be equal to the channels parameter used to create \verb+P+.
    The number of columns gives the number of samples.
   \vname{channels}: number of channels to record.
   \vname{device}: device to be used for output (default value \verb+-1+). The default output 
   device is used if \verb+device+ is equal to \verb+-1+.
   \vname{samplerate} give the audio sample rate (default value is 44100).
  \end{varlist}
\end{parameters}

\begin{mandescription}
  \verb+player_create+ is used to create a player object which can be used to 
  write sound samples to an audio device using portaudio. 
  Sound samples are written to the audio device with the \verb+write+ method 
  (which is blocking). The audio stream is closed using the \verb+stop+ method. 
\end{mandescription}
%--example 
\begin{examples}
  \begin{Verbatim}
    rate=44100;
    t = linspace(0,3,3*rate);
    y = sin(2*%pi*440*t); // mono 
    P = player_create(samplerate=rate,channels=1);
    for freq=440:10:480
      y = sin(2*%pi*freq*t);
      P.write[y]; 
    end
    P.stop[];
  \end{Verbatim}
\end{examples}
\begin{manseealso}
  \manlink{record}{record} 
  \manlink{play}{play}  
  \manlink{playfile}{playfile}  
\end{manseealso}

