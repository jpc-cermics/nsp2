% -*- mode: latex -*-

\mansection{play}
\begin{mandesc}
  \short{play}{play audio samples using portaudio} 
\end{mandesc}
%-- Calling sequence section
\begin{calling_sequence}
\begin{verbatim}
  play(y, sync=bool,device=int,samplerate=num,nocb=bool)  
\end{verbatim}
\end{calling_sequence}
%-- Parameters
\begin{parameters}
  \begin{varlist}
    \vname{y}: audio sample given as a real matrix. The number of 
    rows gives the number of channels. The number of columns gives 
    the number of samples.
    \vname{sync}: optional boolean argument (\verb+%f+ by default). If true the call to \verb+play+ 
    will block until the end of the play. The waiting time is spent
    in a gtk loop. The audio play may be interupted by 
    Ctrl-C, or by using the Stop Audio submenu of the Control menu. 
    If sync is false, the \verb+play+ is run in a non-blocking way, the 
    \verb+play+ may be stopped by the  Stop Audio submenu or by an other call 
    to \verb+play+. 
   \vname{device}: device to be used for output (default value \verb+-1+). The default output 
   device is used if \verb+device+ is equal to \verb+-1+.
   \vname{samplerate} give the audio sample rate (default value is 44100).
   \vname{nocb} if true then the portaudio API without callback is used 
   (default value is \verb+%f+).
  \end{varlist}
\end{parameters}

\begin{mandescription}
  plays audio samples given by \verb+y+ using portAudio (portable cross-platform Audio API).
\end{mandescription}
%--example 
\begin{examples}
  \begin{Verbatim}
    rate=44100;
    t = linspace(0,3,3*rate);
    y = sin(2*%pi*440*t); // mono 
    play(y,sync=%t);
    y = [y,sin(2*%pi*220*t)] // stereo
    play(y,sync=%t);
    // a more complex sound 
    p=[1,3;5,3;8,3]; 
    freq=[440*(2.^(p(:,2)-3+(p(:,1)-10)/12))]; // C3,E3,G3
    y= [0.4,0.3,0.1]*sin(2*%pi*freq*t);
    play(y,sync=%t);
  \end{Verbatim}
\end{examples}
\begin{manseealso}
  \manlink{record}{record} 
  \manlink{player\_create}{player_create}  
  \manlink{playfile}{playfile}  
\end{manseealso}

