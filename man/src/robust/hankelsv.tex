% -*- mode: latex -*-
\mansection{hankelsv}
\begin{mandesc}
  \short{hankelsv}{Hankel singular values} \\ % 
\end{mandesc}
%\index{hankelsv}\label{hankelsv}
%-- Calling sequence section
\begin{calling_sequence}
\begin{verbatim}
  [nk2,W]=hankelsv(sl [,tol])  
  [nk2]=hankelsv(sl [,tol])  
\end{verbatim}
\end{calling_sequence}
%-- Parameters
\begin{parameters}
  \begin{varlist}
    \vname{sl}:  a linear system (state-space).
    \vname{tol}: tolerance parameter for detecting imaginary axis modes 
    (default value is \verb!1000*%eps!).
  \end{varlist}
\end{parameters}
\begin{mandescription}
  returns \verb!nk2!, the squared Hankel singular values of \verb!sl!
  and \verb!W = P*Q! = controllability gramian times observability
  gramian.\verb!nk2! is the vector of eigenvalues of \verb!W!.
\end{mandescription}
%--example 
\begin{examples}
  \begin{mintednsp}{nsp}
    A=diag([-1,-2,-3]);
    sl=syslin('c',A,rand(3,2),rand(2,3));
    [nk2,W]=hankelsv(sl)
    [Q,M]=pbig(W,nk2(2)-%eps,'c');
    slr=projsl(sl,Q,M);hankelsv(slr)
  \end{mintednsp}
\end{examples}
%-- see also
\begin{manseealso}
  \manlink{balreal}{balreal} \manlink{equil}{equil} \manlink{equil1}{equil1}  
\end{manseealso}
