% -*- mode: latex -*-
\mansection{riccati}
\begin{mandesc}
  \short{riccati}{Riccati equation} \\ % 
\end{mandesc}
%\index{riccati}\label{riccati}
%-- Calling sequence section
\begin{calling_sequence}
\begin{verbatim}
  X=riccati(A,B,C,dom,[typ])  
  [X1,X2]=riccati(A,B,C,dom,[typ])  
\end{verbatim}
\end{calling_sequence}
%-- Parameters
\begin{parameters}
  \begin{varlist}
    \vname{A,B,C}: real matrices nxn, \verb!B! and  \verb!C! symetric.
    \vname{dom}: \verb!'c'! or \verb!'d'! for the time domain (continuous or discrete)
    \vname{typ}: string: \verb!'eigen'! for block diagonalization or \verb!schur'! for  Schur method.
    \vname{X1,X2,X}: square real matrices (X2 invertible), X symmetric
  \end{varlist}
\end{parameters}
\begin{mandescription}
  \verb!X=riccati(A,B,C,dom,[typ])! solves the Riccati equation:
\begin{verbatim}
  A'*X+X*A-X*B*X+C=0 
\end{verbatim}
  in continuous time case, or:
\begin{verbatim}
  A'*X*A-(A'*X*B1/(B2+B1'*X*B1))*(B1'*X*A)+C-X
\end{verbatim}
  with \verb!B=B1/B2*B1'! in the discrete time case.
  If called with two output arguments, \verb!riccati! returns \verb!X1,X2!
  such that \verb!X=X1/X2!.
\end{mandescription}
% -- see also
\begin{manseealso}
  \manlink{ricc}{ricc} \manlink{ric\_desc}{ric-desc}  
\end{manseealso}
