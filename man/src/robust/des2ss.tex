% -*- mode: latex -*-
\mansection{des2ss}
\begin{mandesc}
  \short{des2ss}{descriptor to state-space} \\ % 
\end{mandesc}
%\index{des2ss}\label{des2ss}
%-- Calling sequence section
\begin{calling_sequence}
\begin{verbatim}
  [Sl]=des2ss(A,B,C,D,E [,tol])  
  [Sl]=des2ss(Des)  
\end{verbatim}
\end{calling_sequence}
%-- Parameters
\begin{parameters}
  \begin{varlist}
    \vname{A,B,C,D,E}: real matrices of appropriate dimensions
    \vname{Des}: list
    \vname{Sl}: \verb!syslin! list
    \vname{tol}: real parameter (threshold) (default value \verb!100*%eps!).
  \end{varlist}
\end{parameters}
\begin{mandescription}
  Descriptor to state-space transform.\verb!Sl=des2ss(A,B,C,D,E)! returns a
  linear system \verb!Sl! equivalent to the descriptor system
  \verb!(E,A,B,C,D)!.  For index one \verb!(E,A)! pencil, explicit formula is
  used and for higher index pencils \verb!rowshuff! is
  used.\verb!Sl=des2ss(Des)! with \verb!Des=list('des',A,B,C,D,E)! returns a
  linear system \verb!Sl! in state-space form with possibly a polynomial
  \verb!D! matrix.  A generalized Leverrier algorithm is used.
\end{mandescription}
%--example 
\begin{examples}
  \begin{Verbatim}
    s=poly(0,'s');G=[1/(s-1),s;1,2/s^3];
    S1=tf2des(G);S2=tf2des(G,"withD");
    W1=des2ss(S1);W2=des2ss(S2);
    clean(ss2tf(W1))
    clean(ss2tf(W2))
  \end{Verbatim}
\end{examples}
%-- see also
\begin{manseealso}
  \manlink{des2tf}{des2tf} \manlink{glever}{glever} \manlink{rowshuff}{rowshuff}  
\end{manseealso}
