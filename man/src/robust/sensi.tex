% -*- mode: latex -*-
\mansection{sensi}
\begin{mandesc}
  \short{sensi}{sensitivity functions} \\ % 
\end{mandesc}
%\index{sensi}\label{sensi}
%-- Calling sequence section
\begin{calling_sequence}
\begin{verbatim}
  [Se,Re,Te]=sensi(G,K)  
  [Si,Ri,Ti]=sensi(G,K,flag)  
\end{verbatim}
\end{calling_sequence}
%-- Parameters
\begin{parameters}
  \begin{varlist}
    \vname{G}: standard plant (\verb!syslin! list)
    \vname{K}: compensator (\verb!syslin! list)
    \vname{flag}: character string \verb!'o'! (default value) or \verb!'i'!
    \vname{Se}: output sensitivity function \verb!(I+G*K)^-1!
    \vname{Re}: \verb!K*Se!
    \vname{Te}: \verb!G*K*Se! (output complementary sensitivity function)
  \end{varlist}
\end{parameters}
\begin{mandescription}
  \verb!sensi! computes sensitivity functions. If \verb!G! and \verb!K! are
  given in state-space form, the systems returned are generically minimal. 
  Calculation is made by \verb!lft!, e.g., 
  \verb!Se! can be given by the commands 
  \verb! P = augment(G,'S')!, \verb!Se=lft(P,K)!.
  If \verb!flag! = \verb!'i'!, \verb![Si,Ri,Ti]=sensi(G,K,'i')!
  returns the input sensitivity functions.
\begin{verbatim}
  [Se;Re;Te]= [inv(eye()+G*K);K*inv(eye()+G*K);G*K*inv(eye()+G*K)];
  [Si;Ri;Ti]= [inv(eye()+K*G);G*inv(eye()+K*G);K*G*inv(eye()+K*G)];
\end{verbatim}
\end{mandescription}
%--example 
\begin{examples}
  \begin{mintednsp}{nsp}
    G=ssrand(1,1,3);K=ssrand(1,1,3);
    [Se,Re,Te]=sensi(G,K);
    Se1=inv(eye(size(G*K))+G*K);  //Other way to compute
    ss2tf(Se)    //Se seen in transfer form
    ss2tf(Se1)
    ss2tf(Te)
    ss2tf(G*K*Se1)
    [Si,Ri,Ti]=sensi(G,K,'i');
    w1=[ss2tf(Si);ss2tf(Ri);ss2tf(Ti)]
    KG=K*G;
    w2=[ss2tf(inv(eye(size(KG))+KG));
        ss2tf(G*inv(eye(size(KG))+KG));
        ss2tf(K*G*inv(eye(size(KG))+KG))];
    clean(w1-w2)
  \end{mintednsp}
\end{examples}
%-- see also
\begin{manseealso}
  \manlink{augment}{augment} \manlink{lft}{lft} \manlink{h\_cl}{h-cl}  
\end{manseealso}
