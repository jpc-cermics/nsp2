% -*- mode: latex -*-
\mansection{dhinf}
\begin{mandesc}
  \short{dhinf}{H-infinity design of discrete-time systems} \\ % 
\end{mandesc}
% \index{dhinf}\label{dhinf}
% -- Calling sequence section
\begin{calling_sequence}
\begin{verbatim}
  [AK,BK,CK,DK,(RCOND)] = dhinf(A,B,C,D,ncon,nmeas,gamma)  
\end{verbatim}
\end{calling_sequence}
%-- Parameters
\begin{parameters}
  \begin{varlist}
    \vname{A}: the n-by-n system state matrix A.
    \vname{B}: the n-by-m system input matrix B.
    \vname{C}: the p-by-n system output matrix C.
    \vname{D}: the p-by-m system matrix D.
    \vname{ncon}: the number of control inputs. m $>$= ncon $>$= 0, p-nmeas $>$= ncon.
    \vname{nmeas}: the number of measurements. p $>$= nmeas $>$= 0, m-ncon $>$= nmeas.
    \vname{gamma}: the parameter gamma used in \verb!H_infinity! design. It is assumed that
    gamma is sufficiently large so that the controller is admissible. gamma $>$= 0.
    \vname{AK}: the n-by-n controller state matrix AK.
    \vname{BK}: the n-by-nmeas controller input matrix BK.
    \vname{CK}: the ncon-by-n controller output matrix CK.
    \vname{DK}: the ncon-by-nmeas controller matrix DK.
    \vname{RCOND}: a vector containing estimates of the reciprocal condition numbers of
    the matrices which are to be inverted and estimates of the reciprocal condition numbers
    of the Riccati equations which have to be solved during the computation of the controller.
    (See the description of the algorithm in [1].)
    \begin{varlist}
      \vname{RCOND} (1) contains the reciprocal condition number of the  matrix R3,
      \vname{RCOND} (2) contains the reciprocal condition number of the  matrix R1 - R2'*inv(R3)*R2
      \vname{RCOND} (3) contains the reciprocal condition number of the matrix V21,
      \vname{RCOND} (4) contains the reciprocal condition number of the   matrix St3,
      \vname{RCOND} (5) contains the reciprocal condition number of the  matrix V12,
      \vname{RCOND} (6) contains the reciprocal condition number of the matrix Im2 + DKHAT*D22,
      \vname{RCOND} (7) contains the reciprocal condition number of the  X-Riccati equation,
      \vname{RCOND} (8) contains the reciprocal condition number of the  Z-Riccati equation.
    \end{varlist}
  \end{varlist}
\end{parameters}
\begin{mandescription}
  \verb![AK,BK,CK,DK,(RCOND)] = dhinf(A,B,C,D,ncon,nmeas, gamma)! 
  To compute the matrices of an H-infinity (sub)optimal n-state
  controller
\begin{verbatim}
      | AK | BK |
  K = |----|----|,
      | CK | DK |
\end{verbatim}
for the discrete-time system
\begin{verbatim}
      | A  | B1  B2  |   | A | B |
  P = |----|---------| = |---|---|,
      | C1 | D11 D12 |   | C | D | 
      | C2 | D21 D22 |
\end{verbatim}
and for a given value of gamma, where B2 has column size of the
number of control inputs (ncon) and C2 has row size of the number
of measurements (nmeas) being provided to the controller.
\end{mandescription}
%-- section-References
\paragraph{References}
[1] P.Hr. Petkov, D.W. Gu and M.M. Konstantinov. Fortran 77 routines
for Hinf and H2 design of linear discrete-time control systems.
Report99-8, Department of Engineering, Leicester University, April 1999.
% --example 
\begin{examples}
  \begin{mintednsp}{nsp}
    //example from Niconet report SLWN1999-12
    //Hinf
    A=[-0.7  0    0.3  0   -0.5 -0.1
      -0.6  0.2 -0.4 -0.3  0    0
      -0.5  0.7 -0.1  0    0   -0.8
      -0.7  0    0   -0.5 -1    0
      0    0.3  0.6 -0.9  0.1 -0.4
      0.5 -0.8  0    0    0.2 -0.9];
    B=[-1 -2 -2  1  0
      1  0  1 -2  1
      -3 -4  0  2 -2
      1 -2  1  0 -1
      0  1 -2  0  3
      1  0  3 -1 -2];
    C=[ 1 -1  2 -2  0 -3
      -3  0  1 -1  1  0
      0  2  0 -4  0 -2
      1 -3  0  0  3  1
      0  1 -2  1  0 -2];
    D=[1 -1 -2  0  0
      0  1  0  1  0
      2 -1 -3  0  1
      0  1  0  1 -1
      0  0  1  2  1];
    ncon=2
    nmeas=2
    gam=111.30;
    [AK,BK,CK,DK] = dhinf(A,B,C,D,ncon,nmeas,gam)
  \end{mintednsp}
\end{examples}
%-- see also
\begin{manseealso}
  \manlink{hinf}{hinf} \manlink{h\_inf}{h-inf}  
\end{manseealso}
