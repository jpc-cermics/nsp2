% -*- mode: latex -*-
\mansection{h\_inf}
\begin{mandesc}
  \shortunder{h\_inf}{h_inf}{H-infinity (central) controller} \\ % 
\end{mandesc}
%\index{h\_inf}\label{h-inf}
%-- Calling sequence section
\begin{calling_sequence}
\begin{verbatim}
  [Sk,ro]=h_inf(P,r,romin,romax,nmax)  
  [Sk,rk,ro]=h_inf(P,r,romin,romax,nmax)  
\end{verbatim}
\end{calling_sequence}
%-- Parameters
\begin{parameters}
  \begin{varlist}

    \vname{P}: \verb!syslin! list: continuous-time linear system (``augmented''
    plant given in state-space form or in transfer form)

    \vname{r}: size of the \verb!P22! plant i.e. 2-vector
    \verb![#outputs,#inputs]!
    
    \vname{romin,romax}: a priori bounds on \verb!ro! with \verb!ro=1/gama^2!;
    (\verb!romin=0!  usually)

    \vname{nmax}: integer, maximum number of iterations in the gama-iteration.
  \end{varlist}
\end{parameters}
\begin{mandescription}
  \verb!h_inf! computes H-infinity optimal controller for the continuous-time
  plant \verb!P!.  The partition of \verb!P! into four sub-plants is given
  through the 2-vector \verb!r! which is the size of the \verb!22! part of
  \verb!P!. \verb!P! is given in state-space e.g. \verb!P=syslin('c',A,B,C,D)!
  with \verb!A,B,C,D! = constant matrices or \verb!P=syslin('c',H)! with
  \verb!H! a transfer matrix. 

  \verb![Sk,ro]=H_inf(P,r,romin,romax,nmax)! returns
  \verb!ro! in \verb![romin,romax]! and the central controller \verb!Sk! in the
  same representation as \verb!P!. 

  All calculations are made in state-space,
  i.e conversion to state-space is done by the function, if necessary. 

  \verb![Sk,rk,ro]=H_inf(P,r,romin,romax,nmax)! 
  returns \verb!ro! and the Parameterization of all stabilizing controllers: a
  stabilizing controller \verb!K! is obtained by \verb!K=lft(Sk,r,PHI)! where
  \verb!PHI! is a linear system with dimensions \verb!r'! and
  satisfy:\verb!H_norm(PHI) < gamma!.  \verb!rk (=r)! is the size of the
  \verb!Sk22! block and \verb!ro = 1/gama^2!  after \verb!nmax! iterations.
  Algorithm is adapted from Safonov-Limebeer. Note that \verb!P! is assumed to
  be a continuous-time plant.
\end{mandescription}
%-- see also
\begin{manseealso}
  \manlink{gamitg}{gamitg} \manlink{ccontrg}{ccontrg} \manlink{leqr}{leqr}  
\end{manseealso}
%-- Author
\begin{authors}
  F.Delebecque INRIA (1990)  
\end{authors}
