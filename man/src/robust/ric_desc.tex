% -*- mode: latex -*-
\mansection{ric\_desc}
\begin{mandesc}
  \shortunder{ric\_desc}{ric_desc}{Riccati equation} \\ % 
\end{mandesc}
% \index{ric\_desc}\label{ric-desc}
% -- Calling sequence section
\begin{calling_sequence}
\begin{verbatim}
  X=ric_desc(H [,E))  
    [X1,X2,zero]=ric_desc(H [,E])  
\end{verbatim}
\end{calling_sequence}
% -- Parameters
\begin{parameters}
  \begin{varlist}
    \vname{H,E}: real square matrices
    \vname{X1,X2}: real square matrices
    \vname{zero}: real number
  \end{varlist}
\end{parameters}
\begin{mandescription}
  Riccati solver with hamiltonian matrices as inputs.
  In the continuous time case calling sequence is \verb!ric_descr(H)! (one input):
  Riccati equation is:
\begin{verbatim}
  (Ec)   A'*X + X*A + X*R*X -Q = 0.
\end{verbatim}
  Defining the hamiltonian matrix \verb!H! by:
\begin{verbatim}
  H = [A  R;
       Q -A']
\end{verbatim}
  with the calling sequence \verb![X1,X2,zero]=ric_descr(H)!, the 
  solution \verb!X! is given by \verb!X=X1/X2!.\verb!zero! = L1 norm of rhs of (\verb!Ec!)
  The solution \verb!X! is also given by \verb!X=riccati(A,Q,R,'c'))!
  In the discrete-time case calling sequence is \verb!ric_descr(H,E)! (two inputs):
  The Riccati equation is:
\begin{verbatim}
  (Ed)  A'*X*A-(A'*X*B*(R+B'*X*B)^-1)*(B'*X*A)+C-X = 0.
\end{verbatim}
  Defining \verb!G=B/R*B'! and the hamiltonian pencil \verb!(E,H)! by:
\begin{verbatim}
  E=[eye(n,n),G;               H=[A, 0*ones(n,n);
     0*ones(n,n),A']             -C, eye(n,n)];
\end{verbatim}
  with the calling sequence \verb![X1,X2,err]=ric_descr(H,E)!, the 
  solution \verb!X! is given by \verb!X=X1/X2!.\verb!zero!= L1 norm of rhs of (\verb!Ed!)
  The solution \verb!X! is also given by \verb!X=riccati(A,G,C,'d')!  
  with \verb!G=B/R*B'!
\end{mandescription}
% -- see also
\begin{manseealso}
  \manlink{riccati}{riccati}  
\end{manseealso}
