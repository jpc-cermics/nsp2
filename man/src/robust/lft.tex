% -*- mode: latex -*-
\mansection{lft}
\begin{mandesc}
  \short{lft}{linear fractional transformation} \\ % 
\end{mandesc}
%\index{lft}\label{lft}
%-- Calling sequence section
\begin{calling_sequence}
\begin{verbatim}
  [P1]=LFT(P,K)  
  [P1]=LFT(P,r,K)  
  [P1,r1]=LFT(P,r,Ps,rs)  
\end{verbatim}
\end{calling_sequence}
%-- Parameters
\begin{parameters}
  \begin{varlist}
    \vname{P}: linear system (\verb!syslin! list), the ``augmented'' plant, implicitly partitioned into four blocks (two input ports and two output ports).
    \vname{K}: linear system (\verb!syslin! list), the controller (possibly an ordinary gain).
    \vname{r}: 1x2 row vector, dimension of \verb!P22!
    \vname{Ps  }: linear system (\verb!syslin! list), implicitly partitioned into four blocks (two input ports and two output ports).
    \vname{rs  }: 1x2 row vector, dimension of \verb!Ps22!
  \end{varlist}
\end{parameters}
\begin{mandescription}
  Linear fractional transform between two standard plants
  \verb!P! and \verb!Ps! in state space form or in
  transfer form (\verb!syslin! lists).\verb!r= size(P22) rs=size(P22s)!\verb!LFT(P,r, K)! is the linear fractional transform
  between \verb!P! and a controller \verb!K!
  (\verb!K! may be a gain or a controller in state space form
  or in transfer form);\verb!LFT(P,K)! is \verb!LFT(P,r,K)! with
  \verb!r!=size of \verb!K! transpose;\verb!P1= P11+P12*K* (I-P22*K)^-1 *P21!\verb![P1,r1]=LFT(P,r,Ps,rs)! returns the generalized (2
  ports) lft of \verb!P! and \verb!Ps!.\verb!P1! is the pair two-port interconnected plant and the
  partition of \verb!P1! into 4 blocks in given by
  \verb!r1! which is the dimension of the \verb!22!
  block of \verb!P1!.\verb!P! and \verb!R! can be PSSDs i.e. may admit a
  polynomial \verb!D! matrix.
\end{mandescription}
%--example 
\begin{examples}
  \begin{Verbatim}
    s=poly(0,'s');
    P=[1/s, 1/(s+1); 1/(s+2),2/s]; K= 1/(s-1);
    lft(P,K)
    lft(P,[1,1],K)
    P(1,1)+P(1,2)*K*inv(1-P(2,2)*K)*P(2,1)   //Numerically dangerous!
    ss2tf(lft(tf2ss(P),tf2ss(K)))
    lft(P,-1)
    f=[0,0;0,1];w=P/.f; w(1,1)
    //Improper plant (PID control)
    W=[1,1;1,1/(s^2+0.1*s)];K=1+1/s+s
    lft(W,[1,1],K); ss2tf(lft(tf2ss(W),[1,1],tf2ss(K)))
  \end{Verbatim}
\end{examples}
%-- see also
\begin{manseealso}
  \manlink{sensi}{sensi} \manlink{augment}{augment} \manlink{feedback}{feedback} \manlink{sysdiag}{sysdiag}  
\end{manseealso}
