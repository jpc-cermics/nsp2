% -*- mode: latex -*-
%
% Scilab ( http://www.scilab.org/ ) - This file is part of Scilab
% Copyright (C) 2008 - INRIA -  
% 
% This file must be used under the terms of the CeCILL.
% This source file is licensed as described in the file COPYING, which
% you should have received as part of this distribution.  The terms
% are also available at    
% http://www.cecill.info/licences/Licence_CeCILL_V2-en.txt
%



% -*- mode: latex -*-
\mansection{augment}
\begin{mandesc}
  \short{augment}{augmented plant} \\ % 
\end{mandesc}
%\index{augment}\label{augment}
%-- Calling sequence section
\begin{calling_sequence}
\begin{verbatim}
  [P,r]=augment(G)  
  [P,r]=augment(G,flag1)  
  [P,r]=augment(G,flag1,flag2)  
\end{verbatim}
\end{calling_sequence}
%-- Parameters
\begin{parameters}
  \begin{varlist}
    \vname{G}: linear system (\verb!syslin! list), the nominal plant
    \vname{flag1}: one of the following (upper case) character string: \verb! 'S' !, \verb! 'R' !, \verb! 'T' !\verb! 'SR' !, \verb! 'ST' !, \verb! 'RT' !\verb! 'SRT' !
    \vname{flag2}: one of the following character string: \verb! 'o' ! (stands for 'output', this is the default value) or \verb!'i'! (stands for 'input').
    \vname{P}: linear system (\verb!syslin! list), the ``augmented'' plant
    \vname{r}: 1x2 row vector, dimension of \verb!P22 = G!
  \end{varlist}
\end{parameters}
\begin{mandescription}
  If \verb!flag1='SRT'! (default value), returns the "full" augmented plant
\begin{verbatim}
  [ I | -G]   --$>$'S'
  [ 0 |  I]   --$>$'R'
  P = [ 0 |  G]   --$>$'T'
  [-------]
  [ I | -G]
\end{verbatim}
\verb! 'S' !, \verb! 'R' !, \verb! 'T' ! refer to the first three (block) rows
of \verb!P! respectively.
If one of these letters is absent in \verb!flag1!, the corresponding
row in \verb!P! is missing.
If \verb!G! is given in state-space form, the returned \verb!P! is minimal.
\verb!P! is calculated by: \verb![I,0,0;0,I,0;-I,0,I;I,0,0]*[I,-G;0,I;I,0]!.
The augmented plant associated with input sensitivity functions, namely
\begin{verbatim}
  [ I | -I]   --$>$'S'  (input sensitivity)
  [ G | -G]   --$>$'R'  (G*input sensitivity)
  P = [ 0 |  I]   --$>$'T'  (K*G*input sensitivity)
  [-------]
  [ G | -G]
\end{verbatim}
is obtained by the command \verb![P,r]=augment(G,flag,'i')!. For
state-space \verb!G!, this \verb!P!
is calculated by: \verb![I,-I;0,0;0,I;0,0]+[0;I;0;I]*G*[I,-I]!
and is thus generically minimal.
Note that weighting functions can be introduced by left-multiplying
\verb!P! by a diagonal system of appropriate dimension, e.g.,
\verb! P = sysdiag(W1,W2,W3,eye(G))*P!.
Sensitivity functions can be calculated by \verb!lft!. One has:
For output sensitivity functions [P,r]=augment(P,'SRT'):
lft(P,r,K)=[inv(eye()+G*K);K*inv(eye()+G*K);G*K*inv(eye()+G*K)];
For input sensitivity functions [P,r]=augment(P,'SRT','i'):
lft(P,r,K)=[inv(eye()+K*G);G*inv(eye()+K*G);K*G*inv(eye()+G*K)];
\end{mandescription}
%--example 
\begin{examples}
  \begin{Verbatim}
    G=ssrand(2,3,2); //Plant
    K=ssrand(3,2,2); //Compensator
    [P,r]=augment(G,'T');
    T=lft(P,r,K);   //Complementary sensitivity function
    Ktf=ss2tf(K);Gtf=ss2tf(G);
    Ttf=ss2tf(T);T11=Ttf(1,1);
    Oloop=Gtf*Ktf;
    Tn=Oloop*inv(eye(size(Oloop))+Oloop);
    clean(T11-Tn(1,1));
    //
    [Pi,r]=augment(G,'T','i');
    T1=lft(Pi,r,K);T1tf=ss2tf(T1); //Input Complementary sensitivity function
    Oloop=Ktf*Gtf;
    T1n=Oloop*inv(eye(size(Oloop))+Oloop);
    clean(T1tf(1,1)-T1n(1,1))
  \end{Verbatim}
\end{examples}
%-- see also
\begin{manseealso}
  \manlink{lft}{lft} \manlink{sensi}{sensi}  
\end{manseealso}
