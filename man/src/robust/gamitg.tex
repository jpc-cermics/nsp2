% -*- mode: latex -*-
\mansection{gamitg}
\begin{mandesc}
  \short{gamitg}{H-infinity gamma iterations} \\ % 
\end{mandesc}
%\index{gamitg}\label{gamitg}
%-- Calling sequence section
\begin{calling_sequence}
\begin{verbatim}
  [gopt]=gamitg(G,r,prec [,options]);  
\end{verbatim}
\end{calling_sequence}
%-- Parameters
\begin{parameters}
  \begin{varlist}
    \vname{G}: \verb!syslin! list (plant realization )
    \vname{r}: 1x2 row vector (dimension of \verb!G22!)
    \vname{prec}: desired relative accuracy on the norm
    \vname{option}: string \verb!'t'!
    \vname{gopt}: real scalar, optimal H-infinity gain
  \end{varlist}
\end{parameters}
\begin{mandescription}
  \verb!gopt=gamitg(G,r,prec [,options])!
  returns the H-infinity optimal gain \verb!gopt!.\verb!G! contains the state-space matrices \verb![A,B,C,D]! of
  the plant with the usual partitions:
\begin{verbatim}
  B = ( B1 , B2 ) ,    C = ( C1 ) ,    D = ( D11  D12)
                           ( C2 )          ( D21  D22)
\end{verbatim}
  These partitions are implicitly given in \verb!r!: \verb!r(1)!
  and \verb!r(2)! are the dimensions of \verb!D22! (rows x columns)
  With \verb!option='t'!, \verb!gamitg! traces each bisection step, i.e., 
  displays the lower and upper bounds and the current test point.
\end{mandescription}
% -- see also
\begin{manseealso}
  \manlink{ccontrg}{ccontrg} \manlink{h\_inf}{h-inf}  
\end{manseealso}
% -- Author
\begin{authors}
  P. Gahinet
\end{authors}
