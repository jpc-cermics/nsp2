% -*- mode: latex -*-
\mansection{hinf}
\begin{mandesc}
  \short{hinf}{H-infinity design of continuous-time systems} \\ % 
\end{mandesc}
%\index{hinf}\label{hinf}
%-- Calling sequence section
\begin{calling_sequence}
\begin{verbatim}
  [AK,BK,CK,DK,(RCOND)] = hinf(A,B,C,D,ncon,nmeas,gamma)  
\end{verbatim}
\end{calling_sequence}
%-- Parameters
\begin{parameters}
  \begin{varlist}
    \vname{A}: the n-by-n system state matrix A.
    \vname{B}: the n-by-m system input matrix B.
    \vname{C}: the p-by-n system output matrix C.
    \vname{D}: the p-by-m system matrix D.
    \vname{ncon}: the number of control inputs. m $>$= ncon $>$= 0, p-nmeas $>$= ncon.
    \vname{nmeas}: the number of measurements. p $>$= nmeas $>$= 0, m-ncon $>$= nmeas.
    \vname{gamma}: the parameter gamma used in \verb!H_infinity! design. 
    It is assumed that gamma is sufficiently large so that the controller is admissible. gamma $>$= 0.
    \vname{AK}: the n-by-n controller state matrix AK.
    \vname{BK}: the n-by-nmeas controller input matrix BK.
    \vname{CK}: the ncon-by-n controller output matrix CK.
    \vname{DK}: the ncon-by-nmeas controller matrix DK.
    \vname{RCOND}: a vector containing estimates of the reciprocal condition numbers of the matrices which are to be inverted and estimates of the reciprocal condition numbers of the Riccati equations which have to be solved during the computation of the controller. (See the description of the algorithm in [1].)
    \begin{varlist}
      \vname{RCOND} (1) contains the reciprocal condition number of the  control transformation matrix TU,
      \vname{RCOND} (2) contains the reciprocal condition number of the  measurement transformation matrix TY,
      \vname{RCOND} (3) contains an estimate of the reciprocal condition number of the X-Riccati equation,
      \vname{RCOND} (4) contains an estimate of the reciprocal condition number of the Y-Riccati equation.
    \end{varlist}
  \end{varlist}
\end{parameters}
\begin{mandescription}
  \verb![AK,BK,CK,DK,(RCOND)] = hinf(A,B,C,D,ncon,nmeas,gamma)!
  To compute the matrices of an H-infinity (sub)optimal n-state
  controller
\begin{verbatim}
      | AK | BK |
  K = |----|----|,
      | CK | DK |
\end{verbatim}
for the continuous-time system
\begin{verbatim}
      | A  | B1  B2  |   | A | B |
  P = |----|---------| = |---|---|,
      | C1 | D11 D12 |   | C | D | 
      | C2 | D21 D22 |
\end{verbatim}
and for a given value of gamma, where B2 has column size of the
number of control inputs (ncon) and C2 has row size of the number
of measurements (nmeas) being provided to the controller.
\end{mandescription}
%-- section-References
\paragraph{References}
[1] P.Hr. Petkov, D.W. Gu and M.M. Konstantinov. Fortran 77 routines for Hinf and H2 design of continuous-time linear control systems.     Report98-14, Department of Engineering, Leicester University,     August 1998.
%--example 
\begin{examples}
  \begin{mintednsp}{nsp}
    //example from Niconet report SLWN1999-12
    //Hinf
    A=[-1  0  4  5 -3 -2
      -2  4 -7 -2  0  3
      -6  9 -5  0  2 -1
      -8  4  7 -1 -3  0
      2  5  8 -9  1 -4
      3 -5  8  0  2 -6];
    B=[-3 -4 -2  1  0
      2  0  1 -5  2
      -5 -7  0  7 -2
      4 -6  1  1 -2
      -3  9 -8  0  5
      1 -2  3 -6 -2];
    C=[ 1 -1  2 -4  0 -3
      -3  0  5 -1  1  1
      -7  5  0 -8  2 -2
      9 -3  4  0  3  7
      0  1 -2  1 -6 -2];
    D=[ 1 -2 -3  0  0
      0  4  0  1  0
      5 -3 -4  0  1
      0  1  0  1 -3
      0  0  1  7  1];
    Gamma=10.18425636157899;
    [AK,BK,CK,DK] = hinf(A,B,C,D,2,2,Gamma)
  \end{mintednsp}
\end{examples}
%-- see also
\begin{manseealso}
  \manlink{dhinf}{dhinf}  
\end{manseealso}
