% -*- mode: latex -*-
%% Scilab ( http://www.scilab.org/ ) - This file is part of Scilab
%% Copyright (C) 1987-2016 -  (INRIA)
%%
%% This program is free software; you can redistribute it and/or modify
%% it under the terms of the GNU General Public License as published by
%% the Free Software Foundation; either version 2 of the License, or
%% (at your option) any later version.
%%
%% This program is distributed in the hope that it will be useful,
%% but WITHOUT ANY WARRANTY; without even the implied warranty of
%% MERCHANTABILITY or FITNESS FOR A PARTICULAR PURPOSE.  See the
%% GNU General Public License for more details.
%%
%% You should have received a copy of the GNU General Public License
%% along with this program; if not, write to the Free Software
%% Foundation, Inc., 59 Temple Place, Suite 330, Boston, MA  02111-1307  USA
%%

\mansection{diophant}
\begin{mandesc}
  \short{diophant}{diophantine (Bezout) equation} \\ % 
\end{mandesc}
%\index{diophant}\label{diophant}
%-- Calling sequence section
\begin{calling_sequence}
\begin{verbatim}
  [x,err]=diophant(p1p2,b)  
\end{verbatim}
\end{calling_sequence}
%-- Parameters
\begin{parameters}
  \begin{varlist}
    \vname{p1p2}: polynomial vector \verb!p1p2 = [p1 p2]!
    \vname{b}: polynomial
    \vname{x}: polynomial vector [x1;x2]
  \end{varlist}
\end{parameters}
\begin{mandescription}
  \verb! diophant! solves the bezout equation:\verb!p1*x1+p2*x2=b! with  \verb!p1p2! a polynomial vector.
  If the equation is not solvable
  else \verb!err=0!
\end{mandescription}
%--example 
\begin{examples}
  \begin{Verbatim}
    s=poly(0,'s');p1=(s+3)^2;p2=(1+s);
    x1=s;x2=(2+s);
    [x,err]=diophant([p1,p2],p1*x1+p2*x2);
    p1*x1+p2*x2-p1*x(1)-p2*x(2)
  \end{Verbatim}
\end{examples}
