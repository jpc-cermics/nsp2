% -*- mode: latex -*-
%% Scilab ( http://www.scilab.org/ ) - This file is part of Scilab
%% Copyright (C) 1987-2016 -  (INRIA)
%%
%% This program is free software; you can redistribute it and/or modify
%% it under the terms of the GNU General Public License as published by
%% the Free Software Foundation; either version 2 of the License, or
%% (at your option) any later version.
%%
%% This program is distributed in the hope that it will be useful,
%% but WITHOUT ANY WARRANTY; without even the implied warranty of
%% MERCHANTABILITY or FITNESS FOR A PARTICULAR PURPOSE.  See the
%% GNU General Public License for more details.
%%
%% You should have received a copy of the GNU General Public License
%% along with this program; if not, write to the Free Software
%% Foundation, Inc., 59 Temple Place, Suite 330, Boston, MA  02111-1307  USA
%%

\mansection{hrmt}
\begin{mandesc}
  \short{hrmt}{gcd of polynomials} \\ % 
\end{mandesc}
%\index{hrmt}\label{hrmt}
%-- Calling sequence section
\begin{calling_sequence}
\begin{verbatim}
  [pg,U]=hrmt(v)  
\end{verbatim}
\end{calling_sequence}
%-- Parameters
\begin{parameters}
  \begin{varlist}
    \vname{v}: row of polynomials i.e. \verb!1!x\verb!k! polynomial matrix
    \vname{pg}: polynomial
    \vname{U}: unimodular matrix polynomial
  \end{varlist}
\end{parameters}
\begin{mandescription}
  \verb![pg,U]=hrmt(v)! returns a unimodular matrix \verb!U! and  \verb!pg! = gcd of row of 
  polynomials \verb!v! such that \verb!v*U = [pg,0]!.
\end{mandescription}
%--example 
\begin{examples}
  \begin{Verbatim}
    x=poly(0,'x');
    v=[x*(x+1),x^2*(x+1),(x-2)*(x+1),(3*x^2+2)*(x+1)];
    [pg,U]=hrmt(v);U=clean(U)
    det(U)
  \end{Verbatim}
\end{examples}
%-- see also
\begin{manseealso}
  \manlink{gcd}{gcd} \manlink{htrianr}{htrianr}  
\end{manseealso}
%-- Author
\begin{authors}
  S. Steer INRIA
\end{authors}
