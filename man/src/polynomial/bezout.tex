% -*- mode: latex -*-
%% Scilab ( http://www.scilab.org/ ) - This file is part of Scilab
%% Copyright (C) 1987-2016 - (INRIA)
%%
%% This program is free software; you can redistribute it and/or modify
%% it under the terms of the GNU General Public License as published by
%% the Free Software Foundation; either version 2 of the License, or
%% (at your option) any later version.
%%
%% This program is distributed in the hope that it will be useful,
%% but WITHOUT ANY WARRANTY; without even the implied warranty of
%% MERCHANTABILITY or FITNESS FOR A PARTICULAR PURPOSE.  See the
%% GNU General Public License for more details.
%%
%% You should have received a copy of the GNU General Public License
%% along with this program; if not, write to the Free Software
%% Foundation, Inc., 59 Temple Place, Suite 330, Boston, MA  02111-1307  USA
%%

\mansection{bezout}
\begin{mandesc}
  \short{bezout}{Bezout equation for polynomials or integers} \\ % 
\end{mandesc}
%\index{bezout}\label{bezout}
%-- Calling sequence section
\begin{calling_sequence}
\begin{verbatim}
  [thegcd,U]=bezout(p1,p2)   
\end{verbatim}
\end{calling_sequence}
%-- Parameters
\begin{parameters}
  \begin{varlist}
    \vname{p1, p2}: two real polynomials or two integer scalars (type equal to 8)
  \end{varlist}
\end{parameters}
\begin{mandescription}
  \verb![thegcd,U]=bezout(p1,p2)! computes GCD \verb!thegcd! of \verb!p1! and \verb!p2! and in addition a (2x2) 
  unimodular matrix \verb!U! such that:\verb![p1,p2]*U = [thegcd,0]!
  The lcm of \verb!p1! and \verb!p2! is given by:\verb!p1*U(1,2)! (or \verb!-p2*U(2,2)!)
\end{mandescription}
%--example 
\begin{examples}
  \begin{itemize}
  \item polynomial case 
    \begin{mintednsp}{nsp}
      x=poly(0,'x');
      p1=(x+1)*(x-3)^5;p2=(x-2)*(x-3)^3;
      [thegcd,U]=bezout(p1,p2) 
      det(U)
      clean([p1,p2]*U)
      thelcm=p1*U(1,2)
      lcm([p1,p2])
    \end{mintednsp}
  \item integer case 
    \begin{mintednsp}{nsp}
      i1=m2i(2*3^5); i2=m2i(2^3*3^2);
      [thegcd,U]=bezout(i1,i2) 
      V=m2i([2^2*3^5, 2^3*3^2,2^2*3^4*5]);
      [thegcd,U]=gcd(V)
      V*U
      lcm(V)
    \end{mintednsp}
  \end{itemize}
\end{examples}
%-- see also
\begin{manseealso}
  \manlink{poly}{poly} \manlink{roots}{roots} \manlink{simp}{simp} \manlink{clean}{clean} \manlink{lcm}{lcm}  
\end{manseealso}
%-- Author
\begin{authors}
  S. Steer INRIA
\end{authors}
