% -*- mode: latex -*-
%% Scilab ( http://www.scilab.org/ ) - This file is part of Scilab
%% Copyright (C) 1987-2016 -  (INRIA)
%%
%% This program is free software; you can redistribute it and/or modify
%% it under the terms of the GNU General Public License as published by
%% the Free Software Foundation; either version 2 of the License, or
%% (at your option) any later version.
%%
%% This program is distributed in the hope that it will be useful,
%% but WITHOUT ANY WARRANTY; without even the implied warranty of
%% MERCHANTABILITY or FITNESS FOR A PARTICULAR PURPOSE.  See the
%% GNU General Public License for more details.
%%
%% You should have received a copy of the GNU General Public License
%% along with this program; if not, write to the Free Software
%% Foundation, Inc., 59 Temple Place, Suite 330, Boston, MA  02111-1307  USA
%%

\mansection{htrianr}
\begin{mandesc}
  \short{htrianr}{triangularization of polynomial matrix} \\ % 
\end{mandesc}
%\index{htrianr}\label{htrianr}
%-- Calling sequence section
\begin{calling_sequence}
\begin{verbatim}
  [Ar,U,rk]=htrianr(A)  
\end{verbatim}
\end{calling_sequence}
%-- Parameters
\begin{parameters}
  \begin{varlist}
    \vname{A}: polynomial matrix
    \vname{Ar}: polynomial matrix
    \vname{U}: unimodular polynomial matrix
    \vname{rk}: integer, normal rank of \verb!A!
  \end{varlist}
\end{parameters}
\begin{mandescription}
  triangularization of polynomial matrix \verb!A!.\verb!A! is \verb![m,n]! ,   \verb!m $<$= n!.\verb!Ar=A*U!
  Warning: there is an elimination of "small" terms (see function code).
\end{mandescription}
%--example 
\begin{examples}
  \begin{Verbatim}
    x=poly(0,'x');
    M=[x;x^2;2+x^3]*[1,x-2,x^4];
    [Mu,U,rk]=htrianr(M)
    det(U)
    M*U(:,1:2)
  \end{Verbatim}
\end{examples}
%-- see also
\begin{manseealso}
  \manlink{hrmt}{hrmt} \manlink{colcompr}{colcompr}  
\end{manseealso}
