% -*- mode: latex -*-
%% Scilab ( http://www.scilab.org/ ) - This file is part of Scilab
%% Copyright (C) 1987-2016 -  (INRIA)
%%
%% This program is free software; you can redistribute it and/or modify
%% it under the terms of the GNU General Public License as published by
%% the Free Software Foundation; either version 2 of the License, or
%% (at your option) any later version.
%%
%% This program is distributed in the hope that it will be useful,
%% but WITHOUT ANY WARRANTY; without even the implied warranty of
%% MERCHANTABILITY or FITNESS FOR A PARTICULAR PURPOSE.  See the
%% GNU General Public License for more details.
%%
%% You should have received a copy of the GNU General Public License
%% along with this program; if not, write to the Free Software
%% Foundation, Inc., 59 Temple Place, Suite 330, Boston, MA  02111-1307  USA
%%

\mansection{invr}
\begin{mandesc}
  \short{invr}{inversion of (rational) matrix} \\ % 
\end{mandesc}
%\index{invr}\label{invr}
%-- Calling sequence section
\begin{calling_sequence}
\begin{verbatim}
  F = invr(H)  
\end{verbatim}
\end{calling_sequence}
%-- Parameters
\begin{parameters}
  \begin{varlist}
    \vname{H}: polynomial or rational matrix
    \vname{F}: polynomial or rational matrix
  \end{varlist}
\end{parameters}
\begin{mandescription}
  This function is deprecated, you can use \verb!inv! directly.
\end{mandescription}
%--example 
\begin{examples}
  \begin{mintednsp}{nsp}
    s=poly(0,'s')
    H=[s,s*s+2;1-s,1+s]; 
    inv(H)
    H=[1/s,(s+1);1/(s+2),(s+3)/s];
    inv(H)
  \end{mintednsp}
\end{examples}
%-- see also
\begin{manseealso}
  \manlink{glever}{glever} \manlink{coffg}{coffg} \manlink{inv}{inv}  
\end{manseealso}
