% -*- mode: latex -*-
%% Scilab ( http://www.scilab.org/ ) - This file is part of Scilab
%% Copyright (C) 1987-2016 -  (INRIA)
%%
%% This program is free software; you can redistribute it and/or modify
%% it under the terms of the GNU General Public License as published by
%% the Free Software Foundation; either version 2 of the License, or
%% (at your option) any later version.
%%
%% This program is distributed in the hope that it will be useful,
%% but WITHOUT ANY WARRANTY; without even the implied warranty of
%% MERCHANTABILITY or FITNESS FOR A PARTICULAR PURPOSE.  See the
%% GNU General Public License for more details.
%%
%% You should have received a copy of the GNU General Public License
%% along with this program; if not, write to the Free Software
%% Foundation, Inc., 59 Temple Place, Suite 330, Boston, MA  02111-1307  USA
%%

\mansection{gcd}
\begin{mandesc}
  \short{gcd}{gcd calculation} \\ % 
\end{mandesc}
%\index{gcd}\label{gcd}
%-- Calling sequence section
\begin{calling_sequence}
\begin{verbatim}
  [pgcd,U]=gcd(p)  
\end{verbatim}
\end{calling_sequence}
%-- Parameters
\begin{parameters}
  \begin{varlist}
    \vname{p}: polynomial row vector \verb!p=[p1,..,pn]! or integer row
    vector (type equal to 8)
  \end{varlist}
\end{parameters}
\begin{mandescription}
  computes the gcd  of components of \verb!p! and a unimodular 
  matrix (with polynomial inverse) \verb!U!, with minimal degree such that\verb!p*U=[0 ... 0 pgcd]!
\end{mandescription}
%--example 
\begin{examples}
  \begin{itemize}
  \item polynomial case
    \begin{Verbatim}
      s=poly(0,'s');
      p=[s,s*(s+1)^2,2*s^2+s^3];
      [pgcd,u]=gcd(p);
      clean(p*u)
    \end{Verbatim}
  \item integer case
    \begin{Verbatim}
      V=m2i([2^2*3^5, 2^3*3^2,2^2*3^4*5]);
      [thegcd,U]=gcd(V)
      V*U
    \end{Verbatim}
  \item scalar case
    \begin{Verbatim}
      V=[2^2*3^5, 2^3*3^2,2^2*3^4*5];
      [thegcd,U]=gcd(V)
      V*U
    \end{Verbatim}
  \end{itemize}
\end{examples}
%-- see also
\begin{manseealso}
  \manlink{bezout}{bezout} \manlink{lcm}{lcm} \manlink{hermit}{hermit}  
\end{manseealso}
