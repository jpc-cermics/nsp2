% -*- mode: latex -*-
%% Scilab ( http://www.scilab.org/ ) - This file is part of Scilab
%% Copyright (C) 1987-2016 -  (INRIA)
%%
%% This program is free software; you can redistribute it and/or modify
%% it under the terms of the GNU General Public License as published by
%% the Free Software Foundation; either version 2 of the License, or
%% (at your option) any later version.
%%
%% This program is distributed in the hope that it will be useful,
%% but WITHOUT ANY WARRANTY; without even the implied warranty of
%% MERCHANTABILITY or FITNESS FOR A PARTICULAR PURPOSE.  See the
%% GNU General Public License for more details.
%%
%% You should have received a copy of the GNU General Public License
%% along with this program; if not, write to the Free Software
%% Foundation, Inc., 59 Temple Place, Suite 330, Boston, MA  02111-1307  USA
%%

\mansection{coffg}
\begin{mandesc}
  \short{coffg}{inverse of polynomial matrix} \\ % 
\end{mandesc}
%\index{coffg}\label{coffg}
%-- Calling sequence section
\begin{calling_sequence}
\begin{verbatim}
  [Ns,d]=coffg(Fs)   
\end{verbatim}
\end{calling_sequence}
%-- Parameters
\begin{parameters}
  \begin{varlist}
    \vname{Fs}: square polynomial matrix
  \end{varlist}
\end{parameters}
\begin{mandescription}
  \verb!coffg! computes \verb!Fs^-1! where \verb!Fs! is a polynomial
  matrix by co-factors method.\verb!Fs! inverse = \verb!Ns/d!\verb!d! = common denominator; \verb!Ns! =  numerator (a polynomial matrix)
  (For large matrices,be patient...results are generally reliable)
\end{mandescription}
%--example 
\begin{examples}
  \begin{Verbatim}
    s=poly(0,'s')
    a=[ s, s^2+1; s  s^2-1];
    [a1,d]=coffg(a);
    (a1/d)-inv(a)
  \end{Verbatim}
\end{examples}
%-- see also
\begin{manseealso}
  \manlink{determ}{determ} \manlink{detr}{detr} \manlink{invr}{invr} \manlink{penlaur}{penlaur} \manlink{glever}{glever}  
\end{manseealso}
%-- Author
\begin{authors}
  F. D.; ;   
\end{authors}
