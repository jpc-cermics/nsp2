% -*- mode: latex -*-
%% Scilab ( http://www.scilab.org/ ) - This file is part of Scilab
%% Copyright (C) 1987-2016 -  (INRIA)
%%
%% This program is free software; you can redistribute it and/or modify
%% it under the terms of the GNU General Public License as published by
%% the Free Software Foundation; either version 2 of the License, or
%% (at your option) any later version.
%%
%% This program is distributed in the hope that it will be useful,
%% but WITHOUT ANY WARRANTY; without even the implied warranty of
%% MERCHANTABILITY or FITNESS FOR A PARTICULAR PURPOSE.  See the
%% GNU General Public License for more details.
%%
%% You should have received a copy of the GNU General Public License
%% along with this program; if not, write to the Free Software
%% Foundation, Inc., 59 Temple Place, Suite 330, Boston, MA  02111-1307  USA
%%

\mansection{lcmdiag}
\begin{mandesc}
  \short{lcmdiag}{least common multiple diagonal factorization} \\ % 
\end{mandesc}
%\index{lcmdiag}\label{lcmdiag}
%-- Calling sequence section
\begin{calling_sequence}
\begin{verbatim}
  [N,D]=lcmdiag(H)  
  [N,D]=lcmdiag(H,flag)  
\end{verbatim}
\end{calling_sequence}
%-- Parameters
\begin{parameters}
  \begin{varlist}
    \vname{H}: rational matrix
    \vname{N}: polynomial matrix
    \vname{D}: diagonal polynomial matrix
    \vname{flag}: character string: \verb!'row'! or \verb!'col'! (default)
  \end{varlist}
\end{parameters}
\begin{mandescription}
  \verb![N,D]=lcmdiag(H,'row')! computes a factorization \verb!D*H=N!,
  i.e.  \verb!H=D^(-1)*N!  where D is a diagonal matrix with D(k,k)=lcm of 
  kth row of H('den').\verb![N,D]=lcmdiag(H)! or \verb![N,D]=lcmdiag(H,'col)! returns
  \verb!H=N*D^(-1)! with diagonal D and D(k,k)=lcm of kth col of H('den')
\end{mandescription}
%--example 
\begin{examples}
  \begin{mintednsp}{nsp}
    s=poly(0,'s');
    H=[1/s,(s+2)/s/(s+1)^2;1/(s^2*(s+2)),2/(s+2)];
    [N,D]=lcmdiag(H);
    N/D-H
  \end{mintednsp}
\end{examples}
%-- see also
\begin{manseealso}
  \manlink{lcm}{lcm} \manlink{gcd}{gcd} \manlink{bezout}{bezout}  
\end{manseealso}
