% -*- mode: latex -*-
\mansection{horner}
\begin{mandesc}
  \short{hornerm}{substituting matrix in polynomial indeterminate}
\end{mandesc}
%-- Calling sequence section
\begin{calling_sequence}
\begin{verbatim}
  R=hornerm(P,A)
\end{verbatim}
\end{calling_sequence}
%-- Parameters
\begin{parameters}
  \begin{varlist}
    \vname{P}: polynomial matrix
    \vname{A}: a square numerical matrix
    \vname{R}: a cell 
  \end{varlist}
\end{parameters}
\begin{mandescription}
  Evaluates the polynomial matrix \verb!P! at values given by the 
  matrix \verb!A! using the Horner algorithm. The returned values 
  are stored in a cell with same dimensions as \verb!P!.
  The cell element \verb!R{i,j}! is the numerical matrix obtained 
  by replacing the \verb!(i,j)! polynomial indeterminate variable by
  the numerical variable \verb!A!
\end{mandescription}
%--example 
\begin{examples}
  \begin{mintednsp}{nsp}
    p=m2p(1:3);
    A=testmatrix('magic',3);
    R=hornerm(p,A); 
    R{1}.equal[A^0 + 2*A^1 + 3*A^2]
  \end{mintednsp}
\end{examples}
%-- see also
%\begin{manseealso}
%\end{manseealso}
