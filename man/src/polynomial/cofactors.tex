% -*- mode: latex -*-
\mansection{cofactors}
\begin{mandesc}
  \short{cofactors}{gcd and cofactors of polynomials}\\ % @mandesc@
\end{mandesc}
%-- Calling sequence section
\begin{calling_sequence}
\begin{verbatim}
 [g,p,q]=cofactors(u,v,td, iter=%t|%f,ftol=1.e-14)   
\end{verbatim}
\end{calling_sequence}
%-- Parameters
\begin{parameters}
  \begin{varlist}
    \vname{u,v}: polynomials. 
    \vname{td}: an integer. 
    \vname{g,p,q}: polynomials.
  \end{varlist}
\end{parameters}
\begin{mandescription}
  This function is used by the function \verb+gcd+. 
  Given an integer \verb+td+ as a tentative gcd degree, the function \verb+cofactors+ 
  computes the two cofactors \verb+p+ and \verb+q+ such that \verb+q*u - p*v=0+ and a 
  polynomial \verb+g+ of \verb+td+ degree (using \verb+epdiv_lsq+) such that:
\begin{verbatim}
     p*g - u = 0 
     q*g - v = 0
\end{verbatim} 
  If \verb+iter+ is true, iterative refinement is performed to minimize the euclidian 
  norm of \verb+(p*g - u,q*g - v)+ using the function \verb+fsolve_lsq+ \verb+ftol+ 
  being transmited to \verb+fsolve_lsq+. The 
  companion function which performs iterative refinement is called 
  \verb+cofactors_iter+. 
\end{mandescription}
%--example 
\begin{examples}
  \begin{Verbatim}
    x=poly(0);
    u=(1+x+x^3+x^4)*(1+6*x);
    v=(1+x+x^3+x^4)*(1+7*x);
    k=4; // tentative degree 
    [g,p,q]=cofactors(u,v,k,iter=%f)
    [norm( g*p -u ), norm(g*q -v )]
    [g,p,q]=cofactors(u,v,k)
    [norm( g*p -u ), norm(g*q -v )]
  \end{Verbatim}
\end{examples}
%-- see also
\begin{manseealso}
  \manlink{Pmat}{Pmat}, \manlink{epdiv\_lsq}{epdiv_lsq}
\end{manseealso}
%-- Author
\begin{authors}
  J.Ph Chancelier. 
\end{authors}

