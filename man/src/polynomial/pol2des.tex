% -*- mode: latex -*-
%% Scilab ( http://www.scilab.org/ ) - This file is part of Scilab
%% Copyright (C) 1987-2016 -  (INRIA)
%%
%% This program is free software; you can redistribute it and/or modify
%% it under the terms of the GNU General Public License as published by
%% the Free Software Foundation; either version 2 of the License, or
%% (at your option) any later version.
%%
%% This program is distributed in the hope that it will be useful,
%% but WITHOUT ANY WARRANTY; without even the implied warranty of
%% MERCHANTABILITY or FITNESS FOR A PARTICULAR PURPOSE.  See the
%% GNU General Public License for more details.
%%
%% You should have received a copy of the GNU General Public License
%% along with this program; if not, write to the Free Software
%% Foundation, Inc., 59 Temple Place, Suite 330, Boston, MA  02111-1307  USA
%%

\mansection{pol2des}
\begin{mandesc}
  \short{pol2des}{polynomial matrix to descriptor form} \\
\end{mandesc}
% \index{pol2des}\label{pol2des}
% -- Calling sequence section
\begin{calling_sequence}
\begin{verbatim}
  [N,B,C]=pol2des(Ds)  
\end{verbatim}
\end{calling_sequence}
%-- Parameters
\begin{parameters}
  \begin{varlist}
    \vname{Ds}: a polynomial matrix
    \vname{N, B, C}: three real matrices
  \end{varlist}
\end{parameters}
\begin{mandescription}
  Given the polynomial matrix \verb!Ds=D_0 +D_1 s +D_2 s^2 +... +D_k s^k!,
  \verb!pol2des! returns three  matrices \verb!N, B, C!, with \verb!N! nilpotent 
  such that:\verb!Ds = C (s*N-eye())^-1 B!
\end{mandescription}
%--example 
\begin{examples}
  \begin{Verbatim}
    s=poly(0,'s');
    G=[1,s;1+s^2,3*s^3];
    [N,B,C]=pol2des(G);
    G1=clean(C*inv(s*N-eye(size(N)))*B),
    G2=G1.num
  \end{Verbatim}
\end{examples}
%-- see also
\begin{manseealso}
  \manlink{ss2des}{ss2des} \manlink{tf2des}{tf2des}  
\end{manseealso}
%-- Author
\begin{authors}
  F.D.;   
\end{authors}
