% -*- mode: latex -*-
%% Scilab ( http://www.scilab.org/ ) - This file is part of Scilab
%% Copyright (C) 1987-2016 -  (INRIA)
%%
%% This program is free software; you can redistribute it and/or modify
%% it under the terms of the GNU General Public License as published by
%% the Free Software Foundation; either version 2 of the License, or
%% (at your option) any later version.
%%
%% This program is distributed in the hope that it will be useful,
%% but WITHOUT ANY WARRANTY; without even the implied warranty of
%% MERCHANTABILITY or FITNESS FOR A PARTICULAR PURPOSE.  See the
%% GNU General Public License for more details.
%%
%% You should have received a copy of the GNU General Public License
%% along with this program; if not, write to the Free Software
%% Foundation, Inc., 59 Temple Place, Suite 330, Boston, MA  02111-1307  USA
%%

\mansection{determ}
\begin{mandesc}
  \short{determ}{determinant of polynomial matrix} \\ % 
\end{mandesc}
%\index{determ}\label{determ}
%-- Calling sequence section
\begin{calling_sequence}
\begin{verbatim}
  res=determ(W [,k])
\end{verbatim}
\end{calling_sequence}
%-- Parameters
\begin{parameters}
  \begin{varlist}
    \vname{W}: real square polynomial matrix
    \vname{k}: optional integer argument giving an upper bound for the degree of the determinant of \verb!W!.
  \end{varlist}
\end{parameters}
\begin{mandescription}
  This function is deprecated and replaced by \verb!det(W,k=)!. It returns the determinant of a real polynomial matrix 
  (computation made by FFT). The integer !k! when given is an integer larger than the actual degree of the determinant
  of \verb!W!.
  The default value of \verb!k! is the smallest power of 2 which is larger
  than \verb!n*maxi(degree(W))!.
  Method: evaluate the determinant of \verb!W! for the Fourier frequencies
  and apply inverse FFT to the coefficients of the determinant.
\end{mandescription}
%--example 
\begin{examples}
  \begin{Verbatim}
    s=poly(0,'s');
    w=s*rand(10,10);
    det(w)
    det(coeff(w,1))*s^10
  \end{Verbatim}
\end{examples}
%-- see also
\begin{manseealso}
  \manlink{det}{det} \manlink{detr}{detr} \manlink{coffg}{coffg}  
\end{manseealso}
%-- Author
\begin{authors}
  F.D.  
\end{authors}
