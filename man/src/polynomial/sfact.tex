% -*- mode: latex -*-

%%
%% This program is free software; you can redistribute it and/or modify
%% it under the terms of the GNU General Public License as published by
%% the Free Software Foundation; either version 2 of the License, or
%% (at your option) any later version.
%%
%% This program is distributed in the hope that it will be useful,
%% but WITHOUT ANY WARRANTY; without even the implied warranty of
%% MERCHANTABILITY or FITNESS FOR A PARTICULAR PURPOSE.  See the
%% GNU General Public License for more details.
%%
%% You should have received a copy of the GNU General Public License
%% along with this program; if not, write to the Free Software
%% Foundation, Inc., 59 Temple Place, Suite 330, Boston, MA  02111-1307  USA
%%

%% Scilab ( http:%%                                                                                                                                                          
\mansection{sfact}
\begin{mandesc}
  \short{sfact}{discrete time spectral factorization} \\ % 
\end{mandesc}
%\index{sfact}\label{sfact}
%-- Calling sequence section
\begin{calling_sequence}
\begin{verbatim}
  F=sfact(P)  
\end{verbatim}
\end{calling_sequence}
%-- Parameters
\begin{parameters}
  \begin{varlist}
    \vname{P}: real polynomial matrix
  \end{varlist}
\end{parameters}
\begin{mandescription}
  Finds \verb!F!, a spectral factor of
  \verb!P!. \verb!P! is a polynomial matrix such that
  each root of \verb!P! has a mirror image w.r.t the unit
  circle. Problem is singular if a root is on the unit circle.\verb!sfact(P)! returns a polynomial matrix
  \verb!F(z)! which is antistable and such that\verb!P = F(z)* F(1/z) *z^n!
  For scalar polynomials a specific algorithm is implemented.
  Algorithms are adapted from Kucera's book.
\end{mandescription}
%--example 
\begin{examples}
  \begin{itemize}
  \item polynomial (\verb1x1!) example
    \begin{Verbatim}
      z=poly(0,'z');
      p=(z-1/2)*(2-z)
      w=sfact(p);
      w* (horner(w,1/z,ttmode=%t)).num
    \end{Verbatim}
  \item polynomial matrix example
    \begin{Verbatim}
      F1=[z-1/2,z+1/2,z^2+2;1,z,-z;z^3+2*z,z,1/2-z];
      P=F1*gtild(F1,'d'); 

\end{itemize}
\end{examples}
%-- see also
\begin{manseealso}
  \manlink{gtild}{gtild} \manlink{fspecg}{fspecg}  
\end{manseealso}
