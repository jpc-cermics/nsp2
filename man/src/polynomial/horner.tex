% -*- mode: latex -*-
\mansection{horner}
\begin{mandesc}
  \short{horner}{polynomial evaluation} \\ % 
\end{mandesc}
%-- Calling sequence section
\begin{calling_sequence}
\begin{verbatim}
  R=horner(P,Q, vdim=%f, ttmode=%f)
\end{verbatim}
\end{calling_sequence}
%-- Parameters
\begin{parameters}
  \begin{varlist}
    \vname{P}: polynomial matrix
    \vname{Q}: a polynomial or numerical matrix
    \vname{vdim, ttmode}: optional named boolean arguments.
    \vname{R}: a cell whose elements are  polynomial or numerical matrices 
    if \verb!ttmode! is false and a  polynomial or a numerical matrix if 
    \verb!ttmode! is true.
  \end{varlist}
\end{parameters}
\begin{mandescription}
  For a polynomial matrix \verb!P! returns \verb!P(Q)!. The evaluation 
  is performed using the Horner algorithm. When \verb!Q! is a 
  numerical matrix the Horner algorithm is a builtin function 
  and when \verb!Q! is a  polynomial matrix the computation is 
  performed with a library nsp function. 
  
  When \verb!ttmode! is false the returned value is a cell matrix 
  \begin{itemize}
    \item if \verb!vdim! is false the cell matrix has same dimension 
      as \verb!P! and the cell element \verb!(i,j)! is the polynomial
      matrix \verb!P(i,j)(Q)!. 
    \item if \verb!vdim! is true the cell matrix has same dimension 
      as \verb!Q! and the cell element \verb!(i,j)! is the polynomial
      matrix \verb!P(Q(i,j))!.
  \end{itemize}
  
  When \verb!ttmode! is true then the operation is a term to term 
  horner which implies that both matrices have same dimensions (with 
  the usual convention that \verb!1x1! matrices are promoted 
  to any dimensions).  The returned result is then a polynomial matrix. 
  
\end{mandescription}
%--example 
\begin{examples}
  \begin{itemize}
  \item \verb!ttmode=%f!
    \begin{Verbatim}
      p=m2p(1:3);
      q=m2p(1:2);
      r=testmatrix('magic',3);
      R=horner([p,q],r); // a 1x2 cell 
      R1=horner([p,q],r,vdim=%t); // a 3x3 cell
      A=ce2m(R1,indice=1); // a 3x3 matrix 
      B=ce2m(R1,indice=2); // a 3x3 matrix 
      R.equal[{A,B}]  // recovering R from R1.
    \end{Verbatim}
  \item \verb!ttmode=%t!
    \begin{Verbatim}
      p=m2p(1:3);
      q=m2p(1:2);
      r=[1,2;3,7];
      R=horner([p,q;q,p],r,ttmode=%t);
    \end{Verbatim}
  \end{itemize}
\end{examples}

%-- see also
%\begin{manseealso}
%\end{manseealso}
