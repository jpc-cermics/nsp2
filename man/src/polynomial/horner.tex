% -*- mode: latex -*-
\mansection{horner}
\begin{mandesc}
  \short{horner}{polynomial evaluation} \\ % 
\end{mandesc}
%-- Calling sequence section
\begin{calling_sequence}
\begin{verbatim}
  R=horner(P,Q, vdim=%f, ttmode=%f)
\end{verbatim}
\end{calling_sequence}
%-- Parameters
\begin{parameters}
  \begin{varlist}
    \vname{P}: polynomial matrix
    \vname{Q}: a polynomial or numerical matrix
    \vname{vdim, ttmode}: optional named boolean arguments.
    \vname{R}: a cell if \verb!ttmode! is false and a 
    polynomial or a numerical matrix if \verb!ttmode! is true.
  \end{varlist}
\end{parameters}
\begin{mandescription}
  Evaluates the polynomial matrix \verb!P! at values given by \verb!Q! 
  using the Horner algorithm. When \verb!Q! is a 
  numerical matrix the Horner algorithm is a builtin function 
  and when \verb!Q! is a  polynomial matrix the computation is 
  performed with a library nsp function. 
  
  \begin{itemize} 
  \item When \verb!ttmode! is true then the performed operation is a term to term 
    horner which implies that both matrices must have same dimensions (with 
    the usual convention that \verb!1x1! matrices are promoted 
    to any dimensions). The returned result, \verb!R! is a polynomial matrix if 
    \verb!Q! is a polynomial matrix or a numerical matrix if \verb!Q! is numeric 
    which is given by \verb!R(i,j) = horner(P(i,j),Q(i,j)\verb!
  \item When \verb!ttmode! is false the returned value, \verb!R! is a cell matrix 
    \begin{itemize}
    \item if \verb!vdim! is false the cell matrix has same dimension 
      as \verb!P! and the cell element \verb!(i,j)! is the polynomial or 
      numerical matrix \verb!P(i,j)(Q)!. 
    \item if \verb!vdim! is true the cell matrix has same dimension 
      as \verb!Q! and the cell element \verb!(i,j)! is the polynomial
      or numerical matrix \verb!P(Q(i,j))!.
    \end{itemize}
  \end{itemize}
\end{mandescription}
%--example 
\begin{examples}
  \begin{itemize}
  \item \verb!ttmode=%f!
    \begin{mintednsp}{nsp}
      p=m2p(1:3);
      q=m2p(1:2);
      r=testmatrix('magic',3);
      R=horner([p,q],r); // a 1x2 cell 
      R1=horner([p,q],r,vdim=%t); // a 3x3 cell
      A=ce2m(R1,indice=1); // a 3x3 matrix 
      B=ce2m(R1,indice=2); // a 3x3 matrix 
      R.equal[{A,B}]  // recovering R from R1.
    \end{mintednsp}
  \item \verb!ttmode=%t!
    \begin{mintednsp}{nsp}
      P = ce2p({1:2,1;2,[1,-1]})
      Q1 = ce2p({1,[1,4];[0,1],[1,-1]})
      R1=horner(P,Q1,ttmode=%t);
      
      Q2 =[1,2;3,7];
      R2=horner(P,Q2,ttmode=%t);
    \end{mintednsp}
  \end{itemize}
\end{examples}

%-- see also
%\begin{manseealso}
%\end{manseealso}
