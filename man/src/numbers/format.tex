% -*- mode: latex -*-

\mansection{format}

\begin{mandesc}
  \short{format}{number printing and display format}
\end{mandesc}


%-- Calling sequence section
\begin{calling_sequence}
   \begin{verbatim}
     format()
     format(str)
     format(field)
     format(field, prec)
     format(field, prec, eflag)
   \end{verbatim}
\end{calling_sequence}

%-- Parameters

\begin{parameters}

  \begin{varlist}
   \vname{str} : a string among "long", "medium", "short", "long e", "medium e", "short e"
   \vname{field} : integer max size of characters to print the number
   \vname{prec} : min number of significative digits (default is 4)
   \vname{eflag} : boolean scalar, should be \%t to force number
   display with the form mantissa and exponent part (default is \%f).
  \end{varlist}

  \end{parameters}

  \begin{mandescription}
    Sets the current printing format. Default is "short" and
    correspond to prec=4, field=11, eflag=\%f (see the following
    table). Using  "long", "medium", "short" or the form \verb+format(field, prec)+
    or \verb+format(field, prec, %f)+ leads to a ``variable format'' 
    which tries to use a fixed point notation when possible (that
    is if the fixed point notation with at least $prec$ significant digits
    inside the max $field$ chars is possible). Otherwise the number is
    written with a mantissa plus an exponent part.

    On the other hand using  "long e", "medium e", "short e" or
     \verb+format(field, prec, %t)+  displays numbers always
    with a mantissa and an exponent part.

    \verb+format()+ lets recover the default format.

  \begin{tabular}{|c|c|c|c|}
     string  & field & prec & eflag \\
  \hline
   "short"   & 11    & 4    &  \%f \\
   "short e" & 11    & 4    &  \%t \\
   "medium"  & 18    & 11   &  \%f \\
   "medium e"& 18    & 11   &  \%t \\
   "long"    & 23    & 16   &  \%f \\
   "long e"  & 23    & 16   &  \%t \\
   \end{tabular}

  \end{mandescription}

  %--example 

\begin{examples}

\begin{program}\HCode{\%pi\Hnewline
format("medium")\Hnewline
\%pi\Hnewline
format("long")\Hnewline
\%pi\Hnewline
format()\Hnewline
// default format should output the next number as is\Hnewline
// (because 11 char could be used to print it in fixed\Hnewline
//  point notation together with 4 significant digits)\Hnewline
x = 0.000001234\Hnewline
// but should use notation with exponent on the following number\Hnewline
x = 0.0000001234}
\end{program}

\end{examples}

  %-- see also

%\begin{manseealso}
% \manlink{write}{write} \manlink{disp}{disp} \manlink{print}{print}  
%\end{manseealso}

