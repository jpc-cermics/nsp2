% -*- mode: latex -*-

\mansection{format}

\begin{mandesc}
  \short{format}{number printing and display format}
\end{mandesc}


%-- Calling sequence section
\begin{calling_sequence}
   \begin{verbatim}
     format()
     format(str)
     format(field, prec)
     format(field, prec, eflag)
   \end{verbatim}
\end{calling_sequence}

%-- Parameters

\begin{parameters}

  \begin{varlist}
   \vname{str} : a string among "long", "medium", "short", "long e", "medium e", "short e"
   \vname{field} : integer max size of characters to print the number
   \vname{prec} : min number of significative digits
   \vname{eflag} : boolean scalar.
  \end{varlist}

  \end{parameters}

  \begin{mandescription}
    Sets the current printing format. Default is "short" and
    correspond to prec=4, field=11, eflag=\%f. Using
    "long", "medium", "short" or the form \verb+format(field, prec)+
    or \verb+format(field, prec, %f)+ leads to a ``variable format'' 
    which tries to use a fixed point notation when possible (that
    is when the number magnitude is not too large or too small) and 
    otherwise writes the number with a mantissa plus an exponent
    part.

    On the other hand using  "long e", "medium e", "short e" or
     \verb+format(field, prec, %t)+  displays always numbers
    with mantissa plus exponent part.

    \verb+format()+ lets recover the default format.

  \begin{tabular}{|c|c|c|}
     string & field & prec \\
  \hline
   "short"   & 11 & 4 \\
   "medium"  & 18 & 11 \\
   "long"    & 23 & 16 \\
   \end{tabular}


  \end{mandescription}

  %--example 

\begin{examples}

\begin{program}\HCode{\%pi\Hnewline
format("medium")\Hnewline
\%pi\Hnewline
format("long")\Hnewline
\%pi\Hnewline
format()\Hnewline
// default format should output the next number as is\Hnewline
// (because 11 char could be used to print it in fixed\Hnewline
//  point notation together with 4 significant digits)\Hnewline
x = 0.000001234\Hnewline
// but should use notation with exponent on the following number\Hnewline
x = 0.0000001234}
\end{program}

\end{examples}

  %-- see also

\begin{manseealso}
% \manlink{write}{write} \manlink{disp}{disp} \manlink{print}{print}  
\end{manseealso}

