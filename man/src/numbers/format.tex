% -*- mode: latex -*-

\mansection{format}

\begin{mandesc}
  \short{format}{number printing and display format for floating point numbers}
\end{mandesc}


%-- Calling sequence section
\begin{calling_sequence}
   \begin{verbatim}
     // first form: set the current format
     format(str)
     format(field, prec, eflag)

     // second form: get the current format
     [field, prec, eflag] = format("get")
   \end{verbatim}
\end{calling_sequence}

%-- Parameters

\begin{parameters}

  \begin{varlist}
   \vname{str}: a string among "long", "medium", "short", "long e", "medium e", "short e", "default"
   \vname{field}: integer max size of characters to print the number
   \vname{prec}: min number of significative digits (default is 4)
   \vname{eflag}: boolean scalar, should be \%t to force number display always with an exponent part (default is \%f).
  \end{varlist}

  \end{parameters}

  \begin{mandescription}
  \begin{description}
  \item[form 1]  The first form lets to set the current printing format for floating point
    numbers. Default is "short" and correspond to prec=4, field=11, eflag=\%f (see the following
    table). Using  "long", "medium", "short" or the form \verb+format(field, prec)+
    or \verb+format(field, prec, %f)+ leads to a ``variable format'' 
    which tries to use a fixed point notation when possible (that
    is if the fixed point notation with at least $prec$ significant digits
    inside the max $field$ chars is possible). Otherwise the number is
    written with a mantissa plus an exponent part.

    On the other hand using  "long e", "medium e", "short e" or
     \verb+format(field, prec, %t)+  displays numbers always
    with a mantissa and an exponent part.

    \verb+format("default")+ lets recover the default format (which is "short")

  \begin{tabular}{|c|c|c|c|}
     string  & field & prec & eflag \\
  \hline
   "short"   & 11    & 4    &  \%f \\
   "short e" & 11    & 4    &  \%t \\
   "medium"  & 18    & 11   &  \%f \\
   "medium e"& 18    & 11   &  \%t \\
   "long"    & 23    & 16   &  \%f \\
   "long e"  & 23    & 16   &  \%t \\
   \end{tabular}

  \item[form 2] lets to retrieve the 3 parameters field, prec, eflag which define the display format.
      This is useful if you want to modify locally the format :
\begin{mintednsp}{nsp}
// save current format
[save_field,save_prec,save_eflag] = format("get");
// set a different format and do some operations
.....
// reset initial format
format(save_field,save_prec,save_eflag)
\end{mintednsp}
      
  \end{description} 
    
  \end{mandescription}

  %--example 

\begin{examples}

\begin{mintednsp}{nsp}
%pi
format("medium")
%pi
format("long")
%pi
format("default")
// default format should output the next number as is
// (because 11 char could be used to print it in fixed
//  point notation together with 4 significant digits)
x = 0.000001234
// but should use notation with exponent on the following number
x = 0.0000001234

// print 0.2. (0.2 could not exactly be coded as a double float number)
format("default")
0.2
format("long")
0.2
// now discover that fl(0.2) is not exactly 0.2
// but  0.200000000000000011102230246251565404236316680908203125
format(57,55,%f)
0.2
\end{mintednsp}
\end{examples}

  %-- see also

%\begin{manseealso}
% \manlink{write}{write} \manlink{disp}{disp} \manlink{print}{print}  
%\end{manseealso}

