% -*- mode: latex -*-


\mansection{isreal}

\begin{mandesc}
\short{isreal}{check if a variable has real or complex entries}
\end{mandesc}


%-- Calling sequence section
\begin{calling_sequence}
    \begin{verbatim}
       b=isreal(x)  
       b=isreal(x,strict)  
    \end{verbatim}
\end{calling_sequence}

%-- Parameters
\begin{parameters}
  \begin{varlist}
   \vname{x} : numerical scalar vector or matrix (full or sparse).
   \vname{strict} : boolean scalar (default is \%f).
   \vname{b} : boolean scalar
  \end{varlist}
\end{parameters}

\begin{mandescription}
In nsp numerical matrices are formed with real or complex coefficients
(in fact floating point numbers see
\manlink{number\_properties}{number_properties}). This function
lets distinguish between these two cases:
\begin{itemize}
\item \verb!isreal(x)! (or \verb!isreal(x,%f)!) returns true when 
\verb!x! is stored as a real variable or when it is stored 
as a complex variable with all imaginary parts equal to zero. 
\item \verb!isreal(x,\%t)! returns true when \verb!x! is stored as a real
variable and otherwise returns false.
\end{itemize}

\end{mandescription}

%--example 
\begin{examples}

\begin{program}\HCode{isreal([1,2])\Hnewline
isreal(1+0*\%i)\Hnewline
isreal(1+0*\%i,\%t)\Hnewline
A = sprand(3,3,0.4)\Hnewline
isreal(A)\Hnewline
B = A + \%i*(0*A)\Hnewline
isreal(B)\Hnewline
isreal(B,\%t)}
\end{program}

\end{examples}

