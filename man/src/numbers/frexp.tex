% -*- mode: latex -*-

\mansection{frexp}

\begin{mandesc}
  \short{frexp}{dissect floating-point numbers into base 2 exponent and mantissa}
\end{mandesc}


%-- Calling sequence section
\begin{calling_sequence}
    \begin{verbatim}
       [m,e]=frexp(x)  
       [m,e]=frexp(x,output=str)  
    \end{verbatim}
\end{calling_sequence}


%-- Parameters
\begin{parameters}
  \begin{varlist}
   \vname{x}: real scalar, vector or matrix
   \vname{output=str}: a string equal to \verb+"usual"+ (default value) or  \verb+"int"+
   \vname{m}: array of real values, in the range $0.5 \le |m| < 1$ when \verb+output="usual"+
              for finite real. 
   \vname{e}: array of integers that satisfy the equation: $x = m
   \times 2^e$ (\verb+ x = m.*2.^e+ is nsp language).
  \end{varlist}
\end{parameters}

\begin{mandescription}
    This function corresponds to the ANSI C function frexp().
  Any zeros in $x$ produce $m=0$ and $e=0$. When \verb+output="int"+ each
  mantissa part is multiplied by the smallest power $p$ of 2 such that it becomes
  an integer (the new exponent being the old one minus $p$).
\end{mandescription}


%--example 
\begin{examples}

\begin{mintednsp}{nsp}
[m,e]=frexp(0)

[m,e]=frexp(0.5*2^(-3))

tiny = number_properties("tiny")
[m,e]=frexp(tiny)

tiniest = number_properties("tiniest")
[m,e]=frexp(tiniest)
[m,e]=frexp(tiniest,output="int")

[m,e]=frexp(%inf)
[m,e]=frexp(%nan)

// rmq : 0.2 is not a floating point number
format(24,17)
[m,e]=frexp(0.2)
[m,e]=frexp(0.2,output="int")
\end{mintednsp}

\end{examples}

%-- see also

\begin{manseealso}
\manlink{number\_properties}{number_properties} %, \manlink{log2}{log2}  
\end{manseealso}

