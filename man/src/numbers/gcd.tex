% -*- mode: latex -*-
\mansection{gcd}
\begin{mandesc}
  \short{gcd}{gcd of integers or polynomials}\\
  \short{lcm}{lcm of integers or polynomials}\\
  \short{euclide}{extended euclidian algorithm}\\
\end{mandesc}
\begin{calling_sequence}
\begin{verbatim}
 [g]=gcd(p,q)  
 [g]=lcm(p,q)  
 [g,a,b]=euclide(p,q)

 [z,f1,f2,res]= gcd(u,v,delta=1.e-9)
\end{verbatim}
\end{calling_sequence}
%-- Parameters
\begin{parameters}
  \begin{varlist}
    \vname{p,q}: both polynomials or both numbers (double or int32).
    \vname{g,a,b}: polynomials or numbers (double or int32).
    \vname{u,v}: both polynomials. 
    \vname{z,f1,f2}: polynomials. 
  \end{varlist}
\end{parameters}
\begin{mandescription}
  \verb+gcd(p,q)+, \verb+lcm(p,q)+, \verb+euclide(p,q)+ respectively compute 
  the gcd, the lcm and the \verb+(g,a,b)+ integers which are solutions of the 
  Bezout's identity \verb+ap + bq = g = gcd(p, q)+.
  
  \verb+[z,f1,f2,res]= gcd(u,v,delta=1.e-9)+ computes an epsilon-gcd of 
  two polynomials \verb+u+ and \verb+v+ (epsilon given by the delta parameter). 
  \verb+f1+ and \verb+f2+ are the associated cofactors \verb+u/norm(u) = z*f1+ and 
  \verb+v/norm(v)= z*f2+ and \verb+res+ gives the euclidian norm of \verb+u/norm(u) - z*f1+ 
  and \verb+v/norm(v)- z*f2+.
\end{mandescription}
%--example 
\begin{examples}
  \begin{Verbatim}
    x= poly(0);
    p1=(1+x);  p2=(2+x);  p3=(3+x);  p4=(x);
    cp=[1,2,5,1];
    cq=[1,0,2,0];
    u= p1^cp(1)*p2^cp(2)*p3^cp(3)*p4^cp(4);
    v= p1^cq(1)*p2^cq(2)*p3^cq(3)*p4^cq(4);
    cpq = min(cq,cp);
    g = p1^cpq(1)*p2^cpq(2)*p3^cpq(3)*p4^cpq(4);
    [z,f1,f2,res] = gcd(p,q,delta=1.e-9);
    norm(u/norm(u) - z*f1)
    norm(v/norm(v)- z*f2)

    z.normalize[];
    norm(z -g) 
  \end{Verbatim}
\end{examples}
%-- see also
\begin{manseealso}
  \manlink{Pmat}{Pmat}
\end{manseealso}
\begin{authors}
  Paola Boito, Jean-Philippe Chancelier.
\end{authors}

