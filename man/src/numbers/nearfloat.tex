% -*- mode: latex -*-

\mansection{nearfloat}

\begin{mandesc}
   \short{nearfloat}{get previous or next floating-point number}
\end{mandesc}

%-- Calling sequence section
\begin{calling_sequence}
    \begin{verbatim}
      xnear = nearfloat(dir, x)  
    \end{verbatim}
\end{calling_sequence}

%-- Parameters
\begin{parameters}
  \begin{varlist}
   \vname{dir}: string ("succ" or "pred")
   \vname{x}: real scalar, vector or matrix
   \vname{xnear}: real scalar, vector or matrix
  \end{varlist}
\end{parameters}

\begin{mandescription}
    This function computes, in the element wise meaning, the corresponding neighbours 
    of the elements of \verb!x! (in the underlying floating point set,
    see \manlink{number\_properties}{number_properties}).
    The successors if \verb!dir = "succ"! and the predecessors if  \verb!dir = "pred"!
\end{mandescription}

%--example 
\begin{examples}

\paragraph{example 1} Here we show the fact that the distance
between 2 neighboring floating numbers increases by a factor
of $b$ at each positive or negative power of the base $b$.
Here $b$ is assumed to be 2.   
\begin{mintednsp}{nspxxx}
format("long e")
// 0.125 = 2^(-3)
inc_before = 0.125 - nearfloat("pred",0.125)
inc_after = nearfloat("succ",0.125) - 0.125
inc_after/inc_before
// 1 = 2^0
inc_before = 1 - nearfloat("pred",1)
inc_after = nearfloat("succ",1) - 1
inc_after/inc_before
// 8 = 2^3
inc_before = 8 - nearfloat("pred",8)
inc_after = nearfloat("succ",8) - 8
inc_after/inc_before;
\end{mintednsp}

\paragraph{example 2}
\begin{mintednsp}{nspxxx}
format("long e")
// successor of huge (should be Inf)
huge = number_properties("huge")
nearfloat("succ",huge)
// predecessor of Inf (should be huge)
nearfloat("pred",%inf) - huge
\end{mintednsp}

\end{examples}

%-- see also
\begin{manseealso}
\manlink{number\_properties}{number_properties}
\end{manseealso}

%-- Author
\begin{authors}
Bruno Pincon
\end{authors}

