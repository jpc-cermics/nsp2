% -*- mode: latex -*-

\mansection{tests for special floating-point numbers}

\begin{mandesc}
\short{isfinite}{check for "Finite" entries}\\
\short{isnan}{check for "Not a Number" entries}\\
\short{isinf}{check for "Infinite" entries}
\end{mandesc}

%-- Calling sequence section
\begin{calling_sequence}
    \begin{verbatim}
      b=isfinite(x)  
      b=isnan(x)  
      b=isinf(x)  
    \end{verbatim}
\end{calling_sequence}


%-- Parameters
\begin{parameters}
  \begin{varlist}
   \vname{x}: real or complex vector or matrix
   \vname{b}: boolean vector or matrix (of same size than x).
  \end{varlist}
\end{parameters}

\begin{mandescription}
With the IEEE 754 floating point normalisation special
``floating-point'' numbers were defined. For instance 
when the magnitude of a result is (enough) outside the
floating-point max limit (the huge number see 
\manlink{number\_properties}{number_properties}) then
the special number $-Inf$ or $Inf$ is returned. Also
the special number $NaN$ is returned for instance
as a result for $0 / 0$ or $Inf - Inf$, etc...(in fact
several digit patterns correspond to $NaN$).     

\begin{description}
\item[isfinite(x)]: 
    returns a boolean vector or matrix which contains true
    entries for corresponding finite entries (that is entries
    not $Inf$, $-Inf$ or $NaN$).
\item[isnan(x)]: 
    returns a boolean vector or matrix which contains true
    entries for corresponding $NaN$ entries.
\item[isinf(x)]: 
    returns a boolean vector or matrix which contains true
    entries for corresponding $Inf$ and $-Inf$ entries.
\end{description}

When applied on a complex vector the real and imaginary parts should
be both finite to get \%t with isfinite and only one should be $Inf$
or $NaN$ to get \%t with isinf or isnan. 
  
\end{mandescription}

%--example 

\begin{examples}

\begin{program}\HCode{x = [0, \%inf, -8e34, \%nan, 67, -\%inf]\Hnewline
isfinite(x)\Hnewline
isnan(x)\Hnewline
isinf(x)}
\end{program}

\end{examples}

%-- see also
\begin{manseealso}
\manlink{number\_properties}{number_properties}
\end{manseealso}

