% -*- mode: latex -*-

\mansection{number\_properties}
\begin{mandesc}
  \shortunder{number\_properties}{number_properties}{determine floating-point parameters}
\end{mandesc}

%-- Calling sequence section

\begin{calling_sequence}
    \begin{verbatim}
       pr = number_properties(prop)  
    \end{verbatim}
\end{calling_sequence}

%-- Parameters

\begin{parameters}
  \begin{varlist}
   \vname{prop} : string
   \vname{pr} : real or boolean scalar
  \end{varlist}
\end{parameters}

\begin{mandescription}

    This function may be used to get the characteristic
    numbers/properties of the floating point set denoted here by
    $F(b,p,emin,emax)$ (usually the 64 bits float numbers set prescribed 
    by IEEE 754). Numbers of $F$ are of the form:
    $$
     sign * m * b^e
    $$
    $e$ is the exponent and $m$ the mantissa:
    $$
     m = d_1 b^{-1} + d_2 b^{-2} + .... + d_p b^{-p}
    $$
    where $d_i$, the $p$ digits, are integers in $[0, b-1]$ and $e$ is in 
    $[emin, emax]$. The number is said \textbf{"normalised"} if $d_1
    \ne 0$. The following may be got: 
  \begin{tabular}{|r|l|}
    input parameter (prop) & output (pr) \\
  \hline
   "radix"  & $b$ \\
   "digits" & $p$ (number of digits) \\
   "huge"   & max positive float of $F$ \\
   "tiny"   & min positive normalised float of $F$ \\
   "denorm" & \%t when denormalised numbers are used \\
   "tiniest"& min positive denormalised number (*) \\ 
   "eps"    & epsilon machine (see Remarks here after) \\
   "minexp" & $emin$ \\
   "maxexp" & $emax$ \\
   \end{tabular}

(*) return the tiny number when denormalised numbers are not used.

  %-- section-Remarks

  \paragraph{Remarks}
    \begin{itemize}
    \item CAUTION: there are different definitions for the epsilon
      machine. In this function we use the one from the
      Goldberg 's paper ``What Every Computer Scientist Should Known 
      About Floating-Point Arithmetic'' which  is the max
    relative bound between a real and its floating point approximation
    (assuming that the absolute value of the real is in $[tiny,
    huge]$). In more recent works this quantity
    tends to be call \textbf{u} (probably to avoid
    confusion). Generally if normal rounding occurs then :
    $$
      eps = \frac{1}{2} b^{1-p}
    $$
    \item Currently nsp uses only the so called ``double
      floating-point'' numbers.
    \item This function uses the lapack routine dlamch to get the
      machine parameters. The names (radix, digit, huge, etc...) 
      are those recommended  by the LIA 1 standard and are different 
      from the corresponding lapack's  ones.
  \end{itemize}

\end{mandescription}
  
% --example 

\begin{examples}

\begin{program}\HCode{b = number_properties("radix")\Hnewline
eps = number_properties("eps")}
\end{program}

\end{examples}

%-- see also
\begin{manseealso}
\manlink{nearfloat}{nearfloat}, \manlink{frexp}{frexp}, \manlink{isfinite}{isfinite}, \manlink{isnan}{isnan}, \manlink{isinf}{isinf}  % \manlink{frexp}{frexp}  
\end{manseealso}

%-- Author
\begin{authors}
Bruno Pincon (dlamch 's nsp interface) 
\end{authors}

