% -*- mode: latex -*-
\mansection{rounding}
\begin{mandesc} 
  \short{ceil}{smallest integral value not less than argument} \\
  \short{fix}{integer part}\\
  \short{floor}{largest integral value not greater than argument}
  \short{int}{integer part} \\
  \short{round}{round to nearest integer, away from zero}
\end{mandesc}
\begin{calling_sequence}
\begin{verbatim}
  y=ceil(x)  
  y=fix(x)  
  y=floor(x)  
  y=int(x)  
  y=round(x)  
\end{verbatim}
\end{calling_sequence}
%-- Parameters
\begin{parameters}
  \begin{varlist}
    \vname{x,y}: numerical or polynomial matrices.
  \end{varlist}
\end{parameters}
\begin{mandescription}
  these functions return a new matrix with same type as \verb!x! filled 
  with rounded elements of \verb!x!. For complex numbers, roundind is mapped to the 
  real and complex part of the number. For polynomial matrices round is 
  mapped to the  coefficients of each polynomial.
  \begin{itemize}
  \item \verb!ceil! returns the smallest integral value that is not less than its
    argument.
  \item \verb!fix!, \verb!int! round their argument to their integer part i.e using \verb!floor! for 
    positive values and \verb!ceil! for strictly negative value.
  \item \verb!floor! returns the largest integral value that is not greater than
    its argument.
  \item \verb!round! rounds its argument to the nearest integers where halfway cases are rounded
    away from zero.
  \end{itemize}
\end{mandescription}
%% \begin{examples}
%%   \begin{Verbatim}
    
%%   \end{Verbatim}
%% \end{examples}
