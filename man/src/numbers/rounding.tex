% -*- mode: latex -*-
\mansection{rounding}
\begin{mandesc} 
  \short{ceil}{smallest integer value not less than argument (round toward $+\infty$) } \\
  \short{floor}{largest integer value not greater than argument (round toward $-\infty$)}\\
  \short{int}{or} \short{fix}{integer part (round toward $0$)}\\
  \short{round}{round to nearest integer}
\end{mandesc}
\begin{calling_sequence}
\begin{verbatim}
  y=ceil(x)  
  y=int(x)   // or  y=fix(x)
  y=floor(x)  
  y=round(x)  
\end{verbatim}
\end{calling_sequence}
%-- Parameters
\begin{parameters}
  \begin{varlist}
    \vname{x,y}: numerical or polynomial matrices.
  \end{varlist}
\end{parameters}
\begin{mandescription}
  these functions return a new matrix with same type as \verb!x! filled 
  with rounded elements (to integers) of \verb!x!. For complex numbers, roundind is mapped to the 
  real and complex part of the number. For polynomial matrices round is 
  mapped to the  coefficients of each polynomial.
  \begin{itemize}
  \item \verb!ceil! returns the smallest integer value that is not less than its
    argument, this is equivalent to rounding toward positive infinity.
  \item \verb!floor! returns the largest integer value that is not greater than
    its argument. This is equivalent to rounding toward negative infinity.
  \item \verb!int! or \verb!fix! round their argument to their integer part i.e using \verb!floor! for 
    positive values and \verb!ceil! for strictly negative value. This is equivalent to rounding toward zero.
  \item \verb!round! rounds its argument to the nearest integers where halfway cases are rounded
    away from zero.
  \end{itemize}
\end{mandescription}

\begin{examples}
  \begin{Verbatim}
x = [-3.2, -2.8, -1.5, -0.9, 0.9, 1.5, 2.8, 3.2]

// round toward infinity
ceil(x)

// round toward negative infinity
floor(x)

// round toward zero
int(x) // or fix(x)

// round to nearest integer
round(x)   
  \end{Verbatim}
\end{examples}
