% -*- mode: latex -*-

\mansection{fsolve}
\begin{mandesc}
  \short{fsolve}{non linear equations solver}
\end{mandesc}

% -- Calling sequence section
\begin{calling_sequence}
\begin{verbatim}
r = fsolve(x0, f)
[r, fr ] = fsolve(x0, f, xtol=erx, ftol=ea, args=obj, jac=Jf, maxfev=n, scale=vect, warn= bool) 
[r, fr, info, nfev, njev ] = fsolve(x0, f, xtol=erx, ftol=ea, args=obj, jac=Jf, maxfev=p, scale=vect, warn= bool) 
[fr,Jr]= fsolve_jac(x0,f);
\end{verbatim}
\end{calling_sequence}
% -- Parameters
\begin{parameters}
  \begin{varlist}
    \vname{x0}:real vector (says of length $n$), initial estimate of the solution vector
    \vname{f}:nsp function (or string for an C or fortran external)
    \vname{xtol=erx, ftol=eaf}: optional named arguments relative to the stopping test (defaults:
    \verb+xtol=sqrt(2*%eps)+, \verb+ftol=(2*%eps)^(2/3)+)
    \vname{jac=Jf}: optional named argument. Jf should be a nsp function (the Jacobian of \verb+f+)
    \vname{args=obj}: optional named argument (could be any nsp object), useful to pass additional parameters for \verb+f+ (and
    Jf if provided).
    \vname{maxfev=p}: optional named argument, maximum number of function evaluations (default is $100(n+1)$ if Jf is
    provided and $200(n+1)$ otherwise)
    \vname{scale}: vector of length $n$, scaling factors.
    \vname{r}: vector of length $n$, should be an approximated root/zero of \verb+f+ 
    \vname{fr}: vector of length $n$, the value of \verb+f+  at \verb+r+ (ideally a vector of zeros).
    \vname{info}: integer scalar, should be 1 in case of success.
    \vname{warn}: boolean. When the function is called with \verb+lhs <= 2+, then a message describing the 
    stopping condition is displayed except if \verb+warn=%f+.
    \vname{nfev, njev}: integer scalar, respectively the number of calls to \verb+f+and Jf (njev=0 when jac is not provided)
  \end{varlist}
\end{parameters}

\begin{mandescription}
This function tries to find a zero of a mathematical function $f: {\R}^n \rightarrow  {\R}^n$.
That is, find a point $r_e \in {\R}^n$ such that $f(r_e)=0$, by an iterative algorithm starting from an initial guess \verb+x0+.
The function \verb+fsolve+ uses the Powell hybrid method from the Minpack package (\verb+hybrj+, \verb+hybrd+). 
In case of success the function returns an approximation $r$ of an exact root $r_e$ satisfying the following condition:  
$$
|| r_e - r || \le xtol || r_e || \mbox{ or } \max_i | f_i(r) | \le ftol\,.
$$
In this case, the output variable \verb+info+ is set to $1$ (note that, as the exact root  $r_e$
is not known, the first test condition is obtained via an estimation \ldots). Other possible values
for \verb+info+ are:
\begin{itemize}
\item \verb+info<0+: execution aborted because of a problem in the evaluation of the given function \verb+f+ 
  (for example, error in the code of \verb+f+ or missing additional parameters requested by \verb+f+).
\item \verb+info=0+: improper input parameters.
\item \verb+info=2+: number of calls to \verb+f+ has reached or exceeded \verb+maxfev+.
\item \verb+info=3+: \verb+xtol+ and \verb+ftol+ are too small. No further improvement in the approximate solution is possible
\item \verb+info=4+ or \verb+5+: the method is not making good progress. This error often arises when the method
  is trapped near a minimum of $||f(x)||^2$ which is not a zero.
\end{itemize}
Note that, when \verb+fsolve+ is called with only 1 or 2 outputs arguments, an error is set 
when \verb+info <= 0+, and a warning is displayed in the nsp console when \verb+info >= 2+ except 
if \verb+warn+ is set to \verb+%f+. 

\itemdesc{form of the function {\tt f}}
The function \verb+f+ takes one argument, a vector $x$ of length $n$ and possibly an additional
argument which will be passed using the \verb+args=obj+ named optional argument. The function
should return a vector of length $n$. It is possible to use a C or fortran function in place
of an nsp function, this will be described in a next version of this manual page.

\itemdesc{optional argument {\tt args}}
Additional parameters to the function \verb+f+ (and also to the function which computes the Jacobian 
if it is provided by the user) can be transmited using \verb+args=obj+. Where obj can be any nsp 
object (see examples). In this case \verb+f+ has the form:
\begin{quote}
{\tt function y = f(x,obj) \\
      ....}
\end{quote}

\itemdesc{optional {\tt jac} parameter}
The solver needs $Jf$ the jacobian of $x \mapsto f(x)$ and this function can be given using 
\verb+jac=Jf+. When jac is not provided the solver computes an approximation using finite differences. 
The header of the nsp function which defines the Jacobian of \verb+f+ should be of the form:
\begin{quote}
{\tt function J = Jf(x) \\
      ....}\\
or:\\
{\tt function J = Jf(x,obj) \\
     ....}
\end{quote}
when {\tt obj} is provided with {\tt args=obj}. 
Note that when the Jacobian is not given it is approximated by finite differences and it is 
possible to get the result of the evaluated Jacobian using the function \verb+J=fsolve_jac(x,f)+.
This gives a feature to test your own Jacobian if provided by comparing its evaluation with the 
\verb+fsolve_jac+ evaluation. Note that the same possibility is also available using the 
\manlink{derivative}{derivative} function. 

\itemdesc{scale optional parameter}
With \verb+scale=vect+ you can provide to the function a vector of $n$ scale factors.
Scale factors are positive numbers such that the scaled variables $s_i r_i$ are of same order
(says they are roughly between $-1$ and $1$. If you know the approximate magnitude of the components
of the solution vector, for instance you know that $r_1$ is between $-10^{-7}$ and $10^{-6}$, 
$r_2$ is between $-50$ and $80$ and $r_3$ is around  $10^{5}$ then you can choose $s_1 = 10^6$, 
$s_2 = 0.02$ and $s_3 = 10^{-5}$. When the scale factors are not provided by the user, the code 
computes them automatically (using the euclidian norm of the columns of the Jacobian). Scaling is 
important when the magnitude of the variables are really different (as in the example before). In 
this case if you known (even very approximatively) the scale of each parameter it is better
to provide them.  

\end{mandescription} 

\begin{examples}
  
\paragraph{example 1} An example with one equation, one unknown. The function:
$$
  f(x) = x - e^{-(1+x)}, \; f'(x) = 1 + e^{-(1+x)}
$$ 
has exactly one root located in $(0,1)$.
\begin{program}\HCode{function y=f(x);y = x - exp(-1-x);endfunction\Hnewline
// simple call to fsolve\Hnewline
[r,fr,info] = fsolve(0,f)\Hnewline
// using fsolve with providing the Jacobian\Hnewline
function y=df(x);y = 1 + exp(-1-x);endfunction\Hnewline
[r,fr,info] = fsolve(0,f,jac=df)\Hnewline
// playing with xtol, ftol\Hnewline
[r,fr,info] = fsolve(0,f,jac=df,ftol=\%eps,xtol=0)}
\end{program}
 
\paragraph{example 2} two equations, two unknowns. In the second example
of the \manlink{ode}{ode} help page (The Brusselator) one can looks for
the stationnary point(s) of the ode. In fact it is straitforward to see
that there is one unique such point $(2;(5+eps)/2)$ but lets use fsolve
anyway.
\begin{program}\HCode{function [f] = brusselator(u,eps)\Hnewline
 f = [ 2 - (6+eps)*u(1) + u(1)^2*u(2);...\Hnewline
       (5+eps)*u(1) - u(1)^2*u(2)]\Hnewline
endfunction\Hnewline
eps = 2;\Hnewline
[z,fz,info] = fsolve([1;1],brusselator,args=eps)\Hnewline
// compute the relative error\Hnewline
ze = [2 ; (5+eps)/2];\Hnewline
norm(z-ze)/norm(ze)}
\end{program}
\end{examples}

\begin{manseealso}
  \manlink{derivative}{derivative}
\end{manseealso}

% -- Authors
\begin{authors}
 hybrj, hybrd: Jorge More, Burt Garbow, and Ken Hillstrom. Nsp interface: Jean-Philippe Chancelier
\end{authors}
