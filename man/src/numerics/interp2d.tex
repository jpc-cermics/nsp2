% -*- mode: latex -*-


\mansection{interp2d}
\begin{mandesc}
\short{interp2d} {bicubic spline or subspline evaluation function}
\end{mandesc}

% -- Calling sequence section
\begin{calling_sequence}
\begin{verbatim}
[zp[,dzpdx,dzpdy]]=interp2d(xp,yp,x,y,C)
[zp[,dzpdx,dzpdy]]=interp2d(xp,yp,x,y,C,outmode=str_outmode)
[zp[,dzpdx,dzpdy]]=interp2d(xp,yp,x,y,C, str_outmode)
\end{verbatim}
\end{calling_sequence}

% -- Parameters

\begin{parameters}
  \begin{varlist}
   \vname{xp, yp}: real vectors or matrices of same size
   \vname{x,y,C}: real vectors defining a bicubic spline or sub-spline function 
                (called $s$ in the following)
   \vname{str\_out\_mode}: (optional) string defining the evaluation of $s$ outside 
                the grid boundary.
   \vname{zp}: vector or matrix of same format than \verb!xp! and \verb!yp!, elementwise 
                evaluation of $s$ on these points.
   \vname{dzpdx, dzpdy}: vectors (or matrices) of same format than \verb!xp! and \verb!yp!, elementwise
                evaluation of the first derivatives of \verb!s! on these points.
  \end{varlist}
\end{parameters}

\begin{mandescription}
  Given three vectors \verb!(x,y,C)! defining a bicubic spline or sub-spline function
  (see \manlink{splin2d}{splin2d} ) this function evaluates $s$ 
  (and $ds/dx, ds/dy$ if needed) at $(xp(i),yp(i))$:
    \begin{verbatim}
      zp(i) = s(xp(i),yp(i))   
      dzpdx(i) = ds/dx(xp(i),yp(i))
      dzpdy(i) = ds/dy(xp(i),yp(i))  
    \end{verbatim}
%       d2zpdxx(i) = d2s/dx2(xp(i),yp(i))
%       d2zpdxy(i) = d2s/dxdy(xp(i),yp(i))  
%       d2zpdyy(i) = d2s/dy2(xp(i),yp(i))

  The \verb!outmode! parameter defines the evaluation rule for extrapolation,
  i.e. for {\em (xp(i),yp(i)) not in [x(1),x(nx)]x[y(1),y(ny)]}: 
  \begin{description}
  \item["by\_zero"]: an extrapolation by zero is done
  \item["by\_nan"]: extrapolation by Nan
  \item["C0"]:  use $s(proj(P))$ where $proj(P)$ the nearest point from P located on the grid boundary.
  \item["natural"]: the extrapolation is done by using the nearest bicubic-patch from (x,y).
  \item["periodic"]: \verb!s! is extended by periodicity. 
  \end{description}

\end{mandescription}

% --example 

\begin{examples}

\paragraph{example 1} this example shows some different extrapolation features
\begin{mintednsp}{nsp}
// interpolation of cos(x)cos(y)
n = 7;  // a n x n interpolation grid
x = linspace(0,2*%pi,n); y = x;
z = cos(x')*cos(y);
C = splin2d(x, y, z, "periodic");
// now evaluate on a bigger domain than [0,2pi]x [0,2pi]
m = 80; // discretisation parameter of the evaluation grid
xx = linspace(-0.5*%pi,2.5*%pi,m); yy = xx;
[XX,YY] = ndgrid(xx,yy);
zz1 = interp2d(XX,YY, x, y, C, outmode="C0"); // or interp2d(XX,YY, x, y, C, "C0");
zz2 = interp2d(XX,YY, x, y, C, outmode="by_zero"); // or interp2d(XX,YY, x, y, C, "by_zero");
zz3 = interp2d(XX,YY, x, y, C, outmode="periodic"); // or interp2d(XX,YY, x, y, C, "periodic");
zz4 = interp2d(XX,YY, x, y, C, outmode="natural"); // or interp2d(XX,YY, x, y, C, "natural");
xbasc()
subplot(2,2,1,a3d=%t)
  plot3d(xx, yy, zz1, flag=[2 6 4])
  xtitle("extrapolation with the C0 outmode")
subplot(2,2,2,a3d=%t)
  plot3d(xx, yy, zz2, flag=[2 6 4])
  xtitle("extrapolation with the by_zero outmode")
subplot(2,2,3,a3d=%t)
  plot3d(xx, yy, zz3, flag=[2 6 4])
  xtitle("extrapolation with the periodic outmode")
subplot(2,2,4,a3d=%t)
  plot3d(xx, yy, zz4, flag=[2 6 4])
  xtitle("extrapolation with the natural outmode")
xselect()
\end{mintednsp}

\paragraph{others} you can see the examples of the
\manlink{splin2d}{splin2d} help page.

\end{examples}

% -- see also
\begin{manseealso}
\manlink{splin2d}{splin2d}  
\end{manseealso}

% -- Author
\begin{authors}
B. Pincon
\end{authors}

