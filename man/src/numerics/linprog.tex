% -*- mode: latex -*-

\mansection{linprog}
\begin{mandesc}
  \short{linprog}{linear programming and mip solver}
\end{mandesc}

% -- Calling sequence section
\begin{calling_sequence}
\begin{verbatim}
[xopt,fopt,flag] = linprog(c, A, b, Ae, be)
[xopt,fopt,flag[,extras]] = linprog(c, A, b, Ae, be, sense="min"|"max",lb=, ub=,
                                    binprog=%t|%f, intprog=%t|%f, var_type=,
                                    heur_sol=, solver_options=, tm_lim=, 
                                    verb=, frq_out=, solver="smplx"|"ipt")
\end{verbatim}
\end{calling_sequence}
% -- Parameters
\begin{parameters}
  \begin{varlist}
    \vname{c}: real vector (says of size $n$) defining the cost function
    \vname{A,b}: real $m_i \times n$ matrix and vector of size $m_i$ defining inequality constraints
    \vname{Ae,be}: real $m_e \times n$ matrix and vector  of size $m_e$ defining equality constraints
    \vname{sense=}: optional, defines the optimisation direction (default is "min")
    \vname{lb=, ub=}: optional, vectors of size $n$ defining lower and upper bounds of the unknown
                      variables (defaults: $0$ for lb and $\infty$ for ub)
    \vname{intprog=}: optional,  set it to true if all variables are integers.
    \vname{binprog=}: optional, set it to true if all variables are binary integers.
    \vname{var_type=}: optional, should be provided only in case you have both continuous (real) 
                       and integer variables.
    \vname{heur_sol=}: optional (only for mips) vector of size $n$ providing a feasible solution.
    \vname{solver_options=}: optional, an hash table for fine tuning of solver parameters.
    \vname{tm_lim=}: optional, real positive scalar defining max computing time in sec (default is 300).
    \vname{verb=}: optional, level of verbosity (0: no output, 1: (default) error messages,
                   2: normal outputs, 3: normal outputs plus initial and final informations).
    \vname{frq_out=}: time between two normal outputs (information on current iteration is 
                      displayed every frq\_out sec) default is 5 sec.
    \vname{solver=}: for a LP problem you can choose between a simplex solver (default) or an 
                     interior point solver.
    \vname{xopt}: optimal solution (flag=0) or best feasible solution found or empty vector
                  for unbounded or unfeasible cases.
    \vname{fopt}: optimal cost function value.
    \vname{flag}: solution status.
    \vname{extras}: hash table with additional informations (reduced cost, lagrange
                    multipliers, basis information).
  \end{varlist}
\end{parameters}

\begin{mandescription}
This function tries to solve the linear programming problem (LP):
$$
\begin{array}{l}
\min \mbox{ or } \max f(x) = c^{\top} x\\
x \in \R^n \mbox{ such that } \left\{ \begin{array}{l} 
  A x \le b \\
  A_e x = b_e \\
  lb \le x \le ub 
\end{array}\right.
\end{array}
$$
with possibly some (or all) variables restricted to be integers, in which
case the problem is called a MIP (mixed integer (linear programming) problem).
If all variables are binary then you can use \verb+binprog=%t+ (in which case
you don't provide lower and upper bounds). If all variables are 
integer use \verb+intprog=%t+. You should provide the optional
\verb+var_type=+ argument only when both type of variables are
present.  The corresponding argument should be an array of $n$ strings 
with \verb+var_type(k)="C"+ to define variable $k$ as continuous 
and \verb+var_type(i)="I"+  to define it as integer.

\itemdesc{remarks about constraints}
\begin{itemize}
\item The constraint matrices $A$ and $A_e$ can be either sparse or full matrix.
\item If all constraints are equalities (respectively inequalities) second 
and third arguments (respectively fourth and fifth arguments) should 
be empty matrices (with first dimension equal to 0). 
\item The (down) concatenation of $A$ and $A_e$ constitute the full
constraint matrix.
\item If your LP problem have a constraint with both under and upper bounds:
$$
   \underline{b} \le a_1 x_1 + a_2 x_2 + \dots + a_n x_n \le \bar{b}
$$
this gives two inequality constraints:
$$
\begin{array}{rcl}
 a_1 x_1 + a_2 x_2 + \dots + a_n x_n  & \le & \bar{b}\\
 -a_1 x_1 - a_2 x_2 + \dots - a_n x_n &  \le & -\underline{b}
\end{array}
$$
\item For glpk the unknowns $x_1, \dots, x_n$ are called structural variables
and each constraint is associated to an auxiliary variable $y_i = (Ax)_i$ 
for $i \in [1,m_i]$ and $y_{i+m_i} = (A_ex)_i$ for  $i \in [1,m_e]$. This is
important when one wants to specify the initial basis.
\end{itemize}

\itemdesc{possible exit flag values}
The following exit flag values can be obtained:
\begin{itemize}
\item \verb+0+: optimal (or integer optimal) solution found.
\item \verb+1+: unbounded solution
\item \verb+2+: unfeasible LP
\item \verb+3+: dual LP is unfeasible
\item \verb+4+: time limit reached with a feasible solution
\item \verb+5+: time limit reached with a dual feasible solution
\item \verb+6+: iteration limit reached with a feasible solution
\item \verb+7+: iteration limit reached with a dual feasible solution
\item \verb+8+: mip gap reached
\end{itemize}

Remarks:
\begin{itemize}
\item In cases 1, 2, 3, \verb+xopt+ is set to \verb+[]+ (\verb+fopt+ is set to
$\pm \infty$ in case 1 and 2 and to $Nan$ in case 3). 

\item In cases 4,5,6,7,8 \verb+xopt+ is the best feasible solution found so far 
(which realizes a mip gap tolerance set by the user in case 8, 
see \verb+mip_gap+ solver options). If you solve a pure LP with the simplex
method it is possible to continue the computation toward the optimal
solution without starting from scratch if you output the fourth \verb+extra+
parameter which contains current basis information. For that purpose you
need to set the \verb+basis_info+ solver option this way:
\begin{Verbatim}
sopt = hash(basis_info=extras.basis_info [, others_options]);
[xopt, fopt, flag, extras] = linprog(....., solver_options=sopt);
\end{Verbatim}
For a MIP the only way to help the solver in a second run, is to provide 
it the feasible solution found in the first call using \verb+heur_sol=xopt+ 
in the new call to \verb+linprog+.
\end{itemize}

\itemdesc{optional {\tt solver} parameter}
For a pure LP you can choose between the glpk simplex solver 
{\tt solver="smplx"} or the glpk interior point solver 
{\tt solver="ipt"}. Note that the interior point solver is
not efficient in some cases (for enough large size problem) 
it don't manage well full columns in constraint matrix, that is if a variable
is effectively present in many constraints).

\itemdesc{optional {\tt heur\_sol} parameter}
MIPs are often quite hard to solve. In some cases an heuristic solution
can be obtained by other means (or from a first run as we have seen before)
and you can provide it to the mip glpk solver using this optional argument 
(this can help much the glpk mip solver in particular if it is closed to 
the optimal solution). Take care: the feasibility of this  heuristic 
solution is not tested.

\itemdesc{optional {\tt solver\_options} parameter}
This argument is an hash table with the possible following entries:
\begin{description}
\item[option common to all solvers]
\begin{itemize}
\item \verb+scale=s+ When not zero this option lets to scale the LP constraint matrix to get better 
      numerical behavior.  s should be a scalar integer in $[0,13]$ (default is 1).
      \begin{tabular}{|c|l|}
        \hline
        0 & no scaling \\
        1 & automatic scaling (correspond to scale=12) \\
        2 & geometric mean scaling\\
        3 & geometric mean scaling + round scale factors to power of 2\\
        4 & geometric mean scaling but skip if LP is well scaled\\
        5 & geometric mean scaling + round scale factors to power of 2 but skip if LP is well scaled\\
        6 & equilibration scaling\\
        7 & equilibration scaling + round scale factors to power of 2\\
        8 & equilibration scaling but skip if LP is well scaled\\
        9 & equilibration scaling + round scale factors to power of 2 but skip if LP is well scaled\\
       10 & geom mean + equil scaling\\
       11 & geom mean + equil scaling  + round scale factors to power of 2\\
       12 & geom mean + equil scaling but skip if LP is well scaled\\
       13 &  geom mean + equil scaling + round scale factors to power of 2 but skip if LP is well scaled\\
       \hline
      \end{tabular}
      Note that scale don't occur if optional presolve argument is true (scaling being done by the presolver). 
\end{itemize}
\item[simplex solver options]
\begin{itemize}
\item \verb+presolve=%f|%t+ (default is false). In case true a presolver tries to reduce
      the LP problem to a easier one.
\item \verb+meth=1|2|3+ (default is 1). simplex algorithm choice: 1 primal simplex, 3 dual simplex, 2 
      dual simplex but switch to primal simplex in case of failure.
\item \verb+r_test=0|1+ (default is 1) ratio test: 0 standard, 1 Harris.
\item \verb+pricing=0|1+
\item \verb+basis_info=+  hash table which defines the initial basis. The hash table has 2 entries named
      \verb+str+ and \verb+aux+ which are respectively arrays of size $n$ and $m_i+m_e$ giving for
      structural and auxiliary variables their statuses :
      \begin{tabular}{|c|l|}
        \hline
        1 & basic variable \\
        2 & non-basic variable on lower bound \\
        3 & non-basic variable on upper bound\\
        4 & non-basic free variable (not used by linprog) \\
        5 & non-basic fixed variable (this corresponds to an equility constraint)\\
       \hline
      \end{tabular}
\end{itemize}
\item[interior point solver options]
\begin{itemize}
\item \verb+ord_alg=+; reordering algorithm
\end{itemize}
\item[mip solver options]
\begin{itemize}
\item \verb+presolve=%f|%t+ (default is true). In case true a presolver tries to reduce
      the MIP problem to a easier one.
\item \verb+gmi_cuts=%t|%f+ (default is false) use Gomory cuts.
\item \verb+mir_cuts=%t|%f+ (default is true) use mixed integer rounding cuts.
\item \verb+cov_cuts=%t|%f+ (default is false) use mixed cover cuts.
\item \verb+clq_cuts=%t|%f+ (default is false) use clique cuts.
\item \verb+fp_heur=%t|%f+ (default is false) if true find an feasible solution using F-pump heuristic.
\item \verb+binarize=%t|%f+ (default is false) if true binarize integer (not binary) variables (this 
      option is used only if the presolver is enable).
\item \verb+br_tech=br_param+ Branching technique option: \verb+br_param+ should be an int in [1,5],
      1: first fractional variable, 2: last fractional variable, 3: most fractional variable, 4:
      heuristic by Driebeck and Tomlin (this is the default), 5: hybrid pseudicost heuristic.
\item \verb+bt_tech=bt_param+ Backtracking technique option:  \verb+br_param+ should be an int in [1,4],
      1: depth first search, 2: breadth first search, 3: best local bound (default), 4: best projection heuristic.
\item \verb+mip_gap=+
\end{itemize}
\end{description}

\itemdesc{terminal outputs}

TO BE DONE: explain the output informations from glpk during computation. 

\end{mandescription} 

\begin{examples}
\paragraph{example 1} We solve the following toy production problem: maximize
$f(x) = 50 x_1 + 40 x_2 + 70 x_3 + 80 x_4 = c^{\top} x$ under the constraints:
$x_1, x_2, x_3, x_4 \ge  0$ and:
$$
\begin{array}{rcl}
   2 x_1 + 4 x_2 +  8 x_3 +  6 x_4 & \le & 100\\
   10 x_1 + 8 x_2 + 6 x_3 + 10 x_4 & \le & 160\\
      x_1 +  x_2  + 2 x_3 +  2 x_4 & \le & 20
\end{array} \iff  A x \le b
$$
\begin{Verbatim}
c= [50; 40; 70; 80];
A = [ 2 4 8  6;
     10 8 6 10;
      1 1 2  2];
b = [100; 160;  20];
[xopt,fopt,flag,extra] = linprog(c,A,b,[],[],sense="max")
\end{Verbatim}
On such problems there is a trivial basis to start simplex : each structural variable
is a non-basic variable at its lower bound 0 (status flag is 2) and each auxiliary variable 
is a basic variable (status flag is 1). 
\begin{Verbatim}
init_basis = hash(str=[2,2,2,2],aux=[1,1,1]);
sopt = hash(basis_info = init_basis)
[xopt,fopt,flag,extra] = linprog(c,A,b,[],[],sense="max",solver_options=sopt)
\end{Verbatim}

\paragraph{example 2} A knapsak problem, see
\href{http://rosettacode.org/wiki/Knapsack_problem/Bounded}{this page} for more explanations.
For a moutain trip one tourist should select, among a list of items, the most interesting
ones for the trip with the constraint that the total weight is under 4 kg (=400 decagram). 
In the following table, second, third and fourth column corresponds respectively to the weights 
(in decagram),  ``values'' and availability of each item (for instance our tourist can take 
until 3 bananas). If we note $c$ the third column, $a$ the second one and $ub$ the last one, 
the MIP problem is:
$$
\begin{array}{c}
   \max c_1 x_1 + c_2 x_2 + \dots + c_n x_n\\
      a_1 x_1 + a_2 x_2 + \dots + a_n x_n \le 400\\
      0 \le x_i \le u_i, \quad x_i \in \N, \quad i = 1, \dots, n
\end{array}
$$
\begin{Verbatim}
Pb = { "map"   	        ,9    ,150   ,1;
       "compass"        ,13   ,35    ,1;
       "water" 	        ,153  ,200   ,2;
       "sandwich"       ,50   ,60    ,2;
       "glucose"        ,15   ,60    ,2;
       "tin"   	        ,68   ,45    ,3;
       "banana"         ,27   ,60    ,3;
       "apple" 	        ,39   ,40    ,3;
       "cheese"         ,23   ,30    ,1;
       "beer"  	        ,52   ,10    ,3;
       "suntan cream"   ,11   ,70    ,1;
       "camera"         ,32   ,30    ,1;
       "T-shirt"        ,24   ,15    ,2;
       "trousers"       ,48   ,10    ,2;
       "umbrella"       ,73   ,40    ,1;
  "waterproof trousers" ,42   ,70    ,1;
"waterproof overclothes",43   ,75    ,1;
       "note-case"      ,22   ,80    ,1;
       "sunglasses"     ,7    ,20    ,1;
       "towel" 	        ,18   ,12    ,2;
       "socks" 	        ,4    ,50    ,1;
       "book"  	        ,30   ,10    ,2};

c = ce2m(Pb(:,3));
A = ce2m(Pb(:,2))';
b = 400;
ub = ce2m(Pb(:,4));

[xopt,fopt, flag] = linprog(c,A,b,[],[],sense="max",intprog=%t,ub=ub,verb=0);
printf("\nMy knapsack will contain:\n")
for k=1:numel(xopt)
   if xopt(k)>0 then
      printf("    %d %s\n",xopt(k),Pb{k,1})
   end
end
\end{Verbatim}

\end{examples}

\begin{manseealso}

\end{manseealso}

% -- Authors
\begin{authors}
 glpk: Andrew Makhorin (and contributors). Nsp interface: Bruno Pin\c{c}on
\end{authors}
