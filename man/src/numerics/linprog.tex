% -*- mode: latex -*-

\mansection{linprog}
\begin{mandesc}
  \short{linprog}{linear programming and mip solver}
\end{mandesc}

% -- Calling sequence section
\begin{calling_sequence}
\begin{verbatim}
[xopt,fopt,flag] = linprog(c, A, b, Ae, be)
[xopt,fopt,flag[,extras]] = linprog(c, A, b, Ae, be, sense="min"|"max",lb=, ub=,
                                    binprog=%t|%f, intprog=%t|%f, var_type=,
                                    heur_sol=, solver_options=, 
                                    tm_lim=, verb=, frq_out=, solver="smplx"|"ipt")
\end{verbatim}
\end{calling_sequence}
% -- Parameters
\begin{parameters}
  \begin{varlist}
    \vname{c}: real vector (says of size $n$) defining the cost function
    \vname{A,b}: real $m_i \times n$ matrix and vector of size $m_i$ defining inequality constraints
    \vname{Ae,be}: real $m_e \times n$ matrix and vector  of size $m_e$ defining equality constraints
    \vname{sense=}: optional, defines the optimisation direction (default is "min")
    \vname{lb=, ub=}: optional, vectors of size $n$ defining lower and upper bounds of the unknown
                      variables (defaults: $0$ for lb and $\infty$ for ub)
    \vname{binprog=}: optional, should be true if all variables are binary integers.
    \vname{intprog=}: optional,  should be true if all variables are integers.
    \vname{var_type=}: optional, should be provided only in case you have both continuous (real) 
                       and integer variables.
    \vname{heur_sol=}: optional (only for mips) vector of size $n$ providing a feasible solution.
    \vname{solver_options=}: optional, an hash table for fine tuning of solver parameters.
    \vname{tm_lim=}: optional, real positive scalar defining the max computing time in sec (default is 300).
    \vname{verb=}: optional, level of verbosity (0 no output, 1 (default) error messages,
                   2 ``normal'' outputs, 3 ``normal'' outputs plus initial and final informations).
    \vname{frq_out=}: time between two ``normal'' outputs (normal output are displayed every
                     frq\_out sec, default is 5).
    \vname{solver=}: for a LP problem you can choose between a simplex solver (default) or an 
                     interior point solver.
    \vname{xopt}: optimal solution (flag=0) or best feasible solution found or empty vector
                  for unbounded or unfeasible cases.
    \vname{fopt}: optimal cost function value.
    \vname{flag}: solution status
    \vname{extras}: hash table with additional informations (reduced cost, lagrange
                    multipliers, basis information).
  \end{varlist}
\end{parameters}

\begin{mandescription}
This function tries to solve the linear programming problem (LP):
$$
\begin{array}{l}
\min \mbox{ or } \max f(x) = c^{\top} x\\
x \in \R^n \mbox{ such that } \left\{ \begin{array}{l} 
  A x \le b \\
  A_e x = b_e \\
  lb \le x \le ub 
\end{array}\right.
\end{array}
$$
with possibly some (or all) variables restricted to be integers, in which
case the problem is called a MIP (mixed integer (linear programming) problem).
If all variables are binary then you can use \verb+binprog=%t+ (in which case
you don't provide lower and upper bounds). If all variables are 
integer use \verb+intprog=%t+. You should only provide the optional
\verb+var_type=+ argument only in case when both type of variables are
present.  The corresponding argument should be an array of $n$ strings 
with \verb+var_type(k)="C"+ to define variable $k$ as continuous 
and \verb+var_type(i)="I"+  to define it as integer.

If all constraints are equalities (respectively inequalities) second 
($A$) and third ($b$) arguments (respectively fourth $A_e$ and fifth ($b_e$)
arguments) should be empty matrices.

The following exit flag values can be obtained:
\begin{itemize}
\item \verb+0+: optimal (or integer optimal) solution found.
\item \verb+1+: unbounded solution
\item \verb+2+: unfeasible LP
\item \verb+3+: dual LP is unfeasible
\item \verb+4+: time limit reached with a feasible solution
\item \verb+5+: time limit reached with a dual feasible solution
\item \verb+6+: iteration limit reached with a feasible solution
\item \verb+7+: iteration limit reached with a dual feasible solution
\item \verb+8+: mip gap reached
\end{itemize}

\itemdesc{optional {\tt solver} parameter}
For a pure LP you can choose between the glpk simplex solver 
{\tt solver="smplx"} or the glpk interior point solver 
{\tt solver="ipt"}. Note that the interior point solver is
not efficient in some cases (in particular it don't manage 
well full columns in constraint matrix, that is if a variable
is effectively present in many constraints).

\itemdesc{optional {\tt heur\_sol} parameter}
MIPs are often quite hard to solve. In some cases an heuristic solution
can be obtained by other means and you can provide it to the mip glpk
solver using this optional argument (this can help much the glpk mip 
solver in particular if it is closed to the optimal solution). 
Take care: the feasibility of this  heuristic solution is 
not tested.

\itemdesc{optional {\tt solver\_options} parameter}
This argument is an hash table with the possible following entries:
\begin{description}
\item[common options]
\begin{itemize}
\item \verb+scale=s+ scales (when not zero) the LP constraint matrix to get better numerical behavior. 
      s should be a scalar integer in $[0,13]$ (default 1).
      \begin{tabular}{|c|l|}
        \hline
        0 & no scaling \\
        1 & automatic scaling (correspond to scale=12) \\
        2 & geometric mean scaling\\
        3 & geometric mean scaling + round scale factors to power of 2\\
        4 & geometric mean scaling but skip if LP is well scaled\\
        5 & geometric mean scaling + round scale factors to power of 2 but skip if LP is well scaled\\
        6 & equilibration scaling\\
        7 & equilibration scaling + round scale factors to power of 2\\
        8 & equilibration scaling but skip if LP is well scaled\\
        9 & equilibration scaling + round scale factors to power of 2 but skip if LP is well scaled\\
       10 & geom mean + equil scaling\\
       11 & geom mean + equil scaling  + round scale factors to power of 2\\
       12 & geom mean + equil scaling but skip if LP is well scaled\\
       13 &  geom mean + equil scaling + round scale factors to power of 2 but skip if LP is well scaled\\
       \hline
      \end{tabular}
      Note that scale don't occur if optional presolve argument is true (scaling is done by the presolver). 
\end{itemize}
\item[simplex solver options]
\begin{itemize}
\item \verb+presolve=%f|%t+ (default is false). In case true a presolver tries to reduce
      the LP problem to a easier one.
\item \verb+meth=1|2|3+ (default is 1). simplex algorithm choice: 1 primal simplex, 3 dual simplex, 2 
      dual simplex but switch to primal simplex in case of failure.
\item \verb+r_test=0|1+ (default is 1) ratio test: 0 standard, 1 Harris.
\item \verb+pricing=0|1+
\item \verb+basis_info=+  hash table which defines the initial basis.
\end{itemize}
\item[interior point solver options]
\begin{itemize}
\item \verb+ord_alg=+; reordering algorithm
\end{itemize}
\item[mip solver options]
\begin{itemize}
\item \verb+presolve=%f|%t+ (default is true). In case true a presolver tries to reduce
      the MIP problem to a easier one.
\item \verb+gmi_cuts=%t|%f+ (default is false) use Gomory cuts.
\item \verb+mir_cuts=%t|%f+ (default is true) use mir cuts.
\item \verb+cov_cuts=%t|%f+ (default is false) use cov cuts.
\item \verb+clq_cuts=%t|%f+ (default is false) use clique cuts.
\item \verb+fp_heur=%t|%f+ (default is false) if true find an feasible solution using F-pump heuristic.
\item \verb+binarize=%t|%f+ (default is false) if true binarize integer (not binary) variables. 
\item \verb+br_tech=+ an int in [1,5]
\item \verb+bt_tech=+ an int in [1,4]
\item \verb+mip_gap=+
\end{itemize}
\end{description}

TO BE CONTINUED

\end{mandescription} 

\begin{examples}
  
\end{examples}

\begin{manseealso}

\end{manseealso}

% -- Authors
\begin{authors}
 glpk: Andrew Makhorin (and contributors). Nsp interface: Bruno Pin\c{c}on
\end{authors}
