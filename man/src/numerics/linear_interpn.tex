% -*- mode: latex -*-

\mansection{linear\_interpn}

\begin{mandesc}
  \shortunder{linear\_interpn}{linear_interpn}{n dimensional linear interpolation}
\end{mandesc}

% -- Calling sequence section
\begin{calling_sequence}
   \begin{verbatim}
       vp = linear_interpn(xp1,xp2,..,xpn, x1, ..., xn, v)
       vp = linear_interpn(xp1,xp2,..,xpn, x1, ..., xn, v, outmode=str_outmode)
       vp = linear_interpn(xp1,xp2,..,xpn, x1, ..., xn, v, str_outmode])
   \end{verbatim}
\end{calling_sequence}

%-- Parameters

\begin{parameters}
  \begin{varlist}
   \vname{xp1, xp2, ;.., xpn}: real vectors (or matrices) of same size
   \vname{x1 ,x2, ..., xn}:  strictly increasing row vectors (with at least 2 components)
               defining the n dimensional interpolation grid
   \vname{v}: vector (case n=1), matrix (case n=2,3,...) with the
               values of the underlying interpolated function at the grid points.
   \vname{str\_out\_mode}: (optional) string defining the evaluation outside the grid (extrapolation)
   \vname{vp}:  vector or matrix of same size than \verb!xp1, ..., xpn!
  \end{varlist}
\end{parameters}

\begin{mandescription}

  Computes values at the points of coordinates defined
  by the vector (or matrices) \verb!xp1, xp2, ..., xpn!, by n-linear
  interpolation from the $n$ dimensional grid defined by the n vectors 
  \verb!x1 ,x2, ..., xn! and the values \verb!v! at the grid points 
  ($v(i_1,i_2,...,i_n)$ being the value at the point
  $(x1(i_1),x2(i_2), ..., xn(i_n))$) (see {\bf remark} for using
  this function for $n \ge 3$).
  The \verb!outmode! parameter set the evaluation rule for
  extrapolation that is when \verb+(xp1(i),xp2(i),...,xpn(i))+ is 
  outside the grid boundary. The different choices are:
  \begin{itemize}
       \item \itemdesc{"by\_zero"}: an extrapolation by zero is done
       \item \itemdesc{"by\_nan"}: extrapolation by Nan
       \item \itemdesc{"C0"}: the extrapolation is the linear
         interpolation of $proj(P)$ the nearest point from P located on the grid boundary.
       \item \itemdesc{"natural"}: the extrapolation is done by using the nearest n-linear patch from the point.
       \item \itemdesc{"periodic"}: the n-linear interpolant is extended by periodicity. 
  \end{itemize}

\end{mandescription}
 
\paragraph{Remark}
  Currently nsp don't provide matrix with more than 2 dimensions. To
  use this function for trilinear interpolation and more, you must 
  ``simulate'' such n-dimensional matrices in the following way:
  assume that there are $ni$ elements along dimension $i$ then :
  \begin{verbatim}
      v(i1,i2,i3) = v(i_1 + n1*((i2-1) + n2*(i3-1)))
 or   v(i1,i2,i3) = v(i_1, i2 + n2*(i3-1))\end{verbatim}
  to simulate 3d matrix and :
  \begin{verbatim}
      v(i1,i2,i3,i4) = v(i_1 + n1*((i2-1) + n2*((i3-1) + n3*(i4-1))))
 or   v(i1,i2,i3,i4) = v(i_1, i2 + n2*((i3-1)+ n3*(i4-1)))\end{verbatim}
 to simulate 4d matrix, and so on.

% --example 

\begin{examples}
\paragraph{example 1}  1d linear interpolation
\begin{mintednsp}{nsp}
xclear();
x = linspace(0,2*%pi,11);
y = sin(x);
xx = linspace(-2*%pi,4*%pi,400)';
yy = linear_interpn(xx, x, y, outmode="periodic");
xbasc()
plot2d(xx,yy,style=2)
plot2d(x,y,style=-9, strf="000")
xtitle("linear interpolation of sin(x) with 11 interpolation points")
\end{mintednsp}

\paragraph{example 2} bilinear interpolation
\begin{mintednsp}{nsp}
xclear()
n = 8;
x = linspace(0,2*%pi,n); y = x;
z = 2*sin(x')*sin(y);
xx = linspace(0,2*%pi, 40);
[xp,yp] = ndgrid(xx,xx);
zp = linear_interpn(xp,yp, x, y, z);
xbasc()
plot3d(xx, xx, zp, flag=[2 6 4])
[xg,yg] = ndgrid(x,x);
param3d1(xg,yg, z, style=-9*ones(1,n), flag=[0 0])
xtitle("Bilinear interpolation of 2sin(x)sin(y)")
xselect()
\end{mintednsp}

\paragraph{example 3} bilinear interpolation and experimentation with all the out\_mode features
\begin{mintednsp}{nsp}
xclear();
nx = 20; ny = 30;
x = linspace(0,1,nx);
y = linspace(0,2, ny);
[X,Y] = ndgrid(x,y);
z = 0.4*cos(2*%pi*X).*cos(%pi*Y);
nxp = 60 ; nyp = 120;
xp = linspace(-0.5,1.5, nxp);
yp = linspace(-0.5,2.5, nyp);
[XP,YP] = ndgrid(xp,yp);
zp1 = linear_interpn(XP, YP, x, y, z, outmode="natural");
zp2 = linear_interpn(XP, YP, x, y, z, outmode="periodic");
zp3 = linear_interpn(XP, YP, x, y, z, outmode="C0");
zp4 = linear_interpn(XP, YP, x, y, z, outmode="by_zero");
zp5 = linear_interpn(XP, YP, x, y, z, outmode="by_nan");
xbasc()
subplot(2,3,1)
   plot3d(x, y, z, leg="x@y@z", flag = [2 4 4])
   xtitle("initial function 0.4 cos(2 pi x) cos(pi y)")
subplot(2,3,2)
   plot3d(xp, yp, zp1, leg="x@y@z", flag = [2 4 4])
   xtitle("Natural")
subplot(2,3,3)
   plot3d(xp, yp, zp2, leg="x@y@z", flag = [2 4 4])
   xtitle("Periodic")
subplot(2,3,4)
   plot3d(xp, yp, zp3, leg="x@y@z", flag = [2 4 4])
   xtitle("C0")
subplot(2,3,5)
   plot3d(xp, yp, zp4, leg="x@y@z", flag = [2 4 4])
   xtitle("by_zero")
subplot(2,3,6)
   plot3d(xp, yp, zp5, leg="x@y@z", flag = [2 4 4])
   xtitle("by_nan")
xselect()
\end{mintednsp}


\paragraph{example 4 : trilinear interpolation}

\begin{mintednsp}{nsp}
xclear();
function v=f(x,y,z)
   v=(x-0.5).^2 + (y-0.5).^3 + (z-0.5).^2
endfunction
func =  "v=(x-0.5).^2 + (y-0.5).^3 + (z-0.5).^2";
n = 5; 
x = linspace(0,1,n); y=x; z=x;
[X,Y,Z] = ndgrid(x,y,z);
V = f(X,Y,Z);
// compute (and display) the linear interpolant on some slices
m = 41;
dir = ["z="  "z="  "z="  "x="  "y="];
val = [ 0.1   0.5   0.9   0.5   0.5];
ebox = [0 1 0 1 0 1];

XF=[]; YF=[]; ZF=[]; VF=[];
for i = 1:length(val)
   [X,Y,Z] = slice_parallelepiped(dir(i), val(i), ebox, m, m, m);
   Vm = linear_interpn(X,Y,Z, x, y, z, V);
   [xf,yf,zf,vf] = nf3d(X,Y,Z,Vm,1);
   XF = [XF xf]; YF = [YF yf]; ZF = [ZF zf]; VF = [VF vf]; 
end
nb_col = 128;
vmin = min(VF); vmax = max(VF);
color = bsearch(VF,linspace(vmin,vmax,nb_col+1));
xbasc();
xset("colormap",jetcolormap(nb_col));
xset("hidden3d",-1)
plot3d(XF, YF, ZF,colors=-color, flag=[-1 6 4])
colorbar(vmin,vmax)
xset("font size",3)
xtitle("tri-linear interpolation of "+func)
xset("font size",1)
xselect()
\end{mintednsp}
 
\end{examples}

% -- see also

\begin{manseealso}
\manlink{splin}{splin} \manlink{splin2d}{splin2d}
\end{manseealso}

  %-- Author

\begin{authors}
  B. Pincon
\end{authors}

