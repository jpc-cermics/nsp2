% -*- mode: latex -*-

\mansection{linear\_interpn}

\begin{mandesc}
  \shortunder{linear\_interpn}{linear_interpn}{n dimensional linear interpolation}
\end{mandesc}
%\index{linear\_interpn}\label{linear_interpn}

% -- Calling sequence section
\begin{calling_sequence}
   \begin{verbatim}
       vp = linear_interpn(xp1,xp2,..,xpn, x1, ..., xn, v [,out_mode])
   \end{verbatim}
\end{calling_sequence}

%-- Parameters

\begin{parameters}
  \begin{varlist}
   \vname{xp1, xp2, .., xpn} : real vectors (or matrices) of same size
   \vname{x1 ,x2, ..., xn} :  strictly increasing row vectors (with at least 2 components)
               defining the n dimensional interpolation grid
   \vname{v} : vector (case n=1), matrix (case n=2,3,...) with the
               values of the underlying interpolated function at the grid points.
   \vname{out\_mode} : (optional) string defining the evaluation outside the grid (extrapolation)
   \vname{vp} :  vector or matrix of same size than \verb!xp1, ..., xpn!
  \end{varlist}
\end{parameters}

\begin{mandescription}

  Computes values at the points of coordinates defined
  by the vector (or matrices) \verb!xp1, xp2, ..., xpn!, by n-linear
  interpolation from the $n$ dimensional grid defined by the n vectors 
  \verb!x1 ,x2, ..., xn! and the values \verb!v! at the grid points 
  ($v(i_1,i_2,...,i_n)$ being the value at the point
  $(x1(i_1),x2(i_2), ..., xn(i_n))$) (see {\bf remark} for using
  this function for $n \ge 3$).
  The \verb!out_mode! parameter set the evaluation rule for
  extrapolation that is when \verb+(xp1(i),xp2(i),...,xpn(i))+ is 
  outside the grid boundary. The different choices are:
  \begin{itemize}
       \item \itemdesc{"by\_zero"} : an extrapolation by zero is done
       \item \itemdesc{"by\_nan"} : extrapolation by Nan
       \item \itemdesc{"C0"} : the extrapolation is the linear
         interpolation of $proj(P)$ the nearest point from P located on the grid boundary.
       \item \itemdesc{"natural"} : the extrapolation is done by using the nearest n-linear patch from the point.
       \item \itemdesc{"periodic"} : the n-linear interpolant is extended by periodicity. 
  \end{itemize}

\end{mandescription}
 
\paragraph{Remark}
  Currently nsp don't provide matrix with more than 2 dimensions. To
  use this function for trilinear interpolation and more, you must 
  ``simulate'' such n-dimensional matrices in the following way:
  assume that there are $ni$ elements along dimension $i$ then :
  \begin{verbatim}
      v(i1,i2,i3) = v(i_1 + n1*((i2-1) + n2*(i3-1)))
 or   v(i1,i2,i3) = v(i_1, i2 + n2*(i3-1))\end{verbatim}
  to simulate 3d matrix and :
  \begin{verbatim}
      v(i1,i2,i3,i4) = v(i_1 + n1*((i2-1) + n2*((i3-1) + n3*(i4-1))))
 or   v(i1,i2,i3,i4) = v(i_1, i2 + n2*((i3-1)+ n3*(i4-1)))\end{verbatim}
 to simulate 4d matrix, and so on.

% --example 

\begin{examples}
\paragraph{example 1}  1d linear interpolation
\begin{program}\HCode{x = linspace(0,2*\%pi,11);\Hnewline
y = sin(x);\Hnewline
xx = linspace(-2*}\verb+%pi+\HCode{,4*\%pi,400)';\Hnewline
yy = linear_interpn(xx, x, y, "periodic");\Hnewline
xbasc()\Hnewline
plot2d(xx,yy,style=2)\Hnewline
plot2d(x,y,style=-9, strf="000")\Hnewline
xtitle("linear interpolation of sin(x) with 11 interpolation points")\Hnewline
}
\end{program}
\paragraph{example 2} bilinear interpolation
\begin{program}\HCode{n = 8;\Hnewline
x = linspace(0,2*\%pi,n); y = x;\Hnewline
z = 2*sin(x')*sin(y);\Hnewline
xx = linspace(0,2*\%pi, 40);\Hnewline
[xp,yp] = ndgrid(xx,xx);\Hnewline
zp = linear_interpn(xp,yp, x, y, z);\Hnewline
xbasc()\Hnewline
plot3d(xx, xx, zp, flag=[2 6 4])\Hnewline
[xg,yg] = ndgrid(x,x);\Hnewline
param3d1(xg,yg, z, style=-9*ones(1,n), flag=[0 0])\Hnewline
xtitle("Bilinear interpolation of 2sin(x)sin(y)")\Hnewline
legends("interpolation points",-9,1)\Hnewline
xselect()\Hnewline
}
\end{program}
\paragraph{example 3} bilinear interpolation and experimentation with all the outmode features
\begin{program}\HCode{nx = 20; ny = 30;\Hnewline
x = linspace(0,1,nx);\Hnewline
y = linspace(0,2, ny);\Hnewline
[X,Y] = ndgrid(x,y);\Hnewline
z = 0.4*cos(2*\%pi*X).*cos(\%pi*Y);\Hnewline
nxp = 60 ; nyp = 120;\Hnewline
xp = linspace(-0.5,1.5, nxp);\Hnewline
yp = linspace(-0.5,2.5, nyp);\Hnewline
[XP,YP] = ndgrid(xp,yp);\Hnewline
zp1 = linear_interpn(XP, YP, x, y, z, "natural");\Hnewline
zp2 = linear_interpn(XP, YP, x, y, z, "periodic");\Hnewline
zp3 = linear_interpn(XP, YP, x, y, z, "C0");\Hnewline
zp4 = linear_interpn(XP, YP, x, y, z, "by_zero");\Hnewline
zp5 = linear_interpn(XP, YP, x, y, z, "by_nan");\Hnewline
xbasc()\Hnewline
subplot(2,3,1)\Hnewline
   plot3d(x, y, z, leg="x@y@z", flag = [2 4 4])\Hnewline
   xtitle("initial function 0.4 cos(2 pi x) cos(pi y)")\Hnewline
subplot(2,3,2)\Hnewline
   plot3d(xp, yp, zp1, leg="x@y@z", flag = [2 4 4])\Hnewline
   xtitle("Natural")\Hnewline
subplot(2,3,3)\Hnewline
   plot3d(xp, yp, zp2, leg="x@y@z", flag = [2 4 4])\Hnewline
   xtitle("Periodic")\Hnewline
subplot(2,3,4)\Hnewline
   plot3d(xp, yp, zp3, leg="x@y@z", flag = [2 4 4])\Hnewline
   xtitle("C0")\Hnewline
subplot(2,3,5)\Hnewline
   plot3d(xp, yp, zp4, leg="x@y@z", flag = [2 4 4])\Hnewline
   xtitle("by_zero")\Hnewline
subplot(2,3,6)\Hnewline
   plot3d(xp, yp, zp5, leg="x@y@z", flag = [2 4 4])\Hnewline
   xtitle("by_nan")\Hnewline
xselect()\Hnewline
}
\end{program}


\paragraph{example 4 : trilinear interpolation}
\begin{program}\HCode{function v=f(x,y,z)\Hnewline
   v=(x-0.5).^2 + (y-0.5).^3 + (z-0.5).^2\Hnewline
endfunction\Hnewline
func =  "v=(x-0.5).^2 + (y-0.5).^3 + (z-0.5).^2";\Hnewline
n = 5; \Hnewline
x = linspace(0,1,n); y=x; z=x;\Hnewline
[X,Y,Z] = ndgrid(x,y,z);\Hnewline
V = f(X,Y,Z);\Hnewline
// compute (and display) the linear interpolant on some slices\Hnewline
m = 41;\Hnewline
dir = ["z="  "z="  "z="  "x="  "y="];\Hnewline
val = [ 0.1   0.5   0.9   0.5   0.5];\Hnewline
ebox = [0 1 0 1 0 1];\Hnewline
\Hnewline
XF=[]; YF=[]; ZF=[]; VF=[];\Hnewline
for i = 1:length(val)\Hnewline
   [X,Y,Z] = slice_parallelepiped(dir(i), val(i), ebox, m, m, m);\Hnewline
   Vm = linear_interpn(X,Y,Z, x, y, z, V);\Hnewline
   [xf,yf,zf,vf] = nf3d(X,Y,Z,Vm,1);\Hnewline
   XF = [XF xf]; YF = [YF yf]; ZF = [ZF zf]; VF = [VF vf]; \Hnewline
end\Hnewline
nb_col = 128;\Hnewline
vmin = min(VF); vmax = max(VF);\Hnewline
color = bsearch(VF,linspace(vmin,vmax,nb_col+1));\Hnewline
xset("colormap",jetcolormap(nb_col));\Hnewline
xbasc()\Hnewline
xset("hidden3d",-1)\Hnewline
colorbar(vmin,vmax)\Hnewline
plot3d(XF, YF, ZF,colors=-color, flag=[-1 6 4])\Hnewline
xset("font size",3)\Hnewline
xtitle("tri-linear interpolation of "+func)\Hnewline
xset("font size",1)\Hnewline
xselect()}
\end{program}
 
\end{examples}

% -- see also

\begin{manseealso}
\manlink{splin}{splin} \manlink{splin2d}{splin2d}
\end{manseealso}

  %-- Author

\begin{authors}
  B. Pincon
\end{authors}

