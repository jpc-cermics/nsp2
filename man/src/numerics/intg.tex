% -*- mode: latex -*-

\mansection{intg}
\begin{mandesc}
  \short{intg}{one dimensional integration over a finite or infinite interval}
\end{mandesc}

% -- Calling sequence section
\begin{calling_sequence}
\begin{verbatim}
Ia = intg(a,b,f)
[Ia [,ea_estim [,ier [,neval]]]]  = intg(a, b, f, atol=ea, rtol=er, args=fpar,... 
                                         limit=maxsub, sing=points, vecteval=b)
\end{verbatim}
\end{calling_sequence}
% -- Parameters
\begin{parameters}
  \begin{varlist}
    \vname{a,b}:  real scalars (including \verb+-%inf+ and\verb+%inf+) defining the integration range
    \vname{f}: nsp function defining the function to integrate
    \vname{atol=ea}: optional named argument, requested absolute error
                      on the result (default is 1e-14)
    \vname{rtol=er}: optional named argument, requested relative error
                      on the result (default is 1e-8).
    \vname{args=fpar}: optional named argument, useful to pass
    additional parameters needed by the function f.
    \vname{limit=maxsub}: optional named argument, max number of
    subdivisions allowed (default 750).
    \vname{sing=points}: optional named argument, array of reals where the integrand
    is known to be singular (note that you cannot use sing with an infinite range).
    \vname{vecteval=b}: optional named argument, boolean scalar
    (default \verb+%f+), use \verb+%t+ if you write the function
                   \verb+f+ such that it could be evaluated on a
                   vector argument (this speed up the computation).
    \vname{Ia}: computed approximate result
    \vname{ea_estim}: estimated error
    \vname{ier}: scalar given some information about the computation (see below)
    \vname{neval}: number of function evaluations.
  \end{varlist}
\end{parameters}

\begin{mandescription}
This function computes an approximation $I_a$ of:
$$
   I = \int_a^b f(x) dx
$$
such that:
$$
   | I - I_a | \le  \max (atol, rtol*|I|)
$$
together with an estimation \verb+ea_estim+ of the error  $| I - I_a |$.

The \verb+ier+ variable could take values from 0 to 5: 
\begin{itemize}
\item ier = 0,  normal and reliable termination of the routine. It is assumed that the
      requested  accuracy has been achieved.
\item other values correspond to an abnormal termination of the routine. The estimates
      for integral and error are less reliable. It is assumed that the  requested accuracy
      has not been achieved:
      \begin{itemize}
      \item ier = 1: maximum number of subdivisions (the limit
            parameter) allowed has been achieved. You can try to
            increase limit but generally it is advised to have a close look at the
            integrand, in order to determine the integration
            difficulties. If the position of a local difficulty can
            be determined (i.e.  singularity, discontinuity within
            the interval) one will probably gain by given the singular
            point(s) using the \verb+sing=+ optional argument (or
            by using an appropriate special-purpose integrator 
            designed for handling the type  of difficulty involved).
      \item ier = 2: the occurrence of roundoff error is detected
            which prevents the requested tolerance from being
            achieved. The error may be under-estimated.
      \item ier = 3: extremely bad integrand behaviour occurs at 
            some interior points of the integration interval.
      \item ier = 4: it is presumed that the requested tolerance 
            cannot be achieved, and that the returned result is the
            best which can be obtained.
      \item ier = 5: the integral is probably divergent, or slowly
            convergent. It must be noted that divergency can occur with
            any other value of ier.
      \end{itemize}
\end{itemize}

Additional parameters for the function $f$ could be passed using the
named optional argument \verb+args=fpar+ where \verb+fpar+ is a 
cell array with the additional parameters of $f$ : if $f$ is of the form 
\verb+function z=f(x,p1,p2,...,pn)+ then \verb+args={p1,p2,...,pn}+. 
Note that if $f$ has only one additional parameter it is not mandatory 
to encapsulate it in a cell array (except when the parameter is it 
self a cell array !). The additional parameters could be any nsp object. 
See example 3 (and example 5) here after. 


While the \verb+sing=+ optional parameter is first devoted to
integrate functions with (integrable) singularities it can
be also use to improve the first discretization in the intg 
computing process. When the internal routine (dqagpe) computes 
a first approximation of the integral it uses a 21 points 
Gaussien formula on the whole $[a,b]$ interval. So if these
points are all located on a part where the function is constant 
then intg may return a completly false value. To begin the 
integration approximation using 10 intervals (instead
of just one) you can use \verb+sing=linspace(a,b,11)+.

\end{mandescription}

\begin{examples}
  
\paragraph{example 1} a simple example 
\begin{mintednsp}{nsp}
function y=f1(x);y=exp(x);endfunction
[I,ea] = intg(0,1,f1)
abs(I-(%e-1)) // exact absolute error
\end{mintednsp}
  
\paragraph{example 2} comparizon with the error function
\begin{mintednsp}{nsp}
function y=f2(x);y=2*exp(-x^2)/sqrt(%pi);endfunction
[Ia,ea] = intg(0,0.5,f2)
I = erf(0.5)
abs(I - Ia)
\end{mintednsp}
  
\paragraph{example 3} passing extra parameters (see also example 5). 
\begin{mintednsp}{nsp}
// a function with one parameter
function y=g(x,a);y=a*x;endfunction
// in this case the usual syntax :
[Ia,ea] = intg(0,2*%pi,g,args={1})  // exact result is 2*pi^2
// could be replaced by the short cut :
[Ia,ea] = intg(0,2*%pi,g,args=1)
abs(Ia - 2*%pi^2)

// a function with 2 parameters
function y=g(x,a,w);y=a*x + sin(w*x);endfunction
[Ia,ea] = intg(0,2*%pi,g,args={1,1}) // exact result is 2*pi^2
abs(Ia - 2*%pi^2)
\end{mintednsp}

\paragraph{example 4} a difficult problem. Here we try to integrate:
$$
    f(x) = 2 x \cos \left(\frac{1}{x}\right) +  \sin \left(\frac{1}{x}\right)
$$ 
on $[0,1]$. $f$ is not continuous at $0$ and oscillate more and more
rapidly as $x \rightarrow 0$. The antiderivative of $f$
is:
$$
    F(x) = x^2  \cos \left(\frac{1}{x}\right)
$$
so we know the exact result ($I=\cos(1)$). We write the function in a vectorial
manner so as to experiment also with the \verb+vecteval+ option. 
\begin{mintednsp}{nsp}
function y=f(x);y = 2*x.*cos(1./x) + sin(1./x); endfunction

// 1/ reduce the tolerance
[Ia,ea,ier] = intg(0,1,f, rtol=1e-6)

// 2/ increase the limit parameter (trying to reach the requested precision)
[Ia,ea,ier] = intg(0,1,f, rtol=1e-6, limit=10000)

// 3/ compare ea the estimated error with the (near) exact error:
ea_exact = abs(Ia-cos(1))

// 4/ increase rtol and measure the computing time
tic(); [Ia,ea,ier] = intg(0,1,f, rtol=1e-7, limit=10000); toc()

// 5/ the same using vecteval=%t (it must be faster)
tic(); [Ia,ea,ier] = intg(0,1,f, rtol=1e-7, limit=10000, vecteval=%t); toc()
\end{mintednsp}

  
\paragraph{example 5} In this example we try to integrate probability 
density functions, theorical results must be all equal to 1 (see the
\manlink{pdf}{pdf} help page). 
\begin{mintednsp}{nsp}
// a function to handle all pdf functions with intg
function y=f(x,p);y=pdf(p{1},x,p{2:$});endfunction

// integrating the N(0,1) density
[Ia, ea, ier] = intg(-%inf, %inf, f, args={{"nor",0,1}})

// integrating the gamma (3,1) density
[Ia, ea, ier] = intg(0, %inf, f, args={{"gam",3,1}})

// integrate the Cauchy (100) density
[Ia, ea, ier] = intg(-%inf, %inf, f, args={{"cau",100}})

// integrate the Laplace (1000) density
[Ia, ea, ier] = intg(-%inf, %inf, f, args={{"lap",1000}})

// integrate the beta (2,6) density
[Ia, ea, ier] = intg(0, 1, f, args={{"bet",2,6}})
\end{mintednsp}
  
\paragraph{example 6} Integration with singularities.
\begin{mintednsp}{nsp}
function y=f(x);y = log(abs(x)); endfunction
[Ia,ea,ier] = intg(-1,1,f, sing=0) // exact result is -2
Ia+2
\end{mintednsp}


\end{examples}

\begin{manseealso}
  \manlink{intg\_splin}{intg_splin}, \manlink{int2d}{int2d},  \manlink{int3d}{int3d}  
\end{manseealso}

% -- Authors
\begin{authors}
  uses the routine dqagpe and dqagie (Robert Piessens and Elise de Doncker) from
  quadpack. Nsp interface: Bruno Pincon.
\end{authors}
