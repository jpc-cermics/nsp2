% -*- mode: latex -*-

\mansection{int2d}
\begin{mandesc}
  \short{int2d}{two dimensional integration over a rectangle or a set of triangles}
\end{mandesc}

% -- Calling sequence section
\begin{calling_sequence}
\begin{verbatim}
Ia  = int2d(X, Y, f)
[Ia, [,ea_estim, [,ier [,neval]]]] = int2d(X, Y, f, tol=e, tolflag=b, args=obj, limit=maxtri, iclose=b, vecteval=b)
\end{verbatim}
\end{calling_sequence}
% -- Parameters
\begin{parameters}
  \begin{varlist}
    \vname{X,Y}: vectors with 2 components $X=[a,b]$ and $Y=[c,d]$ for
    the rectangle case or  $3 \times n$ arrays, respectively the
    abscissae and ordinates of the triangles.
    \vname{f}: nsp function
    \vname{tol=e}: optional named argument, the requested absolute or
    relative error on the result (default is 1e-10).
    \vname{tolflag=b}: optional named argument, boolean scalar. Should be
    \verb+%t+ if the requested tolerance \verb+tol+ concerns the absolute
     error and \verb+%f+ for a relative error request (default is \verb+%f+).
    \vname{args=obj}: optional named argument, useful to pass
    additional parameters needed by the function f.
    \vname{limit=maxtri}: optional named argument, max number of
    triangles allowed in the subdivision process of the initial set of
    triangles (default 100).
    \vname{iclose=b}: optional named argument, boolean scalar. The
    default is \verb+%t+ and the basic integration rule (Lyness and 
    Jespersen of degree 11) uses some integration points on the
    boundary of the triangle. With \verb+iclose=%f+, the basic
    integration rule (Lyness and Jespersen of degree 8) don't
    use integration points on the boundary.
    \vname{vecteval=b}: optional named argument, boolean scalar
    (default \verb+%f+), use \verb+%t+ if you write the function
    \verb+f+ such that it could be evaluated on vectors arguments 
    (this speed up the computation).
    \vname{Ia}: computed approximate result
    \vname{ea_estim}: estimated absolute error
    \vname{ier}: scalar given some information about the computation.
    \vname{neval}: number of function evaluations used for the computation.
  \end{varlist}
\end{parameters}

\begin{mandescription}
This function computes an approximation $I_a$ of:
$$
   I = \int_a^b \int_c^d f(x,y) dxdy
$$
or
$$
   I = \sum_k \int_{T_k} f(x,y) dxdy
$$
such that:
$$
   | I - I_a | \le  tol, \mbox{ or } | I - I_a | \le  tol*|I|
$$
depending on the \verb+tolflag+ optional parameter.
An additional parameters for the function $f$ could be passed using the
optional argument \verb+args+ which can be any nsp object (see
examples). \verb+ea_estim+ is an estimation of $| I - I_a |$. The
\verb+ier+ variable could take the following values: 
\begin{itemize}
\item ier = 0,  normal and reliable termination of the routine. It is assumed that the
      requested  accuracy has been achieved.
\item ier = 1: maximum number of triangles (the limit
      parameter) allowed has been achieved. You can try to
      increase limit.
\item ier = 2: the occurrence of roundoff error is detected
      which prevents the requested tolerance from being
      achieved ; the error may be under-estimated.
\item ier = 3: program ended with relative error less than five times the machine epsilon.
      (this can be obtained when the input tolerance is too stringent).
\item ier = 4: max number of function calls have been reached ;
      in the actual interface implementation this other limit is set
      from the \verb+limit=maxtri+ parameter and should rarely be
      reached. If this is the case you could increase the
      \verb+limit=maxtri+ parameter.
\end{itemize}

Rmk: the underlying routine (twodq) integrates only on triangles and
the rectangle case is handled in providing a simple triangulation of
the rectangle. 
\end{mandescription}

\begin{examples}
  
\paragraph{example 1} a simple example, integrating on the unit square: 
\begin{Verbatim}
function z=f(x,y);z=cos(x+y);endfunction
[Ia,ea,ier] = int2d([0,1],[0,1],f) 
I = -cos(2)+2*cos(1)-1 // exact result  
abs(Ia-I) //near exact absolute error
\end{Verbatim}

\paragraph{example 1bis} the same example using triangles
\begin{Verbatim}
function z=f(x,y);z=cos(x+y);endfunction
// the unit square is divided in 2 triangles:
X = [0,1;1,1;0,0];
Y = [0,0;0,1;1,1];
[Ia,ea,ier] = int2d(X,Y,f)
I = -cos(2)+2*cos(1)-1 // exact result  
abs(Ia-I) //near exact absolute error
\end{Verbatim}
  
\paragraph{example 2} a more difficult one's, always on the unit square:
\begin{Verbatim}
function z=f(x,y);z=cos(10*x+20*y);endfunction

I = (-cos(30)+cos(10)+cos(20)-1)/200 // exact result  

// first call 
[Ia,ea,ier] = int2d([0,1],[0,1],f)
abs(Ia-I)

// increase limit
[Ia,ea,ier] = int2d([0,1],[0,1],f,limit=1000)
abs(Ia-I)

// the function could be evaluated on vector x and y
// timing with scalar evaluation
tic(); [Ia,ea,ier] = int2d([0,1],[0,1],f,limit=1000); toc()
// timing with vectorial evaluation (this should be faster)
tic(); [Ia,ea,ier] = int2d([0,1],[0,1],f,limit=1000,vecteval=%t); toc()
\end{Verbatim}

\paragraph{example 3} integrating a singularity. Here we try to compute:
$$
    \int_T \frac{dxdy}{x+y} = 1, \; T = \{ x \ge 0, y \ge 0, x+y \le 1 \}
$$ 
In this case we could always use \verb+iclose=%t+ because the singularity is
located only at one vertex $(0,0)$:
\begin{Verbatim}
function z=f(x,y);z = 1./(x+y); endfunction
X = [0;1;0]; Y = [0;0;1];
[Ia,ea,ier] = int2d(X,Y,f,limit=1000)
abs(Ia-1)

// show once again the speed-up using vector evaluation of f
tic(); [Ia,ea,ier] = int2d(X,Y,f,limit=1000); toc()
tic(); [Ia,ea,ier] = int2d(X,Y,f,limit=1000,vecteval=%t); toc()
\end{Verbatim}
Rmk: a singularity at the origin could be removed using polar
coordinates but in case of the unit triangle it may be better
to use $x = u(1-v), y = uv$, $dxdy = u dudv$ because $u$ and $v$
belong in the unit square and so you can use directly int2d
(in example 3 using this transformation leads to integrate $f(u,v)=1$
on the unit square). 
\end{examples}

\begin{manseealso}
  \manlink{intg}{intg}, \manlink{int3d}{int3d}  
\end{manseealso}

% -- Authors
\begin{authors}
  uses the routine \verb!twodq! (D.K. Kahaner, O.W. Rechard, R. Barnhill)
  Nsp interface: Bruno Pincon.
\end{authors}
