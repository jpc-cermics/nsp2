% -*- mode: latex -*-

\mansection{quapro}
\begin{mandesc}
  \short{quapro}{linear quadratic programming solver}
\end{mandesc}
\index{quapro}\label{quapro}
\begin{calling_sequence}
  \begin{verbatim}
    [x,lagr,f]=quapro(Q,p,C,b [,x0,imp])  
    [x,lagr,f]=quapro(Q,p,C,b,ci,cs [,x0,imp])  
    [x,lagr,f]=quapro(Q,p,C,b,ci,cs,me [,x0,imp])  
    [x,lagr,f]=quapro(Q,p,C,b,ci,cs,me,x0 [,imp])  
  \end{verbatim}
\end{calling_sequence}
%-- Parameters
\begin{parameters}
  \begin{varlist}
    \vname{Q} : real symmetric matrix (dimension \verb!n x n!).
    \vname{p} : real (column) vector (dimension \verb! n!)
    \vname{C} : real matrix (dimension \verb! (me + md) x n!) (If no constraints are given, you can set \verb!C = []!)
   \vname{b} : RHS column vector (dimension \verb! (me + md)!) (If no constraints are given, you can set \verb!b = []!)
   \vname{ci} : column vector of lower-bounds (dimension \verb!n!). If there are no lower bound constraints, put \verb!ci = []!. If some components of \verb!x! are bounded from below, set the other (unconstrained) values of \verb!ci! to a very  large negative  number (e.g. \verb!ci(j) = -number_properties('huge')!.
   \vname{cs} : column vector of upper-bounds. (Same remarks as above).
   \vname{me} : number of equality constraints (i.e. \verb!C(1:me,:)*x = b(1:me)!)
   \vname{x0} : either an initial guess for \verb!x!    or one of the character strings \verb!'v'! or \verb!'g'!. If \verb!x0='v'! the calculated initial feasible point is a vertex. If \verb!x0='g'! the calculated initial feasible point is arbitrary.
   \vname{imp} : verbose option (optional parameter)   (Try \verb!imp=7,8,...!). warning the message are output in the window where scilab has been started.
   \vname{x} : optimal solution found.
   \vname{f} : optimal value of the cost function (i.e. \verb!f=0.5*x'*Q*x+p'!).
   \vname{lagr} : vector of Lagrange multipliers.  If lower and upper-bounds \verb!ci,cs! are provided, \verb!lagr! has  \verb!n + me + md! components and \verb!lagr(1:n)! is the Lagrange  vector associated with the bound constraints and  \verb!lagr (n+1 : n + me + md)! is the Lagrange vector associated  with the linear constraints. (If an upper-bound (resp. lower-bound) constraint \verb!i! is active  \verb!lagr(i)! is $>$ 0 (resp. $<$0). If no bounds are provided, \verb!lagr! has only \verb!me + md! components.
  \end{varlist}

\end{parameters}

\begin{mandescription}

Note that \verb!quapro! is not directly accessible in nsp. The \verb!quapro! 
interface is GPL but you have to get the in order to get the source code 
for  \verb!quapro!.

\begin{itemize}
\item \verb![x,lagr,f]=quapro(Q,p,C,b [,x0])!
\begin{equation*}
\begin{split}
  \min_{x}& \hspace{1cm}  \frac{1}{2} x^T Q x + p^T *x \\
  C x \le b  & \\
\end{split}
\end{equation*}
\item  \verb![x,lagr,f]=quapro(Q,p,C,b,ci,cs [,x0])!
\begin{equation*}
\begin{split}
  \min_{x} &\hspace{1cm}  \frac{1}{2} x^T Q x + p^T *x \\
  C x \le b  & \\
  ci \le x \le cs & \\
\end{split}
\end{equation*}
\item \verb![x,lagr,f]=quapro(Q,p,C,b,ci,cs,me [,x0])!
\begin{equation*}
\begin{split}
  \min_{x} &\hspace{1cm}  \frac{1}{2} x^T Q x + p^T *x \\
  C(j,:) x = b(j) , \,  j \in [1,me] & \\
  C(j,:) x \le b(j), \, j \in [me+1,md] & \\
  ci \le x \le cs &\\
\end{split}
\end{equation*}
\end{itemize}
  If no initial point is given the
  program computes a feasible initial point
  which is a vertex of the region of feasible points if
  \verb!x0='v'!.
  If \verb!x0='g'!, the program computes a feasible initial 
  point which is not necessarily a vertex. This mode is
  advisable when the quadratic form is positive
  definite and there are  few constraints in
  the problem or when there are large bounds
  on the variables that are just security bounds and
  very likely not active at the optimal solution.
  Note that \verb!Q! is not necessarily non-negative, i.e.
  \verb!Q! may have negative eigenvalues.
\end{mandescription}
\begin{examples}
  \begin{program}\HCode{//Find x in R^6 such that:\Hnewline
      //C1*x = b1 (3 equality constraints i.e me=3)\Hnewline
      C1= [1,-1,1,0,3,1;\Hnewline
	-1,0,-3,-4,5,6;\Hnewline
	2,5,3,0,1,0];\Hnewline
      b1=[1;2;3];\Hnewline
      //C2*x $<$= b2 (2 inequality constraints)\Hnewline
      C2=[0,1,0,1,2,-1;\Hnewline
	-1,0,2,1,1,0];\Hnewline
      b2=[-1;2.5];\Hnewline
      //with  x between ci and cs:\Hnewline
      ci=[-1000;-10000;0;-1000;-1000;-1000];cs=[10000;100;1.5;100;100;1000];\Hnewline
      //and minimize 0.5*x'*Q*x + p'*x with\Hnewline
      p=[1;2;3;4;5;6]; Q=eye(6,6);\Hnewline
      //No initial point is given;\Hnewline
      C=[C1;C2] ; //\Hnewline
      b=[b1;b2] ;  //\Hnewline
      me=3;\Hnewline
      [x,lagr,f]=quapro(Q,p,C,b,ci,cs,me)\Hnewline
      //Only linear constraints (1 to 4) are active (lagr(1:6)=0):\Hnewline
      [x,lagr,f]=quapro(Q,p,C,b,[],[],me)   //Same result as above\Hnewline}
  \end{program} 
\end{examples}
%-- see also
\begin{manseealso}
%%  \manlink{linpro}{linpro} \manlink{optim}{optim}  
\end{manseealso}

\begin{authors}
  \paragraph{Eduardo Casas  Renteria} , Universidad de Cantabria,
  \paragraph{Cecilia Pola Mendez}  , Universidad de Cantabria 
\end{authors}
