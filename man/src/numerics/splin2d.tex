% -*- mode: latex -*-

\mansection{splin2d}
\begin{mandesc}
  \short{splin2d} {bicubic spline or subspline (interpolation of 2d
    gridded datas)}
\end{mandesc}

%-- Calling sequence section
\begin{calling_sequence}
\begin{verbatim}
C = splin2d(x, y, z, [,spline_type])
\end{verbatim}
\end{calling_sequence}

% -- Parameters
\begin{parameters}
  \begin{varlist}
   \vname{x,y} : strictly increasing vectors (with at least 2 components)
               defining the interpolation grid
   \vname{z} : nx x ny matrix (nx being the length of \verb!x! and 
               ny the length of \verb!y!)
   \vname{spline\_type } : (optionnal) a string selecting the kind of bicubic spline to compute
   \vname{C} : a big vector with the coefficients of the bicubic patches
  \end{varlist}
\end{parameters}

\begin{mandescription}
    This function computes a bicubic spline or sub-spline $s$ which interpolates the 
    $(x_i,y_j,z_{ij})$ points, ie, we have $s(x_i,y_j)=z_{ij}$ for all  $i=1,..,nx$
    and $j=1,..,ny$. 
    The resulting spline $s$ is defined by the triplet $(x,y,C)$ where
    $C$ is the vector (of length 16(nx-1)(ny-1)) with the coefficients of each
    of the (nx-1)(ny-1) bicubic patches : on $[x_i, x_{i+1}] \times
    [y_j, y_{j+1}]$, $s$ is defined by :
    $$
       s(x,y) = \sum_{k=1}^4  \sum_{l=1}^4  C_{i,j}(k,l) (x - x_i)^k
       (y - y_j)^l
    $$

    The evaluation of  $s$ at some points must be done by the \manlink{interp2d}{interp2d}  
    function. Several kind of splines may be computed by selecting the appropriate 
    \verb!spline_type! parameter. The method used to compute the bicubic spline
    (or sub-spline) is the old fashionned one 's, i.e. to compute on each grid point
    $(x_i,y_j)$ an approximation of the first derivatives $ds/dx(x_i,y_j)$
    and $ds/dy(x_i,y_j)$ and of the cross derivative $d2s/dxdy(x_i,y_j)$.
    Those derivatives are computed by the mean of 1d spline schemes leading to a C2
    function ($s$ is twice continuously differentiable) or by the mean of a
    local approximation scheme leading to a C1 function only. This scheme is selected
    with the \verb!spline_type! parameter (see \manlink{splin}{splin}  for details) :  
    
  \begin{itemize}

       \item \itemdesc{"not\_a\_knot"} : this is the default case.
       \item \itemdesc{"natural"} 
       \item \itemdesc{"periodic"} : to use if the underlying function is periodic : you must have {\em z(1,j) = z(nx,j) for
              all j in [1,ny] and  z(i,1) = z(i,ny) for i in [1,nx]} but this is not verified
              by the interface.
       \item \itemdesc{"monotone"} 
       \item \itemdesc{"fast"} 
       \item \itemdesc{"fast\_periodic"}  
  \end{itemize}

  \end{mandescription}

  %-- section-Remarks

  \paragraph{Remarks}
From an accuracy point of view use essentially the {\em not\_a\_knot} type or 
the {\em periodic} type if the underlying interpolated function is periodic.
The {\em monotone}, {\em fast},  (or {\em fast\_periodic}) type may
be useful in some cases, for instance to limit oscillations ({\em monotone} being the
most powerfull for that purpose).
    

% --example 
\begin{examples}
\paragraph{example 1} interpolation of cos(x)cos(y). 
\begin{program}\HCode{n = 7;  // a regular grid with n x n interpolation points will be used;\Hnewline
x = linspace(0,2*\%pi,n); y = x;\Hnewline
z = cos(x')*cos(y);\Hnewline
C = splin2d(x, y, z, "periodic");\Hnewline
m = 50; // discretisation parameter of the evaluation grid\Hnewline
xx = linspace(0,2*\%pi,m); yy = xx;\Hnewline
[XX,YY] = ndgrid(xx,yy);\Hnewline
zz = interp2d(XX,YY, x, y, C);\Hnewline
emax = max(abs(zz - cos(xx')*cos(yy)));\Hnewline
xbasc()\Hnewline
plot3d(xx, yy, zz, flag=[2 4 4])\Hnewline
[X,Y] = ndgrid(x,y);\Hnewline
param3d1(X,Y,z,style=-9*ones(1,n), flag=[0 0])\Hnewline
str = sprintf(" with }\verb+%+\HCode{d x }\verb+%+\HCode{d interpolation points. ermax = }\verb+%+\HCode{g",n,n,emax) \Hnewline
xtitle("spline interpolation of cos(x)cos(y)"+str)}
\end{program}

\paragraph{example 2} different interpolation functions on random datas
\begin{program}\HCode{n = 6;\Hnewline
x = linspace(0,1,n); y = x;\Hnewline
z = rand(n,n);\Hnewline
np = 50;\Hnewline
xp = linspace(0,1,np); yp = xp;\Hnewline
[XP, YP] = ndgrid(xp,yp);\Hnewline
ZP1 = interp2d(XP, YP, x, y, splin2d(x, y, z, "not_a_knot"));\Hnewline
ZP2 = linear_interpn(XP, YP, x, y, z);\Hnewline
ZP3 = interp2d(XP, YP, x, y, splin2d(x, y, z, "natural"));\Hnewline
ZP4 = interp2d(XP, YP, x, y, splin2d(x, y, z, "monotone"));\Hnewline
xset("colormap", jetcolormap(64))\Hnewline
xbasc()\Hnewline
subplot(2,2,1)\Hnewline
   plot3d1(xp, yp, ZP1, flag=[2 2 4])\Hnewline
   xtitle("not_a_knot")\Hnewline
subplot(2,2,2)\Hnewline
   plot3d1(xp, yp, ZP2, flag=[2 2 4])\Hnewline
   xtitle("bilinear interpolation")\Hnewline
subplot(2,2,3)\Hnewline
   plot3d1(xp, yp, ZP3, flag=[2 2 4])\Hnewline
   xtitle("natural")\Hnewline
subplot(2,2,4)\Hnewline
   plot3d1(xp, yp, ZP4, flag=[2 2 4])\Hnewline
   xtitle("monotone")\Hnewline
xselect()}
\end{program}

\paragraph{example 3} ``not a knot'' spline and monotone sub-spline on a step function
\begin{program}\HCode{a = 0; b = 1; c = 0.25; d = 0.75;\Hnewline
// create interpolation grid\Hnewline
n = 11;\Hnewline
x = linspace(a,b,n);\Hnewline
ind = find(c }\verb+<+\HCode{= x  }\verb+&+\HCode{  x }\verb+<+\HCode{= d); \Hnewline
z = zeros(n,n); z(ind,ind) = 1;  // a step inside a square\Hnewline
// create evaluation grid\Hnewline
np = 220;\Hnewline
xp = linspace(a,b, np);\Hnewline
[XP, YP] = ndgrid(xp, xp);\Hnewline
zp1 = interp2d(XP, YP, x, x, splin2d(x,x,z));\Hnewline
zp2 = interp2d(XP, YP, x, x, splin2d(x,x,z,"monotone"));\Hnewline
// plot\Hnewline
xbasc()\Hnewline
xset("colormap",jetcolormap(128))\Hnewline
subplot(1,2,1)\Hnewline
   plot3d1(xp, xp, zp1, flag=[-2 6 4])\Hnewline
   xtitle("spline (not_a_knot)")\Hnewline
subplot(1,2,2)\Hnewline
   plot3d1(xp, xp, zp2, flag=[-2 6 4])\Hnewline
   xtitle("subspline (monotone)")}
\end{program}

\end{examples}

  %-- see also

\begin{manseealso}
\manlink{splin}{splin}  \manlink{interp2d}{interp2d} \manlink{linear\_interpn}{linear_interpn}  
\end{manseealso}

  %-- Author

\begin{authors}
B. Pincon
\end{authors}

