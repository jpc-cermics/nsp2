% -*- mode: latex -*-

\mansection{ode}
\begin{mandesc}
  \short{ode}{ordinary differential equations solvers}
\end{mandesc}

% -- Calling sequence section
\begin{calling_sequence}
\begin{verbatim}
x = ode(x0,t0,t,f) // or  x = ode(x0,t0,t,f,task=1)
[x,tt] = ode(x0, t0, t, f, task=2|3)
[x, ier] = ode(x0,t0,t,f, task=1, type=method, atol=ea, rtol=er, jac=Df, args=fpar, odeoptions=htable, warn=b)
[x, tt, ier] = ode(x0,t0,t,f, task=2|3, type=method, atol=ea, rtol=er, jac=Df, args=obj, odeoptions=htable, warn=b)
\end{verbatim}
\end{calling_sequence}
% -- Parameters
\begin{parameters}
  \begin{varlist}
    \vname{x0}: real vector, initial condition.
    \vname{t0}: real scalar, initial time.
    \vname{t}: when task=1, \verb+t+ should a real vector with time instants at which the solution must
    be computed and recorded (final integration time is \verb+t($)+); when task=2 or 3, \verb+t+ is a 
    scalar giving the final integration time.
    \vname{f}: nsp function or string (name of a C or fortran function dynamicaly linked to nsp) 
               defining the vector field of the differential equation.
    \vname{task=num}: output mode, num could be equal to 1 (the default), 2 or 3. See explanations hereafter.
    \vname{type=method}: ode solver to use, method is a string among  "default", "adams", "stiff" or "rkd5".
    \vname{atol=ea, rtol=er}: optional named argument, relative to the requested maximum error on a time step 
    (default values depends on the solver used). \verb+atol+ can be a vector of same size as 
    the vector \verb+x0+.
    \vname{jac=Df}: \verb+Df+ should be a nsp function or a string (name of a C or fortran function dynamicaly linked 
                    to nsp) defining the Jacobian of function \verb+f+.
    \vname{args=fpar}: optional named argument used to pass additional parameters to the function \verb+f+ 
    (and to the function \verb+Df+ if it is provided). 

    \vname{odeoptions}: a hash table of parameters for fine manual tuning of the choosen ode solver.
    \vname{warn=b}: boolean scalar, default is true. Use \verb+warn=%f+ to remove the display of warning 
                    informations from the ode solvers in the nsp console.
    \vname{x}: solution at time instants \verb+t+ choosen by the user (task=1) or at each time step (task=2 or 3)
    \vname{tt}: time instants choosen by the solver (task=2 or 3).
    \vname{ier}: error/warning indicator 
  \end{varlist}
\end{parameters}

\begin{mandescription}

This function computes an approximate solution of the ordinary differential equation:
$$
\begin{array}{l}
   \dot{x}(t) = f(t, x(t)), \; f: \R \times \R^d \rightarrow  \R^d \\
   x(t_0) = x_0  \in  \R^d
\end{array}
$$
using various solvers (choosen by named optional \verb+type+ argument) 
\begin{itemize}
\item "default": lsoda solver (from the Livermore solvers odepack collection). Default tolerances: atol=1e-9,
  rtol=1e-7. This one could handle both stiff and non stiff ode.
\item "adams": lsode solver (from the Livermore solvers odepack collection) using adams method. Default tolerances:
  atol=1e-9, rtol=1e-7. To use with non stiff equations (can be used on middle stiff ode if the jacobian is provided).
\item "stiff": lsode solver (from the Livermore solvers odepack collection) using bdf method. Default tolerances:
  atol=1e-9, rtol=1e-7. To use with stiff ode.
\item "rkd5": dopri5 (Runge-Kutta) solver. Default tolerances: atol=1e-6, rtol=1e-4. To use with non stiff ode (this
solver has a stiff test and so you could be warned in case your ode is stiff then switch to \verb+type="stiff"+).
\end{itemize}


\itemdesc{header of the function {\tt f}}
The function $f$ takes as first argument a scalar $t$, then a vector $x$ of length $d$ as 2d argument and 
must output a vector of length $d$. Note that even if $f$ doesn't depend explicitly of the time $t$ (case of 
an autonomous differential equation) it should nevertheless have this parameter as first argument. 
In brief \verb+f+ should have the form:
\begin{quote}
{\tt function y = f(t,x) \\
      ....}
  \end{quote}
It is possible to code this function in C or fortran language, see examples. 

\itemdesc{optional argument {\tt args}}
Additional parameters for the function $f$ (and also for the function which computes the Jacobian 
if it is provided by the user) could be passed using \verb+args=fpar+ where \verb+fpar+ is an array of
cells with the additional parameters :
If \verb+f+ has the form:
\begin{quote}
{\tt function y = f(t,x,p1,...,pn) \\
      ....}
\end{quote}
then \verb+args={p1,...,pn}+. Note that if $f$ has only one additional parameter 
it is not mandatory to encapsulate it in a cell array (except when the parameter is it 
self a cell array !). The additional parameters could be any nsp object. 

\itemdesc{optional {\tt atol} and {\tt rtol} parameters: solver accuracy control}
At each time step taken by the solver says from $\tau_{k-1}$ to  $\tau_k = \tau_{k-1} + (\Delta t)_k$
the estimated local error in the i th component $x_i(\tau_k)$ (that is the error in 
$x_i(\tau_k)$ assuming that $x(\tau_{k-1})$ is exact) is controlled so as to be less 
than $etol_i$ where:
$$
etol_i =
\left\{
\begin{array}{l}
 rtol | x_i | + atol  \mbox{ if } atol \mbox { is a scalar} \\
 rtol | x_i | + atol_i  \mbox{ if } atol \mbox { is a vector}
\end{array}
\right.
$$

\itemdesc{output mode {\tt task} and arguments {\tt t} and {\tt tt}}
Each solver integrates the differential equation from \verb+t0+ in
direction of $T$ ($T$ the last or unique component of \verb+t+) taking 
time steps $(\Delta t)_k$ (such as the local error tolerance is respected) 
until it overshoots $T$ (note that integration is forward in time if 
$T>t0$ and backward in time if $T<t0$). The solution can be recorded
differently depending on the parameter task:
\begin{description}
\item[task=1] In this case the approximate solution is output
at the time instants specified in the vector {\tt t}, so \verb+x+ is
a $d \times length(t)$ array (the values are obtained by an interpolation
formula specific to each solver).

\item[task=2 or 3] In this case the approximate solution is given at each
time instants taken by the solver. The slight difference is that the last
value for \verb+task=2+ is the first time the solver overshoots $T$
while for \verb+task=3+ the last value is computed exactly at  $T$
(by interpolation as for task=1) and so \verb+tt($)+$=T$. Notes that the 
initial time \verb+t0+ is the first component of the output times \verb+tt+
and the intial solution \verb+x0+ are the first column of the output
solution \verb+x+.    
\end{description}

\itemdesc{optional jac parameter} Depending on the solver used, the jacobian 
\verb+Df+, of $x \mapsto f(t,x)$ could be required and this function can be 
provided by the user with the optional named argument \verb+jac+. When 
\verb+jac+ is not given (and the jacobian is needed) the solver computes an 
approximation using finite differences. Note that, like for function \verb+f+, 
the formal argument \verb+t+ should be present even if
the Jacobian does not explicitly depend on time. So the header of
the nsp function defining \verb+jac+, the Jacobian of \verb+f+ should be as follows:
\begin{quote}
{\tt function J = Df(t,x) \\
      ....}\\
or:\\
{\tt function J = Df(t,x,p1,...,pn) \\
     ....}
\end{quote}
if additional parameters should be provided (must be the same than for \verb+f+).
It is possible to code this function in C or fortran language, see examples. 

\itemdesc{odeoptions: fine solver tuning}
 It is an hash table which lets to give values to some solvers parameters. 
Common parameters among the solvers are:
\begin{description}
\item[mxstep=value] this sets the maximal number of time steps a solver can take ; default
values are 10000 for lsode,lsoda and 20000 for dopri5. If you encounter such a limitation
you can use \verb+odeoptions=hash(mxstep=50000)+ for instance.
\item[h0=value] this sets the first step size ; don't use this option if you have not a precise
idea, by default each solver have some algorithm to determine some good initial step size.
\end{description}


\itemdesc{error indicator {\tt ier}}
The output variable {\tt ier} can give (together with the warning messages from
the solvers) some information about possible failures of the solver. Note that
if a failure occurs during the integration process while {\tt ier} is required
as output argument, then {\tt ode} returns what have been computed (solution
is output for time instants before the failure). Otherwise ({\tt ier} not in
output arguments) an error is set. 
\begin{description}
\item[ier=0] no error
\item[ier=1] more than mxstep are needed to perform integration (you can increase
             mxstep using odeoptions)
\item[ier=2] excess accuracy requested (the tolerance atol and rtol are too stringent)
\item[ier=3] illegal input detected (see the warning message)
\item[ier=4] repeated error test failure
\item[ier=5] repeated convergence failure
\item[ier=6] an error weight became zero (occurs when atol/atol(i) is zero while
             the component i of the solution is zero).
\item[ier=7] the problem is stiff, or become stiff (this can arise when using dopri5 ;
             in this case use stiff or default).
\item[ier=8] time step become too small
\item[ier=9] evaluation of the rhs function $f$ (or its jacobian) has failed
\end{description}


  

\end{mandescription} 

\begin{examples}
  
\paragraph{example 1} A very simple example. we want to solve the system:
$$
  \dot{x}(t) = - x, \; x(0) = 1\,.
$$ 
The solution is known to be ($x(t) = exp(-t)$);
\begin{mintednsp}{nsp}
function y=f1(t,x);y=-x;endfunction

// output solution at specific time (task=1 (default))
t = linspace(0,4,10);
x = ode(1,0,t,f1);
// error
x - exp(-t)

// output solution at time choosen by the solver (task=2)
[x,tt] = ode(1,0,4,f1,task=2);
x - exp(-tt)

// same thing using the dopri5 solver
[x,tt] = ode(1,0,4,f1,task=2,type="rkd5")
x - exp(-tt)
\end{mintednsp}
  
  
\paragraph{example 2} The Brusselator. We want to solve:
$$
\left\{
\begin{array}{l}
\frac{du_1}{dt} = 2 - (6 + \epsilon) u_1 + u_1^2 u_2 \\
\frac{du_2}{dt} = (5 + \epsilon) u_1 - u_1^2 u_2
\end{array}
\right.
$$
for some initial condition $u(0) = u0$ and some parameter $\epsilon$.
\begin{mintednsp}{nsp}
function [f] = Brusselator(t,u,eps)
    f = [ 2 - (6+eps)*u(1) + u(1)^2*u(2);...
         (5+eps)*u(1) - u(1)^2*u(2)]
endfunction
eps = 0.1 // parameter value
T = 200 ;  // final integration time
t = linspace(0,T,8000); // time instants to record to solution
u0 = [0;0];  // initial condition (starting point)
// solve the ode (only one additional parameter so using args={eps} is not mandatory)
u = ode(u0, 0, t, Brusselator, rtol = 1e-9, atol = 1e-10, args=eps);
// plot solution in the phase space
plot2d(u(1,:)',u(2,:)', style=2, rect=[-0.5,-0.5,6,6])
plot2d(u0(1),u0(2),style=-5, strf="000",leg='starting point')
\end{mintednsp}
  
  
\paragraph{example 3} pure, damped or amplified oscillators. The second order differential
equation:
$$
  \ddot{x}(t) + 2 \alpha \dot{x}(t) + k x(t) = 0, \; x(0)=x_0, \dot{x}(0) = v_0 
$$ 
corresponds for $k > 0$ and $\alpha^2 < k$ (damping or amplification factor not too high) to 
a pure oscillator when $\alpha=0$, a damped oscillator when $\alpha>0$ and to an amplified 
oscillator for $\alpha<0$. Its is easily solved analytically:
$$
  x(t) = e^{-\alpha t}( A \cos(\omega t) + B \sin(\omega t) ) \; \mbox{with: }
\left\{ \begin{array}{l}
\omega = \sqrt{k - \alpha^2} \\
A = x_0, \; B = (v_0 + \alpha x_0)/\omega
\end{array} \right.
$$
To use {\tt ode}, one should put this equation in a system of two first order differential 
equations:
$$
u(t) =  \left[ \begin{array}{c} u_1(t) \\  u_2(t)\end{array} \right]  \equiv  \left[ \begin{array}{c} x(t) \\
    \dot{x}(t)\end{array} \right], \mbox{ and }
\dot{u}(t) =  \left[ \begin{array}{c} \dot{u}_1(t) \\ \dot{u}_2(t) \end{array} \right] =
    \left( \begin{array}{c} u_2(t) \\  -k u_1(t) - 2 \alpha u_2(t) \end{array} \right) = f(u)
$$
\begin{mintednsp}{nsp}
function [f] = oscillator(t,u,k,alpha)
   f = [ u(2); -k*u(1) - 2*alpha*u(2)];
endfunction
T = 16 ;  // final integration time
t = linspace(0,T,1000); // time instants to record to solution
u0 = [1;0];  // initial condition (x(0)=1, dx/dt(0)=0)
// solve the ode
up = ode(u0, 0, t, oscillator, args={1,0});
ud = ode(u0, 0, t, oscillator, args={1,0.02});
ua = ode(u0, 0, t, oscillator, args={1,-0.02});
// plot the 3 solutions
xbasc()
subplot(1,2,1)
plot2d([up(1,:)',ud(1,:)',ua(1,:)'],[up(2,:)',ud(2,:)',ua(2,:)'], style=[2,3,5], leg="pure@damped@amplified")
xtitle("trajectories in phase space","x(t)","dx/dt(t)")
subplot(2,2,2)
plot2d(t',[up(1,:)',ud(1,:)',ua(1,:)'],style=[2,3,5], leg="pure@damped@amplified")
xtitle("position fct of time","t","x(t)")
subplot(2,2,4)
plot2d(t',[up(2,:)',ud(2,:)',ua(2,:)'],style=[2,3,5], leg="pure@damped@amplified")
xtitle("velocity fct of time","t","dx/dt(t)")
\end{mintednsp}
 
\paragraph{example 4} using a function coded in C. To speed up the computation it is possible to code the rhs
function (and its Jacobian if needed) in C or fortran language. To be done !


\end{examples}

\begin{manseealso}
  \manlink{intg}{intg}
\end{manseealso}

% -- Authors
\begin{authors}
 odepack: Linda r. Petzold, Alan C. Hindmarsh. dopri5: H. Hairer (translated in C and adapted/modified
for nsp by B. Pincon).  Nsp interface: Jean-Philippe Chancelier and Bruno Pincon
\end{authors}
