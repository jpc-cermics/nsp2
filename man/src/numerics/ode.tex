% -*- mode: latex -*-

\mansection{ode}
\begin{mandesc}
  \short{ode}{ordinary differential equations solver}
\end{mandesc}

% -- Calling sequence section
\begin{calling_sequence}
\begin{verbatim}
x = ode(x0,t0,t,f)
x = ode(x0,t0,t,f, atol=ea, rtol=er, jac=Df, args=obj, type=str, odeoptions=htable)
\end{verbatim}
\end{calling_sequence}
% -- Parameters
\begin{parameters}
  \begin{varlist}
    \vname{x0}:real column vector, the initial condition
    \vname{t0}:real scalar, the initial time
    \vname{t}: real vector with increasing time instants at which the solution must be recorded
    \vname{f}: nsp function defining the second member of the differential equation
    \vname{atol=ea, rtol=er}: optional named argument, relative to the requested maximum error on a time step 
                     (default is atol=1e-9, rtol=0). atol could be a vector of same size than x0.
    \vname{jac=Df}: optional named argument. Df should be a nsp function (the Jacobian of f)
    \vname{args=obj}: optional named argument (could be any nsp object), useful to pass additional parameters needed by the function f.
    \vname{type=str}: optional named argument (scalar string) could be "default" (which is the default) or "discrete"
    \vname{odeoptions}: an hash table of parameters for fine tuning of the solver.
  \end{varlist}
\end{parameters}

\begin{mandescription}

When type is not "discrete", this function computes an approximate solution of the ordinary differential equation:
$$
\begin{array}{l}
   \dot{x}(t) = f(t, x(t)), \; f: R \times R^d \rightarrow  R^d \\
   x(t_0) = x_0  \in  R^d
\end{array}
$$
using the Livermore solvers odepack. Currently only lsoda is interfaced and this solver begin with the Adams
method (for nonstiff systems) but switches  automatically to bdf method if the differential equation is detected
as stiff. The function $f$ takes on entry a scalar $t$, then a vector $x$ of length $d$ and output a vector of 
length $d$. Note that even if $f$ doesn't depend explicitly of the time $t$ (case of an autonomous differential equation)
the nsp function should have this parameter as first argument. In fact \verb+f+ should have the
form:
\begin{quote}
{\tt function y = f(t,x) \\
      ....}
  \end{quote}

Additional parameters for the function $f$ (and also for the function which computes the Jacobian 
if it is provided by he user) could be passed using \verb+args=obj+ where obj could be any nsp 
object (see examples). In this case \verb+f+ should have the form:
\begin{quote}
{\tt function y = f(t,x,obj) \\
      ....}
\end{quote}

The (approximate) solution is got at the time instants defined by the argument \verb+t+.
If \verb+t+ as $n$ components says  \verb+t = [t1, t2, ..., tn]+ then the output is a
$d \times n$ array and the k th column is an approximation of $x(tk)$.

At each time step the estimated local error in $x_i$ is controlled so as to be less than:
\begin{quote}
   $ewt_i = rtol | x_i | + atol$ ~ if atol is a scalar \\
   $ewt_i = rtol | x_i | + atol_i$ ~ if atol is a vector
\end{quote}
 

The solver needs $Df$ the jacobian of $x \mapsto f(t,x)$ so you could provide it with the optional named argument 
jac. When jac is not given the solver uses finite differences. Note that, like for \verb+f+, the formal argument
\verb+t+ should be present even if the Jacobian don't depends explicitly of the time. So the header of the nsp 
function defining the Jacobian of $f$ should be of the form:
\begin{quote}
{\tt function J = Df(t,x) \\
      ....}\\
or:\\
{\tt function J = Df(t,x,obj) \\
     ....}
\end{quote}
when {\tt obj} is provided with {\tt args=obj}.



odeoptions will be described in a future version of this help page.

\end{mandescription} 

\begin{examples}
  
\paragraph{example 1} the most simple example: we want to integrate:
$$
  \dot{x}(t) = - x, \; x(0) = 1, 
$$ 
the solution is known ($x(t) = exp(-t)$);
\begin{program}\HCode{function y=f(x);y=-x;endfunction\Hnewline
t = linspace(0,4,10)
x = ode(1,0,t,f1)\Hnewline
exp(-t)}
\end{program}
 
\end{examples}

\begin{manseealso}
  \manlink{intg}{intg}, \manlink{int2d}{int2d}  
\end{manseealso}

% -- Authors
\begin{authors}
  uses the routine lsoda of the odepack package. Nsp interface: Jean-Philippe Chancelier
\end{authors}
