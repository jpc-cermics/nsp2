% -*- mode: latex -*-

\mansection{interp}
\begin{mandesc}
  \short{interp}{cubic spline evaluation function}
\end{mandesc}

%-- Calling sequence section
\begin{calling_sequence}
    \begin{verbatim}
[yp [,yp1 [,yp2 [,yp3]]]]=interp(xp, x, y, d [, out_mode])
    \end{verbatim}
\end{calling_sequence}

%-- Parameters
\begin{parameters}
  \begin{varlist}
   \vname{xp}: real vector or matrix
   \vname{x,y,d}: real vectors of the same size defining a cubic spline or sub-spline function 
                (called $s$ in the following)
   \vname{out\_mode}: (optional) string defining the evaluation of $s$ outside the $[x_1,x_n]$ interval
   \vname{yp}: vector or matrix of same size than \verb!xp!, elementwise evaluation of $s$
                on \verb!xp! (\verb!yp(i)=s(xp(i)! or \verb!yp(i,j)=s(xp(i,j)!)
   \vname{yp1, yp2, yp3}: vectors (or matrices) of same size than \verb!xp!, elementwise evaluation of the 
                successive derivatives of $s$ on \verb!xp!
  \end{varlist}
\end{parameters}

\begin{mandescription}
    Given three vectors $(x,y,d)$ defining a spline or sub-spline function
    (see \manlink{splin}{splin} ) with $y_i=s(x_i)$, $d_i = s'(x_i)$ this function
    evaluates $s$ (and $s'$, $s''$, $s'''$ if needed) at \verb!xp(i)!:
    
    \begin{verbatim}
      yp(i) = s(xp(i))    or  yp(i,j) = s(xp(i,j))
      yp1(i) = s'(xp(i))   or  yp1(i,j) = s'(xp(i,j))
      yp2(i) = s''(xp(i))   or  yp2(i,j) = s''(xp(i,j))
      yp3(i) = s'''(xp(i))   or  yp3(i,j) = s'''(xp(i,j))
    \end{verbatim}

    The \verb!out_mode! parameter set the evaluation rule for extrapolation,
    i.e. for $xp(i)  \notin [x_1,x_n]$: 
    
  \begin{itemize}

       \item \itemdesc{"by\_zero"} an extrapolation by zero is done
       \item \itemdesc{"by\_nan"} extrapolation by Nan
       \item \itemdesc{"C0"} the extrapolation is defined as follows:
      $$
      \begin{array}{ll}
      s(x) = y_1  & \mbox{ if }  x < x_1 \\
      s(x) = y_n  & \mbox{ for } x > x_n
      \end{array}
      $$   

       \item \itemdesc{"natural"} the extrapolation is defined as follows ($p_i$ being the polynomial defining 
              $s$ on $[x_i,x_{i+1}]$):
      $$
      \begin{array}{ll}
      s(x) = p_1(x)     &  \mbox{ if }  x < x_1 \\
      s(x) = p_{n-1}(x) &  \mbox{ if }  x > x_n
      \end{array}
      $$   

       \item \itemdesc{"linear"} the extrapolation is defined as follows:
      $$
      \begin{array}{ll}
      s(x) = y_1 + s'(x_1)(x-x_1) &  \mbox{ if }  x < x_1 \\
      s(x) = y_n + s'(x_n)(x-x_n) &  \mbox{ if }  x > x_n
      \end{array}
      $$   

       \item \itemdesc{"periodic"} $s$ is extended by periodicity. 
  \end{itemize}

  \end{mandescription}

  %--example 
  \begin{examples}

\paragraph{Note} See first basic examples on the  \manlink{splin}{splin} help page.

\paragraph{example 1}  illustration of $C^2$ and $C^1$ continuity of spline and subspline
\begin{program}\HCode{a = -8; b = 8;\Hnewline
function y=sinc(x),k=find(x==0),y=sin(x)./x,y(k)=1,endfunction\Hnewline
x = linspace(a,b,20)';\Hnewline
y = sinc(x);\Hnewline
dk = splin(x,y);  // not_a_knot\Hnewline
df = splin(x,y, "fast");\Hnewline
xx = linspace(a,b,800)';\Hnewline
[yyk, yy1k, yy2k] = interp(xx, x, y, dk); \Hnewline
[yyf, yy1f, yy2f] = interp(xx, x, y, df); \Hnewline
xbasc()\Hnewline
subplot(3,1,1)\Hnewline
plot2d(xx, [yyk yyf], style=[2,5], leg="not_a_knot spline@fast sub-spline")\Hnewline
plot2d(x, y, style=-9, leg="interpolation points", leg_pos="ul")\Hnewline
xtitle("spline interpolation")\Hnewline
subplot(3,1,2)\Hnewline
plot2d(xx, [yy1k yy1f], style=[2,5], leg="not_a_knot spline@fast sub-spline")\Hnewline
xtitle("spline interpolation (derivatives)")\Hnewline
subplot(3,1,3)\Hnewline
plot2d(xx, [yy2k yy2f], style=[2,5], leg="not_a_knot spline@fast sub-spline")\Hnewline
xtitle("spline interpolation (second derivatives)")\Hnewline
}
\end{program}

\paragraph{example 2} illustration of the different extrapolation possibilities
\begin{program}\HCode{x = linspace(0,1,11)';\Hnewline
y = cosh(x-0.5);\Hnewline
d = splin(x,y);\Hnewline
xx = linspace(-0.5,1.5,401)';\Hnewline
yy0 = interp(xx,x,y,d,"C0");\Hnewline
yy1 = interp(xx,x,y,d,"linear");\Hnewline
yy2 = interp(xx,x,y,d,"natural");\Hnewline
yy3 = interp(xx,x,y,d,"periodic"); \Hnewline 
xbasc()\Hnewline
plot2d(xx,[yy0 yy1 yy2 yy3],style=2:5,frameflag=2,leg="C0@linear@natural@periodic")\Hnewline
xtitle(" different way to evaluate a spline outside its domain")\Hnewline
} 
\end{program}

\end{examples}

%-- see also
\begin{manseealso}
\manlink{splin}{splin} %\manlink{lsq\_splin}{lsq_splin}  
\end{manseealso}

%-- Author
\begin{authors}
    Bruno Pincon
\end{authors}

