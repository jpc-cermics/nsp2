% -*- mode: latex -*-

\mansection{xgetfile}
\begin{mandesc}
  \short{xgetfile}{dialog to get a file or folder path}\\ % @mandesc@
\end{mandesc}
\index{xgetfile}\label{xgetfile}
%-- Calling sequence section
\begin{calling_sequence}
  \begin{verbatim}
    path=xgetfile(title='string',dir='dirname',masks=SMat,action=%t|%f,save=%t|%,open=%t|%f,folder=%t|%f)
  \end{verbatim}
\end{calling_sequence}
%-- Parameters
\begin{parameters}
  \begin{varlist}
    \vname{masks}: a string matrix which gives the file mask to be used as filters for file selection. This matrix should be of size \verb+2xm+, the first line 
gives a description of each filter and the second line gives each filter. 
filters are given with Unix convention, the default value is '*' which matches all files.
    \vname{dir}: a character string which gives the initial directory used in the dialog. 
    \vname{path}: the selected file or folder path or an empty string if user selects the \verb+Cancel+ 
    button.
    \vname{title='string'}: optional arguments which gives a title for the xgetfile widget.
  \end{varlist}
\end{parameters}
\begin{mandescription}
  Creates and open a dialog window for file or folder selection. One of the optional arguments 
  \verb+save+, \verb+open+, \verb+folder+ can be set to true to specify respectively a dialog 
  for choosing a file path for a save operation, for an open operation or for choosing a folder. 
  If \verb+save+, \verb+open+, \verb+folder+ a general widget is opened (and it contains a exec/load/
  chdir menu if \verb+action+ is set to true), but this usage is not recommended since this 
  widget is deprecated.
\end{mandescription}
%--example 
\begin{examples}
  \begin{Verbatim} 
    xgetfile()
    xgetfile(dir="/tmp",title=['title'],save=%t)
    masks=['C','Fortran','Nsp';'*.c','*.f','*.sce'];
    xgetfile(dir="/tmp",masks=masks,title=['title'],open=%t)
    xgetfile(dir="/tmp",title=['Choose a folder'],folder=%t)
  \end{Verbatim} 
\end{examples}
%-- see also


