% -*- mode: latex -*-
\mansection{x\_choices}
\begin{mandesc}
  \shortunder{x\_choices}{x_choices}{Interactive dialog using a small subset of Gtk widgets}\\ % @mandesc@
\end{mandesc}
%-- Calling sequence section
\begin{calling_sequence}
  \begin{verbatim}
    [Lres,L,rep]=x_choices(title,items [,flag])   
  \end{verbatim}
\end{calling_sequence}
%-- Parameters
\begin{parameters}
  \begin{varlist}
    \vname{title}: string matrix which gives the title of the popup window.
    \vname{items}: a list of items, where each \verb!item! is itself a list 
    describing an entry to be selected. The item list should be given as follows 
    \begin{verbatim}
      item=list(type,title,def,choices) 
    \end{verbatim} 
    where 
    \begin{varlist}
      \vname{type}: is a string among \verb+'combo'+, \verb+'entry'+,\verb+'matrix'+, 
      \verb+'colors'+, \verb+'save'+, \verb+'open'+, \verb+'folder'+,  \verb+'spin'+, \verb+'range'+ and 
      \verb+'button'+.
    \vname{title}: is a string describing the entry. 
    \vname{def}: is an integer which semantics depends on the type field. It is used to describe a default choice or set 
      a default parameter. \verb+def+ is ignored by \verb+'save'+ 
      \verb+'load'+, \verb+'folder'+ \verb+'spin'+ and \verb+'range'+. 
      For \verb+'matrix'+, \verb+'def'+ gives the size of entry used for matrix edition.
    \vname{choices}: is a string matrix, or a vector of scalars (for \verb+range+ and \verb+spin+),
    or a list (for \verb+button+) giving the possible choices or one initial value. For an 
      \verb+'open'+ entry it is a default answer followed be a list of file filters (See example below). For 
      \verb+'spin'+ or \verb+'range'+ it is a vector used to specify the parameters of the spin or range widget.
      For a \verb+'button'+ choices must be a list like a \verb+items+ list since \verb+'button'+ when activated 
      will open a new \verb+x_choices+ panel.
    \end{varlist}
    \vname{flag}: optional boolean which changes the way titles are displayed. 
    \vname{L}: is a copy of \verb+items+ but in which the default values are replaced by the user's choices. 
    \vname{Lres}: is a list filed with the user answers or a null list in case of cancel.
    \vname{rep}: is a vector giving the user choices when this choice can be expressed as a scalar 
    i.e in the \verb+'combo'+, \verb+'colors'+, \verb+'spin'+ and \verb+'range'+ cases. For other widgets 
    the \verb+rep+ entry is not relevant.
  \end{varlist}
\end{parameters}
\begin{mandescription}
  Select items through toggle lists and return in \verb!rep! the selected items
\end{mandescription}
  %--example 
\begin{examples}
  \paragraph{First example}

  \begin{program}\HCode{l1=list('combo','combo title',1,['choice 1','choice 2','choice 3']);\Hnewline
      l2=list('entry','entry title',1,['initial']); // 1 is unused \Hnewline
      l3=list('matrix','enter matrix',10,string(rand(6,2)));// l(3) is for entry size\Hnewline
      l4=list('colors','colors choice 4',29,['']);\Hnewline
      // here choices is used to give a default file name \Hnewline
      l5=list('save','file save',1,['foo.sav']); // initial value \Hnewline
      // the default file name if given l6(4)(1) must be a valid full pathname.\Hnewline
      l6=list('open','file open',1,['','*.eps','*.pdf']); // answer, filter \Hnewline
      // \Hnewline
      l7=list('folder','choose a folder',1,['']); // answer, filter unused \Hnewline
      L= list(l1,l2,l3,l4,l5,l6,l7);\Hnewline
      [Lres,L]=x_choices('Toggle Menu',L,\%t);\Hnewline
    }
  \end{program}

  \paragraph{Second example}

  \begin{program}\HCode{l1=list('combo','combo title',1,['choice 1','choice 2','choice 3']);\Hnewline
      // v8=[value,lower,upper,step_increment,page_increment,page_size,climb_rate,digits] \Hnewline
      v8=[5,0,100,0.5,10,20,3,3];    \Hnewline
      l2=list('spin','double with spin',0,v8);// l(3) is unused\Hnewline
      l3=list('range','double with range',0,v8);// l(3) is unused\Hnewline
      L= list(l1,l2,l3);\Hnewline
      [Lres,L]=x_choices('Toggle Menu',L,\%t);\Hnewline
      [Lres,L]=x_choices('Toggle Menu',L,\%f);\Hnewline
    }
  \end{program}

  \paragraph{Third example with a recursive call}

  \begin{program}\HCode{l1=list('combo','combo title',1,['choice 1','choice 2','choice 3']);\Hnewline
      // v8=[value,lower,upper,step_increment,page_increment,page_size,climb_rate,digits] \Hnewline
      v8=[5,0,100,0.5,10,20,3,3];    \Hnewline
      l2=list('spin','double with spin',0,v8);// l(3) is unused\Hnewline
      l3=list('range','double with range',0,v8);// l(3) is unused\Hnewline
      L= list(l1,l2,l3);\Hnewline
      \Hnewline
      l1=list('matrix','enter matrix',60,string(rand(6,2)));// l(3) is for entry size\Hnewline
      l2=list('button','New panel',0,L);\Hnewline
      L=list(l1,l2);\Hnewline
      [Lres,L]=x_choices('Toggle Menu',L,\%t);\Hnewline
    }
  \end{program}
\end{examples}

