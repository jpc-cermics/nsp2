% -*- mode: latex -*-
%% Scilab ( http://www.scilab.org/ ) - This file is part of Scilab
%% Copyright (C) 1987-2016 - (INRIA)
%%
%% This program is free software; you can redistribute it and/or modify
%% it under the terms of the GNU General Public License as published by
%% the Free Software Foundation; either version 2 of the License, or
%% (at your option) any later version.
%%
%% This program is distributed in the hope that it will be useful,
%% but WITHOUT ANY WARRANTY; without even the implied warranty of
%% MERCHANTABILITY or FITNESS FOR A PARTICULAR PURPOSE.  See the
%% GNU General Public License for more details.
%%
%% You should have received a copy of the GNU General Public License
%% along with this program; if not, write to the Free Software
%% Foundation, Inc., 59 Temple Place, Suite 330, Boston, MA  02111-1307  USA
%%                                                                                

\mansection{remezb}
\begin{mandesc}
  \short{remezb}{Minimax approximation of magnitude response} \\ % 
\end{mandesc}
%\index{remezb}\label{remezb}
%-- Calling sequence section
\begin{calling_sequence}
\begin{verbatim}
  [an]=remezb(nc,fg,ds,wt)  
\end{verbatim}
\end{calling_sequence}
%-- Parameters
\begin{parameters}
  \begin{varlist}
    \vname{nc}: Number of cosine functions
    \vname{fg}: Grid of frequency points in [0,.5)
      \vname{ds}: Desired magnitude on grid \verb!fg!
      \vname{wt}: Weighting function on error on grid \verb!fg!
      \vname{an}: Cosine filter coefficients
  \end{varlist}
\end{parameters}
\begin{mandescription}
  Minimax approximation of a frequency domain
  magnitude response. The approximation takes
  the form \verb!h = sum[a(n)*cos(wn)]!
  for n=0,1,...,nc. An FIR, linear-phase filter
  can be obtained from the the output of the function
  by using the following commands
\begin{verbatim}
  hn(1:nc-1)=an(nc:-1:2)/2;
  hn(nc)=an(1);
  hn(nc+1:2*nc-1)=an(2:nc)/2;
\end{verbatim}
\end{mandescription}
%--example 
\begin{examples}
  \begin{mintednsp}{nsp}
    // Choose the number of cosine functions and create a dense grid 
    // in [0,.24) and [.26,.5)
    nc=21;ngrid=nc*16;
    fg=.24*(0:ngrid/2-1)/(ngrid/2-1);
    fg(ngrid/2+1:ngrid)=fg(1:ngrid/2)+.26*ones(1,ngrid/2);
    // Specify a low pass filter magnitude for the desired response
    ds(1:ngrid/2)=ones(1,ngrid/2);
    ds(ngrid/2+1:ngrid)=zeros(1,ngrid/2);
    // Specify a uniform weighting function
    wt=ones(size(fg));
    // Run remezb
    an=remezb(nc,fg,ds,wt)
    // Make a linear phase FIR filter 
    hn(1:nc-1)=an(nc:-1:2)/2;
    hn(nc)=an(1);
    hn(nc+1:2*nc-1)=an(2:nc)/2;
    // Plot the filter's magnitude response
    xbasc();plot2d(.5*(0:255)/256,frmag(hn,256));
  \end{mintednsp}

  \begin{mintednsp}{nsp}
    // Choose the number of cosine functions and create a dense grid in [0,.5)
    nc=21; ngrid=nc*16;
    fg=.5*(0:(ngrid-1))/ngrid;
    // Specify a triangular shaped magnitude for the desired response
    ds(1:ngrid/2)=(0:ngrid/2-1)/(ngrid/2-1);
    ds(ngrid/2+1:ngrid)=ds(ngrid/2:-1:1);
    // Specify a uniform weighting function
    wt=ones(size(fg));
    // Run remezb
    an=remezb(nc,fg,ds,wt)
    // Make a linear phase FIR filter 
    hn(1:nc-1)=an(nc:-1:2)/2;
    hn(nc)=an(1);
    hn(nc+1:2*nc-1)=an(2:nc)/2;
    // Plot the filter's magnitude response
    xbasc();   plot2d(.5*(0:255)/256,frmag(hn,256));
    // --example 
    xbasc();plot2d([],sincd(10,1)) 
  \end{mintednsp}
\end{examples}
%-- see also
\begin{manseealso}
  \manlink{eqfir}{eqfir}  
\end{manseealso}
%-- Author
\begin{authors}
  Carey Bunks
\end{authors}
