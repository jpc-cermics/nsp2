% -*- mode: latex -*-
%% Scilab ( http://www.scilab.org/ ) - This file is part of Scilab
%% Copyright (C) 1987-2016 - (INRIA)
%%
%% This program is free software; you can redistribute it and/or modify
%% it under the terms of the GNU General Public License as published by
%% the Free Software Foundation; either version 2 of the License, or
%% (at your option) any later version.
%%
%% This program is distributed in the hope that it will be useful,
%% but WITHOUT ANY WARRANTY; without even the implied warranty of
%% MERCHANTABILITY or FITNESS FOR A PARTICULAR PURPOSE.  See the
%% GNU General Public License for more details.
%%
%% You should have received a copy of the GNU General Public License
%% along with this program; if not, write to the Free Software
%% Foundation, Inc., 59 Temple Place, Suite 330, Boston, MA  02111-1307  USA
%%                                                                                                

\mansection{yulewalk}
\begin{mandesc}
  \short{yulewalk}{ least-square filter design} \\ % 
\end{mandesc}
%\index{yulewalk}\label{yulewalk}
%-- Calling sequence section
\begin{calling_sequence}
\begin{verbatim}
  Hz = yulewalk(N,frq,mag)   
\end{verbatim}
\end{calling_sequence}
%-- Parameters
\begin{parameters}
  \begin{varlist}
    \vname{N}: integer (order of desired filter)
    \vname{frq}: real row vector (non-decreasing order), frequencies.
    \vname{mag}: non negative real row vector (same size as frq), desired magnitudes.
    \vname{Hz}: filter \verb!B(z)/A(z)!
  \end{varlist}
\end{parameters}
\begin{mandescription}
  Hz = yulewalk(N,frq,mag) finds the N-th order iir filter
\[
H(z) = \frac{B(z)}{A(z)} = 
\frac{ b(1)z^{n-1}+b(2)z^{n-2}+\cdots + b(n)}%
{ z^{n-1}+a(2)z^{n-2}+\cdots + a(n)}
\]
% \begin{verbatim}
%                   n-1         n-2            
%       B(z)   b(1)z     + b(2)z    + .... + b(n)
% H(z)= ---- = ---------------------------------
%                 n-1       n-2
%       A(z)    z   + a(2)z    + .... + a(n)
% \end{verbatim}

which matches the magnitude frequency response given by vectors frq and mag.
Vectors frq and mag specify the frequency and magnitude of the desired
frequency response. The frequencies in frq must be between 0.0 and 1.0,
with 1.0 corresponding to half the sample rate. They must be in
increasing order and start with 0.0 and end with 1.0.
\end{mandescription}
%--example 
\begin{examples}
  \begin{Verbatim}
    f=[0,0.4,0.4,0.6,0.6,1];H=[0,0,1,1,0,0];Hz=yulewalk(8,f,H);
    fs=1000;fhz = f*fs/2;  
    xbasc(0);xset('window',0);plot2d(fhz',H');
    xtitle('Desired Frequency Response (Magnitude)')
    [frq,repf]=repfreq(Hz.num,Hz.den,frq=0:0.001:0.5);
    xbasc(1);xset('window',1);plot2d(fs*frq',abs(repf'));
    xtitle('Obtained Frequency Response (Magnitude)')
  \end{Verbatim}
\end{examples}
