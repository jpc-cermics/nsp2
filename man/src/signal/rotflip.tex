% -*- mode: latex -*-

\mansection{utility functions for signal processing}
\begin{mandesc}
  \short{rot90}{rotate a matrix by increment of 90 degrees}\\
  \short{fliplr}{reverse the columns of a matrix}\\
  \short{flipud}{reverse the rows of a matrix}\\
\end{mandesc}

% -- Calling sequence section
\begin{calling_sequence}
\begin{verbatim}
B = rot90(A)
B = rot90(A,n)
B = fliplr(A)
B = flipud(A)
\end{verbatim}
\end{calling_sequence}
% -- Parameters
\begin{parameters}
  \begin{varlist}
    \vname{A}: real or complex matrix
    \vname{n}: number of increments of 90 degrees to apply for the matrix rotation (default 1)
    \vname{B}: resulting matrix
  \end{varlist}
\end{parameters}

\begin{mandescription}

\verb+rot90(A,n)+ returns a matrix obtained by a rotation of $n \times 90$ degrees of the
matrix $A$ (in the counterclockwise direction for positive $n$).

\verb+fliplr(A)+ returns a matrix obtained by reversing the columns of the matrix $A$.

\verb+flipud(A)+ returns a matrix obtained by reversing the rows of the matrix $A$.

\end{mandescription}
 
\begin{examples}
  \begin{mintednsp}{nsp}
    A = [0, 1, 2, 3; 4, 5, 6, 7; 8, 9, 10, 11]
    rot90(A)
    rot90(A,-1)
    rot90(A,2)
    fliplr(A) 
    A(:,$:-1:1)
    flipud(A) 
    A($:-1:1,:)
  \end{mintednsp}
\end{examples}


\begin{manseealso}
  \manlink{fftshift}{fftshift}, \manlink{ifftshift}{ifftshift}
\end{manseealso}

