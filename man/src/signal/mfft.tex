% -*- mode: latex -*-
%% Scilab ( http://www.scilab.org/ ) - This file is part of Scilab
%% Copyright (C) 1987-2016 - (INRIA)
%%
%% This program is free software; you can redistribute it and/or modify
%% it under the terms of the GNU General Public License as published by
%% the Free Software Foundation; either version 2 of the License, or
%% (at your option) any later version.
%%
%% This program is distributed in the hope that it will be useful,
%% but WITHOUT ANY WARRANTY; without even the implied warranty of
%% MERCHANTABILITY or FITNESS FOR A PARTICULAR PURPOSE.  See the
%% GNU General Public License for more details.
%%
%% You should have received a copy of the GNU General Public License
%% along with this program; if not, write to the Free Software
%% Foundation, Inc., 59 Temple Place, Suite 330, Boston, MA  02111-1307  USA
%%                                                                                

\mansection{mfft}
\begin{mandesc}
  \short{mfft}{multi-dimensional fft} \\ % 
\end{mandesc}
%\index{mfft}\label{mfft}
%-- Calling sequence section
\begin{calling_sequence}
\begin{verbatim}
  [xk]=mfft(x,flag,dim)  
\end{verbatim}
\end{calling_sequence}
%-- Parameters
\begin{parameters}
  \begin{varlist}
    \vname{x}: \verb!x(i,j,k,...)! input signal in the form of a row vector whose values are arranged so that the i index runs the quickest, followed by the j index, etc.
    \vname{flag}: (-1) FFT or (1) inverse FFT
    \vname{dim}: dimension vector which gives the number of values of \verb!x! for each of its indices
    \vname{xk}: output of multidimensional fft in same format as for \verb!x!
  \end{varlist}
\end{parameters}
\begin{mandescription}
  FFT for a multi-dimensional signal
  For example for a three dimensional vector which has three points
  along its first dimension, two points along its second
  dimension and three points along its third dimension the row
  vector is arranged as follows
\begin{verbatim}
  x=[x(1,1,1),x(2,1,1),x(3,1,1),
    x(1,2,1),x(2,2,1),x(3,2,1),
    x(1,1,2),x(2,1,2),x(3,1,2),
    x(1,2,2),x(2,2,2),x(3,2,2),
    x(1,1,3),x(2,1,3),x(3,1,3),
    x(1,2,3),x(2,2,3),x(3,2,3)]
\end{verbatim}
and the \verb!dim! vector is: \verb!dim=[3,2,3]!
\end{mandescription}
%-- Author
\begin{authors}
  Carey Bunks
\end{authors}
