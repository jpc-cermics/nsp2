% -*- mode: latex -*-
%% Scilab ( http://www.scilab.org/ ) - This file is part of Scilab
%% Copyright (C) 1987-2016 - (INRIA)
%%
%% This program is free software; you can redistribute it and/or modify
%% it under the terms of the GNU General Public License as published by
%% the Free Software Foundation; either version 2 of the License, or
%% (at your option) any later version.
%%
%% This program is distributed in the hope that it will be useful,
%% but WITHOUT ANY WARRANTY; without even the implied warranty of
%% MERCHANTABILITY or FITNESS FOR A PARTICULAR PURPOSE.  See the
%% GNU General Public License for more details.
%%
%% You should have received a copy of the GNU General Public License
%% along with this program; if not, write to the Free Software
%% Foundation, Inc., 59 Temple Place, Suite 330, Boston, MA  02111-1307  USA
%%                                                                                

\mansection{lindquist}
\begin{mandesc}
  \short{lindquist}{Lindquist's algorithm} \\ % 
\end{mandesc}
%\index{lindquist}\label{lindquist}
%-- Calling sequence section
\begin{calling_sequence}
\begin{verbatim}
  [P,R,T]=lindquist(n,H,F,G,R0)  
\end{verbatim}
\end{calling_sequence}
%-- Parameters
\begin{parameters}
  \begin{varlist}
    \vname{n}: number of iterations.
    \vname{H, F, G}: estimated triple from the covariance sequence of \verb!y!.
    \vname{R0}: E(yk*yk')
    \vname{P}: solution of the Riccati equation after n iterations.
    \vname{R, T}: gain matrices of the filter.
  \end{varlist}
\end{parameters}
\begin{mandescription}
  computes iteratively the minimal solution of the algebraic
  Riccati equation and gives the matrices \verb!R! and \verb!T! of the 
  filter model, by the Lindquist's algorithm.
\end{mandescription}
%-- see also
\begin{manseealso}
  \manlink{srfaur}{srfaur} \manlink{faurre}{faurre} \manlink{phc}{phc}  
\end{manseealso}
%-- Author
\begin{authors}
  Georges Le Vey
\end{authors}
