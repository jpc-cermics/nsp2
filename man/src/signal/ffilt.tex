% -*- mode: latex -*-
%% Scilab ( http://www.scilab.org/ ) - This file is part of Scilab
%% Copyright (C) 1987-2016 - (INRIA)
%%
%% This program is free software; you can redistribute it and/or modify
%% it under the terms of the GNU General Public License as published by
%% the Free Software Foundation; either version 2 of the License, or
%% (at your option) any later version.
%%
%% This program is distributed in the hope that it will be useful,
%% but WITHOUT ANY WARRANTY; without even the implied warranty of
%% MERCHANTABILITY or FITNESS FOR A PARTICULAR PURPOSE.  See the
%% GNU General Public License for more details.
%%
%% You should have received a copy of the GNU General Public License
%% along with this program; if not, write to the Free Software
%% Foundation, Inc., 59 Temple Place, Suite 330, Boston, MA  02111-1307  USA
%%                                                                                

\mansection{ffilt}
\begin{mandesc}
  \short{ffilt}{coefficients of FIR low-pass} \\ % 
\end{mandesc}
%\index{ffilt}\label{ffilt}
%-- Calling sequence section
\begin{calling_sequence}
\begin{verbatim}
  [x]=ffilt(ft,n,fl,fh)  
\end{verbatim}
\end{calling_sequence}
%-- Parameters
\begin{parameters}
  \begin{varlist}
    \vname{ft}: filter type where \verb!ft! can take the values
    \begin{varlist}
      \vname{"lp"  }: for low-pass filter
      \vname{"hp"  }: for high-pass filter
      \vname{"bp"  }: for band-pass filter
      \vname{"sb"  }: for stop-band filter
    \end{varlist}
    \vname{n}: integer (number of filter samples desired)
    \vname{fl}: real (low frequency cut-off)
    \vname{fh}: real (high frequency cut-off)
    \vname{x}: vector of filter coefficients
  \end{varlist}
\end{parameters}
\begin{mandescription}
  Get \verb!n! coefficients of a FIR low-pass, high-pass, band-pass, or
  stop-band filter.  For low and high-pass filters one cut-off frequency must be
  specified whose value is given in \verb!fl!. For band-pass and stop-band
  filters two cut-off frequencies must be specified for which the lower value is
  in \verb!fl! and the higher value is in \verb!fh!
\end{mandescription}
%-- Author
\begin{authors}
    Carey Bunks  
\end{authors}
