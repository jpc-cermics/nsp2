% -*- mode: latex -*-
%% Scilab ( http://www.scilab.org/ ) - This file is part of Scilab
%% Copyright (C) 1987-2016 - (INRIA)
%%
%% This program is free software; you can redistribute it and/or modify
%% it under the terms of the GNU General Public License as published by
%% the Free Software Foundation; either version 2 of the License, or
%% (at your option) any later version.
%%
%% This program is distributed in the hope that it will be useful,
%% but WITHOUT ANY WARRANTY; without even the implied warranty of
%% MERCHANTABILITY or FITNESS FOR A PARTICULAR PURPOSE.  See the
%% GNU General Public License for more details.
%%
%% You should have received a copy of the GNU General Public License
%% along with this program; if not, write to the Free Software
%% Foundation, Inc., 59 Temple Place, Suite 330, Boston, MA  02111-1307  USA
%%                                                                                

\mansection{mese}
\begin{mandesc}
  \short{mese}{maximum entropy spectral estimation} \\ % 
\end{mandesc}
%\index{mese}\label{mese}
%-- Calling sequence section
\begin{calling_sequence}
\begin{verbatim}
  [sm,fr]=mese(x [,npts]);  
\end{verbatim}
\end{calling_sequence}
%-- Parameters
\begin{parameters}
  \begin{varlist}
    \vname{x}: Input sampled data sequence
    \vname{npts}: Optional parameter giving number of points of \verb!fr! and \verb!sm! (default is 256)
    \vname{sm}: Samples of spectral estimate on the frequency grid \verb!fr!
    \vname{fr}: npts equally spaced frequency samples in \verb![0,.5)!
  \end{varlist}
\end{parameters}
\begin{mandescription}
  Calculate the maximum entropy spectral estimate of \verb!x!
\end{mandescription}
%-- Author
\begin{authors}
  Carey Bunks
\end{authors}
