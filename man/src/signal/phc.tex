% -*- mode: latex -*-
%% Scilab ( http://www.scilab.org/ ) - This file is part of Scilab
%% Copyright (C) 1987-2016 - (INRIA)
%%
%% This program is free software; you can redistribute it and/or modify
%% it under the terms of the GNU General Public License as published by
%% the Free Software Foundation; either version 2 of the License, or
%% (at your option) any later version.
%%
%% This program is distributed in the hope that it will be useful,
%% but WITHOUT ANY WARRANTY; without even the implied warranty of
%% MERCHANTABILITY or FITNESS FOR A PARTICULAR PURPOSE.  See the
%% GNU General Public License for more details.
%%
%% You should have received a copy of the GNU General Public License
%% along with this program; if not, write to the Free Software
%% Foundation, Inc., 59 Temple Place, Suite 330, Boston, MA  02111-1307  USA
%%                                                                                

\mansection{phc}
\begin{mandesc}
  \short{phc}{Markovian representation} \\ % 
\end{mandesc}
%\index{phc}\label{phc}
%-- Calling sequence section
\begin{calling_sequence}
\begin{verbatim}
  [H,F,G]=phc(hk,d,r)  
\end{verbatim}
\end{calling_sequence}
%-- Parameters
\begin{parameters}
  \begin{varlist}
    \vname{hk}: hankel matrix
    \vname{d}: dimension of the observation
    \vname{r}: desired dimension of the state vector for the approximated model
    \vname{H, F, G}: relevant matrices of the Markovian model
  \end{varlist}
\end{parameters}
\begin{mandescription}
  Function which computes the matrices \verb!H, F, G! of a Markovian 
  representation by the principal hankel
  component approximation method, from the hankel matrix built
  from the covariance sequence of a stochastic process.
\end{mandescription}
%--example 
\begin{examples}
  \begin{Verbatim}
    //
    //This example may usefully be compared with the results from 
    //the 'levin' macro (see the corresponding help and example)
    //
    //We consider the process defined by two sinusoids (1Hz and 2 Hz) 
    //in additive Gaussian noise (this is the observation); 
    //the simulated process is sampled at 10 Hz.
    //
    t=0:.1:100;
    y=sin(2*%pi*t)+sin(2*%pi*2*t);y=y+rand(y);
    xbasc();plot2d(t,y)
    //
    //covariance of y
    //
    nlag=128;
    c=corr(y,nlag);
    //
    //hankel matrix from the covariance sequence
    //(we can choose to take more information from covariance
    //by taking greater n and m; try it to compare the results !
    //
    n=20;m=20;
    h=hank(n,m,c);
    //
    //compute the Markov representation (mh,mf,mg)
    //We just take here a state dimension equal to 4:
    //this is the rather difficult problem of estimating the order !
    //Try varying ns ! 
    //(the observation dimension is here equal to one)
    ns=4;
    [mh,mf,mg]=phc(h,1,ns);
    //
    //verify that the spectrum of mf contains the 
    //frequency spectrum of the observed process y
    //(remember that y is sampled -in our example 
    //at 10Hz (T=0.1s) so that we need 
    //to retrieve the original frequencies through the log 
    //and correct scaling by the frequency sampling)
    //
    s=spec(mf);s=log(s);
    s=s/2/%pi/.1;
    //
    //now we get the estimated spectrum
    imag(s),
  \end{Verbatim}
\end{examples}
%-- see also
\begin{manseealso}
  \manlink{levin}{levin}  
\end{manseealso}
