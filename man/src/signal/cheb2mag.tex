% -*- mode: latex -*-
%% Scilab ( http://www.scilab.org/ ) - This file is part of Scilab
%% Copyright (C) 1987-2016 - (INRIA)
%%
%% This program is free software; you can redistribute it and/or modify
%% it under the terms of the GNU General Public License as published by
%% the Free Software Foundation; either version 2 of the License, or
%% (at your option) any later version.
%%
%% This program is distributed in the hope that it will be useful,
%% but WITHOUT ANY WARRANTY; without even the implied warranty of
%% MERCHANTABILITY or FITNESS FOR A PARTICULAR PURPOSE.  See the
%% GNU General Public License for more details.
%%
%% You should have received a copy of the GNU General Public License
%% along with this program; if not, write to the Free Software
%% Foundation, Inc., 59 Temple Place, Suite 330, Boston, MA  02111-1307  USA
%%                                                                                

\mansection{cheb2mag}
\begin{mandesc}
  \short{cheb2mag}{response of type 2 Chebyshev filter} \\ % 
\end{mandesc}
%\index{cheb2mag}\label{cheb2mag}
%-- Calling sequence section
\begin{calling_sequence}
\begin{verbatim}
  [h2]=cheb2mag(n,omegar,A,sample)  
\end{verbatim}
\end{calling_sequence}
%-- Parameters
\begin{parameters}
  \begin{varlist}
    \vname{n}: integer ; filter order
    \vname{omegar}: real scalar: cut-off frequency
    \vname{A}: attenuation in stop band
    \vname{sample}: vector of frequencies where cheb2mag is evaluated
    \vname{h2}: vector of Chebyshev II filter values at sample points
  \end{varlist}
\end{parameters}
\begin{mandescription}
  Square magnitude response of a type 2 Chebyshev filter.\verb!omegar! = stopband edge, \verb!sample! = vector of 
  frequencies where the square magnitude \verb!h2! is desired.
\end{mandescription}
%--example 
\begin{examples}
  Chebyshev; ripple in the stopband
  \begin{mintednsp}{nsp}
    n=10;omegar=6;A=1/0.2;sample=0.0001:0.05:10;
    h2=cheb2mag(n,omegar,A,sample);
    xbasc();plot2d(sample,log(h2)/log(10),style=2);
    xtitle('','frequencies','magnitude in dB')
    //Plotting of frequency edges
    minval=(-max(-log(h2)))/log(10);
    plot2d([omegar;omegar],[minval;0]);
    //Computation of the attenuation in dB at the stopband edge
    attenuation=-log(A*A)/log(10);
    plot2d(sample',attenuation*ones(size(sample))');
  \end{mintednsp}
\end{examples}
%-- see also
\begin{manseealso}
  \manlink{cheb1mag}{cheb1mag}  
\end{manseealso}
