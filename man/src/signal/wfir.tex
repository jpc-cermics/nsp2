% -*- mode: latex -*-
%% Scilab ( http://www.scilab.org/ ) - This file is part of Scilab
%% Copyright (C) 1987-2016 - (INRIA)
%%
%% This program is free software; you can redistribute it and/or modify
%% it under the terms of the GNU General Public License as published by
%% the Free Software Foundation; either version 2 of the License, or
%% (at your option) any later version.
%%
%% This program is distributed in the hope that it will be useful,
%% but WITHOUT ANY WARRANTY; without even the implied warranty of
%% MERCHANTABILITY or FITNESS FOR A PARTICULAR PURPOSE.  See the
%% GNU General Public License for more details.
%%
%% You should have received a copy of the GNU General Public License
%% along with this program; if not, write to the Free Software
%% Foundation, Inc., 59 Temple Place, Suite 330, Boston, MA  02111-1307  USA
%%                                                                                                

\mansection{wfir}
\begin{mandesc}
  \short{wfir}{linear-phase FIR filters} \\ % 
\end{mandesc}
%\index{wfir}\label{wfir}
%-- Calling sequence section
\begin{calling_sequence}
\begin{verbatim}
  [wft,wfm,fr]=wfir(ftype,forder,cfreq,wtype,fpar)  
\end{verbatim}
\end{calling_sequence}
%-- Parameters
\begin{parameters}
  \begin{varlist}
    \vname{ftype}: string: \verb!'lp','hp','bp','sb'! (filter type)
    \vname{forder}: Filter order (pos integer)(odd for \verb!ftype='hp'! or \verb!'sb'!)
    \vname{cfreq}: 2-vector of cutoff frequencies (\verb!0$<$cfreq(1),cfreq(2)$<$.5!)  only \verb!cfreq(1)! is used when \verb!ftype='lp'! or \verb!'hp'!
    \vname{wtype}: Window type (\verb!'re','tr','hm','hn','kr','ch'!)
    \vname{fpar}: 2-vector of window parameters. Kaiser window \verb!fpar(1)$>$0 fpar(2)=0!. Chebyshev window  \verb!fpar(1)$>$0, fpar(2)$<$0! or \verb!fpar(1)$<$0, 0$<$fpar(2)$<$.5!
    \vname{wft}: time domain filter coefficients
    \vname{wfm}: frequency domain filter response on the grid fr
    \vname{fr}: Frequency grid
  \end{varlist}
\end{parameters}
\begin{mandescription}
  Function which makes linear-phase, FIR low-pass, band-pass,
  high-pass, and stop-band filters
  using the windowing technique.
  Works interactively if called with no arguments.
\end{mandescription}
%-- Author
\begin{authors}
  Carey Bunks
\end{authors}
