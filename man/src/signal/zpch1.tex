% -*- mode: latex -*-
%% Scilab ( http://www.scilab.org/ ) - This file is part of Scilab
%% Copyright (C) 1987-2016 - (INRIA)
%%
%% This program is free software; you can redistribute it and/or modify
%% it under the terms of the GNU General Public License as published by
%% the Free Software Foundation; either version 2 of the License, or
%% (at your option) any later version.
%%
%% This program is distributed in the hope that it will be useful,
%% but WITHOUT ANY WARRANTY; without even the implied warranty of
%% MERCHANTABILITY or FITNESS FOR A PARTICULAR PURPOSE.  See the
%% GNU General Public License for more details.
%%
%% You should have received a copy of the GNU General Public License
%% along with this program; if not, write to the Free Software
%% Foundation, Inc., 59 Temple Place, Suite 330, Boston, MA  02111-1307  USA
%%                                                                                                

\mansection{zpch1}
\begin{mandesc}
  \short{zpch1}{Chebyshev analog filter} \\ % 
\end{mandesc}
%\index{zpch1}\label{zpch1}
%-- Calling sequence section
\begin{calling_sequence}
\begin{verbatim}
  [poles,gain]=zpch1(n,epsilon,omegac)  
\end{verbatim}
\end{calling_sequence}
%-- Parameters
\begin{parameters}
  \begin{varlist}
    \vname{n}: integer (filter order)
    \vname{epsilon}: real: ripple in the pass band (\verb!0$<$epsilon$<$1!)
    \vname{omegac}: real: cut-off frequency in Hertz
    \vname{poles}: resulting filter poles
    \vname{gain}: resulting filter gain
  \end{varlist}
\end{parameters}
\begin{mandescription}
  Poles of a Type 1 Chebyshev analog filter. The transfer function is given by:
\begin{verbatim}
  H(s)=gain/poly(poles,'s')
\end{verbatim}
\end{mandescription}
%-- Author
\begin{authors}
  Fran\c{c}ois Delebecque  
\end{authors}
