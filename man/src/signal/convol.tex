% -*- mode: latex -*-
%% Scilab ( http://www.scilab.org/ ) - This file is part of Scilab
%% Copyright (C) 1987-2016 - (INRIA)
%%
%% This program is free software; you can redistribute it and/or modify
%% it under the terms of the GNU General Public License as published by
%% the Free Software Foundation; either version 2 of the License, or
%% (at your option) any later version.
%%
%% This program is distributed in the hope that it will be useful,
%% but WITHOUT ANY WARRANTY; without even the implied warranty of
%% MERCHANTABILITY or FITNESS FOR A PARTICULAR PURPOSE.  See the
%% GNU General Public License for more details.
%%
%% You should have received a copy of the GNU General Public License
%% along with this program; if not, write to the Free Software
%% Foundation, Inc., 59 Temple Place, Suite 330, Boston, MA  02111-1307  USA
%%                                                                                

\mansection{convol}
\begin{mandesc}
  \short{convol}{convolution} \\ % 
\end{mandesc}
%\index{convol}\label{convol}
%-- Calling sequence section
\begin{calling_sequence}
\begin{verbatim}
  [y]=convol(h,x)  
  [y,e1]=convol(h,x,e0)  
\end{verbatim}
\end{calling_sequence}
%-- Parameters
\begin{parameters}
  \begin{varlist}
    \vname{x,h}:input sequences (\verb!h! is a "short" sequence, 
    \verb!x! a "long" one)
    \vname{e0}: old tail to overlap add (not used in first call)
    \vname{y}: output of convolution
    \vname{e1}: new tail to overlap add (not used in last call)
  \end{varlist}
\end{parameters}
\begin{mandescription}
  calculates the convolution \verb!y= h*x! of two
  discrete sequences by using the fft.  Overlap add method can be used.
  USE OF OVERLAP ADD METHOD: 
  For x=[x1,x2,...,xNm1,xN]
  First call is [y1,e1]=convol(h,x1); 
  Subsequent calls: [yk,ek]=convol(h,xk,ekm1);
  Final call: [yN]=convol(h,xN,eNm1);
  Finally y=[y1,y2,...,yNm1,yN]
\end{mandescription}
%--example 
\begin{examples}
  \begin{mintednsp}{nsp}
    x=1:3;
    h1=[1,0,0,0,0];h2=[0,1,0,0,0];h3=[0,0,1,0,0];
    x1=convol(h1,x),x2=convol(h2,x),x3=convol(h3,x),
    convol(h1+h2+h3,x)
    p1=poly(x,'x','coeff')
    p2=poly(h1+h2+h3,'x','coeff')
    p1*p2
  \end{mintednsp}
\end{examples}
%-- see also
\begin{manseealso}
  \manlink{corr}{corr} \manlink{fft}{fft} \manlink{pspect}{pspect}  
\end{manseealso}
%-- Author
\begin{authors}
  Fran\c{c}ois D , Carey Bunks Date 3 Oct. 1988; ;   
\end{authors}
