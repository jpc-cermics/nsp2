% -*- mode: latex -*-
%% Scilab ( http://www.scilab.org/ ) - This file is part of Scilab
%% Copyright (C) 1987-2016 - (INRIA)
%%
%% This program is free software; you can redistribute it and/or modify
%% it under the terms of the GNU General Public License as published by
%% the Free Software Foundation; either version 2 of the License, or
%% (at your option) any later version.
%%
%% This program is distributed in the hope that it will be useful,
%% but WITHOUT ANY WARRANTY; without even the implied warranty of
%% MERCHANTABILITY or FITNESS FOR A PARTICULAR PURPOSE.  See the
%% GNU General Public License for more details.
%%
%% You should have received a copy of the GNU General Public License
%% along with this program; if not, write to the Free Software
%% Foundation, Inc., 59 Temple Place, Suite 330, Boston, MA  02111-1307  USA
%%                                                                                

\mansection{srfaur}
\begin{mandesc}
  \short{srfaur}{square-root algorithm} \\ % 
\end{mandesc}
%\index{srfaur}\label{srfaur}
%-- Calling sequence section
\begin{calling_sequence}
\begin{verbatim}
  [p,s,t,l,rt,tt]=srfaur(h,f,g,r0,n,p,s,t,l)  
\end{verbatim}
\end{calling_sequence}
%-- Parameters
\begin{parameters}
  \begin{varlist}
    \vname{h, f, g}: convenient matrices of the state-space model.
    \vname{r0}: E(yk*yk').
    \vname{n}: number of iterations.
    \vname{p}: estimate of the solution after n iterations.
    \vname{s, t, l}: intermediate matrices for  successive iterations;
    \vname{rt, tt}: gain matrices of the filter model after \verb!n! iterations.
    \vname{p, s, t, l}: may be given as input if more than one recursion is desired (evaluation of intermediate values of \verb!p!).
  \end{varlist}
\end{parameters}
\begin{mandescription}
  square-root algorithm for the algebraic Riccati equation.
\end{mandescription}
%--example 
\begin{examples}
  \begin{mintednsp}{nsp}
    //GENERATE SIGNAL
    x=%pi/10:%pi/10:102.4*%pi;

    y=[1;1]*sin(x)+[sin(2*x);sin(1.9*x)]+rand(2,1024);
    //COMPUTE CORRELATIONS
    c=[];for j=1:2,for k=1:2,c=[c;corr(y(k,:),y(j,:),64)];end;end
    c=matrix(c,2,128);
    //FINDING H,F,G with 6 states
    hk=hank(20,20,c);
    [H,F,G]=phc(hk,2,6);
    //SOLVING RICCATI EQN
    r0=c(1:2,1:2);
    [P,s,t,l,Rt,Tt]=srfaur(H,F,G,r0,200);
    //Make covariance matrix exactly symetric
    Rt=(Rt+Rt')/2
  \end{mintednsp}
\end{examples}
%-- see also
\begin{manseealso}
  \manlink{phc}{phc} \manlink{faurre}{faurre} \manlink{lindquist}{lindquist}  
\end{manseealso}
