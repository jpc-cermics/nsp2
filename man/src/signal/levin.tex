% -*- mode: latex -*-
%% Scilab ( http://www.scilab.org/ ) - This file is part of Scilab
%% Copyright (C) 1987-2016 - (INRIA)
%%
%% This program is free software; you can redistribute it and/or modify
%% it under the terms of the GNU General Public License as published by
%% the Free Software Foundation; either version 2 of the License, or
%% (at your option) any later version.
%%
%% This program is distributed in the hope that it will be useful,
%% but WITHOUT ANY WARRANTY; without even the implied warranty of
%% MERCHANTABILITY or FITNESS FOR A PARTICULAR PURPOSE.  See the
%% GNU General Public License for more details.
%%
%% You should have received a copy of the GNU General Public License
%% along with this program; if not, write to the Free Software
%% Foundation, Inc., 59 Temple Place, Suite 330, Boston, MA  02111-1307  USA
%%                                                                                

\mansection{levin}
\begin{mandesc}
  \short{levin}{Toeplitz system solver by Levinson algorithm (multidimensional)} \\
\end{mandesc}
% \index{levin}\label{levin}
% -- Calling sequence section
\begin{calling_sequence}
\begin{verbatim}
  [la,sig]=levin(n,cov)  
\end{verbatim}
\end{calling_sequence}
%-- Parameters
\begin{parameters}
  \begin{varlist}
    \vname{n}: maximum order of the filter
    \vname{cov}: matrix containing the
    (d x d matrices for a d-dimensional process). It must be given the following way:
    \vname{la}: list-type variable, giving the successively calculated Levinson polynomials (degree 1 to n), with coefficients \verb!Ak!
    \vname{sig}: list-type variable, giving the successive mean-square errors.
  \end{varlist}
\end{parameters}
\begin{mandescription}
  function which solves recursively on n
  the following Toeplitz system (normal equations)
  where {\verb!Rk;k=1,nlag!} is the sequence of \verb!nlag! empirical covariances
\end{mandescription}
%--example 
\begin{examples}
  \begin{Verbatim}
    //We use the 'levin' macro for solving the normal equations 
    //on two examples: a one-dimensional and a two-dimensional process.
    //We need the covariance sequence of the stochastic process.
    //This example may usefully be compared with the results from 
    //the 'phc' macro (see the corresponding help and example in it)
    //
    //
    //1) A one-dimensional process
    //   -------------------------
    //
    //We generate the process defined by two sinusoids (1Hz and 2 Hz) 
    //in additive Gaussian noise (this is the observed process); 
    //the simulated process is sampled at 10 Hz (step 0.1 in t, underafter).
    //
    t1=0:.1:100;// rand('normal');
    y1=sin(2*%pi*t1)+sin(2*%pi*2*t1);y1=y1+randn(y1);plot2d(t1,y1);
    //
    //covariance of y1
    //
    nlag=128;
    c1=corr(y1,nlag);
    c1=c1';//c1 needs to be given columnwise (see the section PARAMETERS of this help)
    //
    //compute the filter for a maximum order of n=10
    //la is a list-type variable each element of which 
    //containing the filters of order ranging from 1 to n; (try varying n)
    //in the d-dimensional case this is a matrix polynomial (square, d X d)
    //sig gives, the same way, the mean-square error
    //
    n=15;
    [la1,sig1]=levin(n,c1);
    //
    //verify that the roots of 'la' contain the 
    //frequency spectrum of the observed process y
    //(remember that y is sampled -in our example 
    //at 10Hz (T=0.1s) so that we need to retrieve 
    //the original frequencies (1Hz and 2 Hz) through 
    //the log and correct scaling by the frequency sampling)
    //we verify this for each filter order
    //
    for i=1:n, s1=roots(la1(i));s1=log(s1)/2/%pi/.1;
    //
    //now we get the estimated poles (sorted, positive ones only !)
    //
    s1=sort(imag(s1));s1=s1(1:i/2);end;
    //
    //the last two frequencies are the ones really present in the observed 
    //process ---$>$ the others are "artifacts" coming from the used model size.
    //This is related to the rather difficult problem of order estimation.
    //
    //2) A 2-dimensional process 
    //   -----------------------
    //(4 frequencies 1, 2, 3, and 4 Hz, sampled at 0.1 Hz:
    //   |y_1|        y_1=sin(2*Pi*t)+sin(2*Pi*2*t)+Gaussian noise
    // y=|   | with: 
    //   |y_2|        y_2=sin(2*Pi*3*t)+sin(2*Pi*4*t)+Gaussian noise
    //
    //
    d=2;dt=0.1;
    nlag=64;
    t2=0:2*%pi*dt:100;
    y2=[sin(t2)+sin(2*t2)+rand(t2);sin(3*t2)+sin(4*t2)+rand(t2)];
    c2=[];
    for j=1:2, for k=1:2, c2=[c2;corr(y2(k,:),y2(j,:),nlag)];end;end;
    c2=matrix(c2,2,128);cov=[];
    for j=1:64,cov=[cov;c2(:,(j-1)*d+1:j*d)];end;//covar. columnwise
    c2=cov;
    //
    //in the multidimensional case, we have to compute the 
    //roots of the determinant of the matrix polynomial 
    //(easy in the 2-dimensional case but tricky if d$>$=3 !). 
    //We just do that here for the maximum desired 
    //filter order (n); mp is the matrix polynomial of degree n
    //
    [la2,sig2]=levin(n,c2);
    mp=la2(n);determinant=mp(1,1)*mp(2,2)-mp(1,2)*mp(2,1);
    s2=roots(determinant);s2=log(s2)/2/%pi/0.1;//same trick as above for 1D process
    s2=sort(imag(s2));s2=s2(1:d*n/2);//just the positive ones !
    //
    //There the order estimation problem is seen to be much more difficult !
    //many artifacts ! The 4 frequencies are in the estimated spectrum 
    //but beneath many non relevant others.
    //
  \end{Verbatim}
\end{examples}
%-- see also
\begin{manseealso}
  \manlink{phc}{phc}  
\end{manseealso}
%-- Author
\begin{authors}
  Georges Le Vey
\end{authors}
