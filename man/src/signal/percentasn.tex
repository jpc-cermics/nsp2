% -*- mode: latex -*-
%% Scilab ( http://www.scilab.org/ ) - This file is part of Scilab
%% Copyright (C) 1987-2016 - (INRIA)
%%
%% This program is free software; you can redistribute it and/or modify
%% it under the terms of the GNU General Public License as published by
%% the Free Software Foundation; either version 2 of the License, or
%% (at your option) any later version.
%%
%% This program is distributed in the hope that it will be useful,
%% but WITHOUT ANY WARRANTY; without even the implied warranty of
%% MERCHANTABILITY or FITNESS FOR A PARTICULAR PURPOSE.  See the
%% GNU General Public License for more details.
%%
%% You should have received a copy of the GNU General Public License
%% along with this program; if not, write to the Free Software
%% Foundation, Inc., 59 Temple Place, Suite 330, Boston, MA  02111-1307  USA
%%                                                                                

\mansection{\%asn}
\begin{mandesc}
  \short{\%asn}{\%asn}{elliptic integral} \\ % 
\end{mandesc}
%\index{\%asn}\label{-p-asn}
%-- Calling sequence section
\begin{calling_sequence}
\begin{verbatim}
  [y]=%asn(x,m)  
\end{verbatim}
\end{calling_sequence}
%-- Parameters
\begin{parameters}
  \begin{varlist}
    \vname{x}: upper limit of integral (\verb!x$>$0!) (can be a vector)
    \vname{m}: parameter of integral (\verb!0$<$m$<$1!)
    \vname{y}: value of the integral
  \end{varlist}
\end{parameters}
\begin{mandescription}
  Calculates the elliptic integral
  If \verb!x! is a vector, \verb!y! is a vector of same dimension as \verb!x!.
\end{mandescription}
%--example 
\begin{examples}
  \begin{mintednsp}{nsp}
    m=0.8;z=%asn(1/sqrt(m),m);K=real(z);Ktilde=imag(z);
    x2max=1/sqrt(m);
    x1=0:0.05:1;x2=1:((x2max-1)/20):x2max;x3=x2max:0.05:10;
    x=[x1,x2,x3];
    y=%asn(x,m);
    rect=[0,-Ktilde,1.1*K,2*Ktilde];
    xbasc();plot2d(real(y)',imag(y)',rect=rect);
    //
    function y=f(t), y=1/sqrt((1-t^2)*(1-m*t^2));endfunction;
    intg(0,0.9,f)-%asn(0.9,m)  //Works for real case only!
  \end{mintednsp}
\end{examples}
%-- Author
\begin{authors}
  Fran\c{c}ois  Delebecque  
\end{authors}
