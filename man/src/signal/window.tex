% -*- mode: latex -*-
%% Scilab ( http://www.scilab.org/ ) - This file is part of Scilab
%% Copyright (C) 1987-2016 - (INRIA)
%%
%% This program is free software; you can redistribute it and/or modify
%% it under the terms of the GNU General Public License as published by
%% the Free Software Foundation; either version 2 of the License, or
%% (at your option) any later version.
%%
%% This program is distributed in the hope that it will be useful,
%% but WITHOUT ANY WARRANTY; without even the implied warranty of
%% MERCHANTABILITY or FITNESS FOR A PARTICULAR PURPOSE.  See the
%% GNU General Public License for more details.
%%
%% You should have received a copy of the GNU General Public License
%% along with this program; if not, write to the Free Software
%% Foundation, Inc., 59 Temple Place, Suite 330, Boston, MA  02111-1307  USA
%%                                                                                                

\mansection{window}
\begin{mandesc}
  \short{window}{compute symmetric window of various type} \\ % 
\end{mandesc}
%\index{window}\label{window}
%-- Calling sequence section
\begin{calling_sequence}
\begin{verbatim}
  win_l=window('re',n)
  win_l=window('tr',n)
  win_l=window('hn',n)
  win_l=window('hm',n)
  win_l=window('kr',n,alpha)
  [win_l,cwp]=window('ch',n,par)
\end{verbatim}
\end{calling_sequence}
%-- Parameters
\begin{parameters}
  \begin{varlist}
    \vname{n}: window length
    \vname{par}: parameter 2-vector \verb!par=[dp,df])!, where
    \verb!dp!  (\verb!0$<$dp$<$.5!) rules the  main lobe
    width and  \verb!df! rules the side lobe height
    (\verb!df$>$0!).Only one of these two value should be specified the other one
    should set equal to \verb!-1!.
    \vname{alpha}: kaiser window parameter \verb!alpha $>$0!). 
    \vname{win}: window
    \vname{cwp}: unspecified Chebyshev window parameter
  \end{varlist}
\end{parameters}
\begin{mandescription}
  function which calculates various symmetric window for Disgital signal processing
  The Kaiser window is a nearly optimal window function. \verb!alpha!
  is an arbitrary positive real number that determines the shape of the
  window,  and the integer  \verb!n! is the length of the window.
  By construction, this function peaks at unity for \verb! k = n/2! ,
  i.e. at the center of the window, and decays exponentially towards the 
  window edges.   The larger the value of \verb!alpha!, the narrower 
  the window becomes; \verb!alpha = 0! corresponds to a rectangular window.
  Conversely, for larger \verb!alpha! the width of the main lobe
  increases in the Fourier transform, while the side lobes decrease in
  amplitude. 
  Thus, this parameter controls the tradeoff between main-lobe width and
  side-lobe area.
  \begin{tabular}{|c|c|c|c|}\hline 
    alpha & window shape \\ \hline
    0 & Rectangular shape \\ \hline
    5 & Similar to the Hamming window \\ \hline
    6 & Similar to the Hanning window \\ \hline
    8.6 & Similar to the Blackman window \\ \hline
  \end{tabular}
  The Chebyshev window minimizes the mainlobe width, given a particular sidelobe
  height. It is characterized by an equiripple behavior, that is, its
  sidelobes all have the same height.
  The Hanning and Hamming windows are quite similar, they only differ in
  the choice of one parameter \verb!alpha!: 
  \verb! w=alpha+(1 - alpha)*cos(2*%pi*x/(n-1))!\verb!alpha! is equal to 1/2 in Hanning window and to 0.54 in
  Hamming window.
\end{mandescription}
%--example 
\begin{examples}
  \begin{itemize}
  \item Hamming window
    \begin{mintednsp}{nsp}
      N=64;
      w=window('hm',N);
      subplot(121);plot2d(1:N,w,line_color=2);xgrid();
      subplot(122)
      n=256;[W,fr]=frmag(w,n);
      plot2d(fr,20*log(W)/log(10),line_color=2);xgrid();
    \end{mintednsp}
  \item Kaiser window
    \begin{mintednsp}{nsp}
      N=64;
      w=window('kr',N,6);
      subplot(121);plot2d(1:N,w,line_color=2);xgrid();
      subplot(122)
      n=256;[W,fr]=frmag(w,n);
      plot2d(fr,20*log(W)/log(10),line_color=2);xgrid();
    \end{mintednsp}
  \item Chebyshev window
    \begin{mintednsp}{nsp}
      N=64;
      [w,df]=window('ch',N,[0.005,-1]);
      subplot(121);plot2d(1:N,w,line_color=2);xgrid();
      subplot(122)
      n=256;[W,fr]=frmag(w,n);
      plot2d(fr,20*log(W)/log(10),line_color=2);xgrid();
    \end{mintednsp}
  \end{itemize}
\end{examples}
%-- see also
\begin{manseealso}
  \manlink{wfir}{wfir} \manlink{frmag}{frmag} \manlink{ffilt}{ffilt}  
\end{manseealso}
%-- Author
\begin{authors}
  Carey Bunks  
\end{authors}
% -- Biblio 
\begin{secbiblio}
  IEEE. Programs for Digital Signal Processing. IEEE Press. New York: John
  Wiley and Sons, 1979. Program 5.2.
\end{secbiblio}
