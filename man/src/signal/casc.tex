% -*- mode: latex -*-
%% Scilab ( http://www.scilab.org/ ) - This file is part of Scilab
%% Copyright (C) 1987-2016 - (INRIA)
%%
%% This program is free software; you can redistribute it and/or modify
%% it under the terms of the GNU General Public License as published by
%% the Free Software Foundation; either version 2 of the License, or
%% (at your option) any later version.
%%
%% This program is distributed in the hope that it will be useful,
%% but WITHOUT ANY WARRANTY; without even the implied warranty of
%% MERCHANTABILITY or FITNESS FOR A PARTICULAR PURPOSE.  See the
%% GNU General Public License for more details.
%%
%% You should have received a copy of the GNU General Public License
%% along with this program; if not, write to the Free Software
%% Foundation, Inc., 59 Temple Place, Suite 330, Boston, MA  02111-1307  USA
%%                                                                                

\mansection{casc}
\begin{mandesc}
  \short{casc}{ cascade realization of filter from coefficients} \\ % 
\end{mandesc}
%\index{casc}\label{casc}
%-- Calling sequence section
\begin{calling_sequence}
\begin{verbatim}
  [cels]=casc(x,z)  
\end{verbatim}
\end{calling_sequence}
%-- Parameters
\begin{parameters}
  \begin{varlist}
    \vname{x}: (4xn)-matrix where each column is a cascade element, the first two column entries being the numerator coefficients and the second two column entries being the denominator coefficients
    \vname{z}: string representing the cascade variable
    \vname{cels}: resulting cascade representation
  \end{varlist}
\end{parameters}
\begin{mandescription}
  Creates cascade realization of filter from a matrix of coefficients.
\end{mandescription}
%--example 
\begin{examples}
  \begin{Verbatim}
    x=[1,2,3;4,5,6;7,8,9;10,11,12]
    cels=casc(x,'z')
  \end{Verbatim}
\end{examples}
