% -*- mode: latex -*-
%% Scilab ( http://www.scilab.org/ ) - This file is part of Scilab
%% Copyright (C) 1987-2016 - (INRIA)
%%
%% This program is free software; you can redistribute it and/or modify
%% it under the terms of the GNU General Public License as published by
%% the Free Software Foundation; either version 2 of the License, or
%% (at your option) any later version.
%%
%% This program is distributed in the hope that it will be useful,
%% but WITHOUT ANY WARRANTY; without even the implied warranty of
%% MERCHANTABILITY or FITNESS FOR A PARTICULAR PURPOSE.  See the
%% GNU General Public License for more details.
%%
%% You should have received a copy of the GNU General Public License
%% along with this program; if not, write to the Free Software
%% Foundation, Inc., 59 Temple Place, Suite 330, Boston, MA  02111-1307  USA
%%                                                                                

\mansection{fsfirlin}
\begin{mandesc}
  \short{fsfirlin}{design of FIR, linear phase filters, frequency sampling technique} \\ % 
\end{mandesc}
%\index{fsfirlin}\label{fsfirlin}
%-- Calling sequence section
\begin{calling_sequence}
\begin{verbatim}
  [hst]=fsfirlin(hd,flag)  
\end{verbatim}
\end{calling_sequence}
%-- Parameters
\begin{parameters}
  \begin{varlist}
    \vname{hd}: vector of desired frequency response samples
    \vname{flag}: is equal to 1 or 2, according to the choice of type 1 or type 2 design
    \vname{hst}: vector giving the approximated continuous response on a dense grid of frequencies
  \end{varlist}
\end{parameters}
\begin{mandescription}
  function for the design of FIR, linear phase filters
  using the frequency sampling technique
\end{mandescription}
%--example 
\begin{examples}
  Example of how to use the fsfirlin macro for the design 
  of an FIR filter by a frequency sampling technique.
  Two filters are designed: 
  \begin{itemize}
  \item the first (response hst1) with 
    abrupt transitions from 0 to 1 between passbands and stop 
    bands
    \begin{Verbatim}
      hd=[zeros(1,15),ones(1,10),zeros(1,39)];//desired samples
      hst1=fsfirlin(hd,1);//filter with no sample in the transition
      hd(15)=.5;hd(26)=.5;//samples in the transition bands
      hst2=fsfirlin(hd,1);//corresponding filter
      fg=linspace(0,1,prod(size(hst1)))/0.5;//normalized frequencies grid
      xbasc();plot2d([fg',fg'],[hst1' hst2']);
    \end{Verbatim}
  \item the second (response hst2) with one sample in each 
    transition band (amplitude 0.5) for smoothing.
    \begin{Verbatim}
      hd=[0*ones(1,15) ones(1,10) 0*ones(1,39)];//desired samples
      hst1=fsfirlin(hd,1);//filter with no sample in the transition
      hd(15)=.5;hd(26)=.5;//samples in the transition bands
      hst2=fsfirlin(hd,1);//corresponding filter
      n=prod(size(hst1))
      fg=linspace(0,1,prod(size(hst1)))/0.5;
      plot2d(fg',hst1');
      plot2d(fg',hst2',line_color=[3]);
    \end{Verbatim}
  \end{itemize}
\end{examples}
%-- see also
\begin{manseealso}
  \manlink{ffilt}{ffilt} \manlink{wfir}{wfir}  
\end{manseealso}
%-- Author
\begin{authors}
  Georges Le Vey
\end{authors}
