% -*- mode: latex -*-
%% Scilab ( http://www.scilab.org/ ) - This file is part of Scilab
%% Copyright (C) 1987-2016 - (INRIA)
%%
%% This program is free software; you can redistribute it and/or modify
%% it under the terms of the GNU General Public License as published by
%% the Free Software Foundation; either version 2 of the License, or
%% (at your option) any later version.
%%
%% This program is distributed in the hope that it will be useful,
%% but WITHOUT ANY WARRANTY; without even the implied warranty of
%% MERCHANTABILITY or FITNESS FOR A PARTICULAR PURPOSE.  See the
%% GNU General Public License for more details.
%%
%% You should have received a copy of the GNU General Public License
%% along with this program; if not, write to the Free Software
%% Foundation, Inc., 59 Temple Place, Suite 330, Boston, MA  02111-1307  USA
%%                                                                                

\mansection{ell1mag}
\begin{mandesc}
  \short{ell1mag}{magnitude of elliptic filter} \\ % 
\end{mandesc}
%\index{ell1mag}\label{ell1mag}
%-- Calling sequence section
\begin{calling_sequence}
\begin{verbatim}
  [v]=ell1mag(eps,m1,z)  
\end{verbatim}
\end{calling_sequence}
%-- Parameters
\begin{parameters}
  \begin{varlist}
    \vname{eps}: passband ripple=\verb!1/(1+eps^2)!
    \vname{m1}: stopband ripple=\verb!1/(1+(eps^2)/m1)!
    \vname{z}: sample vector of values in the complex plane
    \vname{v}: elliptic filter values at sample points
  \end{varlist}
\end{parameters}
\begin{mandescription}
  Function used for squared magnitude of an elliptic filter.
  Usually \verb!m1=eps*eps/(a*a-1)!. Returns 
  \verb!v=real(ones(z)./(ones(z)+eps*eps*s.*s))! for \verb!s=%sn(z,m1)!.
\end{mandescription}
%--example 
\begin{examples}
  \begin{Verbatim}
    function [alpha,beta]=alpha_beta(n,m,m1)
      if 2*int(n/2)==n then beta=K1; else beta=0;end;
      alpha=%k(1-m1)/%k(1-m);
    endfunction
    epsilon=0.1;A=10;  //ripple parameters
    m1=(epsilon*epsilon)/(A*A-1);n=5;omegac=6;
    m=find_freq(epsilon,A,n);omegar = omegac/sqrt(m)
    %k(1-m1)*%k(m)/(%k(m1)*%k(1-m))-n   //Check...
    [alpha,beta]=alpha_beta(n,m,m1)
    alpha*%asn(1,m)-n*%k(m1)      //Check
    sample=0:0.01:20;
    //Now we map the positive real axis into the contour...
    z=alpha*%asn(sample/omegac,m)+beta*ones(size(sample));
    xbasc();plot2d(sample,ell1mag(epsilon,m1,z))
  \end{Verbatim}
\end{examples}
%-- see also
\begin{manseealso}
  \manlink{buttmag}{buttmag}  
\end{manseealso}
