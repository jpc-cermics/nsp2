% -*- mode: latex -*-
%% Scilab ( http://www.scilab.org/ ) - This file is part of Scilab
%% Copyright (C) 1987-2016 - (INRIA)
%%
%% This program is free software; you can redistribute it and/or modify
%% it under the terms of the GNU General Public License as published by
%% the Free Software Foundation; either version 2 of the License, or
%% (at your option) any later version.
%%
%% This program is distributed in the hope that it will be useful,
%% but WITHOUT ANY WARRANTY; without even the implied warranty of
%% MERCHANTABILITY or FITNESS FOR A PARTICULAR PURPOSE.  See the
%% GNU General Public License for more details.
%%
%% You should have received a copy of the GNU General Public License
%% along with this program; if not, write to the Free Software
%% Foundation, Inc., 59 Temple Place, Suite 330, Boston, MA  02111-1307  USA
%%                                                                                

\mansection{corr}
\begin{mandesc}
  \short{corr}{correlation, covariance} \\ % 
\end{mandesc}
% \index{corr}\label{corr}
% -- Calling sequence section
\begin{calling_sequence}
\begin{verbatim}
  [cov,Mean]=corr(x,[y],nlags)  
  [cov,Mean]=corr('fft',xmacro,[ymacro],n,sect)  
  [w,xu]=corr('updt',x1,[y1],w0)  
  [w,xu]=corr('updt',x2,[y2],w,xu)  
  ...  
  [wk]=corr('updt',xk,[yk],w,xu)  
\end{verbatim}
\end{calling_sequence}
%-- Parameters
\begin{parameters}
  \begin{varlist}
    \vname{x}: a real vector
    \vname{y}: a real vector, default value x.
    \vname{nlags}: integer, number of correlation coefficients desired.
    \vname{xmacro}: a scilab external (see below).
    \vname{ymacro}: a scilab external (see below), default value xmacro
    \vname{n}: an integer, total size of the sequence (see below).
    \vname{sect}: size of sections of the sequence (see below).
    \vname{xi}: a real vector
    \vname{yi}: a real vector,default value xi.
    \vname{cov}: real vector, the correlation coefficients
    \vname{Mean}: real number or vector,  the mean of x and if given y
  \end{varlist}
\end{parameters}
\begin{mandescription}
  Computes
\[
   cov(m) = \sum_{k=1}^{n-m} 
   (x(k)  - xmean) (y(m+k)      - ymean) \frac{1}{n}
\]
for m=0,\ldots,\verb!nlag-1! and two vectors 
\verb!x=[x(1),..,x(n)]!\verb!y=[y(1),..,y(n)]!
Note that if x and y sequences are differents corr(x,y,...) is
different with corr(y,x,...)
\begin{description}
\item[Short sequences] \verb![cov,Mean]=corr(x,[y],nlags)! returns the first nlags
  correlation coefficients and Mean = \verb!mean(x)!
  (mean of \verb![x,y]! if \verb!y! is an argument).
  The sequence \verb!x! (resp. \verb!y!) is assumed real, and \verb!x! 
  and \verb!y! are of same dimension n.
\item[Long sequences] \verb![cov,Mean]=corr('fft',xmacro,[ymacro],n,sect)! Here \verb!xmacro! is either
  \begin{description}
  \item[]  
    a function of type \verb![xx]=xmacro(sect,istart)! which
    returns a vector \verb!xx! of dimension
    \verb!nsect! containing the part of the sequence with
    indices from \verb!istart! to
    \verb!istart+sect-1!.
  \item[] 
    a fortran subroutine or C procedure which performs the same
    calculation. (See the source code of \verb!dgetx! for an
    example). \verb!n! = total size of the
    sequence. \verb!sect! = size of sections of the
    sequence. \verb!sect! must be a power of
    2. \verb!cov! has dimension
    \verb!sect!. Calculation is performed by FFT. 
  \end{description}
\item[Updating method] 
\begin{verbatim}
  [w,xu]=corr('updt',x1,[y1],w0)
  [w,xu]=corr('updt',x2,[y2],w,xu)
  ...
  wk=corr('updt',xk,[yk],w,xu)
\end{verbatim}
With this calling sequence the calculation is updated at each
call to \verb!corr!.
\begin{verbatim}
  w0 = 0*ones(1,2*nlags);
  nlags = power of 2.
\end{verbatim}
\verb!x1,x2,...! are parts of \verb!x! such that
\verb!x=[x1,x2,...]! and sizes of \verb!xi! a power of
2.  To get \verb!nlags! coefficients a final fft must be
performed \verb!c=fft(w,1)/n!; \verb!cov=c(1nlags)!
(\verb!n! is the size of \verb!x (y)!).  Caution: this
calling sequence assumes that \verb!xmean = ymean = 0!. 
\end{description}
\end{mandescription}
%--example 
\begin{examples}
  \begin{mintednsp}{nsp}
    x=%pi/10:%pi/10:102.4*%pi;
    y=[.8*sin(x)+.8*sin(2*x)+randn(x);
       .8*sin(x)+.8*sin(1.99*x)+randn(x)];
    c=[];
    for j=1:2,for k=1:2,c=[c;corr(y(k,:),y(j,:),64)];end;end;
    c=matrix(c,2,128);cov=[];
    for j=1:64,cov=[cov;c(:,(j-1)*2+1:2*j)];end;
  \end{mintednsp}

  \begin{mintednsp}{nsp}
   if %f then
      x=randn(1,256);y=-x;
      function [z]=xx(inc,is)  z=x(is:is+inc-1);endfunction
      function [z]=yy(inc,is), z=y(is:is+inc-1);endfunction
      [c,mxy]=corr(x,y,32);
      x=x-mxy(1)*ones(size(x));y=y-mxy(2)*ones(size(y));  //centering
      c1=corr(x,y,32);c2=corr(x,32);
      norm(c1+c2,1)
      [c3,m3]=corr('fft',xx,yy,256,32);
      norm(c1-c3,1)
      [c4,m4]=corr('fft',xx,256,32);
      norm(m3,1),norm(m4,1)
      norm(c3-c1,1),norm(c4-c2,1);
   end
  \end{mintednsp}

  \begin{mintednsp}{nsp}
    x1=x(1:128);x2=x(129:256);
    y1=y(1:128);y2=y(129:256);
    w0=0*ones(1,64);   //32 coeffs
    [w1,xu]=corr('updt',x1,y1,w0);w2=corr('updt',x2,y2,w1,xu);
    zz=real(ifft(w2))/256;c5=zz(1:32);
    // norm(c5-c1,1)
    [w1,xu]=corr('updt',x1,w0);w2=corr('updt',x2,w1,xu);
    zz=real(ifft(w2))/256;c6=zz(1:32);
    // norm(c6-c2,1)
  \end{mintednsp}

  \begin{mintednsp}{nsp}
   if %f then
      // test for Fortran or C external 
      function [y]=xmacro(sec,ist) y=sin(ist:(ist+sec-1));endfunction;
      x=xmacro(100,1);
      [cc1,mm1]=corr(x,2^3);
      [cc,mm]=corr('fft',xmacro,100,2^3);
      [cc2,mm2]=corr('fft','corexx',100,2^3);
      [maxi(abs(cc-cc1)),maxi(abs(mm-mm1)),maxi(abs(cc-cc2)),maxi(abs(mm-mm2))]
      function [y]=ymacro(sec,ist) y=cos(ist:(ist+sec-1));endfunction;
      y=ymacro(100,1);
      [cc1,mm1]=corr(x,y,2^3);
      [cc,mm]=corr('fft',xmacro,ymacro,100,2^3);
      [cc2,mm2]=corr('fft','corexx','corexy',100,2^3);
      [maxi(abs(cc-cc1)),maxi(abs(mm-mm1)),maxi(abs(cc-cc2)),maxi(abs(mm-mm2))]
    end
  \end{mintednsp}
\end{examples}
%-- see also
\begin{manseealso}
  \manlink{fft}{fft}  
\end{manseealso}
