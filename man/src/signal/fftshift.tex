% -*- mode: latex -*-

\mansection{utility functions for fast Fourier transforms}
\begin{mandesc}
  \short{fftshift}{reorder the Fourier coefficients computed by fft}\\
  \short{ifftshift}{inverse operation}\\
\end{mandesc}

% -- Calling sequence section
\begin{calling_sequence}
\begin{verbatim}
d = fftshift(c, dim=mode)
d = fftshift(c, mode)
d = ifftshift(c, dim=mode)
d = ifftshift(c, mode)
\end{verbatim}
\end{calling_sequence}
% -- Parameters
\begin{parameters}
  \begin{varlist}
    \vname{c, d}: real or complex vectors or matrices
    \vname{mode}: an integer or a string chosen among \verb+'M'+, \verb+'m'+, \verb+'full'+, \verb+'FULL'+, \verb+'row'+,
    \verb+'ROW'+, \verb+'col'+, \verb+'COL'+ or an non ambiguous abbreviation. 
    This argument is optional and if omitted 'full' is assumed.
  \end{varlist}
\end{parameters}

\begin{mandescription}

\verb+fftshift+ reorder Fourier coefficients computed the \verb+fft+ function (see \manlink{fft}{fft}
for some explanations).

\verb+ifftshift+ do the inverse operation (usual order \verb+->+ ``fft order'').

The optional named argument \verb+dim+ have the following meaning:
\begin{itemize}
\item dim=0, (or "full", "FULL", "*") reorder row or column vectors (if $c$ is a 
matrix, it is considered as a big column vector (using the column major order) on which 
the reordering is applied). This is the default.
\item dim=1, (or "row", "ROW") reorder each column vectors of the matrix $c$.
\item dim=2, (or "col", "COL") reorder each row vectors of the matrix $c$.
\item dim="m" (or "M") for Matlab compatibility: if $c$ is a vector reorder it but if
$c$ is a matrix each column vector is reordered.
\end{itemize}

\end{mandescription}
 
\begin{examples}
 \paragraph{simple example}
  \begin{Verbatim}
    c = [0, 1, 2, 3, -3, -2, -1]
    d = fftshift(c)
    ifftshift(d)

    c = [0, 1, 2, -3, -2, -1]
    d = fftshift(c)
    ifftshift(d)
  \end{Verbatim}
\end{examples}

% -- Authors
\begin{authors}
   Bruno Pincon
\end{authors}
