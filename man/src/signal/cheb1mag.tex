% -*- mode: latex -*-
%% Scilab ( http://www.scilab.org/ ) - This file is part of Scilab
%% Copyright (C) 1987-2016 - (INRIA)
%%
%% This program is free software; you can redistribute it and/or modify
%% it under the terms of the GNU General Public License as published by
%% the Free Software Foundation; either version 2 of the License, or
%% (at your option) any later version.
%%
%% This program is distributed in the hope that it will be useful,
%% but WITHOUT ANY WARRANTY; without even the implied warranty of
%% MERCHANTABILITY or FITNESS FOR A PARTICULAR PURPOSE.  See the
%% GNU General Public License for more details.
%%
%% You should have received a copy of the GNU General Public License
%% along with this program; if not, write to the Free Software
%% Foundation, Inc., 59 Temple Place, Suite 330, Boston, MA  02111-1307  USA
%%                                                                                

\mansection{cheb1mag}
\begin{mandesc}
  \short{cheb1mag}{ response of Chebyshev type 1 filter} \\ % 
\end{mandesc}
%\index{cheb1mag}\label{cheb1mag}
%-- Calling sequence section
\begin{calling_sequence}
\begin{verbatim}
  [h2]=cheb1mag(n,omegac,epsilon,sample)  
\end{verbatim}
\end{calling_sequence}
%-- Parameters
\begin{parameters}
  \begin{varlist}
    \vname{n}: integer: filter order
    \vname{omegac}: real: cut-off frequency
    \vname{epsilon}: real: ripple in pass band
    \vname{sample}: vector of frequencies where \verb!cheb1mag! is evaluated
    \vname{h2}: Chebyshev I filter values at sample points
  \end{varlist}
\end{parameters}
\begin{mandescription}
  Square magnitude response of a type 1 Chebyshev filter.\verb!omegac!=passband
  edge.\verb!epsilon!: such that \verb!1/(1+epsilon^2)!=passband
  ripple.\verb!sample!: vector of frequencies where the square magnitude is
  desired.
\end{mandescription}
%--example 
\begin{examples}
\paragraph{Chebyshev, ripple in the passband}
  \begin{mintednsp}{nsp}
    n=13;epsilon=0.2;omegac=3;sample=0:0.05:10;
    h=cheb1mag(n,omegac,epsilon,sample);
    xbasc();plot2d(sample,h,style=2);xtitle('','frequencies','magnitude')
  \end{mintednsp}
\end{examples}
%-- see also
\begin{manseealso}
  \manlink{buttmag}{buttmag}  
\end{manseealso}
