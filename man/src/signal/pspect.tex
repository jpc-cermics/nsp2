% -*- mode: latex -*-
%% Scilab ( http://www.scilab.org/ ) - This file is part of Scilab
%% Copyright (C) 1987-2016 - (INRIA)
%%
%% This program is free software; you can redistribute it and/or modify
%% it under the terms of the GNU General Public License as published by
%% the Free Software Foundation; either version 2 of the License, or
%% (at your option) any later version.
%%
%% This program is distributed in the hope that it will be useful,
%% but WITHOUT ANY WARRANTY; without even the implied warranty of
%% MERCHANTABILITY or FITNESS FOR A PARTICULAR PURPOSE.  See the
%% GNU General Public License for more details.
%%
%% You should have received a copy of the GNU General Public License
%% along with this program; if not, write to the Free Software
%% Foundation, Inc., 59 Temple Place, Suite 330, Boston, MA  02111-1307  USA
%%                                                                                

\mansection{pspect}
\begin{mandesc}
  \short{pspect}{cross-spectral estimate between 2 series} \\ % 
\end{mandesc}
%\index{pspect}\label{pspect}
%-- Calling sequence section
\begin{calling_sequence}
\begin{verbatim}
  [sm,cwp]=pspect(sec_step,sec_leng,wtype,x,y,wpar)  
\end{verbatim}
\end{calling_sequence}
%-- Parameters
\begin{parameters}
  \begin{varlist}
    \vname{x}: data if vector, amount of input data if scalar
    \vname{y}: data if vector, amount of input data if scalar
    \vname{sec\_step}: offset of each data window
    \vname{sec\_leng}: length of each data window
    \vname{wtype}: window type \verb!(re,tr,hm,hn,kr,ch)!
    \vname{wpar}: optional parameters  for \verb!wtype='kr', wpar$>$0!  for \verb!wtype='ch', 0$<$wpar(1)$<$.5, wpar(2)$>$0!
    \vname{sm}: power spectral estimate in the interval \verb![0,1]!
    \vname{cwp}: unspecified Chebyshev window parameter
  \end{varlist}
\end{parameters}
\begin{mandescription}
  Cross-spectral estimate between \verb!x! and \verb!y! if both are given
  and auto-spectral estimate of \verb!x! otherwise.
  Spectral estimate obtained using the modified periodogram method.
\end{mandescription}
%-- section-Reference
\paragraph{Reference}
Digital Signal Processing by Oppenheim and Schafer
%--example 
\begin{examples}
  \begin{mintednsp}{nsp}
    x=randn(1:1024-33+1);
    //make low-pass filter with eqfir
    nf=33;bedge=[0 .1;.125 .5];des=[1 0];wate=[1 1];
    h=eqfir(nf,bedge,des,wate);
    //filter white data to obtain colored data 
    h1=[h 0*ones(1,max(size(x))-1)];
    x1=[x 0*ones(1,max(size(h))-1)];
    hf= fft(h1);   xf=fft(x1);yf=hf.*xf;y=real(ifft(yf));
    //plot magnitude of filter
    //h2=[h 0*ones(1:968)];hf2=fft(h2,-1);hf2=real(hf2.*conj(hf2));
    //hsize=maxi(size(hf2));fr=(1:hsize)/hsize;plot(fr,log(hf2));
    //pspect example
    sm=pspect(100,200,'tr',y);smsize=max(size(sm));fr=(1:smsize)/smsize;
    xbasc();plot2d(fr,log(sm));
  \end{mintednsp}
\end{examples}
%-- see also
\begin{manseealso}
  \manlink{cspect}{cspect}  
\end{manseealso}
%-- Author
\begin{authors}
  Carey Bunks
\end{authors}
