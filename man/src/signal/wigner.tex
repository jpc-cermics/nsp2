% -*- mode: latex -*-
%% Scilab ( http://www.scilab.org/ ) - This file is part of Scilab
%% Copyright (C) 1987-2016 - (INRIA)
%%
%% This program is free software; you can redistribute it and/or modify
%% it under the terms of the GNU General Public License as published by
%% the Free Software Foundation; either version 2 of the License, or
%% (at your option) any later version.
%%
%% This program is distributed in the hope that it will be useful,
%% but WITHOUT ANY WARRANTY; without even the implied warranty of
%% MERCHANTABILITY or FITNESS FOR A PARTICULAR PURPOSE.  See the
%% GNU General Public License for more details.
%%
%% You should have received a copy of the GNU General Public License
%% along with this program; if not, write to the Free Software
%% Foundation, Inc., 59 Temple Place, Suite 330, Boston, MA  02111-1307  USA
%%                                                                                                

\mansection{wigner}
\begin{mandesc}
  \short{wigner}{'time-frequency' wigner spectrum} \\ % 
\end{mandesc}
%\index{wigner}\label{wigner}
%-- Calling sequence section
\begin{calling_sequence}
\begin{verbatim}
  [tab]=wigner(x,h,deltat,zp)  
\end{verbatim}
\end{calling_sequence}
%-- Parameters
\begin{parameters}
  \begin{varlist}
    \vname{tab}: wigner spectrum (lines correspond to the time variable)
    \vname{x}: analyzed signal
    \vname{h}: data window
    \vname{deltat}: analysis time increment (in samples)
    \vname{zp}: length of FFT's. \verb!%pi/zp! gives the frequency increment.
  \end{varlist}
\end{parameters}
\begin{mandescription}
  function which computes the 'time-frequency' wigner
  spectrum of a signal.
\end{mandescription}
