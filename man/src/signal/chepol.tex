% -*- mode: latex -*-
%% Scilab ( http://www.scilab.org/ ) - This file is part of Scilab
%% Copyright (C) 1987-2016 - (INRIA)
%%
%% This program is free software; you can redistribute it and/or modify
%% it under the terms of the GNU General Public License as published by
%% the Free Software Foundation; either version 2 of the License, or
%% (at your option) any later version.
%%
%% This program is distributed in the hope that it will be useful,
%% but WITHOUT ANY WARRANTY; without even the implied warranty of
%% MERCHANTABILITY or FITNESS FOR A PARTICULAR PURPOSE.  See the
%% GNU General Public License for more details.
%%
%% You should have received a copy of the GNU General Public License
%% along with this program; if not, write to the Free Software
%% Foundation, Inc., 59 Temple Place, Suite 330, Boston, MA  02111-1307  USA
%%                                                                                

\mansection{chepol}
\begin{mandesc}
  \short{chepol}{Chebychev polynomial} \\ % 
\end{mandesc}
%\index{chepol}\label{chepol}
%-- Calling sequence section
\begin{calling_sequence}
\begin{verbatim}
  [Tn]=chepol(n,var)  
\end{verbatim}
\end{calling_sequence}
%-- Parameters
\begin{parameters}
  \begin{varlist}
    \vname{n}: integer: polynomial order
    \vname{var}: string: polynomial variable
    \vname{Tn}: polynomial in the variable \verb!var!
  \end{varlist}
\end{parameters}
\begin{mandescription}
  Recursive implementation of Chebychev polynomial.
  \verb!Tn=2*poly(0,var)*chepol(n-1,var)-chepol(n-2,var)! with
  \verb!T0=1! and \verb!T1=poly(0,var)!.
\end{mandescription}
%--example 
\begin{examples}
  \begin{Verbatim}
    chepol(4,'x')
  \end{Verbatim}
\end{examples}
%-- Author
\begin{authors}
  Fran\c{c}ois  Delebecque  
\end{authors}
