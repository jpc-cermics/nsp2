% -*- mode: latex -*-
%% Scilab ( http://www.scilab.org/ ) - This file is part of Scilab
%% Copyright (C) 1987-2016 - (INRIA)
%%
%% This program is free software; you can redistribute it and/or modify
%% it under the terms of the GNU General Public License as published by
%% the Free Software Foundation; either version 2 of the License, or
%% (at your option) any later version.
%%
%% This program is distributed in the hope that it will be useful,
%% but WITHOUT ANY WARRANTY; without even the implied warranty of
%% MERCHANTABILITY or FITNESS FOR A PARTICULAR PURPOSE.  See the
%% GNU General Public License for more details.
%%
%% You should have received a copy of the GNU General Public License
%% along with this program; if not, write to the Free Software
%% Foundation, Inc., 59 Temple Place, Suite 330, Boston, MA  02111-1307  USA
%%                                                                                

\mansection{buttmag}
\begin{mandesc}
  \short{buttmag}{ response of Butterworth filter} \\ % 
\end{mandesc}
% \index{buttmag}\label{buttmag}
% -- Calling sequence section
\begin{calling_sequence}
\begin{verbatim}
  [h]=buttmag(order,omegac,sample)  
\end{verbatim}
\end{calling_sequence}
% -- Parameters
\begin{parameters}
  \begin{varlist}
    \vname{order}: integer: filter order
    \vname{omegac}: real: cut-off frequency in Hertz
    \vname{sample}: vector of frequency where \verb!buttmag! is evaluated
    \vname{h}: Butterworth filter values at sample points
  \end{varlist}
\end{parameters}
\begin{mandescription}
  squared magnitude response of a Butterworth filter
  \verb!omegac! = cutoff frequency ; \verb!sample! = sample of frequencies
\end{mandescription}
% --example 
\begin{examples}
  \begin{Verbatim}
    //squared magnitude response of Butterworth filter
    h=buttmag(13,300,1:1000);
    mag=20*log(h)'/log(10);
    xbasc();
    plot2d((1:1000)',mag,line_color=2,rect=[0,-180,1000,20]);
  \end{Verbatim}
\end{examples}
% -- Author
\begin{authors}
  Fran\c{c}ois  Delebecque  
\end{authors}
