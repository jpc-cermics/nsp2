% -*- mode: latex -*-
%% Scilab ( http://www.scilab.org/ ) - This file is part of Scilab
%% Copyright (C) 1987-2016 - (INRIA)
%%
%% This program is free software; you can redistribute it and/or modify
%% it under the terms of the GNU General Public License as published by
%% the Free Software Foundation; either version 2 of the License, or
%% (at your option) any later version.
%%
%% This program is distributed in the hope that it will be useful,
%% but WITHOUT ANY WARRANTY; without even the implied warranty of
%% MERCHANTABILITY or FITNESS FOR A PARTICULAR PURPOSE.  See the
%% GNU General Public License for more details.
%%
%% You should have received a copy of the GNU General Public License
%% along with this program; if not, write to the Free Software
%% Foundation, Inc., 59 Temple Place, Suite 330, Boston, MA  02111-1307  USA
%%                                                                                

\mansection{sinc}
\begin{mandesc}
  \short{sinc}{ samples of sinc function} \\ % 
\end{mandesc}
%\index{sinc}\label{sinc}
%-- Calling sequence section
\begin{calling_sequence}
\begin{verbatim}
  [x]=sinc(n,fl)  
\end{verbatim}
\end{calling_sequence}
%-- Parameters
\begin{parameters}
  \begin{varlist}
    \vname{n}: number of samples
    \vname{fl}: cut-off frequency of the associated low-pass filter in Hertz.
    \vname{x}: samples of the sinc function
  \end{varlist}
\end{parameters}
\begin{mandescription}
  Calculate n samples of the function \verb!sin(2*pi*fl*t)/(pi*t)!
  for \verb!t=-(n-1)/2:(n-1)/2! (i.e. centred around the origin).
\end{mandescription}
%--example 
\begin{examples}
  \begin{Verbatim}
    plot2d([],sinc(100,0.1))
  \end{Verbatim}
\end{examples}
%-- see also
\begin{manseealso}
  \manlink{sincd}{sincd}  
\end{manseealso}
%-- Author
\begin{authors}
  Carey Bunks
\end{authors}
