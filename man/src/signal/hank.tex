% -*- mode: latex -*-
%% Scilab ( http://www.scilab.org/ ) - This file is part of Scilab
%% Copyright (C) 1987-2016 - (INRIA)
%%
%% This program is free software; you can redistribute it and/or modify
%% it under the terms of the GNU General Public License as published by
%% the Free Software Foundation; either version 2 of the License, or
%% (at your option) any later version.
%%
%% This program is distributed in the hope that it will be useful,
%% but WITHOUT ANY WARRANTY; without even the implied warranty of
%% MERCHANTABILITY or FITNESS FOR A PARTICULAR PURPOSE.  See the
%% GNU General Public License for more details.
%%
%% You should have received a copy of the GNU General Public License
%% along with this program; if not, write to the Free Software
%% Foundation, Inc., 59 Temple Place, Suite 330, Boston, MA  02111-1307  USA
%%                                                                                

\mansection{hank}
\begin{mandesc}
  \short{hank}{covariance to hankel matrix} \\ % 
\end{mandesc}
%\index{hank}\label{hank}
%-- Calling sequence section
\begin{calling_sequence}
\begin{verbatim}
  [hk]=hank(m,n,cov)  
\end{verbatim}
\end{calling_sequence}
%-- Parameters
\begin{parameters}
  \begin{varlist}
    \vname{m}: number of bloc-rows
    \vname{n}: number of bloc-columns
    \vname{cov}: sequence of covariances; it must be given as:[R0 R1 R2...Rk]
    \vname{hk}: computed hankel matrix
  \end{varlist}
\end{parameters}
\begin{mandescription}
  this function builds the hankel matrix of size \verb!(m*d,n*d)!
  from the covariance sequence of a vector process
\end{mandescription}
%--example 
\begin{examples}
   Example of how to use the hank macro for 
   building a Hankel matrix from multidimensional 
   data (covariance or Markov parameters e.g.)
    
   This is used e.g. in the solution of normal equations
   by classical identification methods (Instrumental Variables e.g.)
    
   1)let's generate the multidimensional data under the form:
   $C=[c_0 c_1 c_2 \ldots c_n]$
   where each bloc $c_k$ is a d-dimensional matrix (e.g. the k-th correlation 
   of a d-dimensional stochastic process $X(t)$ 
   $[c_k = E(X(t) X'(t+k)]$.
    
   we take here d=2 and n=64 
   generate the hankel matrix H (with 4 bloc-rows and 5 bloc-columns)
   from the data in c
   \begin{Verbatim}
     c=rand(2,2*64);
     H=hank(4,5,c);
   \end{Verbatim}
\end{examples}
%-- see also
\begin{manseealso}
  \manlink{toeplitz}{toeplitz}  
\end{manseealso}
%-- Author
\begin{authors}
  Georges Le Vey
\end{authors}
