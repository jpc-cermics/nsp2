% -*- mode: latex -*-
%% Scilab ( http://www.scilab.org/ ) - This file is part of Scilab
%% Copyright (C) 1987-2016 - (INRIA)
%%
%% This program is free software; you can redistribute it and/or modify
%% it under the terms of the GNU General Public License as published by
%% the Free Software Foundation; either version 2 of the License, or
%% (at your option) any later version.
%%
%% This program is distributed in the hope that it will be useful,
%% but WITHOUT ANY WARRANTY; without even the implied warranty of
%% MERCHANTABILITY or FITNESS FOR A PARTICULAR PURPOSE.  See the
%% GNU General Public License for more details.
%%
%% You should have received a copy of the GNU General Public License
%% along with this program; if not, write to the Free Software
%% Foundation, Inc., 59 Temple Place, Suite 330, Boston, MA  02111-1307  USA
%%                                                                                                

\mansection{zpbutt}
\begin{mandesc}
  \short{zpbutt}{Butterworth analog filter} \\ % 
\end{mandesc}
%\index{zpbutt}\label{zpbutt}
%-- Calling sequence section
\begin{calling_sequence}
\begin{verbatim}
  [pols,gain]=zpbutt(n,omegac)  
\end{verbatim}
\end{calling_sequence}
%-- Parameters
\begin{parameters}
  \begin{varlist}
    \vname{n}: integer (filter order)
    \vname{omegac}: real (cut-off frequency in Hertz)
    \vname{pols}: resulting poles of filter
    \vname{gain}: resulting gain of filter
  \end{varlist}
\end{parameters}
\begin{mandescription}
  computes the poles of a Butterworth analog
  filter of order \verb!n! and cutoff frequency omegac
  transfer function H(s) is calculated by \verb!H(s)=gain/real(poly(pols,'s'))!
\end{mandescription}
%-- Author
\begin{authors}
  Fran\c{c}ois Delebecque INRIA
\end{authors}
