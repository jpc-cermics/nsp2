% -*- mode: latex -*-
\mansection{convhull}
\begin{mandesc}
  \short{convhull}{convex hull of a set of 2d-points}\\
\end{mandesc}
%-- Calling sequence section
\begin{calling_sequence}
\begin{verbatim}
  I=convhull(x,y);
\end{verbatim}
\end{calling_sequence}
%-- Parameters
\begin{parameters}
  \begin{varlist}
    \vname{x,y}: two vectors or matrices of the same size
    \vname{I}: a vector of indices 
  \end{varlist}
\end{parameters}

\begin{mandescription}
  The function \verb+convhull+ computes the convex hull of 
  a set of 2d points given by \verb!(x(.),y(.))!. The convex 
  hull is returned as a vector of indices to be used to extract 
  the convex hull from the given points. That is, the convex hull 
  is given by the points \verb!(x(I(.)),y(I(.)))!.
\end{mandescription}
%--example 
\begin{examples}
  \begin{Verbatim}
    x=rand(1,50);
    y=rand(1,50);
    I=convhull(x,y);
    rect=[-0.2,-0.2,1.2,1.2]
    plot2d(x(I),y(I),line_color=3,rect=rect);
    plot2d(x,y,line_color=-2,mark=1,rect=rect);
  \end{Verbatim}
\end{examples}
% -- see also
% \begin{manseealso}
%   \manlink{indexing arrays}{indexing arrays}
% \end{manseealso}

