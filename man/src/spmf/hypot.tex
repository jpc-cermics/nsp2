% -*- mode: latex -*-
\mansection{hypot}
\begin{mandesc}
  \short{hypot}{length of hypotenuse.}\\
\end{mandesc}
%-- Calling sequence section
\begin{calling_sequence}
\begin{verbatim}
  z = hypot(x,y);
\end{verbatim}
\end{calling_sequence}
%-- Parameters
\begin{parameters}
  \begin{varlist}
    \vname{x,y,z}: real vectors or matrices of the same size.
  \end{varlist}
\end{parameters}

\begin{mandescription}
  This function computes \verb!z=sqrt(x.^2+y.^2)! with accuracy and without
  spurious underflow / overflow problems.
\end{mandescription}
% -- Authors
\begin{authors}
  Algorithm by William Kahan, which
  appears in his article \emph{Branch cuts for complex elementary functions, or
  much ado about nothing's sign bit},
  in \emph{States of the Art in Numerical Analysis} Oxford, Clarendon Press, 1987 ISBN 0-19-853614-3 
  Editors: Iserles, A. and Powell, M. J. D. 
\end{authors}

