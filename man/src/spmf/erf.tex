% -*- mode: latex -*-
\mansection{erf,erfc,erfcx}
\begin{mandesc}
  \short{erf}{The error function.}\\
  \short{erfc}{The complementary error function.}\\
  \short{erfcx}{The scaled complementary error function.}\\
\end{mandesc}
%-- Calling sequence section
\begin{calling_sequence}
\begin{verbatim}
  y = erf(x);
  y = erfc(x);  
  y = erfcx(x);  
\end{verbatim}
\end{calling_sequence}
%-- Parameters
\begin{parameters}
  \begin{varlist}
    \vname{x}: real vector or matrix
    \vname{y}: real vector or matrix (of same size than x)
  \end{varlist}
\end{parameters}

\begin{mandescription}
  The error function \verb+erf+ is defined by:
\[
  \operatorname{erf}(x) :=\frac{2}{\sqrt{\pi}} \int_0^x \exp(-y^2) dy \,.
\]
  The complementary error function \verb+erfc+ is defined by:
\[
\operatorname{erfc}(x) := 1 - \operatorname{erf}(x) = \frac{2}{\sqrt{\pi}} \int_x^{\infty} \exp(-y^2)dy \,.
\]
The scaled complementary error function \verb+erfcx+ is defined by:
\[
\operatorname{erfcx}(x) := \exp(x^2) \operatorname{erfc}(x)\,.
\]
\end{mandescription}

%--example 
\begin{examples}
\begin{Verbatim}
  function y=f(x); y = exp(-x.^2);endfunction;
  intg(0,x,f)*2/sqrt(%pi) - erf(x);
\end{Verbatim}
\end{examples}
%-- see also
%\begin{manseealso}
%\manlink{indexing arrays}{indexing arrays}
%\end{manseealso}

