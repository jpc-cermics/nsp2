% -*- mode: latex -*-
\mansection{bincoeff}
\begin{mandesc}
  \short{bincoeff}{binomial coefficients}\\
\end{mandesc}
%-- Calling sequence section
\begin{calling_sequence}
\begin{verbatim}
  c = bincoeff(n,k);
\end{verbatim}
\end{calling_sequence}
%-- Parameters
\begin{parameters}
  \begin{varlist}
    \vname{n,k}: real vectors or matrices of the same size.
  \end{varlist}
\end{parameters}

\begin{mandescription}
  This function returns in \verb+c(i)+ the binomial coefficients 
  of \verb+n(i)+ and \verb+k(i)+ defined by: 
\[
   \operatorname{bincoeff}(n,k) = \binom{n}{k} := \frac{n!}{k! (n-k)!}
\]
\end{mandescription}
% -- Authors
\begin{examples}

\begin{mintednsp}{nsp}
  N=10; n= N*ones(1,N+1),k=0:N;
  c=bincoeff(n,k);
  d= gamma(n+1) ./ (gamma(k+1).*gamma(n-k+1));
  max(abs(c-d))
\end{mintednsp}

\paragraph{we check the following identity}
\[
(1+x)^n = \sum_{k=0}^n \binom{n}{k} x^k
\]
\begin{mintednsp}{nsp}
  n=10;
  c=bincoeff(n*ones(1,n+1),0:n);
  x=2; 
  y=sum( c .* x.^(0:n));
  y == (1+x)^n
\end{mintednsp}

\end{examples}
