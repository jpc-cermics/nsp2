\chapter*{basic array functions}\addcontentsline{toc}{chapter}{basic array functions}

\hypertarget{arrays}{}

This chapter is devoted to basic operations, functions or methods which apply on
any kind of matrices (numerical, strings, arrrays of cells, ...) and sometimes
on other nsp objects (list, hash table,...).

\begin{quote}
\noindent
\hyperlink{indexing arrays}{indexing arrays} - various ways of indexing arrays \\
\hyperlink{dollar}{dollar} - shortcut notation for the last element\\
\hyperlink{extraction assignment deletion}{extraction assignment deletion} - basic array operations\\
\hyperlink{concatenation}{concatenation} - concatenation of matrices\\
\hyperlink{concatr}{concatr} - method, catenate a matrix by another one on the right side\\   
\hyperlink{concatd}{concatd} - method, catenate a matrix by another one on the down side\\   
\hyperlink{set_diag}{set\_diag} - (method) set a diagonal of a matrix \\
\hyperlink{size}{size} - dimensions of an array, size of some objects \\
\hyperlink{length}{length} - length of an object  \\
\hyperlink{numel}{numel} - length of an object, number of elements of an object  \\
\hyperlink{repmat}{repmat} - replicate a matrix\\
\hyperlink{perms}{perms} - generate all permutations of a vector\\
\hyperlink{nchoosek}{nchoosek} - compute a binomial coefficient or all k-subsets of a set \\
\hyperlink{perm_elem}{perm\_elem} - (method) apply (in place) an elementary permutation to a vector or matrix \\
\hyperlink{redim}{redim} - reshape a matrix\\
\hyperlink{matrix}{matrix} - reshape a matrix\\
\hyperlink{reshape}{reshape} - reshape a matrix\\
\hyperlink{isvector}{isvector} - vector test\\
\hyperlink{isscalar}{isscalar} - scalar test\\
\hyperlink{isempty}{isempty} - empty matrix or object test\\
\end{quote}

\input arrays/indexing.tex
\input arrays/matconcat.tex
\input basicnumarrays/set_diag.tex  
\input arrays/size.tex
\input arrays/length.tex
\input arrays/repmat.tex
\input arrays/perms.tex
\input arrays/nchoosek.tex
\input arrays/perm_elem.tex
\input arrays/redim.tex
\input arrays/isvector.tex
