% -*- mode: latex -*-

\mansection{sub2ind, ind2sub}
\begin{mandesc}
   \short{sub2ind}{convert multi-dimensional indexing to linear indexing}\\
   \short{ind2sub}{convert linear indexing to multi-dimensional indexing}\\
\end{mandesc}

%-- Calling sequence section
\begin{calling_sequence}
\begin{verbatim}
 // two dimensional indexing <-> linear indexing
 Ind = sub2ind(dims, I, J)
 Ind = sub2ind(dims, I, J, ind_type=str)
 [I,J] = ind2sub(dims, Ind)
 [I,J] = ind2sub(dims, Ind, ind_type=str)

 // multi-dimensional indexing <-> linear indexing
 Ind = sub2ind(ndims, I1, I2, ..., In)
 Ind = sub2ind(ndims, I1, I2, ..., In, ind_type=str)
 [I1,...,In] = ind2sub(ndims, Ind)
 [I1,...,In] = ind2sub(ndims, Ind, ind_type=str)
\end{verbatim}
\end{calling_sequence}

%-- Parameters
\begin{parameters}
  \begin{varlist}
   \vname{dims}: vector with the 2 dimensions of a two-dimensional array.
   \vname{I,J}: vectors of indices of same size.
   \vname{ndims}: vector with the n dimensions of a n-dimensional array.
   \vname{I1,...,In}: vectors of indices of same size.
   \vname{Ind}: vector of indices.
   \vname{ind_type=str}: named optional, str can be equal to \verb"double" (default) in which case
          the resulting indices are stored as double floating point numbers or equal to  \verb"int"
          to be stored as int32 numbers.
  \end{varlist}
\end{parameters}

\begin{mandescription}
This function lets to convert between linear indexing and two-dimensional indexing
(see \manlink{indexing arrays}{indexing arrays} to understand the various
ways of indexing arrays). It can be also used for conversion between linear
indexing and multi-dimensional indexing. 

\itemdesc{linear $\leftrightarrow$ two-dimensional indexing conversions}

Denoting by $m$ the number of rows and $n$ the number of columns of a matrix, 
a two dimensional index is a couple $(i,j)$ with $1 \le i \le m$ and
$1 \le j \le n$ and corresponds to the linear index $k = (j-1) \times m + i$
with $1 \le k \le m \times n$ (see \manlink{indexing arrays}{indexing arrays} 
about the various ways of indexing arrays). 

Given the vector \verb+dims = [m,n]+ and $p$ such couples $(i,j)$ under the form 
of two vectors \verb+I+ and \verb+J+ each with $p$ components \verb+Ind=sub2ind(dims,I,J)+
computes the $p$ corresponding linear index and given \verb+dims+ and \verb+Ind+
\verb+[I,J]=ind2sub(dims,Ind)+ do the inverse operation.

\itemdesc{linear  $\leftrightarrow$ multi-dimensional indexing conversions}

The same conversion can be done for multi-dimensional arrays (even if nsp
doesn't yet support arrays with more than 2 dimensions). Considering a $n$
dimensional arrays with dimensions $m_1 \times m_2 \times \dots \times m_n$,
a nd-index is a $n$-tuple $(i_1,i_2,\dots, i_n)$ with $1 \le i_k \le m_k$
for $k=1,\dots,n$ and corresponds to the linear index:
$$
   k = i_1 + m_1 \left( i_2-1 + m_2 \left( i_3 - 1 + m_3 \left( \dots \right) \right) \right)  
$$ 

In this case \verb+dims=[m1,m2,...,mn]+ and the conversion can be done between $p$ such $n$-tuple
under the form of $n$ vectors \verb+I1+, \dots, \verb+IN+ with $p$ components each one, and 
the vector \verb+Ind+ with $p$ components with the corresponding linear indices.    

\itemdesc{named optional arg ind\_type}

Output indices vector(s) can be stored with doubles float number by default or using \verb+ind_type="double"+
or with 32 bits integer using \verb+ind_type="int"+.

\end{mandescription}

%--example 
\begin{examples}
\begin{Verbatim}
// this is the example of the indexing help page
Ind = sub2ind([3,4],2,2)   // should be 5
[I,J] = ind2sub([3,4],5)   // should be 2 and 2

// how to add fastly 1 to the diagonal of a matrix
A = rand(5,5)
Ind = sub2ind([5,5],1:5,1:5);
// now add 1 on the diagonal of A
A(Ind) = A(Ind) + 1
\end{Verbatim}
\end{examples}


%-- see also
\begin{manseealso}
\manlink{indexing arrays}{indexing arrays}
\end{manseealso}

