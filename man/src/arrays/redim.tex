% -*- mode: latex -*-

\mansection{redim, reshape, matrix}
\begin{mandesc}
  \short{redim}{reshape a matrix} \\
  \short{reshape}{reshape a matrix} \\
  \short{matrix}{reshape a matrix}
\end{mandesc}
% -- Calling sequence section
\begin{calling_sequence}
\begin{verbatim}
A.redim[mm, nn]   // method form
A.redim[newdims]  // method form

B = redim(A, mm, nn)  // function form
B = redim(A, newdims) // function form

B = reshape(A,mm, nn) // matlab synonym of redim
B = reshape(A,ndims)

B = matrix(A,mm, nn)  // scilab synonym of redim
B = matrix(A,ndims)
\end{verbatim}
\end{calling_sequence}

% -- Parameters
\begin{parameters}
  \begin{varlist}
    \vname{A, B}: matrices
    \vname{ndims}:  vector made of 2 integers
    \vname{mm, nn}: integers
  \end{varlist}
\end{parameters}

\begin{mandescription}
\verb+redim+  (or \verb+reshape+ or \verb+matrix+) lets to reshape of matrix $A$ with dimensions
$m \times n$ as a matrix of sizes $mm \times nn$ such that 
$mm \times nn = m \times n$ (that is, they have the same number of elements).
The internal ``one dimensionnal'' order of the elements matrix (column major order) are 
not changed. 

\verb+redim+ could be use as a ``method'' (but not \verb+reshape+ and \verb+matrix+). In this case it 
modifies the matrix in place.

Remark:
\begin{itemize}
  \item the new dimensions could be provided either as 2 scalar integer arguments or as a vector with 2 integers.
  \item if you want to reshape the matrix as a row vector you can use \verb+-1+ as second dimension number (this
        avoid to remember or ask how many elements the matrix has) and equivalently you can specify \verb+-1+ as 
        first dimension to reshape the matrix as a column vector, so:
        \begin{verbatim}
        A.redim[1,-1]  // reshape A in place as a row vector
        A.redim[-1,1]  // reshape A in place as a column vector
        \end{verbatim}
\end{itemize}

\end{mandescription}

\begin{examples}
\begin{Verbatim}
A = rand(3,4)
A.redim[2,6]; A
B = redim(A,2,6)    // or reshape(A,2,6) or matrix(A,2,6)

A.redim[-1,1]; A    

B = redim(A,-1,1)   // or reshape(A,-1,1) or matrix(A,-1,1)
B = redim(A,2,6)    // or reshape(A,2,6) or matrix(A,2,6)
\end{Verbatim}
\end{examples}

% -- see also
\begin{manseealso}
   \manlink{size}{size}, \manlink{numel}{numel}, \manlink{length}{length} 
\end{manseealso}

