% -*- mode: latex -*-

\mansection{nchoosek}
\begin{mandesc}
   \short{nchoosek}{compute a binomial coefficient or all k-subsets of a set}
\end{mandesc}

%-- Calling sequence section
\begin{calling_sequence}
\begin{verbatim}
 cnk = nchoosek(n,k)  // first form
 Ek = nchoosek(E,k)   // second form
\end{verbatim}
\end{calling_sequence}

%-- Parameters
\begin{parameters}
  \begin{varlist}
   \vname{n}: non negative integer scalar
   \vname{E}: vector (of numbers, strings, cells,...) with at least 2 elements (if $E$ reduces to a 
              scalar then the first form is considered)
   \vname{k}: non negative integer scalar less or equal to $n$ or to the number of elements of $E$
   \vname{cnk}: binomial coefficient
   \vname{Ek}: matrix with all k-subsets elements of $E$
  \end{varlist}
\end{parameters}

\begin{mandescription}
  This function has two differents behaviors depending of the size of the first argument:
\begin{enumerate}
\item if it is a scalar it should be a non negative integer and the binomial coefficient :
$$
  \left( \begin{array}{c} n \\ k \end{array} \right) = \frac{n!}{k!(n-k)!} =  \left( \begin{array}{c} n \\ n-k \end{array} \right)
$$
is computed.
\item if it is not a scalar then all the subsets with $k$ elements of the vector $E$ are computed
      (it is supposed that all components of $E$ are unique) ; $Ek$ should have 
      $ \left( \begin{array}{c} n \\ k \end{array} \right)$ rows and $k$ columns, each
      row corresponding to a subset.
\end{enumerate}

\end{mandescription}

%--example 
\begin{examples}
\paragraph{first form examples}
\begin{Verbatim}
nchoosek(6,4)

// should be equal to 6!/(4!*2!) and n! could be computed by prod(1:n)
prod(1:6)/(prod(1:4)*prod(1:2))

// test symmetry
nchoosek(6,2)
\end{Verbatim}
\end{examples}

\begin{examples}
\paragraph{second form}
\begin{Verbatim}
// all subsets with 3 elements of {1,2,3,4,5}
nchoosek(1:5,3)

// all subsets with 2 elements of {"biniou","grand-mere","gratte"}
nchoosek(["biniou","grand-mere","gratte"],2)
\end{Verbatim}
\end{examples}


%-- see also
\begin{manseealso}
\manlink{perms}{perms}
\end{manseealso}

%-- Author
\begin{authors}
F. Delebecque, B. Pincon
\end{authors}

