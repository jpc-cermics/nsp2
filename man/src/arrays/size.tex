% -*- mode: latex -*-

\mansection{size}
\begin{mandesc}
  \short{size}{dimensions of an array, size of some objects}
\end{mandesc}
% -- Calling sequence section
\begin{calling_sequence}
\begin{verbatim}
dims = size(A)
[m,n] = size(A)
m = size(A,1)  // or m = size(A,"r")
n = size(A,2)  // or m = size(A,"c")
mn = size(A,0) // or m = size(A,"*")   
\end{verbatim}
\end{calling_sequence}

% -- Parameters
\begin{parameters}
  \begin{varlist}
    \vname{A}: matrix
    \vname{dims}:  row vector (\verb+1x2+) of integers
    \vname{m, n, mn}: integers
  \end{varlist}
\end{parameters}

\begin{mandescription}
\verb+size+ gives the dimensions of an array with the following features:
\begin{itemize}
  \item \verb!dims = size(A)! (that is a call with one output argument) returns the matrix dimensions in a row vector 
        (the first element is the number of rows $m$, the second the number of columns $n$) ;
  \item \verb![m,n] = size(A)! (a call with 2 output arguments) returns the number of rows $m$ in the first output arg
        and the number of columns $n$ in the second output arg.
  \item \verb!m = size(A,1)! (or equivalently \verb!m = size(A,"r")! returns the number of rows
  \item \verb!n = size(A,2)! (or equivalently \verb!n = size(A,"c")! returns the number of columns
  \item \verb!n = size(A,0)! (or equivalently \verb!n = size(A,"*")! returns the product of the dimensions (this is
        also equivalent to \manlink{numel}{numel}).
\end{itemize}

Note also that like \manlink{numel}{numel} and \manlink{length}{length}, if \verb+L+ is a list, 
\verb+size(L)+ returns the number of its elements, and if \verb+H+ is an hash table \verb+size(H)+ 
returns the number of its entries.
\end{mandescription}

\begin{examples}
\begin{mintednsp}{nsp}
A = rand(3,4)
dims = size(A)
[m,n] = size(A)
size(A,1)
size(A,2)
size(A,"*")

// for list and hash tables, size behaves like length and numel
L = list(1,2); L(5)=78; 
size(L)
length(L)
numel(L)

H = hash_create(A=9,B=%t);
size(H)
length(H)
numel(H)
\end{mintednsp}
\end{examples}

% -- see also
\begin{manseealso}
  \manlink{numel}{numel}, \manlink{length}{length} 
\end{manseealso}

