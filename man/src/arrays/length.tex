% -*- mode: latex -*-
\mansection{length, numel}
\begin{mandesc}
  \short{length}{length of object} \\
  \short{numel}{length of object, number of elements of object}
\end{mandesc}
% -- Calling sequence section
\begin{calling_sequence}
\begin{verbatim}
n = length(M)
nel = numel(M)
\end{verbatim}
\end{calling_sequence}

% -- Parameters
\begin{parameters}
  \begin{varlist}
    \vname{M}: an object.
    \vname{n}: integer or integer matrix
    \vname{nel}: integer
    \vname{mode}: an optional string argument which can be \verb!'bytes'! or \verb!'utf8'!
    (default value is \verb!'bytes'!). It is used when \verb!M! is a string matrix.
  \end{varlist}
\end{parameters}

\begin{mandescription}
\begin{itemize}
  \item \verb!length(M)! is equal to \verb!size(M,'*')!, the number of entries of \verb!M! for all matrices
    or for cells {\bf except} string matrices.
  \item If \verb+M+ is a string matrix \verb+n=length(M)+ returns in a  numerical matrix of same size
    as \verb+M+ the length (\verb+strlen+) of each entries of \verb+M+. i.e, if \verb+L=length(M)+,
    \verb+n(i,j)+ is the length of string \verb+M(i,j)+.

    The length of a string is counted in bytes or in utf8 characters accoding
    to the optional \verb!mode! value.

  \item If \verb+M+ is a list, \verb+length+ returns as \verb+size+ the length of list \verb+M+.
  \item If \verb+M+ is a hash table \verb+length+ returns the number of entries in the hash table.
  \item \verb!numel(M)! is the same than \verb!length(M)! without the exception of string matrices, that
        is \verb!numel(M)! {\bf always} gives the number of elements of any array, list or hash table.
\end{itemize}
\end{mandescription}

\begin{examples}
\begin{mintednsp}{nsp}
// for matrices of strings length and numel are different:
S = string(rand(3,4))
length(S)
numel(S)

// for other objects, length and numel behave identically:
L = list(1,2); L(5)=78;
length(L)
numel(L)

A = rand(5,7);
length(A)
numel(A)

H = hash_create(A=9,B=%t);
length(H)
numel(H)
\end{mintednsp}
\end{examples}

% -- see also
\begin{manseealso}
  \manlink{size}{size}
\end{manseealso}
