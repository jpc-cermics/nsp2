% -*- mode: latex -*-

\mansection{findparam}
\begin{mandesc}
  \short{findparam}{utility function for fsqp solver}\\ % @mandesc@
\end{mandesc}
%-- Calling sequence section
\begin{calling_sequence}
\begin{verbatim}
  [nf,nineqn,nineq,neqn,neq,nfsr,ncsrl,ncsrn,mesh_pts,nf0,ng0,nc0,nh0,na0]=findparam(list_obj,list_cntr,x0)  
\end{verbatim}
\end{calling_sequence}

%-- Parameters
\begin{parameters}
  \begin{varlist}
    \vname{list\_objs}
    : list made of two lists.
    \vname{list\_cntr}
    : list made of six lists.
    \vname{x0}
    : real column vector (e.g. initial guess)
    \vname{nf, ..., ncsrn}
    : integers (see fsqp-doc).
    \vname{mesh\_pts}
    : real vector.
  \end{varlist}
\end{parameters}

\begin{mandescription}
  
  Given a list of objective functions and a list of constraints
  functions, and a vector x0, findparam returns useful fsqp
  parameters.
  The list of objectives should be: (use empty list when necessary: list() )
  
  \begin{mintednsp}{nsp}
    list_obj=list(...
    list(f_1,...,f_nf0),...     //regular objectives
    list(F_1,...,F_nfsr)        //SR objectives
    )
  \end{mintednsp}

  The f\_i's are functions: y=f\_i(x) should return the value of the ith
  regular objective as a function of x. y can be a column vector if
  several regular objectives are stacked together.
  y=F\_i(x) is the ith sequentially related objective. y is a column
  vector which contains the ith set of SR-objectives (mesh\_pts(i) is
  set to size(F\_i(x),1) by findparam).
  
  The list of constraints functions should be as follows:
  
  \begin{mintednsp}{nsp}
    list_cntr=list(...
    list(g_1,...,g_ng0),...     //regular nonlinear inequality
    list(G_1,...,G_ncsrn),...   //SR      nonlinear inequality
    list(c_1,...,c_nc0),...     //regular linear    inequality
    list(C_1,...,C_ncsrl),...   //SR      linear    inequality
    list(h_1,...,h_nh0),...     //nonlinear         equality
    list(A_1,...,A_na0)         //linear            equality
    )
  \end{mintednsp}

  Functions g\_i's, c\_i's, h\_1 can return column vectors y (e.g. y=g\_1(x)) 
  if several constraints are stacked together.
  Functions G\_i's, C\_i's, A\_i's should return in a column vector the
  set of appropriate SR constraints.
  Examples are given at the end of examplei.sce files. See
  listutils.sci: generic functions obj, cntr, grob, grcn are constructed
  from the lists list\_obj, list\_cntr, and similar lists list\_grobj
  and list\_cntr which contain the gradients of objectives and constraints
  (matrices whith nparam=dim(x) columns).
  
\end{mandescription}

%-- see also
\begin{manseealso}
  \manlink{srfsqp}{srfsqp} \manlink{x\_is\_new}{x-is-new} \manlink{set\_x\_is\_new}{set-x-is-new} 
\end{manseealso}

