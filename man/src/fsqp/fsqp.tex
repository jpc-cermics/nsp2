% -*- mode: latex -*-

\mansection{fsqp}
\begin{mandesc}
  \short{fsqp}{FSQP solver}\\ % @mandesc@
\end{mandesc}
%-- Calling sequence section
\begin{calling_sequence}
\begin{verbatim}
  x=fsqp(x0,ipar,rpar,[bl,bu],obj,cntr,grob,grcn)  
  x=fsqp(x0,ipar,rpar,[bl,bu],obj,cntr,"grob","grcn")  
  x=fsqp(x0,ipar,rpar,[bl,bu],"obj","cntr","grob","grcn", [cd])  
  x=fsqp(x0,ipar,rpar,[bl,bu],"obj","cntr","grobfd","grcnfd", [cd])  
  [x,inform]=fsqp(...)  
  [x,inform,f,g,lambda]=fsqp(...)  
\end{verbatim}
\end{calling_sequence}

%-- Parameters
\begin{parameters}
  \begin{varlist}
    \vname{x0}
    : real column vector (initial guess)
    \vname{ipar}
    : integer vector of size 8.
    \vname{rpar}
    : real vector of size 4.
    \vname{[bl,bu]}
    : real matrix with 2 columns and same row dimension as x0.
    \vname{obj, cntr}
    : either scilab functions or character strings (names of C functions) defining the objective function and the constraints function.
    \vname{grob,grcn}
    : either scilab functions or character strings (names of C functions) defining the gradient of objective function and the gradient of  constraints function.
    \vname{cd}
    : optional real vector (argument passed to C functions).
  \end{varlist}
\end{parameters}

\begin{mandescription}
  
  fsqp interface. This interface uses the notations of the cfsqp
  user's guide. See doc/c\_manual.ps.
  The four functions (objective, constraints, gradient-of-objective,
  gradient-of-constraints) can be given either as C-functions dynamically
  linked with Scilab or as Scilab functions (mixtures are allowed).
  If a Scilab function is given as argument of fsqp, 
  just use its name (without quotes). If a C function is to be called,
  use a character string to define it. For instance fsqp(...,obj,...)
  invokes fsqp with the Scilab function \verb!obj! and
  \verb!fsqp(...,"obj",...)! invokes fsqp with the C function 
  \verb!obj!, linked with Scilab by e.g. \verb! link("obj.o", "obj") !.
  
  The chains "grobfd" and "grcnfd" can be used as C functions to compute
  the gradients of the functions obj and cntr by finite difference.
  Notations:
  
  \begin{mintednsp}{nsp}
    ipar=[nf,nineqn,nineq,neqn,neq,modefsqp,miter,iprint];
    rpar=[bigbnd,eps,epsneq,udelta];
  \end{mintednsp}

  \begin{mintednsp}{nsp}
    function fj=obj(j,x)
    function  gj=cntr(j,x)
    function gradf=grob(j,x)
    function gradgj=grcn(j,x)
  \end{mintednsp}

  obj(j,x) returns the value of the jth objective, given x.
  cntr(j,x) returns the value of the jth constraint, given x.
  grob(j,x) (resp. grcn(j,x)) is the value at x of the gradient of the 
  function x-$>$obj(j,x) (resp.  x-$>$cntr(j,x)).
  
  It may be useful to make use of the fsqp C variable x\_is\_new and to
  evaluate these functions in a vector way. For instance the function
  cntr can be replaced by the following cntr2 function:
  
  \begin{mintednsp}{nsp}
    function cj=cntr2(j,x)
    if x_is_new() then
    all_objectives=allobj(x);
    all_constraints=allcntr(x);
    all_gradobjectives=allgrob(x);
    all_gradconstraints=allgrcn(x);
    cj=all_constraints(j);
    set_x_is_new(0);  //Use resume to define global Scilab variables
    [all_objectives,all_constraints,all_gradobjectives,all_gradconstraints]=....
    resume(all_objectives,all_constraints,all_gradobjectives,all_gradconstraints);
    else
    cj=all_constraints(j);
    end
  \end{mintednsp}

  Here, the function allcntr(x) returns a vector all\_constraints whose
  jth component all\_constraints(j) is cntr(j).
  A typical example comes from discretized semi-infinite program. 
  See example5.sci.
  
\end{mandescription}

%--example
\begin{examples}
  \begin{mintednsp}{nsp}
    //This is example 1 of the doc.
    deff("fj=obj32(j,x)","fj=(x(1)+3*x(1)+x(2))^2+4*(x(1)-x(2))^2")

    deff("gj=cntr32(j,x)",[
      "select j";
      "case 1"
      "gj=x(1)^3-6*x(2)-4*x(3)+3;"
      "case 2"  
      "gj=1-sum(x);"
      "end"]);

    deff("gradf=grob32(j,x)",...
    "fa=2*(x(1)+3*x(2)+x(3));fb=8*(x(1)-x(2));gradf=[fa+fb;3*fa-fb;fa]");

    deff("gradgj=grcn32(j,x)",[
      "select j";
      "case 1"
      "gradgj=[3*x(1)^2;-6;-4];"
      "case 2"
      "gradgj=[-1;-1;-1];"
      "end"]);
    x0=[0.1;0.7;0.2];

    nf=1; nineqn=1; nineq=1; neqn=0; neq=1; modefsqp=100; miter=500; iprint=1;
    ipar=[nf,nineqn,nineq,neqn,neq,modefsqp,miter,iprint];

    bigbnd=1.e10; eps=1.e-8; epsneq=0.e0; udelta=0.e0;
    rpar=[bigbnd,eps,epsneq,udelta];

    bl=[0;0;0];
    bu=[bigbnd;bigbnd;bigbnd];

    x=fsqp(x0,ipar,rpar,[bl,bu],obj32,cntr32,grob32,grcn32)
  \end{mintednsp}
\end{examples}

%-- see also
\begin{manseealso}
  \manlink{srfsqp}{srfsqp} \manlink{x\_is\_new}{x-is-new} \manlink{set\_x\_is\_new}{set-x-is-new} 
\end{manseealso}

