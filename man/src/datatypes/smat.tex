% -*- mode: latex -*-

\mansection{Smat}
\begin{mandesc}
  \short{Smat}{matrix data type} \\
  \short{smat_create}{string matrix creation}
\end{mandesc}
% -- Calling sequence section
\begin{calling_sequence}
\begin{verbatim}
S=[str11, str12, ..., str1n;
   str21, str22, ..., str2n;
   ...
   strm1, strm2, ...; strmn] 
// 
S=smat_create(2,2,"nsp")
\end{verbatim}
\end{calling_sequence}
% -- Parameters
\begin{parameters}
  \begin{varlist}
    \vname{s11 ...}: strings
  \end{varlist}
\end{parameters}

\begin{mandescription}
The \emph{Smat} data type is the string matrix  object, i;e. a
two dimensional array in which each entry is a string.

Column vectors of strings are considered as \verb!m x 1! matrices and row vectors
as \verb!1 x n! matrices.


\end{mandescription}

\paragraph{Operations on string matrices}
\itemdesc{Methods from the matint interface}
\begin{itemize}
\item \verb+A.redim[m,n]+ reshape matrix to size \verb+m+x\verb+n+. \verb+m+ or \verb+n+ can be set to \verb+-1+ 
\item \verb+A.concatr[B]+ \verb+A = [A,B]+
\item \verb+A.concatd[B]+ \verb+A = [A;B]+
\item \verb+A.perm_elem[p,q[,dim]]+ permute p and q: elements
  (default or dim=0), rows (dim=1) or columns (dim=2).
\end{itemize}

\itemdesc{Extraction, insertion, deletion}
\begin{itemize}
   \item If \verb+I+ and is \verb+J+ are respectively row and column indices \verb+A(I,J)+ is the submatrix of \verb+A+ with entries in \verb+I+ and \verb+j+.
   \item \verb+A(I,:)+ is the submatrix of \verb+A+ with row indices in \verb+I+.
   \item \verb+A(:,J)+ is the submatrix of \verb+A+ with comumn indices in \verb+I+.
   \item \verb+A(:,:)+ is \verb+A+
   \item \verb+A(:)+  is the column matrix obtained by stacking the columns of \verb+A+.
\end{itemize}
\begin{itemize}
   \item If \verb+I+ and is \verb+J+ are row and column indices \verb+A(I,J)=B+ inserts
matrix \verb+B+ into matrix \verb+A+ in rows \verb+I+ and columns \verb!J!.
   \item \verb+A(I,:)=[]+ removes the rows of \verb+A+ with indices in \verb+J+.
   \item \verb+A(:,J)=[]+ removes the columns of \verb+A+ with indices in \verb+J+.
\end{itemize}

\itemdesc{Some string functions}
\begin{itemize}
\item \verb+smat_create+ string matrix creation.
\item \verb+ascii+ ascii conversion 
\item \verb+catenate+ string concatenation 
\item \verb+concat+ string matrices concatenation 
\item \verb+getfile+ return the contents of a file in a colmun string vector 
\item \verb+isalnum+ checks for an alphanumeric character 
\item \verb+isalpha+ checks for an alphabetic character
\item \verb+isascii+ checks for 7-bit unsigned char value that fits into the ASCII character set 
\item \verb+isdigit+ checks for a digit (0 through 9) 
\item \verb+isgraph +checks for any printable character except space 
\item \verb+islower+ checks for a lower-case character 
\item \verb+isprint+  checks for any printable character including space 
\item \verb+ispunct+ checks for any printable character which is not a space or an alphanumeric character 
\item \verb+isspace+  checks for white-space characters 
\item \verb+isupper+  checks for an uppercase letter 
\item \verb+isxdigit+  checks for a hexadecimal digits 
\item \verb+length+  return a numeric matrix filled with lenght of each string od the string matrix
\item \verb+m2s+  convert to string matrix 
\item \verb+string+  convert object to string matrix 
\item \verb+part+ substring extraction from indices 
\item \verb+putfile+  write a string matrix in a file in column order mode 
\item \verb+regexp+  search regular expressions 
\item \verb+regsub+ substitutions based on regular expression pattern matching 
\item \verb+split+ string splitting 
\item \verb+strindex + search occurences of a string in a string 
\item \verb+stripblanks+ removes leading and trailing white-space characters 
\item \verb+strstr+  locate a substring in a string matrix 
\item \verb+strsubst+ sub-string substitution in a string matrix 
\item \verb+toupper+ convert letter to upper case 
\item \verb+tolower+  convert letter to lower case 
\item \verb+capitalize+ capitalize strings 
\end{itemize}

\begin{manseealso}

\end{manseealso}

% -- Authors
\begin{authors}
   Jean-Philippe Chancelier, Bruno Pin�on
\end{authors}
