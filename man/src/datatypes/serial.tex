% -*- mode: latex -*-

\mansection{Serial}
\begin{mandesc}
  \short{Serial}{Container for nsp serialized object} \\
  \short{serialize}{serialize an nsp object}
\end{mandesc}
% -- Calling sequence section
\begin{calling_sequence}
\begin{verbatim}
S = serialize(Obj);
A = serialize(Obj,'m')
\end{verbatim}
\end{calling_sequence}
% -- Parameters
\begin{parameters}
  \begin{varlist}
    \vname{Obj}:  any nsp objects. 
    \vname{S}: a \emph{Serial} object 
    \vname{A}: a numeric matrix.
  \end{varlist}
\end{parameters}

\begin{mandescription}
The \emph{Serial} data type is useful to serialize any nsp object in order 
for example to transmit it through communication networks as an array of 
characters. For example, it can be used with \verb+pvm+ or \verb+mpi+ to transmit any 
nsp object. The function \verb+serialize+ is used to produce a serialized object. 
For some applications it can be useful to hide the serialized object in a 
numerical matrix, this can be done with the extra argument \verb+'m'+. Note 
that an object serialized in a numeric matrix can be corrupted if 
the matrix is changed. Thus this version, which is mainly used in scicos 
which transmit block parameters through int or double array, should be used with care. 
\end{mandescription}

\paragraph{Operations on serialized objects}
\itemdesc{methods}
\begin{itemize}
   \item \verb+O=S.unserialize[]+ unserialize \verb+S+. \verb+unserialize+ can 
     also be used as a function for \emph{Serial} arguments or for numeric 
     argument.
\end{itemize}

\begin{examples}
  \begin{program}\HCode{function y=f(x);y=sin(x);endfunction;\Hnewline
      L = list(1,2,"c",\%f, \%t, rand(2,1), f);\Hnewline
      S = serialize(L);\Hnewline
      L1=unserialize(S);\Hnewline
      A = serialize(L,'m');\Hnewline
      L1=unserialize(A);\Hnewline
      ok = exec('unserialize(rand(1,10))',errcatch=\%t);}
  \end{program}
 \end{examples}

\begin{manseealso}

\end{manseealso}

% -- Authors
\begin{authors}
   jpc
\end{authors}
