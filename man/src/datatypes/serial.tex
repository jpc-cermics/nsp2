% -*- mode: latex -*-

\mansection{Serial}
\begin{mandesc}
  \short{Serial}{Container for nsp serialized object} \\
  \short{serialize}{serialize an nsp object}
\end{mandesc}
% -- Calling sequence section
\begin{calling_sequence}
\begin{verbatim}
S = serialize(Obj);
A = serialize(Obj,'m')
\end{verbatim}
\end{calling_sequence}
% -- Parameters
\begin{parameters}
  \begin{varlist}
    \vname{Obj}:  any nsp objects. 
    \vname{S}: a \emph{Serial} object 
    \vname{A}: a numeric matrix.
  \end{varlist}
\end{parameters}

\begin{mandescription}
The \emph{Serial} data type is useful to serialize any nsp object in order 
for example to transmit it through communication networks as an array of 
characters. For example, it can be used with \verb+pvm+ or \verb+mpi+ to transmit any 
nsp object. The function \verb+serialize+ is used to produce a serialized object. 
For some applications it can be useful to hide the serialized object in a 
numerical matrix, this can be done with the extra argument \verb+'m'+. Note 
that an object serialized in a numeric matrix can be corrupted if 
the matrix is changed. Thus this version, which is mainly used in scicos 
which transmit block parameters through int or double array, should 
be used with care. 
\end{mandescription}

\paragraph{Operations on serialized objects}
\itemdesc{methods}
\begin{itemize}
   \item \verb+O=S.unserialize[]+ unserialize the Serial object \verb+S+ which can be in compressed mode. 
   \item \verb+O=S.compress[]+ returns a serialized object with compressed data. 
   \item \verb+O=S.uncompress[]+ uncompress the Serial object \verb+S+ and returns a new uncompressed serialized object.
   \item \verb+str=S.tobase64[]+ or \verb+str=S.tobase64[n]+ returns a string matrix containing a base64 coding of the 
     serialized object \verb+S+. The string matrix is of size \verb+1x1+ and if an optional argument is given the 
     returned matrix is of size \verb+mx1+ each element of the matrix being a string of length at most \verb+n+.
\end{itemize}

\itemdesc{Some functions}
\begin{itemize}
  \item \verb+S=base64toserial(str)+ back convert a base64 string to a serialized object. 
  \item \verb+O1=unserialize(O)+ unserialize the given argument which can be a Serial object or a scalar matrix. 
    If a scalar matrix is given it must be a scalar matrix obtained with the \verb+serialize+ function with 
    optional argument \verb+'m'+. 
  \item \verb+size(S,'*')+ or \verb+length(S)+ returns the size in bytes used for storing the serialized object.
\end{itemize}

\begin{examples}
\begin{itemize}
\item Basic operations.
\begin{mintednsp}{nsp}
  function y=f(x);y=sin(x);endfunction;
  L = list(1,2,"c",%f, %t, rand(2,1), f);
  S = serialize(L);
  L1=unserialize(S); // or S.unserialize[]
  A = serialize(L,'m');
  L1=unserialize(A);
  ok = execstr('unserialize(rand(1,10))',errcatch=%t);
\end{mintednsp}
\item Serialize and compress.
\begin{mintednsp}{nsp}
  A=1:100;
  S=serialize(A);
  Sc=S.compress[];
  A1=Sc.unserialize[];
\end{mintednsp}
\item Conversion to base64 strings 
\begin{mintednsp}{nsp}
  A=1:10;
  S=serialize(A);
  str=S.tobase64[];
  S=base64toserial(str);
  A1=S.unzerialize[];
\end{mintednsp}
\end{itemize}
\end{examples}

\begin{manseealso}
\end{manseealso}

% -- Authors
\begin{authors}
   Jean-Philippe Chancelier
\end{authors}
