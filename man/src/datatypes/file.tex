% -*- mode: latex -*-

\mansection{File}
\begin{mandesc}
  \short{File}{File data type}
  \short{fopen}{creates a file stream}
\end{mandesc}

% -- Calling sequence section
%\begin{calling_sequence}
\subsection*{Constructor}
\index{fopen}\label{fopen}
\begin{verbatim}
F = fopen(file-name,mode='string',xdr=boolean,swap=boolean)
\end{verbatim}
%\end{calling_sequence}
% -- Parameters
\begin{parameters}
  \begin{varlist}
    \vname{file-name}: a string which gives the pathname of file to be opened. 
    Note that \verb+file-name+ can also be one of the following keywords \verb+"stdin"+, or 
    \verb+"stdout"+, or \verb+"stderr"+ in order to use the standard input, output or output error 
    stream.
    \vname{mode}: The argument \verb+mode+ is a string beginning  with  one  of  the  following  sequences
    \begin{description}
      \item[r]: open text file for reading. The stream is positioned at the beginning of the
        file. 
      \item[r+]: open text file for reading and writing. The stream is positioned at the beginning of the
        file. 
      \item[w] truncate  file  to  zero  length  or create text file for writing.  The stream is
        positioned at the beginning of the file.
      \item[w+] open for reading and writing.  The file is created if it  does  not
        exist,  otherwise it is truncated.  The stream is positioned at the beginning of the file.
      \item[a] open for appending (writing at end of file).  The file is created if it does not 
	exist.  The stream is positioned at the end of the file.
      \item[a+]  Open for reading and appending (writing at end of file).  The  file
        is  created  if  it  does not exist.  The initial file position for
        reading is at the beginning of  the  file,  but  output  is  always
        appended to the end of the file.
    \end{description}
    The mode string can also include the letter \verb+b+ as a last character. 
    Adding \verb+b+ b is used if you do I/O to a binary file 
    (but note that \verb+b+ is ignored on all POSIX conforming systems, including Linux).
    \vname{swap} If true (the default value) then byte swaping is forced when using the \verb+get+ and 
    \verb+put+ methods in order to have binary files always filed with data coded as little endian. 
    If \verb+swap+ is false then data is writen in the machine native mode. 
    \vname{xdr} If true then reading and writing is used using \verb+xdr+.
  \end{varlist}
\end{parameters}

\begin{mandescription}
The \emph{File} data type is used for file I/O operations.
\end{mandescription}

\paragraph{Operations on File objects}
\itemdesc{methods}
\begin{itemize}
  \vname{F.close[]}: close the file stream.
  \vname{F.putstr[str]}: outputs a string. 
  \vname{F.put[M,type='string']}: outputs the contents of real matrix \verb+M+ using 
  the \verb!type! parameter as a conversion specifier. The \verb+type+ argument can be chosen 
  among \verb+"l"+, \verb+"i"+, \verb+"s"+, \verb+"ul"+, \verb+"ui"+, \verb+"us"+, \verb+"d"+, \verb+"f"+, 
  \verb+"c"+,\verb+"uc"+ for writing respectively a long, an int, a short, an unsigned long, 
  an unsigned int, an unsigned short, a double, a float, a char and an unsigned char. 
  The bytes which are read are automatically swapped if necessary. 
  This default swapping mode can be suppressed by adding a flag in the \verb!mopen! function 
  (see the description of the \verb+swap+ parameter above).
  It is also possible to force read in little-endian mode or big-endian mode by adding a \verb+'l'+ or \verb+'b'+  
  character at the end of a type specification. For example \verb+'db'+ can be used to read a double 
  in big-endian mode. 
  \vname{F.get[n=val,type='string']}: reads \verb+n+ values of type given by \verb+type+ 
  (see the above \verb+put+ method for \verb+type+ specification).
  and returns the values in a vector of size \verb+n+. 
  \vname{F.getstr[n=val]}: reads \verb+n+ bytes and returns the a string.
  \vname{F.eof[]}: returns \verb+%t+ if an end-of-file was encountered. 
  \vname{F.seek[value,str]}:  sets  the  file  position indicator. The new position, 
  measured in bytes, is obtained by adding offset bytes to  the
  position  specified by \verb+str+. If \verb+str+ is set to \verb+'set'+, \verb+'cur'+, 
  \verb+'end'+ the offset is relative to the start of the file, 
  the current position indicator, or  end of file,  respectively. The default value 
  is \verb+set+.
  Successful call clears the end of file indicator.
  \vname{F.tell[]}: returns the current value of the file position indicator. 
  \vname{F.clearerr[]}: clears any previously raised error.
  \vname{F.error[]}: returns \verb+%t+ if an error was encountered during previous read/write 
  operations 
  \vname{[M,S]F.get_matrix[format='string']}: read a numerical matrix from file current position. 
  First lines can be non-numeric and are then returned in S. The read stops when current 
  line read is not compatible with previous ones. An optional format to be used for reading
  each matrix entry can be given. 
  \vname{S=F.get_lines[n]}: acquires \verb+n+ lines from the file stream \verb+F+ and returns 
  the result in a string matrix. If \verb+n+ is omitted or negative it is assumed to be equal to 
  one. 
  \vname{S=F.get_smatrix[]}: read the whole file contents from current position
  and return a column string vector. Each line read giving a row of the column vector.
  \vname{F.put_matrix[M,format='string',sep='string',title=S]}: 
    output  matrix contents in a file using the given format for each element 
    and the string given by \verb+sep+ as separator. If \verb+title+ is given 
    its contents is outputed in the file before the matrix. Default value 
    for format is the defualt format used in nsp interactive output and defaut 
    separator is a white space.
  \vname{F.put_smatrix[S]}: output the contents of a string matrix in file \verb+F+ 
  the string matrix is explored columnwise (i.e equivalent to \verb+S(:)+).
  \vname{F.print[Obj,as_read=boolean,depth=integer, indent=integer,latex=boolean,name=string,table=boolean]}: 
  see the print function for a description of optional arguments.
  \vname{F.printf[format,arg1,...,argn]}: Similar to the C \verb+fprintf+ function. 
\end{itemize}

\itemdesc{functions}
\begin{itemize}
  \vname{fopen}: opens a file for I/O operations returning a \verb+File+ object.
  \vname{is_little_endian}: checks the little-endian machine specification.
  \vname{getfile(fname)}: gets the contents of ascii file given by \verb+fname+ 
  in a string column vector.
  \vname{putfile(fname,S)}: put the contents of a string matrix in file named 
  \verb+fname+. If the file already existed its contents is replaced, if not it is 
  created. 
\end{itemize}

\begin{examples}
  \begin{program}\HCode{F=fopen("stdout")\Hnewline
      F.printf["using stdout"] \Hnewline
      F.put_matrix[rand(4,4)]  \Hnewline 
      F.close[] }
  \end{program}
\end{examples}

\begin{manseealso}
  \manlink{print}{print}  
  \manlink{fprint}{fprint}  
  \manlink{sprint}{sprint}  
\end{manseealso}

% -- Authors
\begin{authors}
  Jean-Philippe Chancelier 
\end{authors}

