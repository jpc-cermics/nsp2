% -*- mode: latex -*-

\mansection{list}
\begin{mandesc}
  \short{list}{list data type and list creation}
\end{mandesc}

% -- Calling sequence section
\begin{calling_sequence}
\begin{verbatim}
L = list()
L = list(O1, O2, O3, ...)
\end{verbatim}
\end{calling_sequence}
% -- Parameters
\begin{parameters}
  \begin{varlist}
    \vname{O1, O2, 03, ...}:  any nsp objects (matrix of numbers, of
    strings, of booleans, of cells, list, \ldots
    \vname{L}: list formed with the objects (an empty list in the
    first case).
  \end{varlist}
\end{parameters}

\begin{mandescription}
The \emph{list} data type is useful to collect objects of different
types. Internally nsp lists are implemented as doubled linked lists
and operations (extraction, insertion, deletion) on head and queue
are efficient. The \emph{list} function can be used to create lists.  
\end{mandescription}

\paragraph{Operations on lists}
\itemdesc{methods}
\begin{itemize}
   \item \verb+L.add[e,i]+ inserts the object \verb+e+ in position i of \verb+L+
     (elements previously at positions i,i+1,\ldots are shifted by one on the right). 
   \item \verb+L.add_first[e]+ appends the object \verb+e+ in head of \verb+L+.
   \item \verb+L.add_last[e]+ appends the object \verb+e+ in queue of \verb+L+.
   \item \verb+L.compact[[dir]]+ compacts the list\verb+L+ by trying to concatenate (if possible) 
     successive elements of compatible size and same type. The optional argument \verb+dir+  
     is a string: \verb+"row"+ (or \verb+"r"+) to concatenate by row and \verb+"col"+ 
     (or \verb+"c"+) to concatenate by column (which is the default). 
   \item \verb+L.concat[LL]+ appends the list \verb+LL+ to \verb+L+.
   \item \verb+L.first[]+ returns the first element of \verb+L+.
   \item \verb+L.item[i]+ returns the i th element of \verb+L+.
   \item \verb+L.last[]+ returns the last element of \verb+L+.
   \item \verb+L.remove[i]+ removes the \verb+i+-th element of \verb+L+
     (elements previously at positions following \verb+i+  are shifted to the left).
   \item \verb+L.remove_first[]+ removes the element in head of \verb+L+.
   \item \verb+L.remove_last[]+ removes the element in queue of \verb+L+.
   \item \verb+L.reverse[]+ reverses the elements positions of the list  \verb+L+.
   \item \verb+L.sublist[ind]+ returns a new list (\verb+L+ is not modified) 
     built from the elements of \verb+L+ given by the vector \verb+ind+ (checking validity of indices).
   \item \verb+[found[,index]]=L.has[Obj]+ search if the the object \verb+Obj+ is 
     in the list \verb+L+ and return a scalar boolean (\verb+found+). Additionnaly the 
     (first) \verb+index+ of \verb+Obj+ in  \verb+L+ could be returned  
     (\verb+index=0+ if \verb+found+ is false).
\end{itemize}

\itemdesc{extraction, insertion, deletion}
\begin{itemize}
   \item If verb+i+ is a scalar \verb+L(i)+ returns the i-th element of \verb+L+.
   \item If \verb+ind+ is a vector of indices \verb+L(ind)+ 
     puts the respective elements of \verb+L+ on the 
     calling stack. Thus \verb+L(ind)+ returns \verb+size(ind,'*')+ values that 
     can be assigned to several variables with the standard syntax \verb+[a,b,c,...]=L(ind)+. 
     Another usage is for calling sequence of functions : if \verb+L=list([1,2],%f,"c")+
     then  \verb+f(L([1,3])+ is equivalent to  \verb+f([1,2],"c")+
   \item \verb+L(:)+ puts all the elements of \verb+L+ on the caling stack
     (with the previous list  \verb+f(L(:))+ is be equivalent to \verb+f([1,2],%f,"c")+)
   \item \verb+L(i)=e+ replaces the i-th element of \verb+L+ by
     the object \verb+e+. If $i > n$ where $n=length(L)$ then $i-n-1$ undefined
     elements are added at the end of the list and finally \verb+e+
     is added (and becomes the last element of \verb+L+). 
     So \verb-L(n+1)=e- (or \verb-L($+1) = e-) can be used to append a new element at the tail of the list 
     and \verb+L(0)=e+ can be used to insert \verb+e+ a new element at the head of the list 
     (note that this element is at index 1 after the head insertion).
   \item More generaly \verb+L(ind)=...+ expects \verb+size(ind,'*')+ values returned 
     by the {\em rhs} expression which are used to set the \verb+L+ values at \verb+ind+ positions with 
     the rules explained above. 
   \item \verb+L(i)=null()+ deletes the i-th element of the list.
\end{itemize}

Note that some of these features are useful for scilab compatibility
but we recommend for speed efficiency to use methods when possible.
In particular, \verb+L.remove[i]+ should be preferred to \verb+L(i)=null()+ 
(resp. \verb+L.remove_first[]+ and \verb+L.remove_last[]+ to delete head and queue elements),
Use \verb+L.add_first[e]+ instead of \verb+L(0)=e+, and \verb+L.add_last[e]+ instead of 
\verb-L($+1) = e-.

\itemdesc{For loop control using a list}
With a list \verb+L+:\begin{verbatim}
     for e=L
       ....
     end
\end{verbatim} 
is a loop with $length(L)$ iterations, the loop 
variable  \verb+e+ being equal to \verb+L(i+) at the i-th iteration.

\itemdesc{Some functions}
\begin{itemize}
   \item \verb+length(L)+ or \verb+size(L)+ return the length of \verb+L+
   \item \verb+L=list_concat(O1,..,On)+ is similar to \verb+list+ except that 
     the given objects which are list are concatenated whith the built list. 
     For example \verb+list_concat(5,list(4,7),''foo'')+ will return 
     \verb+list(5,4,7,''foo'')+.
   \item \verb+map(L,function [,args])+: builds a new list using 
     \verb+function(L(i),args)+ to build the \verb+i+-th argument. 
     The optional argument must be a list itself. Note that at the present time,
     the \verb+function+ argument must be a PList object (i.e a nsp writen function).
   \item \verb+foldr(L,function [,args])+. Where the header of \verb+function+ is 
     \verb+z=function(x,y,args)+. Suppose here as an example that \verb+L+ is of 
     length 4 then a \verb+foldr+  call will evaluates as:
     \begin{program}
       function(L(1),function(L(2),function(L(3),L(4),args),args),args)
     \end{program}
   \item \verb+unique+ (\manlink{unique}{unique}) 
\end{itemize}

\begin{examples}
  \begin{itemize}
    \item Creation and methods:
    \begin{program}
    L = list(1,2,"c",%f, %t, rand(2,1))
    n = length(L)
    // adds 0 as first element of L increasing its size
    L.add_first[0]; L
    n = length(L)
    // adds pi at the tail of list increasing its size
    L.add_last[%pi]; L
    // deletes last element
    L.remove_last[]; L
    // 2 ways to extract the third element
    L.item[3]
    L(3)
    // extract 5-th and 6-th elements
    [b1,b2] = L([4,5])
    // reverse the list
    L.reverse[]; L
    // compact the list
    L.compact[]; L
    // build a sublist with the first and third elements
    LL = L.sublist[[1,3]]
  \end{program}
  \item \verb+map+
  \begin{program}
    function y=len(x);y=length(x);endfunction;
    L=list(4,1:10,'foo');
    map(L,len)
  \end{program}
\item \verb+foldr+
  \begin{program}
    // reverse a list using foldr 
    function z=rev(x,y);z=list_concat(y,x);endfunction ;
    L=list(1,2,3,4);
    // sum the elements of a list using foldr 
    function z=lplus(x,y);z=x+y;endfunction ;
    L=list(1,2,3,4);
    S=foldr(L,lplus)
  \end{program}
  \end{itemize}
 \end{examples}

\begin{manseealso}
  \manlink{unique}{unique}  
  \manlink{setdiff}{setdiff}  
\end{manseealso}

% -- Authors
\begin{authors}
   jpc, bp
\end{authors}
