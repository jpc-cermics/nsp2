% -*- mode: latex -*-

\mansection{SpColMat}
\begin{mandesc}
  \short{SpColMat}{Sparse matrix data type}
\end{mandesc}
% -- Calling sequence section
\begin{calling_sequence}
\begin{verbatim}
S = sparse(M)
S = sparse(ij,v,[mn]);
S = sp_create(m,n) // or spcol_create(m,n)
\end{verbatim}

\end{calling_sequence}
% -- Parameters
\begin{parameters}
  \begin{varlist}
    \vname{M}:  full vector or matrix  
    \vname{ij}: matrix with two columns containing the row and column indices of 
                non-zeros entries  
    \vname{v}: vector of the values of nonzeros entries
    \vname{mn}: vector of two integer values indicating the dimension 
    \vname{m,n}: scalars numbers of rows and columns
  \end{varlist}
\end{parameters}

\begin{mandescription}

The \emph{SpColMat} data type is the basic type for handling sparse matrices. 
For these matrices, only non-zeros entries are stored in memory, which enables to 
handle larger-size matrices and reduce complexity on matrices operations (the internal
loops are done only on non-zeros entries).

Internally a sparse matrix is stored as an array of pointers (on columns) , each containing 
two arrays : one for the indices and another for the values of the non-zeros entries of
the column \verb+i+. 
%For compatibility with matlab, there is also a internal storage format
%corresponding to the Compressed Row Storage Format (triplet Ax,Ai,Ap, also used by UMFPACK)
 
The functions \verb+sparse+ and \verb+sp_create+ (or  \verb+spcol_create+) lets to create
sparse matrices:
\begin{itemize}
\item \verb+sparse(M)+ builds a sparse matrix from a full one ;
\item \verb+sparse(ij,v,[mn])+ , probably the most useful form, builds a sparse matrix
   from the list of its non zeros elements. Note that it is possible to have several equal
   $(i,j)$ couples of row, column indices. In this case the corresponding values $v$ are 
   added (this can be useful to build finite element matrix).
\item \verb+sp_create(m,n)+ simply creates a sparse matrix of size $m \times n$ (without
   non zero elements).   
\end{itemize}

\end{mandescription}

\paragraph{Operations on sparse matrices}
\itemdesc{Methods}
\begin{itemize}
\item \verb+S.scale_rows[x]+ multiplies each row i of A with a scalar:
\item \verb+S.scale_cols[x]+ multiplies each column j of A with a scalar:
\item \verb+S.get_nnz[]+ returns the number of non zeros entries of \verb+S+ 
\item \verb+S.set_diag[v[,k]]+ put the \emph{SpColMat} vector $v$ in the diagonal number $k$ of \verb+S+
 (if the diagonal number $k$ is not given, default is 0, ie the main diagonal, see \manlink{tril}{tril} page).

\end{itemize}

\itemdesc{Operators}

Classical operations such as addition (\verb-A+B-), substraction  (\verb+A-B+), product (\verb+A*B+), 
element-wise product (\verb+A.*B+), sub-matrix extraction (\verb+A(indi,indj)+), 
sub-matrix assignment (\verb+A(indi,indj)=B+), right concatenation  (\verb+[A ,B]+), 
down concatenation  (\verb+[A;B]+),  rows deletion (\verb+A(ind,:)=[]+), columns deletion (\verb+A(:,ind)=[]+), 
transposition (\verb+A'+ and \verb+A.'+),  system solving (\verb+A\B+) are available. 
Mixed operations (\verb+A+ sparse and \verb+B+ full or the converse) generally give raise to a 
full matrix (but there are some exceptions like the element-wise product for which it is natural 
to get a sparse one). 

\itemdesc{Some functions}
\begin{itemize}
   \item \verb+size(S)+ returns in a \verb+1 x 2+ vector the dimensions (nb of rows, nb of columns)
of \verb+S+. \verb+size(S,1)+ (resp. \verb+size(S,2)+) returns the number of rows 
(resp. columns) of \verb+S+.
    \item \verb+nnz(S)+ returns the number of nonzero entries in \verb+S+.
    \item \verb+M=full(S)+ converts a sparse matrix \verb+S+ to a full matrix \verb+M+ 
    \item \verb+[I,J,V]=find(S)+ retrieves the \verb+[I,J,V]+ indices triplet 
corresponding to the nonzero entries of \verb+S+ : i.e., the $i^{\mbox{th}}$ nonzero
value \verb+V[i]+ has \verb+(I[i],J[i])+ coordinates.
    \item \verb+[ij,v,mn]=spget(S)+ does the inverse of the call \verb+sparse(ij,v,[mn]);+ 
    \item \verb+umfpack_create(A)+ computes LU factorizations (see \manlink{umfpack_create}{umfpack_create})
    \item \verb+cholmod_create(A)+ computes Cholesky or LDLt factorizations (see
      \manlink{cholmod_create}{cholmod_create})
\end{itemize}

\itemdesc{Some useful tests}
\begin{itemize}
    \item \verb+is(A,%types.SpColMat)+ tests if the argument $A$ is a sparse matrix.
    \item \verb+isreal(A)+ tests if the sparse matrix $A$ is real or complex (see \manlink{isreal}{isreal})
    \item \verb+isvector(A)+ tests if the sparse matrix $A$ is a (sparse) vector (see \manlink{isvector}{isvector})
\end{itemize}


\begin{examples}
\begin{Verbatim}
//
A=sparse((1:10)'*[1 1],ones(1,10));
norm(eye(10,10)-full(A));
B=sparse(diag(ones(10,1));
[I,J,V]=find(A);
[ij,v,mn]=spget(B);
norm(ij- [I' J'])+norm(V'-v);

// built the 1d discrete laplacian matrix
n = 10;
A = sp_create(n,n);
A.set_diag[sparse(2*ones(n,1))]
A.set_diag[sparse(-ones(n-1,1)),1]
A.set_diag[sparse(-ones(n-1,1)),-1]
// built a random rhs
b = randn(n,1)
// solve
x = A\b
// compute residual norm
norm(A*x-b)/norm(b)
\end{Verbatim}
\end{examples}

\begin{manseealso}
\manlink{spones}{spones},
\manlink{spget}{spget},
\manlink{full}{full},
\manlink{sprand}{sprand},
\manlink{speye}{speye},\manlink{norm}{norm},
\manlink{diag}{diag},\manlink{tril}{tril},\manlink{triu}{triu},
\manlink{issymmetric}{issymmetric},\manlink{istriangular}{istriangular},
\manlink{lower_upper_bandwidths}{lower_upper_bandwidths}
\end{manseealso}

% -- Authors
\begin{authors}
   Most of the sparse codes by Jean-Philippe Chancelier and a few by Bruno Pincon
\end{authors}
