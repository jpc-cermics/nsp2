% -*- mode: latex -*-

\mansection{MaxpMat}
\begin{mandesc}
  \short{Mat}{Max plus matrix data type} \\
  \short{m2mp}{Max plus matrix data type}
\end{mandesc}
% -- Calling sequence section
\begin{calling_sequence}
\begin{verbatim}
Ap=maxplus(A)
Ap=m2mp(A)
\end{verbatim}
\end{calling_sequence}
% -- Parameters
\begin{parameters}
  \begin{varlist}
    \vname{A}: matrix numerical object.
  \end{varlist}
\end{parameters}

\begin{mandescription}

The \emph{MaxpMat} data type is the Max plus matrix numerical object. 
Internally numerical matrices \emph{MaxpMat} type are stored by default as 
arrays of \emph{double}.  
\end{mandescription}

\paragraph{Operations on matrices}
\itemdesc{Methods from the matint interface}
\begin{itemize}
\item \verb+A.redim[m,n]+ reshape matrix to size \verb+m+x\verb+n+. \verb+m+ or \verb+n+ can be set to \verb+-1+ 
\item \verb+A.concatr[B]+ \verb+A = [A,B]+
\item \verb+A.concatd[B]+ \verb+A = [A;B]+
\item \verb+A.perm_elem[p,q,dim]+ permute values, rows (\verb+dim=1+) or columns (\verb+dim=2+).
\end{itemize}

\itemdesc{Extraction, insertion, deletion}
\begin{itemize}
   \item If \verb+I+ and is \verb+J+ are respectively row and column indices \verb+A(I,J)+ is the submatrix of \verb+A+ with entries in \verb+I+ and \verb+j+.
   \item \verb+A(I,:)+ is the submatrix of \verb+A+ with row indices in \verb+I+.
   \item \verb+A(:,J)+ is the submatrix of \verb+A+ with comumn indices in \verb+I+.
   \item \verb+A(:,:)+ is \verb+A+
   \item \verb+A(:)+  is the column matrix obtained by stacking the columns of \verb+A+.
\end{itemize}
\begin{itemize}
   \item If \verb+I+ and is \verb+J+ are row and column indices \verb+A(I,J)=B+ inserts
matrix \verb+B+ into matrix \verb+A+ in rows \verb+I+ and columns \verb!J!.
   \item \verb+A(I,:)=[]+ removes the rows of \verb+A+ with indices in \verb+J+.
   \item \verb+A(:,J)=[]+ removes the columns of \verb+A+ with indices in \verb+J+.
\end{itemize}

\itemdesc{Empty matrices}
Empty matrices are matrices with zero rows or zero columns. Operations with empty matrices
are compatible with linear algebra operations. The \verb+m x n+ zero matrix can be 
factorized as the empty \verb+m x 0+ matrix \verb+*+ the \verb+0 x n+ matrix. 

\itemdesc{For loop control}
With a matrix \verb+A+:
\begin{verbatim}
     for col=A
       ....
     end
\end{verbatim} 
is a loop with $size(A,2)$ iterations, the loop 
variable  \verb+col+ being equal to the ith column of \verb+A+ at the i-th iteration.

\itemdesc{Some functions}
\begin{itemize}
   \item \verb+length(A)+ returns the number of entries in \verb+A+.
   \item \verb+size(A)+ returns in a \verb+1 x 2+ vector the dimensions (rows, columns)
of \verb+A+. \verb+size(A,1)+ (resp. \verb+size(A,2)+) retuns the number of rows 
(resp. columns) of \verb+A+.
\end{itemize}


\begin{manseealso}

\end{manseealso}

% -- Authors
\begin{authors}
   jpc, bp
\end{authors}
