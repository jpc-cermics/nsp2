% -*- mode: latex -*-

\mansection{SndFile}
\begin{mandesc}
  \short{SndFile}{SndFile data type}
\end{mandesc}

% -- Calling sequence section
%\begin{calling_sequence}
\subsection*{Constructor}
\begin{verbatim}
L = fopen(file-name,mode='string',xdr=boolean,swap=boolean)
\end{verbatim}
%\end{calling_sequence}
% -- Parameters
\begin{parameters}
  \begin{varlist}
    \vname{mode}: The argument \verb+mode+ is a string beginning  with  one  of  the  following  sequences
    \begin{itemize}
      \item[r]: open text file for reading. The stream is positioned at the beginning of the
        file.
      \item[w] truncate  file  to  zero  length  or create text file for writing.  The stream is
        positioned at the beginning of the file.
      \item[a] open for appending (writing at end of file).  The file is created if it does not 
	exist.  The stream is positioned at the end of the file.
    \end{itemize}
    The mode string can also include the letter b as a last character. 
    Adding the b is used if you do I/O to a binary file 
    (but note that The b is ignored on all POSIX conforming systems, including Linux).
    \vname{swap} If true then byte swaping is forced when using the \verb+mget+ and 
    \verb+mput+ functions. 
    \vname{xdr} If true then reading and writing is used using \verb+xdr+.
  \end{varlist}
\end{parameters}

\begin{mandescription}
The \emph{SndFile} data type is used for file I/O operations.
\end{mandescription}

\paragraph{Operations on SndFile objects}
\itemdesc{attributes}
\begin{itemize}
  \vname{frames}: number of frames in the sound file. 
  \vname{samplerate}: signal sample rate.
  \vname{channels}: number of channels. 
  \vname{format}:  format of sound file. 
  \vname{sections}: number of sections.
  \vname{seekable}: returns a boolean if the file seekable ? 
\end{itemize}
\itemdesc{methods}
\begin{itemize}
  \vname{F.close[]} closes sound file. 
  \vname{F.error[]} gets error code of last error raised. 
  \vname{F.strerror[]} gets error message of last error raised
  \vname{F.error_number[err]} get error message associated to error number.
  \vname{A=F.read[n]}  reads \verb+n+ frames from sound file and returns the 
  result in a numeric matrix of size \verb+mxn+ where \verb+m+ is the 
  number of channels of file
  \vname{F.seek[n,tag]} sets the file position indicator for the stream pointed
       to by stream.  The new position, measured in bytes, is obtained  by  adding
       offset  bytes \verb+n+  to  the  position  specified by \verb+tag+. 
        \verb+tag+ can be set to \verb+"set"+, or \verb+"cur"+, or \verb+"end"+, 
	the offset is relative to the start of the
       file, the current position indicator, or end-of-file, respectively.
  \vname{F.write[A]} write data in sound file from numerical matrix \verb+A+. 
  the number of rows of \verb+A+ must be equal to number of channels of file.
\end{itemize}

\itemdesc{functions}
\begin{itemize}
  \vname{sndfile} opens a sound file. 
    \vname{play(A,job} plays a matrix assuming that the samplerate is 22050.
    The function must be called with job equal to 1 to initialize, to 2 to play 
    and to 3 to close the sound channel.
\end{itemize}

\begin{examples}
  \begin{program}\HCode{rate=22050;n=3;t=linspace(0,n,n*rate);\Hnewline
    x1=sin(2*\%pi*220*t);\Hnewline
    x2=sin(2*\%pi*440*t);\Hnewline
    x3=sin(2*\%pi*660*t);\Hnewline
    x=x1+0.5*x2+0.25*x3;\Hnewline
    F=sndfile('test.wav',mode='w',samplerate=22050,channels=1);\Hnewline
    F.write[x];\Hnewline
    F.close[];\Hnewline
    F=sndfile('test.wav',mode='r');\Hnewline
    F.format\Hnewline
    F.frames }
  \end{program}
\end{examples}

\begin{manseealso}

\end{manseealso}

% -- Authors
\begin{authors}
jpc 
\end{authors}

