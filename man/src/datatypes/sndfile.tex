% -*- mode: latex -*-

\mansection{SndFile}
\begin{mandesc}
  \short{SndFile}{SndFile data type}
\end{mandesc}

% -- Calling sequence section
%\begin{calling_sequence}
\subsection*{Constructor}
\begin{verbatim}
L = sndfile(file, mode=,frames=,samplerate=,channels=,format=,sections=,seekable=)
\end{verbatim}
%\end{calling_sequence}
% -- Parameters
\begin{parameters}
  \begin{varlist}
    \vname{file}: a path name to a file containing sample sounds. 
    \vname{mode}: the argument \verb+mode+ is a string beginning  with  one  of  the  following  sequences
    \begin{itemize}
      \item \verb+"r"+: opens text file for reading. The stream is positioned at the beginning of the  file.
      \item \verb+"w"+: truncates  file  to  zero  length  or create text file for writing.  The stream is
        positioned at the beginning of the file.
      \item \verb+"a"+: opens for appending (writing at end of file).  The file is created if it does not 
	exist.  The stream is positioned at the end of the file.
    \end{itemize}
    The mode string can also include the letter \verb+b+ as a last character. 
    Adding the \verb+b+ is used if you do I/O to a binary file 
    (but note that the  \verb+b+ is ignored on all POSIX conforming systems, including Linux).
  \end{varlist}
\end{parameters}

\begin{mandescription}
The \emph{SndFile} data type is used for file I/O operations.
\end{mandescription}

\paragraph{Operations on SndFile objects}
\itemdesc{attributes}
\begin{itemize}
  \vname{frames}: number of frames in the sound file. 
  \vname{samplerate}: signal sample rate.
  \vname{channels}: number of channels. 
  \vname{format}:  format of sound file. 
  \vname{sections}: number of sections.
  \vname{seekable}: returns a boolean if the file seekable ? 
\end{itemize}
\itemdesc{methods}
\begin{itemize}
  \vname{F.close[]}: closes sound file. 
  \vname{F.error[]}: gets error code of last error raised. 
  \vname{F.strerror[]}: gets error message of last error raised
  \vname{F.error_number[err]}: get error message associated to error number.
  \vname{A=F.read[n]}:  reads \verb+n+ frames from sound file and returns the 
  result in a numeric matrix of size \verb+mxn+ where \verb+m+ is the 
  number of channels of file
  \vname{F.seek[n,tag]}: sets the file position indicator for the stream pointed
       to by stream.  The new position, measured in bytes, is obtained  by  adding
       offset  bytes \verb+n+  to  the  position  specified by \verb+tag+. 
        \verb+tag+ can be set to \verb+"set"+, or \verb+"cur"+, or \verb+"end"+, 
	the offset is relative to the start of the
       file, the current position indicator, or end-of-file, respectively.
  \vname{F.write[A]}: write data in sound file from numerical matrix \verb+A+. 
  the number of rows of \verb+A+ must be equal to number of channels of file.
\end{itemize}

\itemdesc{related functions}
\begin{itemize}
  \vname{sndfile}; opens a sound file. 
  \vname{play}: plays a matrix which can be extract from the sound file, see the example below and the \manlink{play}{play} help page.
  \vname{playfile}: a sound file can be directly played using this function, see \manlink{playfile}{playfile} help page.
\end{itemize}

\begin{examples}
  \begin{Verbatim}
    // write a sound file
    rate=22050;n=3;t=linspace(0,n,n*rate);
    x1=sin(2*%pi*220*t);
    x2=sin(2*%pi*440*t);
    x3=sin(2*%pi*660*t);
    x=x1+0.5*x2+0.25*x3;
    F=sndfile('test.wav',mode='w',samplerate=22050,channels=1);
    F.write[x];
    F.close[];

    // read the sound file
    F=sndfile('test.wav',mode='r');
    F.format
    F.channels
    n=F.frames
    y = F.read[n];
    splrt = F.samplerate
    F.close[]

    // play 
    play(y, samplerate=splrt, sync=%t)
  \end{Verbatim}
\end{examples}

\begin{manseealso}

\end{manseealso}

% -- Authors
\begin{authors}
  Jean-Philippe Chancelier 
\end{authors}

