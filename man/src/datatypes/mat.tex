% -*- mode: latex -*-

\mansection{Mat}
\begin{mandesc}
  \short{Mat}{matrix data type}
\end{mandesc}
% -- Calling sequence section
\begin{calling_sequence}
\begin{verbatim}
A=[a11, a12, ..., a1n;
   a21, a22, ..., a2n;
   ...
   am1, am2, ...; amn]
\end{verbatim}
\end{calling_sequence}
% -- Parameters
\begin{parameters}
  \begin{varlist}
    \vname{a11 ...}: real or complex numbers.
  \end{varlist}
\end{parameters}

\begin{mandescription}

The \emph{Mat} data type is the basic matrix numerical object. 
More generally, matrices are two dimensional arrays which contains
scalar objects (real or complex numbers, boolean numbers, strings,
polynomials). The \emph{Mat} data type is built from real or complex
numbers (real and complex numbers being approximated using
double floating point numbers (see \manlink{number\_properties}{number_properties}).

Column vectors are considered as \verb!m x 1! matrices and row vectors
as \verb!1 x n! matrices.

Internally numerical matrices \emph{Mat} type are stored by default as 
arrays of \emph{double}.

Display of numerical matrices could be changed using the \manlink{format}{format} function.
\end{mandescription}

\paragraph{Operations on matrices}
\itemdesc{Matrix methods}
\begin{itemize}
\item \verb+A.add[B]+  add matrix \verb+B+ to matrix \verb+A+.
\item \verb+A.blas_axpy[alpha, x [i1,i2 [,j1,j2]]]+ constant times a vector plus a vector i.e \verb!A = A + alpha*X!
\item \verb+A.blas_ger[alpha, x, y [,i1,i2,j1,j2]]+  rank one update \verb!A = A + alpha*x*y'!
  on a submatrix of \verb+A+.
\item \verb+A.scale_rows[x]+ multiplies each row i of A with a scalar:
  \verb+A(i,:) = x(i) A(i,:)+. %\verb+x+ must be a (row or column) vector
  % with m components (m number of rows of A).
\item \verb+A.scale_cols[x]+ multiplies each column j of A with a scalar:
  \verb+A(:,j) = x(j) A(:,j)+. %\verb+x+ must be a (row or column) vector
  %with n components (n number of columns of A).
\item \verb+b=A.has[x]+ look for each component of \verb+x+, 
  if it is in \verb+A+: \verb+b+ is a boolean vector 
  (of same dim than \verb+x+) with \verb+b(k)+ true if
  \verb+x(k)+ is in \verb+A+. Additionaly the first index of
  \verb+x(k)+ in \verb+A+ could be provided, either in the one index way
  when 2 output arguments are requested (call of the form \verb+[b,i]=A.has[x]+) or
  either in the two indices way if 3 output arguments are requested
  (\verb+[b,i,j]=A.has[x]+). If \verb+x(k)+ is not in \verb+A+ then
  \verb+i(k)=0+ and \verb+j(k)=0+.
\end{itemize}
\itemdesc{Methods from the matint interface}
\begin{itemize}
\item \verb+A.redim[m,n]+ reshape matrix to size \verb+m+x\verb+n+. \verb+m+ or \verb+n+ can be set to \verb+-1+ 
\item \verb+A.concatr[B]+ \verb+A = [A,B]+
\item \verb+A.concatd[B]+ \verb+A = [A;B]+
\item \verb+A.perm_elem[p,q[,dim]]+ permute p and q: elements
  (default or dim=0), rows (dim=1) or columns (dim=2).
\item \verb+A.set_diag[B [,k]]+ sets the \verb+k+-th diagonal of matrix \verb+A+ to \verb+B+.
\item \verb+A.enlarge[m,n]+ sets the size of \verb+A+ to \verb+(Max(m,size(A,'r')),Max(n,size(A,'c')))+ and fill 
  added elements with default value.
\end{itemize}

\itemdesc{Operators}

Usual operations such as addition (\verb-A+B-), substraction  (\verb+A-B+), product (\verb+A*B+), 
kroneker product (\verb+A.*.B+), power (\verb+A^n+) (\verb+A+ should be square and $n$ a positive (or negative if $A$
is inversible) integer, element-wise product (\verb+A.*B+), element-wise division (\verb+A./B+), 
division of a matrix or vector by a scalar  (\verb+A/s+), 
element-wise power (\verb+A.^n+), right concatenation  (\verb+[A ,B]+), 
down concatenation  (\verb+[A;B]+), transposition (\verb+A'+ and \verb+A.'+), 
linear system solving (\verb+A\x+) are available.

It is worth noting that for addition and substraction one of the operands could be a scalar:
\verb=s+M= or \verb=M+s= returns a matrix of same format than \verb+M+
(even if \verb+M+ is an empty matrix) where the scalar \verb=s= has been added to each
element of \verb+M+. Such shortcuts are also available for element-wise power and division. 

Comparizon operators are denoted \verb+==+, \verb+~=+ or  \verb+<>+ (non equality test), 
\verb+<=+, \verb+<+, \verb+>=+, \verb+>+, and like for  addition and substraction one of
the operands could be a scalar. 

\itemdesc{Extraction, insertion, deletion}
\begin{itemize}
   \item If \verb+I+ and is \verb+J+ are respectively row and column indices \verb+A(I,J)+ is the submatrix of \verb+A+ with entries in \verb+I+ and \verb+j+.
   \item \verb+A(I,:)+ is the submatrix of \verb+A+ with row indices in \verb+I+.
   \item \verb+A(:,J)+ is the submatrix of \verb+A+ with comumn indices in \verb+I+.
   \item \verb+A(:,:)+ is \verb+A+
   \item \verb+A(:)+  is the column matrix obtained by stacking the columns of \verb+A+.
\end{itemize}
\begin{itemize}
   \item If \verb+I+ and is \verb+J+ are row and column indices \verb+A(I,J)=B+ inserts
matrix \verb+B+ into matrix \verb+A+ in rows \verb+I+ and columns \verb!J!.
   \item \verb+A(I,:)=[]+ removes the rows of \verb+A+ with indices in \verb+J+.
   \item \verb+A(:,J)=[]+ removes the columns of \verb+A+ with indices in \verb+J+.
\end{itemize}
See also the \hyperlink{indexing arrays}{indexing arrays, extraction, assignment, rows/cols deletion} help page.

\itemdesc{Some useful tests}
\begin{itemize}
    \item \verb+is(A,%types.Mat)+ or \verb+type(A,"short")=="m"+ or \verb+type(A,"string")=="Mat"+ 
     test if the argument $A$ is a Mat matrix (double float real or complex numbers).
    \item \verb+is(A,%types.Mat) && isreal(A,%t)+ tests if the argument $A$ is a real matrix (see
                     \manlink{isreal}{isreal} help page).
    \item \verb+isvector(A)+ tests if the matrix $A$ is a vector (see \manlink{isvector}{isvector})
    \item \verb+isscalar(A)+ tests if the matrix $A$ is a scalar (see \manlink{isscalar}{isscalar})
\end{itemize}

\itemdesc{Empty matrices}

Empty matrices are matrices with zero rows or zero columns. Operations with empty matrices
are compatible with linear algebra operations. The \verb+m x n+ zero matrix can be 
factorized as the empty \verb+m x 0+ matrix \verb+*+ the \verb+0 x n+ matrix. 



\itemdesc{For loop control}

With a matrix \verb+A+:
\begin{verbatim}
     for col=A
       ....
     end
\end{verbatim} 
is a loop with $size(A,2)$ iterations, the loop 
variable  \verb+col+ being equal to the ith column of \verb+A+ at the i-th iteration.

\itemdesc{Some functions}
\begin{itemize}
   \item \verb+numel(A)+ or \verb+length(A)+ returns the number of entries in \verb+A+.
   \item \verb+size(A)+ returns in a \verb+1 x 2+ vector the dimensions (rows, columns)
of \verb+A+. \verb+size(A,1)+ (resp. \verb+size(A,2)+) retuns the number of rows 
(resp. columns) of \verb+A+.
   \item all \hyperlink{arrays}{basic arrays functions or method} (which works on full arrays of any kind) 
     are naturally available for \emph{Mat} 
   \item For basic functions or methods acting on \emph{Mat} matrix (like \verb+max+, \verb+min+, \verb+minmax+,
     \verb+sum+, \verb+prod+, \verb+diff+, etc...) see \hyperlink{basicnumarrays}{basic functions or methods for numerical matrices}.
    \item Sorting, searching, set function are available for \emph{Mat}  (note that they generally don't work for complex matrices
      though) see \hyperlink{searchandsort}{searching and sorting} chapter.
\end{itemize}


\begin{manseealso}

\end{manseealso}

% -- Authors
\begin{authors}
   Jean-Philippe Chancelier, Bruno Pincon
\end{authors}
