% -*- mode: latex -*-

\mansection{Pmat}
\begin{mandesc}
  \short{Pmat}{matrix data type}
\end{mandesc}
% -- Calling sequence section
\begin{calling_sequence}
\begin{verbatim}
P=[p11, p12, ..., p1n;
   p21, p22, ..., p2n;
   ...
   pm1, pm2, ...; pmn]
\end{verbatim}
\end{calling_sequence}
% -- Parameters
\begin{parameters}
  \begin{varlist}
    \vname{p11 ...}: polynomials
  \end{varlist}
\end{parameters}

\begin{mandescription}
The \emph{Pmat} data type matrix polynomial object. 
Each entry of the matrix is an (univariate) polynomial.
\end{mandescription}

\paragraph{Operations on polynomial matrices}
\paragraph{Herited methods}
\itemdesc{Methods from the matint interface}
\begin{itemize}
\item \verb+A.redim[m,n]+ reshape matrix to size \verb+m+x\verb+n+. \verb+m+ or \verb+n+ can be set to \verb+-1+ 
\item \verb+A.concatr[B]+ \verb+A = [A,B]+
\item \verb+A.concatd[B]+ \verb+A = [A;B]+
\item \verb+A.perm_elem[p,q,dim]+ permute values, rows (\verb+dim=1+) or columns (\verb+dim=2+).
\end{itemize}

\itemdesc{Extraction, insertion, deletion}
\begin{itemize}
   \item If \verb+I+ and is \verb+J+ are respectively row and column indices \verb+A(I,J)+ is the submatrix of \verb+A+ with entries in \verb+I+ and \verb+j+.
   \item \verb+A(I,:)+ is the submatrix of \verb+A+ with row indices in \verb+I+.
   \item \verb+A(:,J)+ is the submatrix of \verb+A+ with comumn indices in \verb+I+.
   \item \verb+A(:,:)+ is \verb+A+
   \item \verb+A(:)+  is the column matrix obtained by stacking the columns of \verb+A+.
\end{itemize}

\begin{itemize}
   \item If \verb+I+ and is \verb+J+ are row and column indices \verb+A(I,J)=B+ inserts
     matrix \verb+B+ into matrix \verb+A+ in rows \verb+I+ and columns \verb!J!.
   \item \verb+A(I,:)=[]+ removes the rows of \verb+A+ with indices in \verb+J+.
   \item \verb+A(:,J)=[]+ removes the columns of \verb+A+ with indices in \verb+J+.
\end{itemize}

\itemdesc{Some functions}
\begin{itemize}
   \item \verb+m2p+ : make a polynomial from its coefficient vector.
   \item \verb+pmat_create+ : polynomial matrix creation.
   \item \verb+length(A)+ returns the number of entries in \verb+A+.
   \item \verb+size(A)+ returns in a \verb+1 x 2+ vector the dimensions (rows, columns)
     of \verb+A+. \verb+size(A,1)+ (resp. \verb+size(A,2)+) retuns the number of rows 
     (resp. columns) of \verb+A+.
\end{itemize}

\begin{manseealso}

\end{manseealso}

% -- Authors
\begin{authors}
   Jean-Philippe Chancelier, Bruno Pin�on
\end{authors}
