% -*- mode: latex -*-

\mansection{Object}
\begin{mandesc}
  \short{Object}{Nsp object data type}
\end{mandesc}

\begin{mandescription}
The \emph{Object} data type is the base class of Nsp objects. 
\end{mandescription}

\paragraph{Operations on Objects}
\itemdesc{attributes}
Attributes values are obtained through the dot operator which can be used to get or set 
values. Note that, when implementing a class, attributes may be restricted to be non writable.
When attributes are internally stored as nsp objects its is possible to implement attribute 
access compatible with set and get chaining rules as in \verb+Obj.attr(2,:)=56+. It is also 
possible to overload the \verb+(.)+ operator in order to access to attributes. 
It is possible to obtain the attribute names of a given object with \verb+__attrs+ or using 
a method as described in next paragraph.
\itemdesc{object methods}
\begin{itemize}
\item \verb+O.set[attr1=val1,attr2=val2,...]+  set attributes values for object \verb+O+ 
\item \verb+O.get[smat1,...]+ get attribute values. Note that attributes can also be obtained with the dot 
  operator.
\item \verb+O.get_method_names[]+ get the method names as a string matrix. If an extra argument is given 
  as in \verb+O.get_method_names[level]+ only the method names for a certain class level are returned. If 
  \verb+level=0+ the base class method names are returned, if \verb+level=1+ the method names inherited from the 
  father class are returned and so on.
\item \verb+O.get_attribute_names[]+ get the attribute names as a string matrix.
\item \verb+O.get_name[]+ get name of object. 
\item \verb+O1.equal[O2]+ test equality. 
\item \verb+O1.not_equal[O2]+ test if \verb!O1! and \verb!O2! are different.
\end{itemize}

\itemdesc{Some functions}
All the nsp objects must implements a set of methods requested for all objects. 
Thus a set of interfaced function at nsp level can accept any object as argument. 
The interface just redirect the computation to the object method. 
This is used for object display (\verb+print+, \verb+sprint+, \verb+fprint+, 
\verb+info+, \verb+sinfo+, \verb+finfo+), for read/write operations (\verb+printf+,\ldots, 
\verb+scanf+), for logical operations, for saving and loading objects (\verb+save+, \verb+load+, 
\verb+serialize+), for obtaining size information on objects, \ldots.

\begin{itemize}
\item \verb+length(A)+ returns the number of entries in \verb+A+.
\item \verb+size(A)+ returns in a \verb+1 x 2+ vector the dimensions (rows, columns)
  of \verb+A+. \verb+size(A,1)+ (resp. \verb+size(A,2)+) returns the number of rows 
  (resp. columns) of \verb+A+.
\end{itemize}

\begin{examples}
  \begin{mintednsp}{nsp}
    a=%types.ClassA.new[cla_color=89,cla_thickness=56];
    // ClassA is a predefined toy class.
    is(a,%types.Mat) 
    is(a,%types.ClassA) 
    a.__attrs
    a.cla_color = 32 
    a.cla_val = rand(2,2)
    a.cla_val(3,3) = 1;
    a.set[cla_val=1:5,cla_color=56]
    a('cla_color') = 78 
    a.get_method_names[];
    // ClassB is a predefined toy class which inherits from ClassA
    b=%types.ClassB.new[cla_color=89,cla_thickness=56];
    b.get_attribute_names[];
    is(b,%types.ClassA) 
    is(b,%types.ClassB) 
    test(b);
  \end{mintednsp}
\end{examples}

\begin{manseealso}

\end{manseealso}


