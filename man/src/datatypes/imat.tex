% -*- mode: latex -*-

\mansection{IMat}
\begin{mandesc}
  \short{IMat}{integer matrix data type}
\end{mandesc}
% -- Calling sequence section
\begin{calling_sequence}
\begin{verbatim}
A=[a11, a12, ..., a1n;
   a21, a22, ..., a2n;
   ...
   am1, am2, ...; amn]
\end{verbatim}
\end{calling_sequence}
% -- Parameters
\begin{parameters}
  \begin{varlist}
    \vname{a11 ...}: integer numbers (IMat scalars)
  \end{varlist}
\end{parameters}

\begin{mandescription}

The \emph{IMat} data type lets to use scalars, vectors or matrices of
integers stored as usual C integers (and not as double floating point numbers). 
Several kinds of integers are available (signed and unsigned integers stored
on 8, 16, 32 and 64 bits). Note that numbers written $1$, $13$, $-67$
are recognized as double float by the nsp interpretor (and so as \manlink{Mat}{Mat}).
Currently signed and unsigned 32 and 64 bits integers can be parsed
directly by adding an \verb+i+ for signed int or  \verb+u+ for unsigned int 
followed by \verb+32+ or \verb+64+ (if this last terminaison is not provided
then \verb+32+ bits are considered). So  \verb+1i+,  \verb+13u+, \verb+-67i64+
are recognized respectively as an int32, an unsigned int32 and a signed int64. 

Other kinds of integers can be obtained using a cast from either the \emph{Mat} type
with the function \verb=m2i(x,int_type_str)=, or either from the \emph{IMat} type with
the method \verb+x.retype[int_type_str]+ where \verb+int_type_str+ could be:
\begin{itemize}
\item \verb+"int8"+ , \verb+"uint8"+ for signed and unsigned 8 bits integers
\item \verb+"int16"+ , \verb+"uint16"+ for signed and unsigned 16 bits integers
\item \verb+"int32"+  \verb+"uint32"+ for signed and unsigned 32 bits integers
\item \verb+"int64"+ , \verb+"uint64"+ for signed and unsigned 64 bits integers
\end{itemize}
Note that the functions \verb+intmin+ and  \verb+intmax+  applied on an \emph{IMat} of
integer type \verb+int_type_str+ returns the limits of this integer type.

Additionnally there are also the following integer types: 'int', 'uint', 
'short', 'ushort', 'long', 'ulong'. In most cases 'int' is equivalent
to 'int32' but you cannot (in general) mix operations between  'int' and
'int32', that is 'int' is a proper nsp integer type. We recommend to use
essentially the first 8 mentionned integer type and not these 6 last ones.  

The method itype[] applied on a IMat lets to known the integer type and returns 
one of these 8+6=14 strings.

To summarize here are some basic examples:
\begin{Verbatim}
// define an imatrix of int32 integer type directly
A = [ 1i, -6i, 123i ;
     -5i, 13i,  67i]

// the method itype lets to known the integer type
A.itype[]

// get int8 integer type by a (in place) cast of A
A.retype["int8"]
A.itype[]

// get int16 integer type by a cast from the Mat type
C = m2i(-8:9, "int16")
C.itype[]

// limits
intmin(1i)
intmax(1i)
intmin(1u)
intmax(1u)
intmin(m2i(1,"int8"))
intmax(1u64)
\end{Verbatim}

%// if you mix both i, u, i64 and u64 : the final integer type is given by the first parsed 
%// biggest (in number of bits) integer type, for instance:
%B = [ 1i, 6789u64, 8i64, 6u ]  // should be of uint64 integer type
%B = [ 1i, 6789i64, 8u64, 6u ]  // should be of int64 integer type

\end{mandescription}

\paragraph{Operations on matrices}


\itemdesc{Matrix methods}
\begin{itemize}
\item \verb+A.itype[]+ returns the integer type of the \emph{IMat} A.
\item \verb+A.retype[int_type_str]+ in place cast of A onto the integer type specified by the string \verb+int_type_str+.
\item \verb+A.add[B]+  add the IMat matrix \verb+B+ to the IMat matrix \verb+A+ (\verb+A+ and \verb+B+ must have
the same  integer type).
\item \verb+A.scale_rows[x]+ multiplies each row i of A with the scalar \verb+x(i)+:
  \verb+A(i,:) = x(i) A(i,:)+.  (\verb+A+ and \verb+x+ must have the same  integer type). See \manlink{scale_rows}{scale_rows} help page.
\item \verb+A.scale_cols[x]+ multiplies each column j of A with a scalar:
  \verb+A(:,j) = x(j) A(:,j)+. (\verb+A+ and \verb+x+ must have the same  integer type).See \manlink{scale_cols}{scale_cols} help page.
\item \verb+b=A.has[x]+ look for each component of \verb+x+, 
  if it is in \verb+A+: \verb+b+ is a boolean vector 
  (of same dim than \verb+x+) with \verb+b(k)+ true if
  \verb+x(k)+ is in \verb+A+. Additionaly the first index of
  \verb+x(k)+ in \verb+A+ could be provided, either in the one index way
  when 2 output arguments are requested (call of the form \verb+[b,i]=A.has[x]+) or
  either in the two indices way if 3 output arguments are requested
  (\verb+[b,i,j]=A.has[x]+). If \verb+x(k)+ is not in \verb+A+ then
  \verb+i(k)=0+ and \verb+j(k)=0+. (\verb+A+ and \verb+x+ must have the same  integer type). See
  \manlink{has}{has} help page.
\item \verb+A.nnz[]+ returns the numbers of non zero elements of A.
\end{itemize}

\itemdesc{Methods from the matint interface}
\begin{itemize}
\item \verb+A.redim[m,n]+ reshape matrix to size \verb+m+x\verb+n+. \verb+m+ or \verb+n+ can be set to \verb+-1+. See
  \manlink{redim}{redim} help page.
\item \verb+A.concatr[B]+ \verb+A = [A,B]+. Here mixed integer types are possible, the rule being that the first biggest
 integer type (in number of bits) is the final type (if number of bits of A is equal of greather to the number of bits 
of B then A don't  change of integer type, otherwise A is casted to the integer type of B). 
\item \verb+A.concatd[B]+ \verb+A = [A;B]+. Same remarks than before for the final type of A.
\item \verb+A.perm_elem[p,q[,dim]]+ permute p and q: elements
  (default or dim=0), rows (dim=1) or columns (dim=2).
\item \verb+A.set_diag[B [,k]]+ sets the \verb+k+-th diagonal of matrix \verb+A+ to \verb+B+. See
  \manlink{set_diag}{set_diag} help page.
\end{itemize}

\itemdesc{Extraction, insertion, deletion}
\begin{itemize}
   \item If \verb+I+ and is \verb+J+ are respectively row and column indices \verb+A(I,J)+ is the submatrix of \verb+A+ with entries in \verb+I+ and \verb+j+.
   \item \verb+A(I,:)+ is the submatrix of \verb+A+ with row indices in \verb+I+.
   \item \verb+A(:,J)+ is the submatrix of \verb+A+ with comumn indices in \verb+I+.
   \item \verb+A(:,:)+ is \verb+A+
   \item \verb+A(:)+  is the column matrix obtained by stacking the columns of \verb+A+.
\end{itemize}
\begin{itemize}
   \item If \verb+I+ and is \verb+J+ are row and column indices \verb+A(I,J)=B+ inserts
matrix \verb+B+ into matrix \verb+A+ in rows \verb+I+ and columns \verb!J!.
   \item \verb+A(I,:)=[]+ removes the rows of \verb+A+ with indices in \verb+J+.
   \item \verb+A(:,J)=[]+ removes the columns of \verb+A+ with indices in \verb+J+.
\end{itemize}
See also the \hyperlink{indexing arrays}{indexing arrays, extraction, assignment, rows/cols deletion} help page.

\itemdesc{Operators}

Usual operations such as addition (\verb-A+B-), substraction  (\verb+A-B+), matrix scalar or scalar
matrix product (\verb+A*B+) (matrix product is not defined), element-wise product (\verb+A.*B+), 
kroneker product (\verb+A.*.B+),
element-wise division (\verb+A./B+), right concatenation  (\verb+[A ,B]+), down concatenation  (\verb+[A;B]+), 
transposition (\verb+A'+) are available. 

{\bf Important note:} integer overflow (an operation leading to an integer greather than
the intmax limit) and underflow  (an operation leading to an integer less than
the intmin limit) don't raise any warning or error. Results are computed following the
rules of modular arithmetic, for instance:
\begin{Verbatim}
intmax(1i) + 1i  // should be equal to intmin(1i)
intmax(1u64) + 1u64 // should be equal to 0 (which is intmin(1u64))
m2i(127,"int8") + m2i(1,"int8")  // should be equal to -128  (which is intmin(m2i(1,"int8")))
\end{Verbatim}

It is worth noting that for addition and substraction one of the operands could be a scalar:
\verb=s+M= or \verb=M+s= returns a matrix of same format than \verb+M+
(even if \verb+M+ is an empty matrix) where the scalar \verb=s= has been added to each
element of \verb+M+. Such shortcuts are also available for element-wise power and division. 

Comparizon operators are denoted \verb+==+, \verb+~=+ or  \verb+<>+ (non equality test), 
\verb+<=+, \verb+<+, \verb+>=+, \verb+>+, and like for  addition and substraction one of
the operands could be a scalar. 

\itemdesc{Some useful tests}
\begin{itemize}
    \item \verb+is(A,%types.IMat)+ or \verb+type(A,"short")=="im"+ or \verb+type(A,"string")=="IMat"+ 
     test if the argument $A$ is an IMat matrix.
    \item \verb+is(A,%types.IMat) && A.itype[]=="int32" + tests if the argument $A$ is an IMat matrix of int32 integer type.
    \item \verb+isvector(A)+ tests if the matrix $A$ is a vector (see \manlink{isvector}{isvector})
    \item \verb+isscalar(A)+ tests if the matrix $A$ is a scalar (see \manlink{isscalar}{isscalar})
    \item \verb+is(A,%types.IMat) && A.itype[]=="int32" && isscalar(A)+ tests if the argument $A$ is a IMat scalar of
                     int32 integer type.
\end{itemize}

\itemdesc{For loop control}

With a matrix \verb+A+:
\begin{verbatim}
     for col=A
       ....
     end
\end{verbatim} 
is a loop with $size(A,2)$ iterations, the loop 
variable  \verb+col+ being equal to the ith column of \verb+A+ at the i-th iteration.

\itemdesc{Some functions}
\begin{itemize}
   \item \verb+numel(A)+ or \verb+length(A)+ returns the number of entries in \verb+A+.
   \item \verb+size(A)+ returns in a \verb+1 x 2+ vector the dimensions (rows, columns)
of \verb+A+. \verb+size(A,1)+ (resp. \verb+size(A,2)+) retuns the number of rows 
(resp. columns) of \verb+A+.
   \item \verb+i2m(A)+ returns a new \emph{Mat} matrix from the \emph{IMat} matrix A (this could be useful for
    operations defined for floating point matrices and not for integer's ones).  
   \item all \hyperlink{arrays}{basic arrays functions or method} (which works on full arrays of any kind) 
     are naturally available for \emph{IMat} 
    \item Most sorting, searching, set functions are available for \emph{IMat} see \hyperlink{searchandsort}{searching
      and sorting} chapter.
   \item Also many \hyperlink{basicnumarrays}{basic functions or methods for numerical matrices} like \verb+min+, 
      \verb+max+, \verb+minmax+, \verb+sum+, \verb+cumsum+, \verb+prod+,  \verb+cumprod+, \verb+diff+ are available
       for \emph{IMat}
\end{itemize}


% -- Authors
\begin{authors}
   Jean-Philippe Chancelier, Bruno Pincon
\end{authors}
