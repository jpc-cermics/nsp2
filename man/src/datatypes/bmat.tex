% -*- mode: latex -*-

\mansection{Bmat}
\begin{mandesc}
  \short{Bmat}{boolean matrix data type} \\
\end{mandesc}
% -- Calling sequence section
\begin{calling_sequence}
\begin{verbatim}
A=[a11, a12, ..., a1n;
   a21, a22, ..., a2n;
   ...
   am1, am2, ...; amn]
\end{verbatim}
\end{calling_sequence}
% -- Parameters
\begin{parameters}
  \begin{varlist}
    \vname{a11 ...}: booleans
  \end{varlist}
\end{parameters}

\begin{mandescription}
The \emph{Bmat} data type is a matrix made of booleans. 
A boolean is either \verb+T+ (true) or \verb+F+ (false). 
\verb+T+ is the nsp variable \verb+%t+ (same as \verb+1==1+)
and \verb+F+ is the nsp variable \verb+%f+ (same as \verb+1~=1+).
In conditional expressions \verb+T+ is equivalent to any nonzero
number and \verb+F+ is equivalent to zero.
\end{mandescription}

\paragraph{Operations on boolean matrices}
\itemdesc{Methods from the matint interface}
\begin{itemize}
\item \verb+A.redim[m,n]+ reshape matrix to size \verb+m+x\verb+n+. \verb+m+ or \verb+n+ can be set to \verb+-1+
\item \verb+A.concatr[B]+ \verb+A = [A,B]+
\item \verb+A.concatd[B]+ \verb+A = [A;B]+
\item \verb+A.perm_elem[p,q[,dim]]+ permute p and q: elements
  (default or dim=0), rows (dim=1) or columns (dim=2).
\item \verb+A.set_diag[B [,k]]+ sets the \verb+k+-th diagonal of matrix \verb+A+ to \verb+B+.
\item \verb+A.enlarge[m,n]+ sets the size of \verb+A+ to \verb+(Max(m,size(A,'r')),Max(n,size(A,'c')))+ and fill 
  added elements with default value.
\end{itemize}

\itemdesc{Extraction, insertion, deletion}
\begin{itemize}
   \item If \verb+I+ and is \verb+J+ are respectively row and column indices \verb+A(I,J)+ is the submatrix of \verb+A+ with entries in \verb+I+ and \verb+j+.
   \item \verb+A(I,:)+ is the submatrix of \verb+A+ with row indices in \verb+I+.
   \item \verb+A(:,J)+ is the submatrix of \verb+A+ with comumn indices in \verb+I+.
   \item \verb+A(:,:)+ is \verb+A+
   \item \verb+A(:)+  is the column matrix obtained by stacking the columns of \verb+A+.
\end{itemize}
\begin{itemize}
   \item If \verb+I+ and is \verb+J+ are row and column indices \verb+A(I,J)=B+ inserts
matrix \verb+B+ into matrix \verb+A+ in rows \verb+I+ and columns \verb!J!.
   \item \verb+A(I,:)=[]+ removes the rows of \verb+A+ with indices in \verb+J+.
   \item \verb+A(:,J)=[]+ removes the columns of \verb+A+ with indices in \verb+J+.
\end{itemize}

\itemdesc{Compatibility with numerical matrices}

Boolean matrices are converted into numerical matrices with \verb+T+ converted to 1 
and \verb+F+ converted to 0. 

\itemdesc{Example of boolean function or operators}
\begin{itemize}
\item \verb+bmat_create(m,n)+ make a $m \times n$ boolean matrix filled with \verb+T+ values
  (use \verb+~bmat_create(m,n)+ to make a boolean matrix filled with \verb+F+ values).
\item \verb+b2m+ : explicit conversion from boolean matrix to numeric matrix.
\item \verb+m2b+ : explicit conversion from numeric matrix to boolean matrix.
\item  \manlink{\tt and, \&, \&\&}{and}: AND logical function and operators.
\item  \manlink{\tt or, |, ||}{or}: OR logical function and operators.
\item  \manlink{\tt not, \~}{not}: NOT logical function/operator.
\item  \manlink{find}{find}: give indices corresponding to true values.

\end{itemize}



\itemdesc{Indexing using boolean vectors as index vectors}  

Boolean vectors could be used instead of numerical ones either for usual 
indexing or for linear indexing (see \manlink{indexing arrays}{indexing arrays} help page). 
The convention is simply that true values are replaced by their corresponding index while 
false values are not used. So the boolean vector \verb+[%t, %f, %f, %t, %f, %t]+ is 
converted as the index vector \verb+[1, 4, 6]+. The actual usage is for expression like:
\begin{Verbatim}
     x = randn(1,8)
     x( x < 0 ) = 0  // set all negative values to zero
\end{Verbatim}
 

\begin{manseealso}
\manlink{all}{all},\manlink{any}{any} 
\end{manseealso}

% -- Authors
\begin{authors}
   Jean-Philippe Chancelier, Bruno Pincon
\end{authors}
