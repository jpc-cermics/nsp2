% -*- mode: latex -*-

\mansection{cholmod}
\begin{mandesc}
  \short{Cholmod}{datatype for Cholesky or LDLt factorization of sparse matrices}\\
  \short{cholmod_create}{creates a Cholmod object from a sparse matrix}
\end{mandesc}

% -- Calling sequence section
\begin{calling_sequence}
\begin{verbatim}
C=cholmod_create(A, type=, mode=, beta=, ordering=, perm=)
[C, m] = cholmod_create(A, type=, mode=, beta=, ordering=, perm=)
\end{verbatim}
\end{calling_sequence}
% -- Parameters
\begin{parameters}
  \begin{varlist}
    \vname{A}: a real or complex sparse symetric matrix
    \vname{C}: a Cholmod nsp object.
    \vname{m}: an integer
    \vname{type,mode,beta,ordering, perm}: optional named argument. 
  \end{varlist}
\end{parameters}

\begin{mandescription}

The \emph{Cholmod} data type (\verb+type(C,'short')== 'cholmod'+) is used to 
encapsulate  ``LDLt factorizations'' of real symmetric or complex hermitian
sparse matrix:
$$
P A P' = L D L'
$$
or ``Cholesky factorizations'' of real symmetric or complex hermitian positive definite 
sparse matrix (spd):
$$
P A P' = L L'
$$
using the Tim Davis' Cholmod library. In these expressions $P$ denote a permutation 
matrix, $D$ a diagonal matrix and $L$ a lower triangular matrix (with unit diagonal 
for $LDL'$ factorization). If we use the permutation vector \verb+p+ associated
to the permutation matrix $P'$ then these factorizations reads 
\verb+ A(p,p) = LDL'+ or  \verb+ A(p,p) = LL'+. 

It is possible to check using the boolean variable \verb+%cholmod+ 
if your current nsp version has been compiled with cholmod support.
 
One interesting feature of \emph{Cholmod} is that you can update or downdate
the factorization due to modification of the initial matrix of the form:
$$
  \tilde{A} = A \pm v v^{\top} = A \pm \sum_{k=1}^p v^k (v^k)^{\top}
$$
where $v$ is a (sparse) column vector ($p=1$) for rank one update or a (sparse) matrix
with (generally) a few column vectors ($p$ much less than the matrix order $n$) for multiple 
rank update. This is interesting because updating the factorization of $A$ to get the 
factorization of $\tilde{A}$ is faster than computing it from scratch. Note that updating
or downdating is available only for factorizations of real matrices.

Finally note that the ``LDLt'' method factorize postive definite matrices like the
``LLt'' method but also indefinite matrices whose leading minors are well conditionned
(but it could not handle all kind of symmetric inversible matrices).
\end{mandescription}. 

\paragraph{cholmod\_create}
This function build a Cholmod nsp object. Note that when the factorization could not
be achieved (the matrix is not positive definite for the ``LLt'' method or a zero
pivot occurs in the ``LDLt'' method) \verb+cholmod_create+ raises an error unless 
the second output argument \verb+m+ is required. In this case $m$ equal $0$ if the factorization 
have been completed and $m$ take a value between $1$ and $n$ (the matrix order) stating
that the failure occurs at step $m$ of the factorization, then $D_{m,m}=0$ (in the ``LDLt''
case) or $L_{m,m}=0$  (in the ``LLt'' case).
 
The optional arguments have the following meaning:
\begin{varlist}
  \vname{mode}: a string, should \verb+"ldl'"+ (enter \verb+"ldl''"+ because of the ' character which is
   a string delimiter) for a $LDL'$ factorization (this is the default)
                or \verb+"ll'"+ (enter  \verb+"ll''"+) for a $LL'$ factorization.
  \vname{type}: a string among \verb+"row", "col", "sym", "lo", "up"+ (default is \verb+"up"+).
     \begin{itemize} 
        \item \verb+type="up"+ use only the upper part of $A$ to factorize $A$
        \item  \verb+type="lo"+ use only the lower part of $A$ to factorize $A$.
        \item  \verb+type="sym"+ use all the matrix (verify the matrix symmetry)
        \item  \verb+type="col"+ factorize $A'A$.
        \item  \verb+type="row"+ factorize $AA'$.
     \end{itemize}

  \vname{beta}: real scalar (default 0), use this option to factorize $\beta I + A$ (or $\beta I + A'*A$ if \verb+type="col"+
                or $\beta I + A*A'$ if \verb+type="row"+).

  \vname{ordering}: an int, lets to choose the ordering strategy if perm is not given. Ordering
  corresponds to find a ``good'' permutation $p$ such that the fill-in of the factorization of
  $P' A P$ is not too high ($P$ being the permutation matrix associated to $p$). Indeed
  the factorization of a sparse matrix could have many more non-zero elements than the initial matrix 
  $A$ (this is what is called fill-in) and so use huge memory and time. Hence it is important to look 
  for permutations that minimize this fill-in. There are several possibilities (some
  depending if Cholmod have been compiled with the Metis support or not):
     \begin{itemize} 
        \item $0$ use natural ordering (no reordering)
        \item $-1$ (default) use default strategy which consists in trying AMD then METIS (if AMD
              leads to a too big fill-in)
        \item $-2$ use default strategy with NESDIS in place of METIS
        \item $-3$ use AMD ordering only
        \item $-4$ use METIS ordering only
        \item $-5$ use NESDIS ordering only
        \item $-6$ use natural ordering with etree postreordering
     \end{itemize}
  In fact if Cholmod have been compiled with the Metis support which will be likely the case, both
  METIS and NESDIS are unavailable (in this case the default strategy is used and so only AMD
  ordering). Once the factorization have been computed the method \verb+get_ordering+ lets to
  known what ordering method have been used.
 
  \vname{perm}: a permutation vector. In this case the ordering is set to be the one from perm.

\end{varlist}

\paragraph{Operations on cholmod}
The following methods are available for Cholmod objects~:
\begin{varlist}
  \vname{C.solve[b,mode=string]}: \verb+C+ being the Cholmod object created from a sparse matrix $A$
  this method could solve various linear systems depending on the optional named arg \verb+mode+ (in the
  following reads $D$ as the identity matrix if a $LL'$ factorization have been computed instead of a  
  $LDL'$ one):
     \begin{itemize} 
        \item when \verb+mode="A"+ (default) it solves $Ax=b$ 
        \item when \verb+mode="LDLt"+ it solves $LDL' x = b$
        \item when \verb+mode="Lt"+ it solves $L' x = b$  
        \item when \verb+mode="LD"+ it solves $LD x = b$
        \item when \verb+mode="D"+ it solves $D x = b$
        \item when \verb+mode="DLt"+ it solves $DL' x = b$
        \item when \verb+mode="P"+ it solves $P x = b$
        \item when \verb+mode="L"+ it solves $L x = b$
        \item when \verb+mode="Pt"+ it solves $P' x = b$
     \end{itemize}
  \vname{isreal}: test if the cholmod object is the factorization of a real (sparse) matrix.
                  \verb+b = C.isreal[]+ returns true if the factorization comes from a real
                  sparse matrix and false if it comes from a complex one.
  \vname{get_ld}: returns the $LD$ part of the factorization. When a $LL'$
        factorization have been computed  it returns the matrix $L$ and when a $LDL'$
        factorization have been computed it returns the matrix $L$ (which has a unit diagonal) with
        the diagonal matrix $D$ stored on its diagonal. In this case if you want to retrieve
        the exact matrix $L$ and $D$ use:
        \begin{Verbatim}
        [L,m,p] = C.get_ld[]
        D = diag(L);  // use D = diag(diag(L)) to get D as a (diagonal) matrix 
        L.set_diag[spones(D)]
        \end{Verbatim}
        As the example shows you can get also the minor and the permutation as second and third
        output arguments.
  \vname{get_minor}: returns the value of $m$ (0 if the factorization has been completly built).
  \vname{get_perm}: returns the permutation $p$ of the factorization (as a permutation vector)         
  \vname{update}: update the factorization due to a multiple rank modification
        of the matrix of the form $A + vv'$. The argument of this method is not exactly the vector
        or matrix $v$: its rows should be permuted using \verb+C.update[v(p,:)]+ where $p$ is 
        the permutation of the factorization.
  \vname{downdate}: \verb+C.downdate[v(p,:)]+ update the factorization due to a multiple rank modification
        of the matrix of the form $A - vv'$.
  \vname{resymbol}: to be done          
  \vname{get_rcond}: returns a rough estimate of the reciprocal condition number (using 
        \verb+min(diag(L))/max(diag(L))+ in case an $LL'$ factorization and 
        \verb+min(diag(abs(D))/max(diag(abs(D)))+ for a $LDL'$ factorization.
  \vname{get_ordering}: returns the ordering strategy effectively used to factorize
                        (0: natural ordering, 1: given permutation, 2: AMD, 3: METIS, 4: NESDIS, 5: AMD for A and COLAMD
  for AA', 6: natural ordering postordered)      
  \vname{get_lnz}: returns the number of non zeros elements of matrix $L$.        
  \vname{get_fl}: to be done            
  \vname{get_memory}: to be done         
\end{varlist}

The update/downdate methods works only on ``LDLt'' factorization of real matrices but in case you have 
computed initially a ``LLt'' factorization (of a real matrix) it is silently transformed into a ``LDLt'' one.
 
\begin{examples}
 \paragraph{simple example}
  \begin{Verbatim}
n = 5;
v = sparse(ones(n-1,1));
A = 2*speye(n,n) - diag(v,1) - diag(v,-1);
Af = full(A)

// 1/ try an LL' factorization with natural ordering
C = cholmod_create(A,mode="ll''",ordering=0);
// some verifications
[L,m,p] = C.get_ld[];
// p should be 1:5 and L should correspond to the Cholesky factorization of Af
chol(Af)' - full(L)   // chol provides the L' so transpose

// 2/ try an LDL' factorization
C = cholmod_create(A);
// some verifications
[L,m,p] = C.get_ld[];
D = diag(diag(L));
L.set_diag[sparse(ones(n,1))]
L*D*L' - A(p,p)  // should be a zero matrix (up to some precision)

// solving a linear system
b = randn(n,1);
x = C.solve[b]
norm(A*x-b)/norm(b)

// 3/ multiple rank update
vf = [0,0.2; 0,-0.6; 0.5,0 ;0,0; 1,0];
v = sparse(vf);
Am = A + v*v';
C.update[v(p,:)];
// verification
[L,m,p] = C.get_ld[];
D = diag(diag(L));
L.set_diag[sparse(ones(n,1))]
L*D*L' - Am(p,p)  // should be a zero matrix (up to some precision)

// 4/ now downdate (retrieve the factorization of A)
An = Am - v*v';
C.downdate[v(p,:)];
// verification
[L,m,p] = C.get_ld[];
D = diag(diag(L));
L.set_diag[sparse(ones(n,1))]
L*D*L' - A(p,p)  // should be a zero matrix (up to some precision)


  \end{Verbatim}
 \end{examples}

% -- Authors
\begin{authors}
   cholmod lib: Tim Davis, nsp interface: Jean-Philippe Chancelier (the interface uses also
   some routines from the cholmod matlab interface of the cholmod lib).
\end{authors}
