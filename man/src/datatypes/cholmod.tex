% -*- mode: latex -*-

\mansection{cholmod}
\begin{mandesc}
  \short{Cholmod}{datatype for cholmod factorization of sparse matrices}
  \short{cholmod_create}{creates a Cholmod object from a sparse matrix} 
\end{mandesc}

% -- Calling sequence section
\begin{calling_sequence}
\begin{verbatim}
Au=cholmod_create(As, type=, mode=, beta=, ordering=,perm=)
\end{verbatim}
\end{calling_sequence}
% -- Parameters
\begin{parameters}
  \begin{varlist}
    \vname{As}: a real or complex sparse matrix
    \vname{Au}: a Cholmod nsp object.
    \vname{type,mode,beta,ordering, perm}: optional named argument. 
  \end{varlist}
\end{parameters}

\begin{mandescription}
The \emph{Cholmod} data type (\verb+type(Au,'short')== 'cholmod'+) is used to 
encapsulate Cholmod sparse Cholesky factorization (provided by Tim Davis' Cholmod library). 
\end{mandescription}. It is possible to check using the boolean variable \verb+%cholmod+ 
if your current nsp version has been compiled with cholmod support. The \verb+solve+ method 
can then be used to solve linear systems. Objects of \emph{Cholmod} type are usefull if 
repeated use of the \verb+solve+ method for the same matrix is to be used, 
since the factorization is kept in the \emph{Cholmod} object. 
\paragraph{Operations on cholmod}
The following methods are available for Cholmod objects~:
\begin{varlist}
  \vname{A.solve[B]}
  \vname{isreal}
  \vname{get_ld}
  \vname{get_minor}        
  \vname{get_perm}         
  \vname{update}          
  \vname{downdate}           
  \vname{resymbol}           
  \vname{get_rcond}          
  \vname{get_ordering}       
  \vname{get_lnz}            
  \vname{get_fl}            
  \vname{get_memory}         
\end{varlist}

\begin{examples}
  \begin{program}\HCode{Af = sprand(10,10,0.8);\Hnewline
      A= cholmod_create(Af);\Hnewline
      A.isreal[]\Hnewline
      A.det[] \Hnewline
      b=rand(10,2);\Hnewline 
      xu= A.solve[b];\Hnewline 
      norm(Af*xu- b)\Hnewline 
      xu1 = cholmod_solve(Af,b);} 
  \end{program}
 \end{examples}

% -- Authors
\begin{authors}
  Jean-Philippe Chancelier
\end{authors}
