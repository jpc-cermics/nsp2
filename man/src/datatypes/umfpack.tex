% -*- mode: latex -*-

\mansection{umfpack}
\begin{mandesc}
  \short{Umfpack}{datatype for umfpack factorization of sparse matrices}\\
  \shortunder{umfpack\_create}{umfpack_create}{creates a Umfpack object from a sparse matrix}\\
  \shortunder{umfpack\_solve}{umfpack_solve}{solves a linear system with umfpack}
\end{mandesc}

% -- Calling sequence section
\begin{calling_sequence}
\begin{verbatim}
Au=umfpack_create(As)
xf=umfpack_solve(As,Bf,mode=string,irstep=int)
\end{verbatim}
\end{calling_sequence}
% -- Parameters
\begin{parameters}
  \begin{varlist}
    \vname{As}: a real or complex sparse matrix
    \vname{Au}: An Umfpack nsp object.
    \vname{Bf,xf}: real or complex full matrices 
  \end{varlist}
\end{parameters}

\begin{mandescription}
The \emph{Umfpack} data type (\verb+type(Au,'short')== 'umfpack'+) is used to 
encapsulate Umfpack LU factorization of matrices (provided by Tim Davis' Umfpack library). 
\end{mandescription}. It is possible to check using the boolean variable \verb+%umfpack+ 
if your current nsp version has been compiled with umfpack support. The \verb+solve+ method 
can then be used to solve linear systems. \emph{Umfpack} data type are usefull if 
repeated use of the \verb+solve+ method for the same matrix is to be used, 
since the factorization is kept in the \emph{Umfpack} object. When just one linear 
system is to be solved the simplified version \verb+xf=umfpack_solve(As,Bf)+ can be used. 
The optional arguments of \verb+umfpack_solve+ are described below in the umfpack method
\verb+solve+.

\paragraph{Operations on umfpack}
The following methods are available for Umfpack objects:
\begin{varlist}
  \vname{A.solve[B,mode=string,irstep=int]}: solves the linear system \verb+A*x=B\verb+ (when mode is not given), \verb+B+ is given as a full 
   matrix. The optional parameter \verb+irstep+ can be used to set the maximal number of refinement iterations.
   The optional parameter \verb+mode+ can be used to specify the linear system to solve. Possible values are~:
     \begin{itemize} 
        \item \verb+mode='Pt_L'+ for solving \verb+P'Lx=b+
        \item \verb+mode='L'+for solving \verb+Lx=b+
        \item \verb+mode='Lt_P'+for solving \verb+L'Px = b+
        \item \verb+mode='Lat_P'+for solving \verb+L.'Px=b+
        \item \verb+mode='Lt'+for solving \verb+L.'x=b+
        \item \verb+mode='U_Qt'+for solving \verb+UQ'x=b+
        \item \verb+mode='U'+for solving \verb+Ux=b+
        \item \verb+mode='Q_Ut'+for solving \verb+QU'x=b+
        \item \verb+mode='Q_Uat'+for solving \verb+QU.'x=b+
        \item \verb+mode='Ut'+for solving \verb+U'x=b+
        \item \verb+mode='Uat'+for solving \verb+U.'x=b+
     \end{itemize}
  \vname{[L,U,p,q,r]=A.luget[]}: returns the \verb+LU+ factorization contained in Umfpack object \verb+A+ i.e the following relation should be met~: \verb+Ap= diag(1 ./ r) *A+ and 
  \verb+L*U=Ap(p,q)+ 
  \vname{A.isreal[]}: returns a boolean to check if the given Umfpack object comes from a real 
  (\verb+%t+) or complex (\verb+%f+) matrix. 
  \vname{A.det[]}: returns the determinant of Umfpack object \verb+A+.
  \vname{A.rcond[]}: returns an estimation of the reciprocal condition
  number in 1-norm (using lapack routines dlacon/zlacon) of Umfpack object \verb+A+.
  \end{varlist}

\begin{examples}
\begin{Verbatim}
Af = sprand(10,10,0.8);
A= umfpack_create(Af);
A.isreal[]
A.det[] 
A.rcond[] 
b=rand(10,2); 
xu= A.solve[b]; 
norm(Af*xu- b) 
xu1 = umfpack_solve(Af,b);
\end{Verbatim}
\end{examples}

\begin{manseealso}
\manlink{Cholmod}{Cholmod}, \manlink{lu}{lu}
\end{manseealso}


% -- Authors
\begin{authors}
   umfpack lib: Tim Davis, nsp interface: Jean-Philippe Chancelier,
   Bruno Pincon (rcond)
\end{authors}
